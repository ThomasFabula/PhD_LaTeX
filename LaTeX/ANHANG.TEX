%\pagenumbering{roman}

\chapter*{Anhang:\\ Experimenteller Meßaufbau}
\label{anhang}

Im Rahmen der vorliegenden Arbeit wurden experimentelle Meßaufbauten zur
meßtechnischen Charakterisierung des statischen und dynamischen Verhaltens
mikromechanischer Strukturen konzipiert und aufgebaut. Um den
Entwicklungsprozeß mikromechanisch gefertigter Mikroresonatoren im Rahmen
des BMFT-Verbundprojektes auch meßtechnisch unterstützen zu können, mußten
insbesondere neben den passiven Eigenschaften (Resonanzfrequenz, Amplituden,
Schwingungsgüten) der Mikroresonatoren aus Silizium, die Sensoreffekte
und die Störeinflüsse, wie temperatur- oder technologisch bedingte
Verspannungen, die bei Multilayerstrukturen auftreten, untersucht werden.
Die experimentellen Daten dienten hierbei einerseits dazu um Rückschlüsse auf
technologische Prozeßschritte ziehen zu können, andererseits der
Entwicklung geeigneter FE-Modelle und der meßtechnischen Verifikation der
numerischen Berechnungsergebnisse.\\

In {\bf Abbildung~A} ist der gesamte experimentelle Meßaufbau zur
Charakterisierung mikromechanischer Strukturen schematisch skizziert.
Der gesamte Meßplatz ist modular aufgebaut und im Rahmen verschiedener
Praktikums- und Diplomarbeiten den Meßaufgaben angepaßt und erweitert
worden. Der Meßplatz besteht aus dem optischen Nachweisverfahren mittels
eines Laservibrometers, der mechanischen Druck- bzw.\ hier nicht
dargestellten Kraftbeaufschlagung, einem ansteuerbaren
Temperaturtisch und einem PC-Meßwerterfassungssystem, das eine
rechnergestützte Datenaufnahme und -speicherung über standardisierte
{\sf HP-IB}-Schnittstellen ({\em IEEE488}) ermöglicht \cite{Mue92}.


{\bf Fremdanregung:}

Es mußten experimentelle Möglichkeiten geschaffen werden, die
Mikrostrukturen ohne Dünnschichtsystem mit Hilfe von externen
Energiequellen zum Schwingen anzuregen. Für den niederen Frequenzbereich
bis etwa 20~kHz können kommerzielle elektromagnetische Schwingungserzeuger,
wie beispielsweise der Miniatur-Schwingtisch von Brüel \& Kj\ae r
(Typ B\&K-4809) verwendet werden. Mit Hilfe eines elektrischen
Leistungsverstärkers (Typ B\&K-2706) war es möglich bei
Membranstrukturen auch relativ große dynamische Schwingungsamplituden
von einigen Mikrometern zu erzeugen. Zur Anregung hochfrequenter
Schwingungsmoden, insbesonderer auch Resonatoren geringer Abmessungen
($l$ = 2--3~mm), wurden Piezokeramiken (PZT) extern angebracht, um die
Mikrostrukturen über
Körperschall akustisch zum Schwingen anzuregen. Hierbei war zu beachten,
daß sowohl die Art der Sensorhalterung, als auch das Schwingungsverhalten
der Piezokeramik selbst das Modenspektrum der mikromechanischen Resonatoren
stark beeinflussen können. Es wurden schwingungsfähige Balken- und
Membranstrukturen untersucht, die sowohl im Gesamtwafer als auch einzeln
eingespannt waren. Hierbei zeigte sich, daß Messungen
auf dem Gesamtwafer ungeeignet sind mikromechanische Strukturen quantitativ
zu charakterisieren, da Schwingunsganteile der Piezokeramik, benachbarter
Resonatoren und niederfrequente Waferresonanzen sich im Spektrum überlagern
und über nichtlineare Effekte Modenkopplungen hervorrufen. Im Rahmen der
durchgeführten Vorversuche wurde festgestellt, daß mit Hilfe von
handelsüblichen PZT-Keramiken eine breitbandige akustische Anregung der
Mikrostrukturen bis einige hundert Kilohertz gewährleistet ist. Auf diese
Weise konnten Resonanzfrequenzen und Schwingungsamplituden, sowie
Oberwellenspektren mikromechanischer Strukturen vermessen werden.\\
%
Im Rahmen einer Diplomarbeit \cite{Bra92a} wurden die Auswirkungen der
Anregung und der Resonatorhalterung auf das dynamische Verhalten
von mikromechanischen Membranresonatoren weiter untersucht.
Die Siliziumsensoren wurden einerseits direkt auf die Piezokeramik geklebt,
so daß eine ideal starre Kopplung realisiert wurde und damit eine optimale
Energieüber\-tragung gegeben war. Diese Methode wies allerdings den
Nachteil auf, daß die Siliziumsensoren
anschließend nicht mehr zerstörungsfrei von der Keramik gelöst werden
konnten. Durch Konstruktion einer universellen Einspannhalterung für
Balkenresonatoren \cite{Bra92b} konnten verschiedene Kraftsensoren
vermessen und der Einfluß der Einspannbedingung auf die Schwingungsgüte
und getestet werden. Allerdings war bei dieser Art der
Resonatorfixierung darauf zu achten, daß die Massen- und
Steifigkeitsverhältnisse der mechanischen Halterung und des
Siliziumbalkens entsprechend groß gewählt sind, damit eine Überlagerung
der Schwingungsmoden im Amplitudenspektrum nicht auftritt und eine genügend
große Separation der Modenbeiträge erreicht wird.\\


{\bf Optisches Abtastsystem:}

Die Abtastung der Eigenfrequenzen und Schwingungsformen erfolgte
interferometrisch mit einem kommerziellen Laservibrometer
({\em POLYTEC OFV1102HR}) unter Verwendung der Laser-Doppler-Technik
\cite{Pol91}. Das Laservibrometers besteht aus einem
{\em Mach-Zehnder}-Interferometer und einer elektronischen
Signalverarbeitung, wie in Abbildung~A schematisch dargestellt.
Mit diesem Gerät ist es möglich Biegeschwingungen aus der Strukturebene
heraus zu detektiert und unter Ausnutzung des Dopplereffektes die
Geschwindigkeitsschnelle $v(t)$ oder interferometrisch die
Schwingungsamplitude $A(t)$ zu vermessen.
Der Arbeitsfrequenzbereich des eingesetzten Laservibrometers erstreckt
sich von 0,1~Hz bis maximal 1~MHz. Die Meßbereiche betragen bei der
Vermessung von Geschwindigkeiten 10$^{-6}$--10~m/s bei einer Auflösung
von 0,5~$\mu$m/s und bei Amplituden 10$^{-9}$--10$^{-2}$~m. Die Auflösung
bei der Amplitudenmessung beträgt minimal 8~nm bedingt durch den Einsatz
eines hochauflösenden Interferenzstreifenzählers. Eine genaue
Funktionsbeschreibung des Laservibrometers und seiner Betriebsmodi ist in
\cite{Sel88} und \cite{Lew90} zu finden.\\
Das Ausgangssignal der Signalverarbeitung des Laservibrometers wird
wahlweise in ein Speicheroszilloskop oder in einen Spektrumanalysator
({\em HP3588A}) eingespeißt. Zur Ermittlung der Absolutamplituden wird
das Zeitsignal
$A(t)$ auf dem Oszilloskop herangezogen, während das in den Frequenzbereich
transformierte Signal $A(\omega)$ direkt vom Spektrumanalysator
ausgelesen und weiter verarbeitet werden kann. Hierzu wurde eine
Meß- und Steuerprogramm ({\sf HPMESS.EXE}) unter {\sf Turbo-Pascal 6.0}
geschrieben und ein Konvertier- ({\sf TRANS.EXE}), sowie
ein Batchprogramm ({\sf HPLO.BAT}) zur Dokumentation der aufgenommenen
Spektren erstellt \cite{Mue92}.\\

Die Justierung der Mikrostrukturen unter dem Laserstrahl kann mit Hilfe
einer xy-Schrittmotorsteuerung erfolgen. Desweiteren wird der LWL-Meßkopf
mit Hilfe mechanischer Mikrometerschrauben. Zusätzlich besteht die
Möglichkeit über mehrere Probernadeln die zu vermessenden Strukturen
elektrisch zu kontaktieren \cite{Sch93a}.
Der Gesamtaufbau des optischen Meßplatzes erfolgte auf einem
schwingungsgedämpften Tisch, um äußere Störeinflüsse, wie
Lüftervibrationen, Trittgeräusche, usw.\ möglichst zu eliminieren.


{\bf Meßgrößeneinleitung:}

Zur Druckbeaufschlagung von Membranen ist ein spezieller Druckchuck
konstruiert worden, der sowohl Überdrücke (bis etwa 5~bar) mit Hilfe einer
$N_{2}$-Druckflasche, als auch Unterdrücke durch eine Vorvakuumpumpe
(bis etwa 1~bar) ermöglicht. Die Druckmessung erfolgt mit einem
kommerziellen Digitaldruckmanometer ({\em RUSKA, Serie 6200}) mit einer
Auflösung besser als $\pm1$ mbar. Der Druckchuck ist aus Messing gefertigt,
um eine hohe Wärmekapazität aufzuweisen und bei thermischen Messungen
eine zeitliches Driften möglichst zu vermeiden. \\

Unter dem Druckchuck befindet sich ein heizbarer Temperaturtisch, der in
einem Temperaturbereich von 0--150$^\circ$C extern geregelt werden kann.
Die Halterungen für die Drucksensormebranen
(DUT = \underline{D}evice-\underline{U}nder-\underline{T}est)
wurden aus Aluminium, Messing und Edelstahl gefertigt, um auch
unterschiedliche Temperaturausdehnung untersuchen zu können. Die
Abdichtung der Membranträger erfolgt mit Hilfe eines O-Rings aus Gummi.\\
%
Für die Kraftbaufschlagung von Siliziumstrukturen stand neben der
im Rahmen der Diplomarbeit von {\sl Müller} erstellten Konstruktion
\cite{Mue92} zum Abschluß des BMFT-Verbundprojektes ein kommerzieller
Kraftmeßstand der Fa. {\em MICOS} zur Verfügung. Dieser ist mit einer
Präzisionsreferenzkraftmeßdose (Typ Q11, Fa. {\em Hottinger Baldwin
Meßtechnik}) ausgestattet und ermöglichte durch eine Motorsteuerung die
automatisierte Aufnahme von Frequenz-Kraft-Kennlinien \cite{Wag94}.

\newpage

Abbildung A: Meßplatz zur experimentellen Charakterisierung
mikromechanischer Strukturen\\
