\pagenumbering{arabic}
\chapter*{Zusammenfassung}
\addcontentsline{toc}{chapter}{Zusammenfassung}

% --------------------------------------------
% ----------  Allgemeines --------------------
% ----------  Kapitel: 1,2,3 -----------------
% --------------------------------------------
%
Im Rahmen dieser Arbeit wurde mit numerischen und experimentellen Methoden
das statische und dynamische Verhalten mikromechanischer Strukturen
untersucht.
Da analytische Verfahren in der Regel von vereinfachenden Annahmen ausgehen
und nur idealisierte Randbedingungen berücksichtigen, wurde als
allgemeines numerisches Berechnungsverfahren die Finite-Elemente-Methode
(FEM) eingesetzt und verschiedene Berechnungsmodelle entwickelt.
Den Schwerpunkt bildete hierbei die Simulation frequenzanaloger Druck- und
Kraftsensoren auf der Basis resonanter, mikromechanischer Bimorphstrukturen,
sowie die Charakterisierung der meßgrößenabhängigen Frequenzänderung.
Insbesondere wurden die Einflüsse anisotroper und temperaturabhängiger
Materialeigenschaften, sowie prozeßbedingte Technologieeinflüsse
beim Einsatz piezoelektrischer Dünnschichtstrukturen
untersucht.\\
%
% --------------------------------------------
% ------------ mechan. Simulation: Drucksensor
% ------------ Kapitel: 4  -------------------
% --------------------------------------------
%
Zur Beschreibung des lastabhängigen dynamischen Verhaltens frequenzanaloger
Sensoren wurden FE-Modelle entwickelt und unter Berücksichtigung
geometrischer Nichtlinearitäten, insbesondere spannungsversteifender Effekte,
Dimensionierungsvorschläge erarbeitet. Auf diese Weise konnten
resonante Silizium-Membrandrucksensoren mit piezoelektrischen
$ZnO$-Dnnschichten entworfen und realisiert werden, die sich durch eine
gemessene Druckempfindlichkeit von 12,4~Hz/mbar in einem Druckbereich
bis 500~mbar auszeichnen.\\
Auf der Basis von dünnen Siliziumbiegebalken wurden elektro-thermisch
mit Hilfe von $NiCr$-Dünnfilmwiderständen angetriebene Kraftsensoren
entwickelt, die sich durch eine gemessene Kraftempfindlichkeit von etwa
170~Hz/N auszeichnen.
Insbesondere konnte das temperaturabhängige Resonanzverhalten der
Balkenresonatoren rechnerisch und meßtechnisch charakterisiert und durch
ein FEM-unterstütztes Redesign des Widerstandlayouts die
Temperaturquerempfindlichkeit des Sensors um das Vierfache
erniedrigt werden.\\
%
% --------------------------------------------
% ------------ piezoelektrische Simulationen
% ------------ Kapitel: 5  -------------------
% --------------------------------------------
%
Mit Hilfe gekoppelter elektro-mechanischer Feldberechnungen wurden
piezoelektrische Dnnschichtstrukturen modelliert, mit denen ein Beitrag
zur Optimierung der technologischen Abscheideprozesse geleistet werden
konnte. Hierzu wurden FE-Modelle für piezoelektrische Bimorphwandler
entwickelt, die das dynamische Verhalten unter Berücksichtigung der
elektro-mechanischen Anregung beschreiben, und Entwurfsregeln zur
anwendungsspezifischen Optimierung abgeleitet.
Fr gesputterte $ZnO$-Dünschichtstrukturen wurden durch Vergleich mit
experimentellen Messungen effektive elektro-mechanische Kopplungsfaktoren
bestimmt. Weiterhin wurde für unterschiedliche Piezoelektrika
($AlN, ZnO, PZT$) der Geometrieeinfluß auf den effektiven
elektro-mechanischen Kopplungsfaktor simuliert und optimale
Schichtdickenverhältnisse berechnet. Insbesondere konnte für resonante
Membrandrucksensoren in Bimorphaufbau ein modenselektives
Elektrodenlayout erarbeitet und die Modenselektivität meßtechnisch
verifiziert werden.
Eine gleichzeitige Temperaturkompensation der Frequenz-Druck-Kennlinie
konnte durch laterale Schichtstrukturierung rechnerisch nachgewiesen
werden.\\
%
% --------------------------------------------
% --------  FEM-unterstützte Sensorentwicklung
% ------------ Kapitel: 6  -------------------
% --------------------------------------------
%
Entwurfsunterstützend wurde mit Hilfe der FE-Methode auf der Basis einer
Dreifachbalkenstruktur ein Layout für einen resonanten Kraftsensor
erarbeitet. Eine hohe Uni\-modalität des Resonators wurde erreicht,
indem die beidseitigen Balkeneinspannbereiche geeignet strukturiert wurden.
Hierdurch konnte die Modenaufspaltung zwischen der Grundmode und der
gewnschten, antisymmetrischen Schwingungsmode, bei dem der Kraftsensor
betrieben wird, um einen Faktor 30 verbessert werden.
Der realisierte Sensor zeichnet sich durch eine gemessene
Kraftempfindlichkeit von 8,6~kHz/N im Meßbereich bis 5~N aus.\\
%
Ein neuartiger Drucksensor auf der Basis einer BOD-Struktur
({\em Beam-on-diaphragm}) wurde mit Hilfe der FE-Berechnungen
konzipiert und in Bezug auf Modenentkopplung und
Meßgrößenempfindlichkeit rechnerisch optimiert. Insbesondere wurde
die Wirkung des Hebelmechanismus der druckeinleitenden Membran auf den
schwingenden Balken simuliert und dabei die durch den Herstellprozeß
bedingte komplexe Balkeneinspanngeometrie berücksichtigt.
Der beschriebene BOD-Drucksensor weist mehrere Geometrieparameter zur
variablen Auslegung der Druckempfindlichkeit, des
Druckbereiches und des Überlastverhaltens auf. Rechnerisch konnte
nachgewiesen werden, daß durch Änderung der Membrandicke
Sensoren für einen Druckbereich von 0,5--12~bar mit {\em gleichem Layout}
bei gleicher Überlastsicherheit herzustellen sind.
Der experimentell realisierte BOD-Drucksensor besitzt eine
Druckempfindlichkeit von 4,47~kHz/bar im Druckbereich von -0,8 bis 1,0~bar,
bei einer Kennliniennichtlinearität von etwa $\pm$3,8~\%.


% =================================================================

\chapter*{Vorwort}
\addcontentsline{toc}{chapter}{Vorwort}


% Motivation fr die vorliegende Arbeit\\
% There's plenty of room at the bottom ... \cite{Fey59,Fey83}
% Stand der Forschung und Technik (Univ. Twente)\\
% kommerziell erhältliche resonante Siliziumsensoren
%  \cite{DRUCK,YOKOGAWA}\\
% offene Fragen zu Beginn der Arbeit:\\
% Einsatz der FE-Methode im mikromechanischen Entwurf, speziell bei
% resonanten Sensoren \cite{Tij87}

Die vorliegende Arbeit wurde im Rahmen des BMFT-Forschungsprojektes
\glqq Einsatz der Mikromechanik zur Herstellung frequenzanaloger
Sensoren\grqq \, (Förderkennzeichen: 13~AS~0114) am Institut für Mikro- und
Informationstechnik der Hahn-Schickard-Gesellschaft für angewandte
Forschung durchgeführt. In diesem Verbundprojekt wurden mikromechanische
Sensoren
für verschiedene Meßanwendungen auf der Basis des frequenz\-analogen
Meßprinzips realisiert. Ziel der vorliegenden Arbeit war es,
entwurfsunterstützend numerische Berechnungsmodelle zu
entwickeln und verschiedene experimentelle Meßmethoden einzusetzen,
um das dynamische Verhalten mikromechanischer Strukturen zu
charakterisieren.
Es sollten insbesondere unterschiedliche Strukturgeometrien und
Schichtsysteme, Antriebs- und Detektionsprinzipien, sowie
verschiedene Elektrodenkonfigurationen für die Sensorelemente modelliert
und die daraus
resultierenden Einflüsse auf die Sensoreigenschaften
untersucht werden.  Die Arbeit dokumentiert die Möglichkeiten, statische,
dynamische und gekoppelte {\em Finite-Elemente-Berechnungen} im
Entwurfsprozeß von
resonanten mikromechanischen Sensoren einzusetzen. Ein Vergleich von
numerischen und experimentellen Ergebnissen zeigt die erreichbaren
Modellierungsgenauigkeiten der FE-Berechnungsmethode auf.

{\bf Kapitel~1} führt in die Grundlagen der Mikromechanik, die
Herstellungstechnologien und den Entwurf von mikromechanischen Strukturen
am Beispiel resonanter Druck-, Kraft- und Strömungssensoren ein.\\
%
{\bf Kapitel~2} gibt einen Überblick über verschiedene theoretische
Beschreibungsweisen mikro\-mechanischer Strukturen. Die analytische
Beschreibung mikromechanischer Bauelemente unter Berücksichtigung von
Nichtlinearitäten wird behandelt und die Wechselwirkung
physikalischer Einflußgrößen diskutiert. Eine Abschätzung der Skalierung
von Resonanzfrequenzen und erzielbaren Empfindlichkeiten soll die
physikalischen Grenzen mikromechanischer resonanter Sensoren aufzeigen.\\
%
Die mathematischen Grundlagen und die numerischen Berechnungsverfahren der
FE-Methode werden in {\bf Kapitel~3} vorgestellt. Es werden die verwendeten
statischen, dynamischen und gekoppelten Berechnungsverfahren, sowie
Fehlerabschätzungen vorgestellt.\\ %diskutiert.\\
%
{\bf Kapitel~4} behandelt die Modellierung des
Schwingungsverhaltens von Balken- und Membranresonatoren und die
verschiedenen Einflüsse der Modellparameter. Die Sensorkenn\-linien
von Kraft- und Drucksensoren werden berechnet und statische
Instabilitäten bei Balkenresonatoren infolge von thermischen
Störeinflüssen untersucht. Durch experimentelle Charakterisierung
technologisch realisierter Resonanzsensoren erfolgt die meßtechnische
Verifikation der numerischen Berechnungen.\\
%
Die in {\bf Kapitel~5} durchgeführten gekoppelten Feldberechnungen gestatten
es, das elektro-thermo-mechanische Verhalten mikromechanischer Strukturen zu
untersuchen. Insbesondere wird das piezoelektrische Antriebs\-prinzip unter
Verwendung von piezoelektrischen Dnnschichten modelliert und der Einfluß
der Strukturgeometrie auf den elektro-mechanischen Kopplungsfaktor untersucht
und mit experimentellen Resultaten verglichen.
Weiterhin werden Entwurfsregeln für die Erhöhung der Modenselektivität bei
resonanten Membrandrucksensoren abgeleitet und die
Temperaturquerempfindlichkeit von Membranstrukturen in
Bimorphaufbau\footnote{Strukturen die aus {\em zwei} unterschiedlichen
Materialien bestehen (z.B.\ Siliziumsubstrat mit piezoelektrischer
$ZnO$-Dünnschicht).} minimiert.\\
%
In {\bf Kapitel~6} wird der Entwurf und die Realisierung alternativer
Sensorgeometrien für die Kraft- und Druckmessung vorgestellt.
Der Einsatz eines Dreifachbalkenschwingers erlaubt durch eine geeignete
Strukturierung der Resonatoreinspannung die Modenselektivität zu erhöhen.
Weiterhin wird durch Ausnutzung der dynamischen Momentenkompensation bei
einer gegenphasigen Schwingungsmode die Schwingungsgüte des Sensorelementes
verbessert.
Auf der Basis eines monolithisch in Silizium integrierten
\glqq Balken-auf-Membran\grqq-Schwingers wird die Realisierung eines
resonanten Drucksensors vorgestellt. Mit Hilfe der FE-Berechnungen konnte
das dynamische Verhalten der Strukturen grundlegend untersucht werden. Die
Auswirkungen von Geometrievariationen wurden modelliert, so daß die
Druckempfindlichkeit entscheidend erhöht werden konnte.\\
%
Abschließend werden in {\bf Kapitel~7} weitere mögliche Anwendungen
aufgezeigt und ein Ausblick auf zukünftige Entwicklungen resonanter
Sensoren gegeben.

