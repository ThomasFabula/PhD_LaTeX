\chapter{Grundlagen der Mikromechanik}
\label{grundlagen}
\pagenumbering{arabic}

Die Mikromechanik ist ein junger interdisziplinärer Forschungsbereich, der
den Technologievorrat der Mikroelektronik nutzt, um miniaturisierte
Bauelemente hauptsächlich aus Silizium für verschiedene
Anwendungsbereiche zu realisieren. Sie befaßt
sich mit dem Entwurf, der Herstellung und der Anwendung mechanischer
Strukturen und Systeme, die zumindest in einer Dimension eine Abmessung
im Mikrometerbereich aufweisen, so daß feinmechanische Formgebungsverfahren
nicht mehr sinnvoll eingesetzt werden können \cite{Bue91a}. %Heu89}.
Die wesentlichen Vorteile der Mikromechanik gegenber den konventionellen
Fertigungstechniken sind:
%
\begin{itemize}
\item
Ausnutzung der Vorteile des einkristallinen Siliziummaterials
\item
hochgenaue Strukturübertragung mittels lithographischer Verfahren und damit
gute Reproduzierbarkeit der mikro\-mechanischen Bauelemente
\item
hoher Integrationsgrad durch minimale Strukturgrößen, verbunden mit
einer wesentlichen Erhöhung der Zuverlässigkeit der Bauelemente
\item
Integration mechanischer, elektronischer und optischer Funktionen
in komplexen Mikrosystemen
\item
kostengünstige Herstellung durch Fertigung im Batch-Betrieb.
\end{itemize}
%
Hauptanwendungen mikromechanischer Strukturen sind derzeit Sensoren und
Aktoren, sowie spezielle mechanische Funktionselemente beispielsweise in
der integrierten Optik.
Zum Einsatz mikromechanischer Systeme werden spezielle Anforderungen
an die Aufbau- und Verbindungstechnik gestellt, um die mechanische
Stabilität und gleichzeitig eine einwandfreie Funktion der Bauelemente zu
gewährleisten.  Bei Sensoren stellt die Art der Meßgrößeneinleitung
einerseits, die Sensorgehäusung als Schutz vor unerwünschten
Umwelteinflüssen andererseits oft ein erhebliches Problem dar, das in
der Regel nur durch anwendungsspezifische Aufbau-, Kontaktierungs-
und Gehäusungstechniken gelöst werden kann. \\
%
Die Funktionsweise
mikromechanischer Sensoren und Aktoren beruht auf verschiedenen
physikalischen Wirkprinzipien, bei denen meist eine Energieumwandlung
vom Mechanischen ins Elektrische stattfindet.  Als
Detektionsprinzipien finden piezoresistive, piezoelektrische,
kapazitive, chemische und optische Sensoreffekte Anwendung, während bei
Aktoren z.B.\ elektrostatische, -magnetische, -thermische und
piezoelektrische Antriebsprinzipien eingesetzt werden.  Mikromechanische
Aktoranwendungen sind beispielsweise Mikroventile und Mikropumpen für
miniaturisierte Dosiersysteme, Mikroschalter, sowie aktive
Positionierelemente für die integrierte Optik (z.B.\ für die
Glasfaser-Chip-Kopplung).\\
Das zur Zeit am weitesten fortgeschrittene Arbeitsgebiet der Mikromechanik
stellt die Sensorik dar. Insbesondere können die hohen Anforderungen der
Präzisionsmeßtechnik durch mikromechanische Sensoren, auf der Basis des
frequenzanalogen Sensorprinzips, gut abgedeckt werden.\\
Die Weiterentwicklung der Mikromechanik stellt die Mikrosystemtechnik dar,
die sich mit der Realisierung multifunktionaler Mikrosysteme durch
Integration verschiedener Bauelementefunktionen besch#ftigt. Mögliche
industrielle Anwendungsgebiete von Mikrosystemen finden
sich in der Produktions- und Fertigungsautomatisierung,
Konsumgüterindustrie, Rechner- und Kommunikationstechnik, Verkehrs-,
Medizin-, sowie Umweltschutztechnik.


\section{Herstellungstechnologien}
\label{herstellungstechnologien}

Wegen seiner hohen Reinheit, der technischen Verfügbarkeit und der
hochentwickelten Verfahrenstechnik nimmt Silizium eine Sonderstellung
bei den in der Mikromechanik verwendeten Werkstoffen ein. Die Existenz
eines Eigenoxids ermöglicht die Nutzung als Maskierschicht bei
selektiven Dotier- und Ätzprozessen und zusätzlich den Einsatz als
Isolations- und Passivierungsschicht. Im Gegensatz zur Mikroelektronik
werden in der Mikromechanik außer Silizium auch andere
Werkstoffe\footnote{z.B.\ Quarz, Gläser, Keramiken.} mit
unterschiedlichen Eigenschaften
verwendet und neben den elektrischen vor allem die mechanischen,
thermischen und piezoelektrischen Eigenschaften der
Materialien ausgenutzt. Bei der technologischen Herstellung handelt es
sich zum einen um planare Prozesse, bei denen auf einem
einkristallinen Siliziumsubstrat komplexe Dünnschichtfolgen aufgebracht
und strukturiert werden. Verschiedene photolithographische Verfahren,
wie optische oder Elektronenstrahl-Lithographie dienen der Strukturierung
der Schichten. Mit Hilfe der Dünnschicht-Technologien werden Schichtsysteme
unter verschiedenen Bedingungen, beispielsweise aus der Gasphase ({\em CVD
= \underline{C}hemical \underline{V}apor \underline{D}eposition})
oder unter Verwendung physikalischer Abscheideprozesse
({\em PVD = \underline{P}hysical \underline{V}apor
\underline{D}eposition}) abgeschieden. Zur
gezielten Veränderung der Materialeigenschaften werden Implantation,
Diffusion und Oxidation eingesetzt.  Neben den additiven Prozessen der
Schichtabscheidung werden subtraktive Prozesse zur isotropen und
anisotropen Strukturierung der mikromechanischen Bauelemente verwendet.
Insbesondere nimmt das anisotrope Naßätzen\footnote{z.B.\ in Alkalilaugen
({\em KOH-,NaOH-}Lösung).}, bei dem die Ätzraten stark von der
Kristallrichtung des einkristallinen Siliziums abhängig sind, in der
dreidimensionalen Strukturierung eine dominierende Stellung ein. Die
Anisotropie des Ätzverhaltens gestattet eine definierte Strukturierung
in lateraler
Richtung, während eine Selektivität durch Einbau von Ätzstoppschichten
(z.B.\ mittels Dotierung mit Fremdatomen) eine vertikale Strukturierung des
Siliziums erlaubt. Verschiedene Trockenätzverfahren wie das Plasmaätzen
({\em PE = \underline{P}lasma \underline{E}tching}) oder reaktive
Ionenstrahlätzen ({\em RIE = \underline{R}eactive \underline{I}on
\underline{E}tching}) gewinnen aufgrund der erhöhten Prozeßkompatibilität
und besseren Möglichkeit der Integration in der
Batch-Verarbeitung zunehmend an Bedeutung. Andere Prozeßtechnologien,
wie laserunterstütze Verfahren (\cite{Ala92a} und Referenzen darin), die
Kernspurtechnik \cite{Spo90} und die {\em LIGA}-Technik
({\em \underline{Li}thographie-\underline{G}alvanoformung-\underline{A}bformung},
\cite{Bec86}) sind zur Zeit im
Forschungsstadium und erweitern zukünftig das technologische Spektrum.
Die Technologien zur Herstellung von dreidimensionalen Mikrostrukturen in
Silizium unter Verwendung von anisotropen Ätztechniken mit oder ohne
Ätzstoppverfahren
werden unter dem Oberbegriff Bulk-Mikromechanik ({\em BMM}) zusammengefaát,
da die Siliziumscheibe (Wafer) i.a.\ ber ihre gesamte Dicke strukturiert
wird. Neue technologische Entwicklungen, wie beispielsweise die
Oberflächenmikromechanik ({\em OFM}) mit Hilfe von
Opferschichten\footnote{engl.: {\em Sacrificial-Layer-Method}.},
sollen der weiteren Miniaturisierung und
dem Bedarf an beweglichen Teilen in komplexen Mikrosystemen Rechnung
tragen \cite{How87, Fan87}. Hierbei werden auf der Basis z.B.\
von Poly-Silizium ({\em poly-Si})
freitragende Strukturen auf der Waferoberfläche realisiert, bei der
Mehrlagenstrukturen aus Siliziumoxid (Opferschicht) und {\em poly-Si} mit
Schichtdicken im Mikrometerbereich verwendet werden. Statt {\em poly-Si}
kann auch Siliziumnitrid als Material der Mikrostruktur eingesetzt
werden \cite{Smi92a}.\\
%
Die in der vorliegenden Arbeit experimentell untersuchten mikromechanischen
Bauelemente wurden ausnahmslos in Bulk-Mikromechanik ohne
Verwendung von Ätzstoppschichten hergestellt und weisen als minimale
Strukturabmessungen Balken- und Membrandicken von etwa 20~$\mu$m auf.


\section{Entwurf mikromechanischer Systeme}
\label{entwurf}

Der Entwurf von mikromechanischen Systemen muß in das
Gesamtherstellungskonzept integriert werden und rechnerunterstützt
erfolgen, da das Gesamtsystemverhalten „äßerst komplex ist und durch das
unterschiedliche Verhalten der Einzelkomponenten bestimmt wird. Die
Vorteile des rechneruntersttzten Entwurfs ({\em CAD = \underline{C}omputer
\underline{A}ided \underline{D}esign}) bestehen im möglichen Einsatz
verschiedener Entwicklungswerkzeuge\footnote{Layout-Editoren,
Volumenmodellierer, Simulatoren, Datenbanken, etc.},
den schnellen Žnderungsmöglichkeiten
und den damit verbundenen kürzeren Entwicklungszeiten. Die Analyse des
Funktionsprinzips und der Bauelementeeigenschaften mit Hilfe der Simulation,
beispielsweise der {\em Finite-Elemente-Methode} ({\em FEM}), fördert das
Verständnis und erlaubt die Prüfung von Teilfunktionen bereits
in einer sehr frühen Entwicklungsphase. Ferner sind Parametervariationen
ein effektives Hilfsmittel zur Optimierung und gleichzeitig zur
Reduktion der Proto\-typenanzahl, so daß die Gesamtkosten der
Bauelementeentwicklung niedrig gehalten werden können. Ein weiterer
wichtiger Punkt ist die Schaffung einer rechnergesttzten
Dokumentationsgrundlage zur Hinterlegung des Entwurfswissens und zur
Überprüfung der in der Optimierung vollführten Variationen.\\
In der Mikroelektronik ist eine
Durchgängigkeit der Entwicklungswerkzeuge, z.B.\ für {\em ASICs (=
\underline{A}pplication \underline{S}pecific \underline{I}ntegrated
\underline{C}ircuits)} gegeben und Stand der kommerziellen
Technik.  Es existieren hochentwickelte rechnergestützte Werkzeuge für
Layout, Simulation und Verifikation, so daß die gesamte Fabrikation der
elektronischen Bauelemente am Rechner simuliert werden kann. Während
die Mikroelektronik durch eine starke Standardisierung der
Herstellungsverfahren gekennzeichnet ist, gibt es in der Mikromechanik
eine Vielzahl von flexiblen Technologien und anwendungsspezifischen
Anforderungen. Zudem befindet sich der Entwicklungsstand mikromechanischer
Entwurfswerkzeuge heute noch am Anfang. Daher beschränkt sich der
Einsatz von rechnergestützten Werkzeugen zur Zeit im
wesentlichen auf die Maskenerstellung mit Layout-Editoren und die
Bauelemente\-simulation mit Hilfe der FE-Methode. Zur Verifikation der
Simulationsmodelle sind daher experimentelle Messungen {\em parallel}
zu den Berechnungen erforderlich, um die Eckdaten des Modellparameterraumes
abzustecken.\\
%
In {\bf Abbildung~\ref{abbmems}} ist der schematische Ablauf des
mikromechanischen Entwurfsprozesses für
{\em Mikro-elektro-mechanische-Systeme (MEMS)}
dargestellt, wie er sich in einer geschlossenen rechnergesttzten
Entwurfsumgebung zukünftig darstellen könnte \cite{Wac93}.
Der Entwurf geht von
den Anforderungen und Spezifikationen aus, die in einem Pflichtenheft
definiert werden. Für einen Sensor enthält das Pflichtenheft
beispielsweise Forderungen bezüglich der Empfindlichkeit gegenüber der
Meßgröße, der Kennlinienlinearität, der Querempfindlichkeiten und des
šberlastverhaltens. Im Layout-Design ({\sl Maskenentwurf}) wird unter
Zugrundelegung der zur Verfügung stehenden Herstellungstechnologien die
Bauelementegeometrie und das Funktionsprinzip, d.h.\ die nutzbaren
physikalischen Wirkprinzipien, festgelegt. Hierfür müssen unter Verwendung
prozeßspezifischer Entwurfsregeln ({\sl Design Rules}) die
{\em zweidimensionalen}
Maskensätze für die verschiedenen benötigten technologischen Schritte
erstellt werden.
%
Anschließend wird mit Hilfe der Prozeßsimulation unter Berücksichtigung
prozeßspezifischer technologischer Parameter und Verwendung geeigneter
physikalischer Modelle die {\em dreidimensionale} Bauelementestruktur
abgeleitet. Als wichtiges Beispiel für die Mikromechanik ist hier die
Simulation des anisotropen Naßätzens zu nennen, auf die in
Kapitel~\ref{processmodeling} kurz eingegangen wird.
%
Die Bauelementesimulation modelliert das Betriebsverhalten
der mikromechanischen Struktur unter Bercksichtigung der speziellen
Materialeigenschaften und „äußeren Betriebsbedingungen, unter Zugrundelegung
der in der Prozeásimulation abgeleiteten Bauelementestruktur. Als Ergebnis
wird die \glqq Ein-/Ausgabe-Charakteristik\grqq \, (z.B.\ Sensorkennlinie)
des Bauelementes zur Verfügung gestellt.
%
Auf einer nächst höheren Modellebene wird das Systemverhalten ausgehend von
dem elektrischen Ersatzschaltbild des mikromechanischen Bauelementes auf
Schaltungsebene mit einem Netzwerksimulator (z.B.\ {\sf SPICE})
modelliert. Der Entwurfsvorgang wird solange iterativ durchlaufen, bis
die Anforderungen erfüllt sind und ein optimiertes Design eines
Prototyps erreicht ist. Anschließend erfolgt die technologische
Realisation und Vermessung des Prototyps. Wenn eine experimentelle
Verifikation der Anforderungen nicht erreicht werden kann, ist eine
Entwurfsverbesserung, ein sogenanntes {\em Redesign}, notwendig. Die
Wiederaufnahme der Entwurfsschleife kann hierbei durch einen Rücksprung
auf verschiedene Entwurfsebenen erfolgen.
%----------------------- Beginn: Figure-Environment ----------------------
\begin{figure}[htb]
\begin{center}
% --- Dateiname des Bildes
\input{abbee.tex}
\setabbee
\end{center}
\caption{\label{abbmems}
 Rechnergesttztes Ablaufschema beim mikromechanischen Entwurfsprozeá
 (nach [Wachutka 93])}
\end{figure}
%----------------------- Ende: Figure-Environment ----------------------

\clearpage

%\newpage
\subsection{Maskenlayout}
\label{maskenlayout}

Bei mikromechanischen Komponenten erfolgt die Strukturübertragung von
der Maske auf das Substrat (z.B.\ Siliziumwafer)
auf photolithographischem Wege. Die
Erstellung der Masken wird mit kommerziellen Layout-Editoren
vorgenommen, die in der Mikroelektronik eingesetzt werden.  Für die
gesamttechnologische Herstellung der Strukturen ist ein Maskensatz
notwendig, der in der Regel aus fünf bis acht Maskenebenen bestehen
kann. Für jeden einzelnen Prozeßschritt ist eine entsprechende Maske,
bei einigen sogar Vorder- und Rückseitenmaske erforderlich.
Insbesondere sind die prozeßspezifischen Eigenheiten und deren
Auswirkungen bereits im Entwurf zu berücksichtigen. Beim Maskenentwurf muß
sichergestellt werden, daß die erlaubten Prozeßtoleranzen nicht
überschritten werden. Beim anisotropen Naßätzen muß beispielsweise die
laterale Unterätzung konvexer Ecken berücksichtigt werden, so daß
entsprechende Kompensationsstrukturen im Maskenlayout vorzusehen sind.
Unter Verwendung verschiedener Kompensationsmodelle werden
Entwurfsregeln erarbeitet und Hinweise zur Dimensionierung abgeleitet
\cite{Haf92}. Zusätzlich spielen die Reproduzierbarkeit und die
Maßhaltigkeit der technologischen Prozesse bei der Herstellung der
mikromechanischen Strukturen eine wesentliche Rolle. Hierfür müssen im
Maskenlayout Teststrukturen vorgesehen werden, die an verschiedenen
Stellen des Wafers angeordnet sind, um gleichzeitig Informationen
bezüglich der Prozeßhomogenität über den Gesamtwafer zu erhalten. Die
Teststrukturen gestatten das Prozeáverhalten quantitativ zu
charakterisieren, indem Eigenschaften der mikromechanischen Elemente
meátechnisch bestimmt werden, um so Rückschlüsse auf technologie- oder
entwurfsbedingte Fehler zu ermöglichen.


\subsection{Simulation des Herstellungsprozesses}
\label{processmodeling}

Eine systematische und gezielte Entwicklung von komplexen Mikrosystemen
ist ohne ein grundlegendes Verständnis ihrer inneren Funktionsmechanismen
nicht möglich.
Die Komplexität des Systemverhaltens nimmt mit der Zahl der technologischen
Herstellungsschritte und der stark ausgeprägten Wechselwirkungen infolge
hoher Integrationsdichten ständig zu. Aus diesem Grunde ist die
Modellierung und numerische Simulation des Herstellungsprozesses und des
Betriebsverhaltens bei komplexer werdenden Bauelementen unerläßlich.
Statt des herkömmlichen experimentellen Weges
{\em \glqq by trial and error\grqq} \/ kann durch Nachbildung
am Rechner das Design zielgenauer optimiert werden. Ein weiterer Vorteil
der Modellierung ist die Möglichkeit meßtechnisch nur schwer oder
überhaupt nicht zugängliche physikalische Parameter zu bestimmen und
dadurch die Funktionsweise der Mikrosysteme besser verstehen zu können.
Dies hilft die Zahl der Entwicklungszyklen zu reduzieren und somit Zeit
und Kosten einzusparen.\\
Da der Zusammenhang zwischen den zweidimensionalen Maskenebenen und dem
drei\-dimensionalen mikromechanischen Bauelement nicht eindeutig umkehrbar
ist,
müssen Berechnungswerkzeuge fr die Prozeßsimulation ({\sl Process Modeling})
eingesetzt werden, um die Strukturgeometrie abzuleiten.
Ferner sind die Materialeigenschaften der mikrotechnisch hergestellten
Schichtsysteme zu bestimmen, die extrem stark von den technologischen
Herstellungsprozessen abhängig sind. Für die Technologien der
Mikroelektronik existieren bereits kommerziell erhältliche
Simulationswerkzeuge für optische Lithographie, Ionenimplantation,
Diffusion, Oxidation, Epitaxie, Schichtabscheidung und Trockenätzen.
Sie gestatten es, technologische Prozeßschritte mittels
geeigneter physikalischer Modelle zu simulieren, und stellen
Eingangsdaten beispielsweise in Form von Implantations- und Dotierprofilen
für die anschließende Bauelementesimulation ({\sl Device Modeling}) zur
Verfügung, in der die elektrischen Eigenschaften berechnet werden
\cite{Lor85}. Eindimensionale
Simulationswerkzeuge wie {\sf ICECREAM} \cite{Pic90} und {\sf SUPREM-III}
\cite{TMA} erlauben Tiefenprofile für Diffusion, Implantation und Oxidation
zu bestimmen, während zweidimensionale Simulationsprogramme wie
{\sf COMPOSITE} \cite{COMP} und {\sf SUPREM-IV} \cite{TMA} die flächenhafte
Berechnung von Dotier-, Implantations- und  Photolackprofilen
gestatten.\\
% Die physikalische Modellbildung für Ätzprozesse und
% Schichtabscheidung sind heute noch ungenügend und Gegenstand der
% Forschungsaktivitäten. \\
Die Anforderungen der Mikromechanik an die Prozeßsimulation
gehen darüber hinaus, und machen neben der Ableitung der dreidimensionalen
Bauelementestruktur auch die Vorhersage von Materialeigenschaften in
Abhängigkeit der technologischen Prozeßführung erforderlich.
Die Abgrenzungen zur Mikroelektronik bestehen zusätzlich in
der Berücksichtigung der Tiefenstrukturierung, der beidseitigen
Photolithographie und Scheibenprozessierung, sowie der speziellen Aufbau-
und Verbindungstechniken. Zur Zeit befindet sich die mikromechanische
Prozeásimulation im Forschungsstadium. Für die spezifischen mikromechanischen
Herstellungstechnologien müssen vorhandene Simulationsmodelle stark
erweitert oder ganz neu entwickelt werden. Zur Prozeßmodellierung
werden zum einen atomistische Modelle, %(ab initio),
zum anderen
aufwendige empirische Modellansätze, die auf experimentell ermittelte
Daten zurückgreifen, verwendet. In der Literatur werden verschiedene
Methoden für die Modellierung des anisotropen Ätzens von Silizium
% \cite{Seq91} und Quarz \cite{Dan91}
auf atomistischer Ebene vorgestellt \cite{Cam90, Tha94}, um eine Vorhersage
der Richtungsabhängigkeit des Ätzvorgangs zu ermöglichen. Auf der anderen
Seite greifen geometriebasierte Simulationsprogramme wie {\sf SIMODE}
\cite{Fru90} oder {\sf ASEP} \cite{Bus91b} auf empirisch am Siliziumkristall
ermittelte Ätzraten zurück und sind in der Lage, die dreidimensionale
Geometrie an FE-Programme über ein neutrales Datenaustauschformat
(z.B.\ {\sf DXF}\footnote{{\em
\underline{D}ata-\underline{E}xchange-\underline{F}ormat} ist ursprünglich
das neutrale Datenaustauschformat des CAD-Programms {\sf AutoCAD} und
inzwischen neben {\sf IGES} ein internationales Standardformat zur
Geometriebergabe.}) zu übergeben. Das ursprünglich für den
Entwurf von integrierten Schaltkreisen entwickelte Programm
{\sf OYSTER} \cite{Kop89} erlaubt unter Vorgabe der zweidimensionalen
Maskensätze und der technologischen Prozeßabfolge dreidimensionale
mikromechaniche Strukturen abzuleiten, benötigt jedoch weitere manuelle
Eingaben seitens des Benutzers, so daá nur ein beschränkter praktischer
Einsatz möglich ist. Das Programmsystem {\sf MEMCAD} \cite{Shu91, Sen92}
stellt eine neuere Entwicklung dar, aus
vorgegebenen zweidimensionalen Maskenentwürfen die dreidimensionale
Strukturgeometrie abzuleiten. Zusätzlich sind in einer
objekt-orientierten Materialdatenbank für einzelne technologische
Herstellungsverfahren algebraische Zusammenhänge zwischen den
Materialeigenschaften und bestimmten Prozeßparametern abgelegt.
Programmintern wird hierbei auf empirische Datensammlungen zurückgegriffen,
wobei weiterführende Literaturreferenzen die Angaben
ergänzen. Verschiedene Forschungsgruppen \cite{Cra91a, Joh91, San90}
arbeiten an der Schaffung integrierter Entwicklungsumgebungen fr
komplexe Mikrosysteme. Diese Sensor- und Mikrosystem-Entwurfswerkzeuge
verwenden eine Kombination von analytischen und numerischen Methoden
(meist FEM) und stellen erste Ans„tze in Richtung einer geschlossenen
Entwurfs\-umgebung für Mikrosysteme dar. Diese Systeme sind jedoch in
der Regel auf bestimmte Sensorklassen, beispielsweise Sensoren für
mechanische Größen, beschränkt.


\subsection{Simulation des Bauelementeverhaltens}
\label{devicemodeling}

Fr die Modellierung von Bauelementen werden in der Mikromechanik
verschiedene Methoden eingesetzt. Zur ersten Auslegung der Geometrie und
zur Festlegung des Arbeitspunktes dienen analytische
Abschätzungen, in denen die physikalischen Grenzen, wie maximal
erreichbare Kräfte, Hübe, Empfindlichkeiten und Grenzfrequenzen
ermittelt werden (siehe Kapitel~\ref{skalierungsverhalten}).
Analytische Programmwerkzeuge,
wie {\sf SENSIM} \cite{Lee82} und {\sf CAPSS} \cite{Bin87},
erlauben die Berechnung
von idealisierten Membrangeometrien für Drucksensoranwendungen und
beschränken sich somit auf bestimmte Klassen von Problemfällen. Neben
den analytischen Berechnungsverfahren werden für aufwendige
Parameterstudien auch symbolische Manipulationsverfahren, sogenannte
Computeralgebra-Programme wie {\sf MAPLE} \cite{Cha91} oder {\sf MATHEMATICA}
\cite{Wol91} eingesetzt. Diese Programmsysteme ermöglichen auch „äußerst
komplexe algebraische Zusammenhänge schnell zu bearbeiten und
Parameterabhängigkeiten effizient zu analysieren.\\
  Die Beschreibung
mikromechanischer Bauelemente setzt im allgemeinen eine dreidimensionale
Formulierung voraus, da die geometrischen Abmessungen in allen drei
Raumrichtungen von gleicher Größenordnung sind, so daß die
Randbedingungen einen erheblichen Einfluß auf das Bauelementeverhalten
haben. Daher kommt beim mikromechanischen Entwurf den numerischen Methoden,
wie der Simulation mit Hilfe der Finite-Elemente-Methode, eine wichtige
Bedeutung zu. Diese gestatten bereits in der Entwurfsphase, auch komplexe
Geometrien mit vielfältigen Randbedingungen und die technologisch zu
realisierenden Funktionsprinzipien unter Berücksichtigung prozeßtechnischer
Einschränkungen zu modellieren. Zusätzlich ermöglicht die FE-Methode
unterschiedliche Strukturgeometrien unter Verwendung
verschiedener Materialien in Mehrlagenaufbau, sogenannte
Multilayerstrukturen\footnote {Auch Sandwich-, Komposit- oder
Bimorphstrukturen genannt.}, zu berechnen. Als Ausgangswerte für die
numerischen Modellrechnungen dienen analytische Abschätzungen, die in
der Regel nur für einfache, idealisierte Strukturgeome\-trien durchgeführt
werden können. An komplexen FE-Modellem lassen sich anschließend
Parametervariationen vornehmen, um Toleranzuntersuchungen und
Sensitivitätsanalysen durchzuführen. So kann beispielsweise der Einfluß
der Strukturgeometrie und der Materialeigenschaften der Dünnschichtsysteme
bereits {\em vor} der Herstellung studiert werden. Die FE-Methode erlaubt
somit mikromechanische Strukturen geeignet auszulegen und die
Bauelementeeigenschaften gezielt
im Vorfeld zu optimieren.\\
In dieser Arbeit werden mit Hilfe {\em dynamischer FE-Berechnungen}
die Eigenfrequenzen und Schwingungsformen (rechnerische Modalanalyse)
mikromechanischer Resonatorstrukturen bestimmt
und der Einfluß der zu untersuchenden physikalischen Meßgrößen
ermittelt. Die Berechnung der Meßgrößenempfindlichkeit erfolgt
durch eine nichtlineare, {\em statische FE-Berechnung}, in welcher die
durch die Meßgröße hervorgerufene Steifigkeitsänderung des Gesamtsystems
bestimmt wird. Die geänderte Struktursteifigkeit wird anschließend
herangezogen, um die Eigenfrequenzen des Sensors unter einer äußeren
Belastung zu ermitteln. Durch einen Vergleich der berechneten
Sensorcharakteristiken mit experimentellen Daten mikromechanischer
Bimorphstrukturen kann auf Materialeigenschaften, insbesondere auf eine
Vordehnung oder Spannung der Dünnschichten zurckgeschlossen werden.
{\em  Gekoppelte FE-Berechnungen} unter
Berücksichtigung der elektromechanischen Wechselwirkung ermöglichen eine
gezielte Vorhersage des statischen und dynamischen Verhaltens von
piezoelektrisch betriebenen Sensoren und Aktoren. Das Frequenzgangverhalten
wird ermittelt, indem das mechanische Amplitudenspektrum und der
elektrische Impedanz- und Phasenverlauf unter Einschluß der
piezoelektrischen Anregung modelliert wird. Um die thermische
Fluid-Struktur-Wechselwirkung zwischen einem mikromechanischen
Bauelement und dem umströmenden Fluid zu beschreiben, muß die
Geschwindigkeits- und Temperaturverteilung in der Strömung berechnet
und die Auswirkungen auf die Struktur infolge der temperaturinduzierten
Spannungen betrachtet werden. Auf diese Weise ist es möglich das Verhalten
eines Strömungssensors zu berechnen.\\
Bei der vorliegenden Arbeit wurde für die numerischen Strukturberechnungen
das kommerziell verfügbare Programmsystem {\sf ANSYS} von
{\sl Swanson Analysis Systems, Inc.} \cite{SASI} verwendet.
% und für die Strömungsberechnungen {\sf FIDAP} \cite{FDI}
Bei der Erstellung komplexer dreidimensionaler Geometrien
und zu deren Vernetzung wurde der Volumenmodellierer und Preprozessor
{\sf I-DEAS} \cite{SDRC} eingesetzt. Die analytischen Abschätzungen
erfolgten mit {\sf MAPLE}.


\section{Funktionsprinzip resonanter Sensoren}
\label{funktionsprinzip}

In modernen Meß- und Regelsystemen, die in zunehmendem Maße digitale
elektronische Bauelemente einsetzen, spielt die Klasse der resonanten
Sensoren auf der Basis mechanischer, schwingungsfähiger Strukturen eine
wichtige Rolle \cite{Bue91b}. Im Gegensatz zu
konventionellen Sensoren, die für die Signalaufbereitung einen
Analog-Digital-Wandler benötigen, zeichnen sich resonante
Sensoren\footnote{Auch als Resonanzsensoren bezeichnet.} durch
den Vorteil eines Frequenzausgangs aus, so daß die Meßgröße
frequenzanalog zur Verfügung steht. Weitere Vorteile frequenzanaloger
Sensoren sind die erzielbare Empfindlichkeit und die damit
verbundene hohe Meßgrößnauflösung, sowie die Störsicherheit durch
den Wegfall der analogen Signalübertragung. Weiterhin besteht
die Möglichkeit zur ständigen Funktionskontrolle des Sensors in
sicherheitsrelevanten Anwendungen (z.B.\ Beschleunigungssensor in
Airbag-Steuersystemen), da die Detektion der Resonanzfrequenz die
Unversehrtheit des Sensors sicherstellt. Resonante
Sensoren können für die Messung mechanischer Größen wie Druck, Kraft,
und Beschleunigung, aber auch für die Messung von
Massenanlagerung\footnote{Mikrowägung nach \cite{Sau59}.},
der relativen Feuchte und zur Temperaturmessung verwendet werden.
Zusätzlich besteht die Möglichkeit der Messung abgeleiteter physikalischer
Größen (z.B.\ Strömungsmessung).\\
Das Funktionsprinzip der in dieser Arbeit betrachteten
resonanten Sensoren beruht auf der Abhängigkeit der Eigenfrequenz
des Resonators von einer „äußeren physikalischen Größe, indem der
mechanische Spannungszustand beeinfluát oder die Trägheit des Resonators
ber eine Massenbelegung verändert wird. Die Änderung der
Resonanzfrequenz durch Dämpfungseffekte infolge Druckschwankungen des
umgebenden Gases
können bei Resonatoren mit hoher Schwingungsgüte in erster Näherung
vernachlässigt werden (siehe Kapitel~\ref{daempfungseinfluesse}).
Die Resonanzfrequenz stellt als Meßgröße
ein quasi-digitales Ausgangssignal dar. Die Umsetzung des Frequenzsignals
erfolgt gemäß {\bf Abbildung~\ref{abbprinzip}} durch Zähltechnik
\cite{ABV93}.
%----------------------- Beginn: Figure-Environment ----------------------
\begin{figure}[htb]
\begin{center}
% --- Dateiname des Bildes
\input{resosen.tex}
\setresosen
\end{center}
 \caption{\label{abbprinzip}
 Blockschaltbild eines Resonanzsensors}
\end{figure}
%----------------------- Ende: Figure-Environment ----------------------
Der Sensor wird einerseits
durch seine passiven Resonatoreigenschaften wie Eigenfrequenz,
Schwingungsmode, Schwingungsgüte und der Meßgrößenempfindlichkeit,
andererseits durch die Eigenschaften der Schwingungsanregung und
-detektion, beispielsweise der Modenselektivität, der Effizienz der
Energieeinkopplung und der Auflösung des Abfragesystems charakterisiert.
Infolge der Abhängigkeit der Resonanzfrequenz von verschiedenen
physikalischen Parametern, sind einerseits die Querempfindlichkeiten
(z.B.\ gegenber Temperatur) der zu entwickelnden Sensoren genau zu
untersuchen, andererseits ist die prinzipielle Einsatzmöglichkeit solcher
Resonatorelemente als Multifunktionssensor gegeben. In der vorliegenden
Arbeit wird das dynamische
Verhalten mikromechanischer Strukturen am Beispiel resonanter Sensoren
untersucht, da diese Sensorklasse durch ihr Funktionsprinzip inhärent die
statischen {\em und} dynamischen Eigenschaften des Sensors wiederspiegelt.


\subsection{Drucksensor}
\label{drucksensor}

Im folgenden soll das frequenzanaloge Funktionsprinzip stellvertretend
am Beispiel eines resonanten, mikromechanischen Membrandrucksensors
aufgezeigt werden, der in {\bf Abbildung~\ref{abbdrucksensor}}
schematisch dargestellt ist. Die erste technologische Realisierung
solcher Silizium-Membrandrucksensoren erfolgte von {\em Smits et al.}
unter Verwendung piezoelektrischer $ZnO$-Dünnschichten \cite{Smi83}.
Dieses Sensordesign war Ausgangspunkt für die im Rahmen des
BMFT-Verbundprojektes durchgeführten Arbeiten.
%----------------------- Beginn: Figure-Environment ----------------------
\begin{figure}[htb]
\begin{center}
% --- Dateiname des Bildes
\input{druckse.tex}
\setdruckse
\end{center}
 \caption{\label{abbdrucksensor}
 Schema eines resonanten Silizium-Drucksensors mit piezoelektrischem Antrieb}
\end{figure}
%----------------------- Ende: Figure-Environment ----------------------
Die Anregung des Sensors wird durch eine piezoelektrische Dünnschicht
(hier: Zinkoxid)
bewerkstelligt, die substratseitig eine hochdotierte p$^{++}$--Elektrode
und als obere Deckelektrode eine Aluminiumschicht aufweist.
Mit Hilfe des transversalen piezoelektrischen
Effektes werden bei Anlegen einer elektrischen Wechselspannung infolge
Dehnungen auf der Membranoberfläche Biegemomente erzeugt und perio\-dische
Membranauslenkungen induziert. Durch eine Druckdifferenz zwischen Unter-
und Oberseite der Membran wird der Siliziumresonator deformiert. Der Betrag
der statischen Auslenkung ist hierbei ein vielfaches größer als die
dynamische Schwingungsamplitude. Bei gengend großer statischer
Auslenkung\footnote{Die statische Auslenkung kann bis zu zwei
Größenordnungen größer als die dynamische Schwingungsamplitude sein
(siehe Kapitel~4).} stellen sich durch Reaktionskräfte
Membranspannungen ein, die zu einer Versteifung des Resonators führen,
so daß sich die {\em Frequenz analog} zur Meßgröße ändert. Wird
umgekehrt der piezoelektrische Effekt auch zur Schwingungsdetektion
genutzt, so kann mit einer geeigneten Auswerteschaltung der Sensor in
Eigenresonanz betrieben und die Frequenzänderung über einen Frequenzzähler
ausgelesen werden \cite{Fun93}.


\subsection{Kraftsensor}
\label{kraftsensor}

Frequenzanaloge Silizium-Kraftsensoren beruhen auf beidseitig
eingespannten Balkenstrukturen, die als mechanische Resonatoren
Einfachbalken- \cite{Blo90} oder auch Mehrfachbalken-Strukturen
\cite{Wag94} einsetzen.
Das Funktionsprinzip ist dem der Drucksensoren identisch. Unter einer
äußeren, axialen Zugkraftbeaufschlagung nimmt die Resonanzfrequenz
infolge der Spannungsänderung im Resonator zu, während sie bei
Beaufschlagung mit einer Druckkraft abnimmt. Zur Anregung kann neben
dem piezoelektrischen auch das elektrothermische Antriebsprinzip
eingesetzt werden \cite{Bou90}. Hierzu wird in einem Zweischichtsystem mit
unterschiedlichen Wärmeausdehnungskoeffizienten eine lokale Erwärmung
durch Mikroheizwiderstände erzeugt, das analog zum Bimetalleffekt zu
einem Biegemoment führt und den Balken auslenkt.  Durch Ansteuerung
der Mikroheizer mit Impulsen geeigneter Folgefrequenz und ausreichender
elektrischer Leistung werden die Siliziumbalken in Resonanz versetzt.
%----------------------- Beginn: Figure-Environment ----------------------
\begin{figure}[htb]
\begin{center}
% --- Dateiname des Bildes
\input{balkene.tex}
\setbalkene
\end{center}
\caption{\label{abbgmssensor}
 Geometrie und elektrisches Layout der elektrothermisch angetriebenen
 Kraft- und Strömungssensoren}
\end{figure}
%----------------------- Ende: Figure-Environment ----------------------
In {\bf Abbildung~\ref{abbgmssensor}} ist die Geometrie und das
elektrische Layout eines elektrothermisch angetriebenen Siliziumsensors
dargestellt, der zur Kraft- und Strömungsmessung eingesetzt wird.
Das elektrothermische Antriebsprinzip stellt eine technologisch einfach zu
realisierende Möglichkeit dar, da Anregung und Detektion analog zum
piezoelektrischen Antrieb in der gleichen Technologie realisiert werden
können, wobei eine gute Kompatibilität mit der Silizium-Ätztechnologie
zur Herstellung der Balkenstrukturen gegeben ist.  Die Abtastung der
Schwingung erfolgt resistiv mit Hilfe in Balkenmitte angeordneter
NiCr-Dehnmeßstreifen ({\em DMS\/}), die zu einer {\em Wheatstone}schen
Vollbrücke verschaltet sind. Zur Steigerung der Anregungseffizienz
werden die Mikroheizwiderstände an den Balkenenden plaziert, w„hrend die
DMS-Widerstände in Balkenmitte und in Nähe der Balkenenden so
angeordnet sind, daß eine möglichst große Brückenverstimmung erreicht
wird. Die in dieser Arbeit untersuchten Sensoren setzen sowohl den
piezo\-elektrischen Antrieb auf der Basis von Dreifachbalkenresonatoren
\cite{Wag94}, als auch elektrothermisch angetriebene
Einfachbalkenresonatoren mit NiCr-Widerständen ein, die durchgängig in
Metalldünnfilmtechnologie realisiert wurden \cite{Bar93}.



\subsection{Strömungssensor}
\label{stroemungssensor}

Beim Betrieb der elektrothermisch angeregten Siliziumresonatoren sind
für die Signalgewinnung mittels Metall-DMS, deren
Empfindlichkeit\footnote{Der K-Faktor ist bei Metallwiderständen über
die relative Widerstands„nderung $\Delta R/R \, = \, K \varepsilon$
infolge der mechanischen Dehnung $\varepsilon$
definiert und betr„gt bei Metall-DMS rund zwei. Diffundierte
Siliziumwiderst„nde weisen einen bis zu zwei Größenordnungen höheren
Wert auf.} sehr gering ist,
Schwingungsamplituden in der Größenordnung von einigen hundert
Nanometern erforderlich. Hierfür sind erhöhte Impulsheizleistungen nötig,
die zu einer Erwärmung des Sensors führen. Bei Anlegen einer
Brückenspeisespannung stellen die DMS-Widerstände zusätzliche Wärmequellen
dar, so daß sich eine deutliche Temperaturüberhöhung gegenüber der
Sensorumgebung einstellt. Aufgrund der thermischen Wechselwirkung zwischen
Resonator und einem umströmenden Fluid, erfolgt eine
geschwindigkeitsabhängige Kühlung infolge des thermischen
Anemometerprinzips. Diese bewirkt eine Änderung der mechanischen
Spannung im Siliziumbalken und führt somit zu einer Resonanzfrequenzänderung.
Während bei der Anwendung als Kraftsensor die Temperaturabhängigkeit der
Resonanzfrequenz unerwünscht ist und durch geeignete Maßnahmen kompensiert
werden muá, kann sie gezielt zur Strömungsmessung in einem bekannten Medium
eingesetzt werden. Wird der Sensor in einem Strömungsrohr betrieben, so
lassen sich
Gasströmungen erfassen und die Fluidgeschwindigkeit messen. Der Einfluß der
thermischen Fluid-Struktur-Wechselwirkung wurde numerisch untersucht
\cite{Mes93} und eine universelle elektrische Ansteuerschaltung für die
elektrothermische
Schwingungsanregung entwickelt \cite{Wie93}. Die im Rahmen des
BMFT-Verbundprojektes entwickelten Balkenresonatoren konnte auf diese Weise
meátechnisch weiter charakterisiert und die Strömungssensitivität
nachgewiesen werden \cite{Fab93b}.
