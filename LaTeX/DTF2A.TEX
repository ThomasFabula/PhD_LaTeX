\chapter[Analytische Beschreibung]
 {Analytische Beschreibung mikromechanischer Strukturen}
\label{beschreibungmikromechanischerstrukturen}


\section{Physikalische Beschreibungsebenen}
\label{beschreibungsebenen}

Dieses Kapitel behandelt die grundlegenden physikalischen
Beschreibungsweisen von mikromechanischen Strukturen. Zur gezielten
Vorhersage des Verhaltens ist es notwendig, auf eine der
Problemstellung angepaßte physikalische Beschreibungsweise zu achten.
Daher müssen entsprechende physikalische Theorien und Näherungsmethoden
im Entwurfsprozeß von Mikrostrukturen Anwendung finden, um eine
effiziente Berechnung zu ermöglichen. Die Festkörperphysik beschreibt
die Phänomene von Festkörpern durch die Untersuchung von mikroskopischen
Vorgängen auf atomarer Ebene, während die Beschreibung des makroskopischen
Verhaltens ausgedehnter Materie durch phänomenologische Theorien
wie Kontinuumsmechanik oder Thermodynamik erfolgt. Bei der Vorhersage des
mechanischen Verhaltens von Mikrostrukturen, bei der in der Regel von der
Diskretheit des Kristallgitters abgesehen werden kann, hat die
Elastizitätstheorie eine wichtige Bedeutung gewonnen.
Fr ausgewählte Grundgeometrien mikromechanischer Bauelemente wie
Biegebalken und Membranen werden analytischen Methoden der
Elastizitätstheorie vorgestellt. Diese ermöglichen das statische und
dynamische Verhalten mikromechanischer Strukturen näherungsweise zu beschreiben.
Die Untersuchung grundlegender physikalischer
Einflußgrößen auf das Schwingungsverhalten mikromechanischer
Strukturen erlaubt die Bewertung der Eignung für resonante
Sensoranwendungen. Nichtlineares Bauelementeverhalten, hervorgerufen
durch spannungsversteifende Effekte bei großnen Deformationen, kann für
bestimmte Spezialfälle näherungsweise in einer analytischen Form
dargestellt werden. Die analytischen Berechnungen dienen neben der
Grobdimensionierung der mikromechanischen Bauelemente als Startwerte
für die Finite-Elemente-Simulationen zur sinnvollen Einschränkung des
Parametervariationsraumes.



\subsection{Mikroskopische Modelle}
\label{mikroskopischemodelle}

Das Verhalten mikromechanischer Systeme wird durch die mikroskopischen
Eigenschaften des Festkörpers auf atomarer Ebene, d.h.\ durch den
kristallinen Aufbau festgelegt. Die Festkörperphysik erlaubt durch die
quantenmechanische Beschreibung, die atomistischen Phänomene des
Kristallgitters durch mechanische, elektronische, thermische oder
sonstige Anregungen und deren Wechselwirkung zu erklären. Die
Elastizitätseigenschaften werden in der Gitterdynamik auf die Lage der
Atome und den zwischen ihnen wirkenden Kräften, d.h.\ den
zugrundeliegenden chemischen Bindungstypen der Kristalle zurückgeführt.
Die Bindung der Atome ist hierbei abhängig von der elektrischen
Wechselwirkung der Elektronen und der benachbarten Atomrümpfe, so daß
die auftretenden elektrostatischen Kräfte für die Bindungsstärke
der verschiedenen möglichen Bindungstypen\footnote{Ionenbindung,
kovalente oder metallische Bindung, usw.} verantwortlich sind
\cite{Hel88}.\\
%
Zur Beschreibung der Dynamik eines dreidimensionalen Kristallgitters
kann der Festkörper als eine unendlich ausgedehnte, periodische
Anordnung von Atomen angesehen werden. In diesem einfachsten Fall kann
das diskrete Gitter durch eine eindimensionale, unendlich lange Kette
von gleichartigen Atomen angenähert werden ({\sl Born}\/sche Näherung).
Beim Festkörper ist der interatomare Potentialverlauf für die
Wechselwirkung zwischen den Atomen verantwortlich, wobei das
resultierende elektrostatische Potential eine Überlagerung aus der
{\sl Coulomb}\/schen Anziehung und der des Abstoßungspotentials ist.
Der exakte asymptotische Potentialverlauf kann in der harmonischen
Näherung durch einen quadratischen Potentialverlauf beschrieben werden.
Der mittlere Atomabstand entspricht der Gleichgewichtslage und
zeichnet sich durch das Potentialminimum aus.
Die Kräfte, die auf die Atome wirken, ergeben sich aus dem Gradienten
des Wechselwirkungspotentials und werden in linearer Näherung
({\sl Hooke}\/sches Verhalten) proportional zur Abstandsdifferenz der
Atome angenommen.\\
Unter Zugrundelegung der Linearität der Gitterkräfte, wird bei der
Quantisierung der elastischen Gitterschwingungen von der Annahme
einzelner harmonischer Oszillatoren ausgegangen. Als Träger der
diskreten Schwingungsenergie werden die Phononen (Schallquanten)
eingeführt. Analog zu den quantisierten Energieeigenwerten des
harmonischen Oszillators besitzt ein Phonon in der
Schwingungsmode $k$ mit der Frequenz $\omega_{k}$ die Energie
$E_{k} = \hbar \omega_{k}$.
Die elastische Energie des Festkörpergitters stellt sich als eine
šberlagerung aller Schwingungsmoden $k$ dar, die ihrerseits mit
$n_{k} \, (\, = \, 0,1,2, \ldots , \infty )$ Phononen besetzt
sind. Die Gesamtenergie läßt sich daher als Summe der einzelnen
Oszillatorbeiträge schreiben
$E = \sum_{k} \hbar \omega_{k} \left( n_{k} + \frac{1}{2} \right)$.
Da $n_{k}$  nur ganzzahlige Werte annehmen kann, entspricht dieses der
Quantisierung der Energie elastischer Schwingungszustände in linearen
Festkörpergittern.  Insbesondere bleibt die Nullpunktsenergie
endlich $E(n_{k} = 0) = \frac{1}{2} \hbar \omega_{0}$.
Eine Abweichung vom linearen Kraftgesetz zwischen den Gitterteilchen
tritt bei Amplitudenzunahme der Gitterschwingungen auf
und es stellen sich anharmonische Effekte ein.  Die Eigenschwingungen
des Festkörpers sind nicht mehr entkoppelt (orthogonal) und im
Phonenbild ergeben sich zusätzliche Wechselwirkungen zwischen den
Phononen (z.B.\  Drei-Phononen-Prozesse).  Es findet ein Energieaustausch
statt und der Schwingungszustand wird durch Emission oder Absorption von
Phononen verändert. Als Folge hiervon resultieren die thermische
Ausdehnung und die endliche Wärmeleitfähigkeit in Kristallen \cite{Kit88}.
Eine zusätzliche Auswirkung ist, daá die adiabatischen und isothermen
elastischen Konstanten nicht mehr gleich sind und von Druck oder
Temperatur abhängig werden (siehe Kapitel~2.1.3).  Durch Einführung
weiterer \glqq vermittelnder Teilchen\grqq \, werden zusätzliche Wechselwirkungen
im Festkörper quantenmechanisch beschrieben.
So führt die Elektronen-Phononen-Wechselwirkung\footnote{Wechselwirkendes
Teilchen: Polaron.} durch eine lokale Deformation des Kristallgitters
zum elektrischen Widerstand, während Exzitonen für Polarisationswellen
und Magnonen fr Magnetisierungswellen verantwortlich
sind \cite{Pau75}. \\
Das Elektronenverhalten in periodischen Gitterstrukturen kann durch die
Halbleiterphysik beschrieben werden, wobei mit Hilfe des Bändermodells
die elektronischen Eigenschaften des Festkörpers erklärt werden können.
Aufgrund des periodischen Verlaufs der potentiellen Energie der Atomrümpfe,
spüren die Elektronen im Festkörper ein periodisches Gitterpotential.
Die möglichen Energieeigenwerte der Elektronen werden im Rahmen des
Bändermodells bestimmt und in quantenmechanisch erlaubte und verbotene
Energiebänder eingeschränkt. Die Bandstruktur, die Besetzung der einzelnen
Bänder und die Bandlücke ({\sl Bandgap}), die das
Leitungsband vom Valenzband trennt, ist dafür verantwortlich, ob es sich
um einen Leiter, Halbleiter oder Isolator handelt.  Die Energiebänder
und die Bandlücke sind sowohl temperatur- als auch druckabhängig.  Der
bei mikromechanischen Sensoren wichtige {\em piezoresistive} Effekt,
d.h.\ die druckabhängige
Widerstands- bzw.\ Leitfähigkeitsänderung in Halbleitern,
kann mikroskopisch durch eine Änderung der Bandlücke und der
Besetzungswahrscheinlichkeit infolge eines
äußeren mechanischen Druckes auf den Festkörper erklärt werden \cite{Mid89}.


\subsection{Kontinuumsmechanische Beschreibung}
\label{kontinuumsmechanik}

Beim Übergang vom diskreten Kristallaufbau zum Kontinuum kann im
klassischen Grenzfall die Quantisierung der mikroskopischen Feldgrößen
vernachlässigt werden. Auf makro\-skopischer Ebene wird als phänomenologische
Theorie die Kontinuumsmechanik bzw.\ Elastizitätstheorie zur Beschreibung
elastomechanischer Festigkeitsprobleme eingesetzt.
Ein schwingender elastischer Festkörper ist ein kontinuierliches System
mit nahezu unendlich vielen Massenpunkten (Größenordnung: $10^{23} \,
cm^{-3}$) und besitzt somit unendlich viele Bewegungsfreiheitsgrade. Die
vollständige Bewegung des Systems kann nur durch die Lagekoordinaten
{\em aller} Punkte exakt beschrieben werden.  Die in der klassischen Mechanik
entwickelten {\sl Lagrange}\/schen und {\sl Hamilton}\/schen Formulierungen
ermöglichen die
Beschreibung der statischen (Deformationen) und dynamischen Vorgänge
(z.B.\ Starrkörperbewegungen, elastische Schwingungen) sowohl diskreter
als auch kontinuierlicher Systeme.  Bei letzteren stellt die
{\sl Lagrange}\/-Dichte die Differenz zwischen kinetischer und poten\-tieller
Energiedichte dar und ermöglicht unter Verwendung des {\sl Hamilton}\/schen
Variationsprinzips die Ableitung der {\sl Lagrange}\/schen
Bewegungsgleichungen. Durch diesen Formalismus wird die Aufstellung
allgemeiner Bewegungsgleichungen auch bei nichtmechanischen
Problemstellungen ermöglicht \cite{Gol78}.
In Kapitel~3 wird die Formulierung der FE-Methode auf der Grundlage von
Variationsprinzipien für elastomechanische und elektromechanische
Kontinua dargestellt.\\
In der Kontinuumsmechanik ist die Existenz eines eindeutigen und
stetigen Verschiebungsfeldes eine unabdingbare Voraussetzung. Es handelt
sich hier um das geometrische Grundgesetz der Elastizitätstheorie.
Die Bewegungs- bzw.\ Zustandsgleichungen des elastischen Festkörpers lassen
sich aus einem Variationsprinzip, der Minimierung
der potentiellen Energie\footnote{{\em Prinzip der virtuellen Arbeit} bzw.\
{\em virtuellen Verschiebungen}.} ableiten.
%
%Satz von {\sl Castigliano}
%
Eine hierzu gleichwertige Methode ist es, die Bewegungsgleichungen direkt
mit Hilfe des {\sl Newton}\/schen Gesetzes aus dem lokalen Gleichgewicht
der am Festkörperelement wirkenden inneren und äußeren Kräfte aufzustellen.
Unter Ausnutzung des {\sl d'Alembert}\/schen Prinzips kann unter Einführung
von zusätzlichen Trägheitskräften das dynamische Problem auf ein statisches
reduziert werden \cite{Som49}:
%
\begin{equation}
\label{kontmech}
%\fbox{$
% \displaystyle
% \begin{array}{rcl}
 \rho \displaystyle \frac{ \partial^{2}{u_{i}}}{ \partial t^{2}} = 
 \sum_{j} \frac{\partial \sigma_{ij}}{\partial x_{j}} % div \, \sigma
 \, + \, f_{i}
 % div \, \tau \, +  \, \mbox{externe Volumenkräfte}
 \qquad \mbox{$i$ = 1, 2, 3}
% \end{array}
% $}
\end{equation}
wobei:
\begin{eqnarray*}
 % {\vec u} & = &  {\vec x'} - {\vec x}
  u_i{}        & = &  x'_{i} - x_{i}
  \, : \, \mbox{Verschiebungsvektor}\\
  \sigma_{ij}  & : & \mbox{allgemeiner Spannungs- bzw. Drucktensor}\\
  f_{i}        & : & \mbox{externe Volumenkr„fte}\\
%  \tau    & : & \mbox{generalisierter {\sl Maxwell}\/scher Spannungstensor}\\
  \rho    & : & \mbox{homogene Materialdichte.}
\end{eqnarray*}
Die Gleichung (\ref{kontmech}) wird auch als
\glqq Spannungs-Divergenz-Theorem\grqq \, bezeichnet und beschreibt das
Verhalten elastischer Kontinua. Bei rein statischen Problemstellungen
verschwinden die Trägheitskräfte auf der linken Seite.
Zur vollständigen Beschreibung des kontinuumsmechanischen Randwertproblems
mssen zusätzlich die Verschiebungen bzw.\ Kräfte auf
der Oberfläche des Körpers gewissen Randbedingungen gengen.\\
%
Die Festlegung des Spannungszustandes wird durch den Zusammenhang von
Spannungen und Dehnungen geliefert und ist werkstoffabhängig. Zur
Darstellung eines beliebigen Deformationszustandes sind drei Normalspannungen
($ \sigma_{xx}, \sigma_{yy}, \sigma_{zz}$) sowie drei Schubspannungen
($ \sigma_{xy}, \sigma_{yz}, \sigma_{zx} $)
erforderlich. Zu diesen Spannungen korrespondieren drei Dehnungen
(Dilatationen bzw.\ Kontraktionen:
$ \varepsilon_{xx}, \varepsilon_{yy}, \varepsilon_{zz} $)
und drei Scherungen (Torsion: $ \gamma_{xy}, \gamma_{yz}, \gamma_{zx} $),
durch deren Überlagerung ebenfalls jede beliebige Deformation eines Körpers
erzeugt werden kann. Unter Verwendung des allgemeinen Materialgesetzes,
auch Werkstoffgesetz genannt, gilt für den Zusammenhang zwischen den
mechanischen Spannungen  $\sigma$ und den Dehnungen $ \varepsilon $ im
Festkörper\footnote{Im weitern gilt:
$x \, = \, 1, y \, = \, 2, z \, = \, 3$.}:
%
\begin{eqnarray}
\label{matgesetz}
 \sigma_{ij} & = & \sum_{k,l}^{} C_{ijkl} \, \varepsilon_{kl}
 \qquad \mbox{bzw.} \qquad
 \varepsilon_{ij} \; = \; \sum_{k,l}^{} S_{ijkl} \, \sigma_{kl}
\end{eqnarray}
wobei $i, j, k, l \in \{1,2,3\}$.
Die elastischen Materialeigenschaften werden hierbei entweder durch den
Tensor der Elastizitätsmoduln $C_{ijkl}$ oder durch den der
Elastizitätskoeffizienten $S_{ijkl}$ beschrieben, wobei
$C_{ijkl} \, = \, S_{ijkl}^{-1}$
gilt. Die Spannungen und Dehnungen sind Tensoren 2.\ Stufe während die
Materialeigenschaften durch Tensoren 4.\ Stufe beschrieben werden, was
sich durch ihr Verhalten bei Koordinatentransformationen zeigt.
Aufgrund der Gleichheit der Schubspannungen (Momentengleichgewicht)
ist der Spannungstensor symmetrisch, d.h.\ $\sigma_{ij} \, = \, \sigma_{ji}$.
Da der Spannungstensor von der Wahl des Koordinatensystems abhängig ist, ist
es sinnvoll ihn in drei Hauptspannungsanteile $\sigma^{H}_{i}$ (i = 1,2,3)
zu zerlegen, die invariant unter Koordinatentransformationen sind. Diese
drei Hauptrichtungen des Spannungstensors sind orthogonal zueinander und
erlauben Aussagen über die maximal zulässigen Bauelementebelastungen.\\
%
Die Definition des Dehnungstensors kann unabhängig von den Verschiebungen
über die Änderung der \glqq Quadrate des Abstandes\grqq \,
zweier Massenpunkte bei einer Deformation erfolgen \cite{Lan74}:
%
\begin{eqnarray}
 ds'^{2} - ds^{2} & =: &  2 \sum_{i,j}^{} \varepsilon_{ij} dx_{i}  dx_{j}
\end{eqnarray}
%
wobei:
\begin{eqnarray*}
 ds^{2} & = & \sum_{k,l}^{} \delta_{kl} dx_{k} dx_{l}
\end{eqnarray*}
%
Der kinematische Zusammenhang\footnote{Auch Kompatibilitäts-
bzw.\ Verträglichkeitsbedingung nach {\sl Saint-Venant} genannt.}
zwischen dem Verschiebungsfeld $u$ und den
Dehnungen $\varepsilon$ wird durch den {\sl Green-Lagrange}\/schen
Dehnungstensor bei endlichen Deformationen beschrieben:
%
\begin{eqnarray}
\label{dehn}
 \varepsilon_{ij} & = & \displaystyle \frac{1}{2} \left (
 \underbrace{ \frac{\partial u_{i}}{ \partial x_{j}} +
 \frac{ \partial u_{j}}{ \partial x_{i}} }_{\mbox{linear}} +
 \sum_{k}^{} \underbrace{ \frac{ \partial u_{k}}{ \partial x_{i}}
 \cdot \frac{ \partial u_{k}}{ \partial x_{j} }}_{ \mbox{nichtlinear } }
\right )
\end{eqnarray}
Wichtige Eigenschaften dieses Tensors sind seine Symmetrie
($\varepsilon_{ij} = \varepsilon_{ji}$) und seine Invarianz
gegenber Starrkörperrotationen. Das technische Dehnungsmaß
vernachlässigt bei infinitesimalen Deformationen kleine
Verschiebungsgradienten, so daß in linearer Näherung ({\sl Hooke}\/sches
Elastizitätsverhalten) das gemischte quadratische Glied entfällt
und es sich dann um den linearisierten {\sl Cauchy-Green}\/schen
Dehnungstensor handelt. Bei endlichen Dehnungen korrespondiert zum
nichtlinearen {\sl Green-Lagrange}\/schen Dehnungstensor~(\ref{dehn})
der symmetrische 2.\ {\sl Piola-Kirchhoff}\/sche Spannungstensor
\cite{Bat82}. Auf die numerische
Berücksichtigung des quadratischen Gliedes des Dehnungstensors wird in
Kapitel~3 unter geometrischen Nichtlinearitäten eingegangen.


\subsection{Thermodynamische Beschreibungsebene}
\label{thermodynamik}

Ein mikromechanisches System wechselwirkt fortw„hrend mit seiner
Umgebung und steht im gegenseitigen Energieaustausch.  Dieser Austausch
von Energie führt zu einer physikalischen Zustandsänderung des Systems.
Hierbei ist jeder Energieform vom thermodynamischen Standpunkt aus
eine unabhängige thermodynamische Zustandsgröße (Zustandsvariable)
zugeordnet.  Die intensiven Variablen können als
\glqq treibende Kräfte\grqq \, im
Festkörper angesehen werden, die die extensiven Variablen als
\glqq Reaktionen\grqq \, zur Folge haben.  In diesem Sinne kann jedes
mikromechanische Bauelement als ein elektro-thermo-mechanisches System
aufgefaßt werden. Das Gesamtmikrosystem läßt sich nach {\sl Wachutka}
hierbei aus seinen Teilsystemen, bestehend aus Kristallgitter und
Ladungsträgern (Elektronen und Löcher) zusammengesetzt denken \cite{Wac92}.
In {\bf Abbildung~\ref{abbwachutka}}
sind die allgemeinen intensiven Zustandsvariablen zur Beschreibung
eines mikromechanischen Systems abgebildet.
%----------------------- Beginn: Figure-Environment ----------------------
\begin{figure}[htb]
\begin{center}
% --- Dateiname des Bildes
\input{abbze.tex}
\setabbze
\end{center}
\caption{\label{abbwachutka}
 Allgemeine intensive Zustandsvariablen eines mikromechanischen
 Systems (nach [Wachutka 92])}
\end{figure}
%----------------------- Ende: Figure-Environment ----------------------
Die Zusammenh„nge zwischen intensiven und extensiven Zustandsvariablen
sind dabei durch die thermodynamischen Zustandsgleichungen gegeben, welche
die entsprechenden physikalischen Wechselwirkungseffekte beschreiben. Die
intensiven Zustandsvariablen eines Mikrosystems sind nach
Abbildung~\ref{abbwachutka} durch den mehrdimensionalen Zustandsvektor:
%
\begin{eqnarray}
 X & = & \left ( \sigma, \vec{E}, \vec{H}, T_{L}, \, T_{1}, \ldots ,
         T_{N}, \, \phi_{1}, \ldots , \phi_{N} \right )
\end{eqnarray}
%
gegeben.  Hierbei ist $\sigma$ der bereits oben definierte
Spannungstensor\footnote{Bei fluiden Medien handelt es sich um den
Drucktensor.},
$\vec{E}$ und $\vec{H}$ die elektrische und magnetische Feldstärke,
$T_{L}$ die Temperatur des Substrats (z.B.\ des Kristallgitters),
$T_{1}, \ldots , T_{N}$  und $ \phi_{1}, \ldots , \phi_{N} $ die
Temperatur und das elektrochemische Potential der beweglichen Ladungsträger,
die einen möglichen Massen- oder Ladungstransport im System verursachen
können. Die entsprechend zugeordneten extensiven Zustandsvariablen sind
durch den Zustandsvektor:
%
\begin{eqnarray}
Y & = & \left ( \varepsilon , \vec{D}, \vec{B}, s_{L}, \, s_{1}, \ldots
, s_{N}, \, c_{1}, \ldots , c_{N} \right )
\end{eqnarray}
%
gegeben. Hierbei ist $\varepsilon$ die mechanische Dehnung, $\vec{D}$
die dielektrische Verschiebung (Flußdichte) und $\vec{B}$ die
magnetische Induktion, $s_{L}$ die Entropiedichte des Substratmediums,
$s_{k}$ und $c_{k}$ die Entropiedichte
und Ladungsträgerkonzentration der $k$-ten Teilchensorte
($k \, = \, 1 \ldots N$). Die Zustandsvariablen sind im allgemeinen orts-
und zeitabhängig. Die intensiven und extensiven Zustandsvariablen sind durch
Materialkonstanten verknüpft, die von den äußeren Randbedingungen des
thermodynamischen Systems abhängig sind. Insbesondere unterscheiden sie
sich bei isothermen (Temperatur: $T = const$) und adiabatischen Prozessen
(Entropie: $S = const$).
Die nachfolgenden Betrachtungen gehen in erster Näherung von linearen
Abhängigkeiten, also Wechselwirkungseffekten erster Ordnung aus.
Zusätzlich sollten für eine angepaáte Modellbildung die Zustandsgleichungen
und -variablen auf den das Mikrosystem beschreibenden minimalen
Parametersatz kondensiert werden.\\
In dieser Arbeit wird das Hauptaugenmerk auf die Interaktion des
mechanischen Verschiebungsfeldes mit dem umgebenden Temperaturfeld und
dem elektrischen Feld gelegt (in Abbildung~\ref{abbwachutka}
gestrichelt hinterlegt). Das Temperaturfeld führt infolge der
temperaturabhängigen Materialeigenschaften und den unterschiedlichen
Wärmeausdehnungskoeffizienten mikromechanischer Multilayersysteme zu
thermisch induzierten Spannungen.  Das elektrische Feld verursacht in
piezoelektrischen Medien eine bidirektionale elektromechanische
Kopplung, die zum direkten und reziproken piezoelektrischen Effekt
fhrt. Auf die Behandlung anderer Wechselwirkungsmechanismen die bei
Mikrosystemen Anwendung finden, z.B.\ elektronische, elektrothermische
oder magnetische Effekte, sei auf Arbeiten von {\sl Wachutka}
\cite{Wac90, Wac91} verwiesen.\\
In {\bf Abbildung~\ref{abbnye}} sind die Zusammenhänge zwischen den
intensiven und extensiven Zustandsvariablen bei
elektro-thermo-mechanischer Wechselwirkung in linearer Näherung
dargestellt \cite{Nye57}.
%----------------------- Beginn: Figure-Environment ----------------------
\begin{figure}[htb]
\begin{center}
% --- Dateiname des Bildes
\input{abbzz.tex}
\setabbzz
\end{center}
\caption{\label{abbnye}
 Wechselwirkung der intensiven und extensiven thermodynamischen
 Zustandsvariablen bei elektro-thermo-mechanischer Kopplung (nach
 [Nye 57])}
\end{figure}
%----------------------- Ende: Figure-Environment ----------------------
Die Größen im äußeren Dreieck entsprechen den intensiven Zustandsvariablen
($\sigma, \vec{E}, T$), die extensiven Zustandsvariablen
($\varepsilon, \vec{D}, S$) befinden sich
im inneren Dreieck. Die direkten Verbindungen zwischen
intensiven und extensiven Zustandsvariablen, durch einen dicken Balken
gekennzeichnet, entsprechen Vorg„ngen, die sich auf Wechselwirkungen
innerhalb der drei Subsysteme beschränken.  Sie beschreiben rein
mechanische, thermische
oder elektrische Effekte.  Erst durch die Ankopplung der anderen
Subsysteme treten Wechselwirkungen wie thermoelastische Effekte,
die Piezo- und Pyroelektrizität in Erscheinung.
Werden die intensiven Zustandsgrößen ($\sigma, \vec{E}, T$) als
unabhängige Variablen\footnote{In den
Gleichungen~(\ref{piezo1}~--~\ref{piezo3}) sind
($\varepsilon, \vec{D}, S$) die abhängigen Variablen.}
verwendet, so lassen sich die elektro-thermo-mechanischen
Kopplungseffekte durch die folgenden Zustandsgleichungen eindeutig
beschreiben \cite{Ike90a, Nye57}:
%
\begin{eqnarray}
\label{piezo1}
 \varepsilon_{ij} & = & \sum_{k,l} S_{ijkl}^{E,T} \sigma_{kl} +
                        \sum_{k} d_{kij}^{T} E_{k} +
                        \alpha_{ij}^{E} \delta T \\
\label{piezo2}
 D_{n}            & = & \sum_{k,l} d_{nkl}^{T} \sigma_{kl} +
                        \sum_{k} \epsilon_{nk}^{\sigma, T} E_{k} +
                        p_{n}^{\sigma} \delta T \\
\label{piezo3}
 \delta S         & = & \sum_{i,j} \alpha_{ij}^{E} \sigma_{ij} +
                        \sum_{k} p_{k}^{\sigma} E_{k} +
                        (c^{\sigma, E}/T) \delta T
\end{eqnarray}
%
wobei $\delta T$ die Temperaturänderung und $\delta S$ die
Änderung der Entropiedichte bedeuten. Die Konstanten der jeweiligen
Systeme sind durch:
\begin{eqnarray*}
 S_{ijkl}^{E,T}            & : & \mbox{elastische Steifigkeitskoeffizienten}\\
 \epsilon_{nk}^{\sigma,T}  & : & \mbox{Permittivitäten (bzw.\
                                  Dielektrizitätskonstanten)}\\
 c^{\sigma,E}              & : & \mbox{spezifische Wärmekapazität}
\end{eqnarray*}
gegeben und die Materialkonstanten $d, \alpha$ und $p$ beschreiben
die elektromechanische, thermomechanische und thermoelektrische
Wechselwirkung:
\begin{eqnarray*}
    d_{kij}^{T}     & : &   \mbox{piezoelektrische Moduln}\\
    \alpha_{ij}^{E} & : &   \mbox{thermische Ausdehnungskoeffizienten}\\
    p_{k}^{\sigma}  & : &   \mbox{pyroelektrische Koeffizienten.}
\end{eqnarray*}
Die Zustandsgleichungen lassen sich auch durch die Verwendung der
extensiven Zustandsgrößen ($\varepsilon, \vec{D}, S$) auf der
rechten Seite der Gleichungen~(\ref{piezo1}--\ref{piezo3}) oder
durch eine Kombination beider Variablentypen ausdrcken.
Die Beziehungen zwischen den Zustandsgrößen sind eindeutig,
wenn die äußeren Bedingungen des thermodynamischen Systems festgelegt
sind. Die Materialkonstanten werden hierzu mit dem jeweiligen konstant
gehaltenen Parameter als hochgestellte Symbole, d.h.\ $\sigma$~=~$const$
bzw.\ E~=~$const$ bzw.\ T~=~$const$, gekennzeichnet. Bei kompakter
Darstellung der Tensoren unter Weglassung der Indizes und der jeweiligen
thermodynamischen Randbedingung der Materialkonstanten können obige
Zustandsgleichungen durch das Kopplungsschema gemäß
{\bf Abbildung~\ref{abbkopplungsschema}}
übersichtlich dargestellt werden, wenn die vermittelnden
Materialtensoren zu einer Gesamtmatrix\footnote{Der Superskript t
kennzeichnet die jeweils transponierte Materialmatrix, so daß die
Gesamtmatrix wieder symmetrisch ist. T steht für die absolute
Temperatur.} zusammengefaßt werden \cite{Nye57}:
%----------------------- Beginn: Figure-Environment ----------------------
\begin{figure}[htb]
\begin{center}
% --- Dateiname des Bildes
\input{abbzd.tex}
\setabbzd
\end{center}
\caption{\label{abbkopplungsschema}
 Gesamtmatrix der elektro-thermo-mechanischen
 Wechselwirkungseffekte in einem Festkörper (nach [Nye 57])}
\end{figure}
%----------------------- Ende: Figure-Environment ----------------------
