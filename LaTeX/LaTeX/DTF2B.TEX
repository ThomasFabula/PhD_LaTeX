\section[Materialeigenschaften]
{Materialeigenschaften mikromechanischer Werkstoffe}
\label{materialeigenschaften}

Als Substratmaterial wird in der Mikromechanik vorwiegend
einkristallines Silizium eingesetzt.  Silizium gehört der vierten
Hauptgruppe
des Periodensystems der Elemente an und besitzt gerichtete
tetraedrisch-kovalente Atombindungen.  Die Kristallstruktur weist ein
kubisch flächenzentriertes Raumgitter auf.  Die kubische Einheitszelle
besitzt eine Kantenlänge von 0,543 nm und enthält acht Silizium-Atome.
Kohlenstoff, Germanium und Zinn kristallisieren ebenfalls in dieser
Diamantstruktur, die ein Inversionszentrum aufweist, so daß diese Elemente
aufgrund der Kristallsymmetrie keine piezoelektrischen Eigenschaften
besitzen.
Als nicht zentrosymmetrische Kristalle sind hingegen Quarz ($SiO_{2}$)
und Zinkoxid ($ZnO$) piezoelektrisch und weisen jeweils eine
Vorzugsrichtung\footnote{Kristallographisch wird die Vorzugsrichtung
als $c$--Achse und die dazu senkrechten Richtungen als $a$--Achsen
bezeichnet.} auf. Einkristalliner Quarz spielt für die
Frequenzstabilisierung und resonanten Sensoranwendungen bereits seit
längerem in Form von verschiedenen Schwingergeometrien (z.B.\ Dicken-
und Flächenscherschwinger) eine bedeutende Rolle \cite{Bue91b}. Hierbei hat
lediglich die unterhalb von 573~$^\circ$C stabile Modifikation, der
sogenannte $ \alpha $--Quarz, eine technische Bedeutung.\\
%
Um Siliziumstrukturen als piezoelektrisch angetriebene Sensoren oder
Aktoren einsetzen zu können, bedarf es zusätzlicher aktiver Wandlerschichten,
beispielsweise piezoelektrischer Dünnschichten wie Aluminiumnitrid oder
Zinkoxid. In {\bf Tabelle~\ref{tabsiznoquarz}} sind für Silizium, Zinkoxid
und Quarz grundlegende physikalische Eigenschaften im Vergleich
zusammengestellt.
%----------------------- Beginn: table ---------------------------
\begin{table}[htb]
\caption{\label{tabsiznoquarz}
 Vergleich der grundlegenden physikalischen Eigenschaften von Silizium,
 Zinkoxid und Quarz}
 % [Kit88, LB82, Tic80]}
\begin{center}
\begin{tabular}{|l||c|c|c|} \hline
{\bf Material:} & Silizium & Zinkoxid  & $ \alpha $--Quarz \\  \hline \hline
Kristallart: & kubisch  & hexagonal        & trigonal          \\ \hline
elastische Konstanten:                             & 3 & 5 & 6 \\ \hline
Schönflies:           & T  & $ C_{6v} $ & $ D_{3} $      \\ \hline
Hermann-Mauguin:      & 23 & 6mm             &  32            \\ \hline
Gitterkonstante [nm]: & $ a_{0} \, = \; 0,543 $ &
                        $ a_{0} \, = \, 0,325 $ &
                        $ a_{0} \, = \, 0,491 $                \\
                      &                          &
                        $ c_{0} \, = \, 0,521 $  &
                        $ c_{0} \, = \, 0,540 $                \\ \hline
Charakter:        & ind. Halbleiter &  Halbleiter & Isolator   \\ \hline
Bandlücke [eV]:   & 1,14            &  3,20       & 8,90       \\ \hline
piezoelektrisch:  & nein            &  ja         & ja         \\ \hline
\end{tabular}\\
\end{center}
\end{table}
%----------------------- Ende: table ---------------------------
Die Angaben zu den
Materialeigenschaften wurden verschiedenen Literaturquellen \cite{Kit88,
LB82, Tic80} entnommen. Die Daten beziehen sich auf einkristalline
ausgedehnte Festkörper, sogenanntes Bulk-Material, bei
Zimmertemperatur (T~=~300~K). Aufgrund der unterschiedlichen
Symmetrieeigenschaften des kubischen, hexagonalen und trigonalen
Kristallgitters variiert die Anzahl der unabhängigen elastischen
Konstanten. Die Bezeichnung der Symmetrieklassen ist an {\sl Schönflies}
und {\sl Hermann-Mauguin} (Internationale Punktgruppen-Bezeichnung)
angelehnt. Während Quarz ein ausgezeichneter
Isolator\footnote{Die Leitfähigkeit beträgt bei Quarz weniger als
$10^{-21} \: S/m$.} ist, werden Silizium und Zinkoxid den Halbleitern
zugeordnet. Im Gegensatz zu Zinkoxid weist Silizium einen indirekten
Bandübergang auf.\\
Das anisotrope elastische Verhalten von Festkörpern ist durch die
Zusammenhänge in Gleichung (\ref{matgesetz}) bestimmt.
Zur physikalischen Beschreibung werden hierzu
die Koordinatenachsen in die Symmetrieachsen des Kristalls gelegt.  In
Abhängigkeit der Symmetrieeigenschaften des Kristallgitters reduziert
sich die Anzahl der unabhängigen Materialparameter des beschreibenden
Materialtensors. Der Spannungs- und Dehnungstensor besitzen jeweils 36
Komponenten, die aufgrund der Symmetrie $ \sigma_{ij} =
\sigma_{ji}$ und $\varepsilon_{ij}  =  \varepsilon_{ji} $ auf
21 unabh„ngige Komponenten reduziert werden.  Dieser Fall entspricht der
triklinen Kristallklasse $ C_{1} $, die somit die maximale Unsymmetrie
aufweist. Bei einkristallinem Quarz reduziert sich die Anzahl weiter
auf sechs unabhängige Konstanten \cite{Bri85, Tic80}. In der technischen
Literatur hat sich die einfachere Matrixschreibweise gegenüber der exakten
Tensorschreibweise durchgesetzt, bei der die Komponenten des Spannungs-
und Dehnungstensors durch sechs unabhängige Größen\footnote{Die Zuordnung
der Matrixeinzelindizes zu den Tensordoppelindizes erfolgt nach dem Schema:
$ii \, \rightarrow \, i$ fr $i \, = \,$1--3$, \; 23/32 \, \rightarrow \, 4,
 \; 13/31 \, \rightarrow \, 5$ und $12/21 \, \rightarrow \, 6$.}
ausgedrückt werden.
% Für den Dehnungstensor gilt beim Übergang auf die Matrixschreibweise
% die Identität \cite{Tic80}:
% \begin{eqnarray}
%  \varepsilon_{ij} & = & \displaystyle \frac{1}{2} \left ( 1 \, + \,
%                   \delta_{ij} \right ) \; \varepsilon_{k}
% \end{eqnarray}
% wobei $i,j = 1--3$  und $k = 4--6$.
Der Elastizitätstensor läßt sich daher durch eine symmetrische
6x6--Matrix darstellen, deren
Koeffizienten die {\sl Voigt}schen Elastizitätsmoduln $C_{ij}$ sind. Das
Materialgesetz ergibt sich dann in vereinfachter Matrixnotation zu:
%
\begin{eqnarray}
\label{voigt}
 \sigma_{i} & = & \sum_{j} C_{ij} \, \varepsilon_{j}
\end{eqnarray}
%
Das Besetzungsschema der symmetrischen Elastizitätsmatrix $C_{ij}$ sieht
bei kubischer Kristallsymmetrie des Festkörpers wie folgt aus:
\begin{displaymath}
 \left ( \begin{array}{llllll}
         C_{11}  &  C_{12}   &   C_{12}  &    0   &    0   &   0 \\
                 &  C_{11}   &   C_{12}  &    0   &    0   &   0 \\
                 &           &   C_{11}  &    0   &    0   &   0 \\
                 &           &           & C_{44} &    0   &   0 \\
                 &           &           &        & C_{44} &   0 \\
%\scriptstyle symmetrisch
                 &           &           &        &        & C_{44}\\
 \end{array}
 \right )
\end{displaymath}

Die drei unabhängigen elastischen Konstanten von einkristallinem
Silizium betragen bei Raumtemperatur $ C_{11} = 165,8$~GPa,
$C_{12} = 63,9$~GPa und $C_{44} = 79,6$~GPa \cite{LB82}.
Die Richtungen $x$, $y$ und $z$ sind die kubischen Kristallrichtungen
[100], [010] und [001].  Die Besetzung der $ S_{ij} $--Matrix
ist identisch, wobei für den Zusammenhang der Komponenten gilt
\cite{Nye57}:
%
\begin{eqnarray}
C_{11} & = & \displaystyle \frac { S_{11}+S_{12} }
             {(S_{11}-S_{12}) (S_{11}+2S_{12})}
             \nonumber \\
C_{12} & = & \displaystyle \frac{-S_{12} }
             {(S_{11}-S_{12}) (S_{11}+2S_{12})} \\
C_{44}           & = & \frac{1}{2} (C_{11} - C_{12})
                  = \displaystyle \frac{1}{ S_{44} } \nonumber
\end{eqnarray}
%
Beim Übergang vom kubischen Kristall zum isotropen Körper ergibt sich
aus Symmetriegründen eine Reduktion der Komponenten auf zwei. Die beiden
unabhängigen Konstanten können durch den isotropen Elastizitätsmodul $E$
({\sl Young}scher Modul) und die Querkontraktionszahl $\nu$
({\sl Poisson}sche Zahl) ausgedrückt werden. Zusätzlich ist der Schubmodul
durch $G$ = E/2(1+$\nu$) definiert. Da für erste Abschätzungen in der Regel
eine grobe Näherung ausreichend ist oder die anisotropen Eigenschaften von
Dünnschichtmaterialien nicht bekannt sind, kann mit isotropem
Materialverhalten gerechnet werden \cite{Heu89}:
%
\begin{eqnarray}
 E   & = & C_{11} \, = \, \frac{1}{S_{11}} \nonumber \\
 \nu & = & - \frac{S_{12}}{S_{11}} \\
 G   & = & \frac{1}{S_{44}} \nonumber
\end{eqnarray}
%
Unter Berücksichtigung der elektrischen und thermischen Wechselwirkung
mit dem elastischen Verschiebungsfeld kommen zum Elastizitätstensor gemäß
den Zustandsgleichungen~(\ref{piezo1}--\ref{piezo3})
zusätzliche kopplungsbeschreibende Materialtensoren hinzu.
In {\bf Abbildung~\ref{abbmatrix6mm}} ist das Besetzungsschema der
Materialmatrizen bei elektro-thermo-mechanischer Wechselwirkung für
ein hexagonales Kristallsystem (Punktsymmetrieklasse: 6mm) angegeben
\cite{Nye57}.
%----------------------- Beginn: Figure-Environment ----------------------
\begin{figure}[htb]
\begin{center}
% --- Dateiname des Bildes
\input{abbzv.tex}
\setabbzv
\end{center}
\caption{\label{abbmatrix6mm}
 Besetzungsschema der Materialmatrizen bei hexagonaler
 Kristallsymmetrie (nach [Nye 57])}
\end{figure}
%----------------------- Ende: Figure-Environment ----------------------
Hierbei bedeuten dicke Punkte die Besetzung eines
Matrixplatzes und die Verbindungslinien bedeuten Gleichheit der
Komponenten.  Das Kreuzsymbol bei der (6,6)--Komponente der
Elastizitätsmatrix steht fr $2(S_{11} - S_{12})$ bzw.\
$ \frac{1}{2} (C_{11} - C_{12}) $. Die symmetrische Gesamtmatrix
besitzt maximal $ (6+3+1) \cdot (6+3+1) = 100 $ Komponenten.
Beim hexagonalen Kristallsystem kommen zu den fnf unabhängigen
elastischen Koeffizienten (Tabelle~\ref{tabsiznoquarz}) weitere drei
piezoelektrische, zwei dielektrische Koeffizienten und jeweils ein
pyroelektrischer, sowie ein thermischer Koeffizient hinzu. Diese
insgesamt 14 unabhängigen Materialkennwerte
charakterisieren das mechanische, thermische und elektrische
Materialverhalten vollständig.


\subsection{Strukturmaterialien}
\label{substratmaterialien}

In {\bf Tabelle~\ref{tabsubstrate}} sind die mechanischen und
thermischen Eigenschaften von einigen wichtigen Substratmaterialien
bei Zimmertemperatur (T = 300 K) zusammengestellt, die für die
Realisierung mikromechanischer Bauelemente Verwendung finden.  Neben den
einkristallinen Werkstoffen Silizium und Quarz (siehe
Tabelle~\ref{tabpiezoelektrika})
werden auch verschiedene Siliziumverbindungen, wie z.B.\ Siliziumoxid
($ SiO_{2} $) und Siliziumnitrid ($ Si_{3}N_{4} $) eingesetzt.  Als
Ausgangsmaterial
dienen Siliziumscheiben (Wafer), die je nach Spezifikation mit
verschiedenen Orientierungen (100, 110, 111), Fremdatomdotierungen (Bor,
Phosphor) und Oberflächenbeschaffenheit (z.B.\ einseitig oder doppelseitig
poliert) mit unterschiedlichen Scheibendurchmessern (100--200~mm) bezogen
werden können. Einkristalliner Quarz wird für mikromechanische
Anwendungen in der Form quadratischer Quarzblanks mit 1,5'' Kantenlänge und
genau definierten Kristallorientierungen (z.B.\ AT--, Z--Schnitt) verwendet.
Siliziumnitrid und poly--Silizium ({\em poly--Si}) können durch spezielle
Prozeßverfahren, mit Hilfe von $SiO_{2}$--Opferschichten,
% (engl.: Sacrificial Layer Methode),
als freitragende Strukturen hergestellt werden \cite{Kam90}.
Siliziumnitrid-Schichten können sowohl durch physikalische
{\em PVD}--Verfahren
(Sputtern), als auch durch chemische {\em LPCVD}-- (= \underline{L}ow
\underline{P}ressure {\sl CVD}) und {\em PECVD}--Verfahren
(= \underline{P}lasma \underline{E}nhanced {\em CVD}) bei Temperaturen
von 780--850~$^\circ$C und etwa 350~$^\circ$C aus der Gasphase
abgeschieden werden. Siliziumoxid wird durch thermisches Oxidieren
(trocken oder naß bei 950--1100~$^\circ$C), als {\em LTO}
(=\underline{L}ow \underline{T}emperature \underline{O}xide) bei
Temperaturen von 450--650~$^\circ$C oder als {\em PECVD}--Oxid bei
300~$^\circ$C hergestellt. Die Abscheideraten sind abhängig von
den Prozeßparametern und betragen etwa 0,5--10 nm/min. Typische
Schichtdicken liegen zwischen einigen hundert Nanometern und wenigen
Mikrometern. Die nachfolgende Zusammenstellung gibt einen Überblick über
die mechanischen Eigenschaften (E--Modul, {\sl Poisson}--Zahl $\nu$,
Dichte $\rho$), Wärmeausdehnungskoeffizienten $\alpha$ und die Größenordnung
der prozeßbedingten inneren Spannungen
\cite{Bach, Bue91a, Heu89, Kam90, Pet79, Tab89, Wandt}.
%----------------------- Beginn: table ---------------------------
\begin{table}[htb]
\caption{\label{tabsubstrate}
 Eigenschaften mikromechanischer Strukturmaterialien}
\begin{center}
\begin{small}
\begin{tabular}{|l||c|c|c|c|c||l|}\hline
{\bf Material} & E--Modul & $\nu$ & $\varrho$ & $\alpha$ & Spannung &
                 Bemerkung: \\
                 & [GPa] &  & [kg/m$^{3}$] & [ppm/K] & [MPa]   & \\
\hline \hline
c--$Si_{110}$   &   169   & 0,063  &  2329  & 2,62 & 86 bordot. &
                                               einkristallin \\
\hline
\multicolumn{7}{|c|}{Dnnschichtmaterial} \\
\hline
{\em poly-Si}  & 160--175 & 0,23   &  2330  & 3,30 & 10  &
                                               polykristallin, LPCVD \\
\hline
$SiO_{2}$      & 57--92   &  0,17  &  2220  & 0,3--0,6 & 10 bis 1000 &
                                                therm. Naáoxid, LTO \\
\hline
$Si_{3}N_{4}$  & 140--320 &  0,3   &  3100  & 2,80  & spannungsarm &
                                                      PE-, LPCVD \\
\hline
\end{tabular}
\end{small}
\end{center}
\end{table}
%----------------------- Ende: table ---------------------------
Neben der Anwendung als Mikrostruktur mit Dicken von 1--3~$\mu$m werden die
Schichten vor allem als Passivierung mit Dicken von 100--300~nm eingesetzt.
Da die Materialparameter mikrotechnisch hergestellter Schichtsysteme
stark vom technologischen Herstellungsprozeß beeinflußt werden,
unterscheiden sie sich u.U.\ erheblich von denen des Bulk-Materials und
weisen in der Regel prozeßtechnisch bedingte, innere Spannungen auf.
Diese Spannungen sind i.a.\ nicht homogen über den Gesamtwafer verteilt
und werden zusätzlich von einer anschließenden Strukturierung der
Bauelemente beeinflußt. Die Materialeigenschaften mikrotechnisch
hergestellter Schichten sind zusätzlich von der Schichtdicke abhängig.
Die Gründe hierfür sind die veränderte Gitterstruktur an der Oberfläche,
andere Schichtmorphologie (meist polykristalline Gefügestruktur) und
herstellungsbedingte Fremdatomanteile aus der Restgasatmosphäre, die zu
einer veränderten Stöchiometrie führen können.  Die makroskopischen
Schichteigenschaften ändern sich ab einer Dicke von 10 nm,
dieses entspricht bereits etwa 100 Atomlagen. Neben der
Änderung der morphologieabhängigen Materialeigenschaften treten
zusätzlich innere Schichtspannungen auf.  Diese sind auf die
unterschiedlichen Wärmeausdehnungskoeffizienten der Schicht und des
Substrates, sowie auf innere Spannungen, hervorgerufen durch
Gitterverzerrungen infolge unterschiedlicher Gitterabstände
zurückzuführen. Durch die mechanischen Spannungen kann die
Haftfestigkeit der Dünnschichten stark herabgesetzt werden, so daß es im
Extremfall zur Schichtablösung kommt.  Aufgrund der größeren
Defektdichte im kristallinen Aufbau weisen Dünnschichten in der Regel
eine höhere Zugfestigkeit als Bulk-Material auf \cite{Gra91}.



\subsection{Piezoelektrische Dnnschichten}

Piezoelektrische Dünnschichten können in mikromechanischen Anwendungen
gleichzeitig als sensitive und aktive Wandlerschichten eingesetzt
werden. Durch Ausnutzung des direkten piezoelektrischen Effektes
(Abbildung~\ref{abbnye}) kann eine mechanische Spannung infolge einer
Dehnung in der Schicht
eine Ladungsverschiebung im Material hervorrufen, die über einen
Ladungsverstärker als elektrische Spannung an geeignet angebrachten
Elektroden abgegriffen werden kann.  Hierdurch können mikromechanische
Dünnschicht-Mikrophone \cite{Fra88, Kim87} oder piezoelektrische
Abtastelemente für schwingungsfähige Mikrostukturen \cite{Blo90}
realisiert werden.  Der reziproke piezoelektrische Effekt erlaubt es
durch Anlegen eines elektrischen Feldes die piezoelektrische Schicht
einer Formänderung zu unterwerfen und bildet damit die Grundlage für
statisch oder dynamisch angetriebene Sensoren und Aktoren. Das Ausnutzen
der pyroelektrischen Aktivität von Zinkoxid erlaubt ferner Anwendungen
als Wärmestrahlungsdetektor \cite{Pol84}. \\
{\bf Tabelle~\ref{tabpiezoelektrika}} zeigt einen Vergleich
der Materialeigenschaften verschiedener Piezoelektrika
\cite{Fra88, Ike90a, LB82, Tic80, Sie81} für eine
Bezugstemperatur von 300~K. Im Vergleich zu den in
Dünnschichttechnologie, meist durch Sputterprozesse, hergestellten
polykristallinen Materialien Aluminiumnitrid ({\em AlN}) und {\em ZnO}
zeichnen sich gesinterte {\em PZT}-Piezokeramiken (Blei-Zirkon-Titanat)
durch extrem hohe piezoelektrische Koeffizienten\footnote{Analog zu
Gleichung~(\ref{voigt}) werden die piezoelektrischen Koeffizienten
$d_{ijk}$ bzw.\ Moduln $e_{ijk}$ in vereinfachter Matrixnotation als
$d_{ij}$ bzw.\ $e_{ij}$ geschrieben.}
$d_{ij}$ aus.  Neben {\em ZnO} weisen auch
{\em AlN} und {\em PZT} ein hexagonales Kristallgitter
(Symmetrieklasse: 6mm) auf, so daß die
Materialeigenschaften direkt miteinander verglichen werden können.
Einkristallines Quarz besitzt dagegen eine andere Kristallsymmetrie
und zeichnet sich durch relativ niedrige piezoelektrische Koeffizienten
aus.\\
%
Ein Maß für die Effizienz der Umwandlung von elektrischer in mechanische
Energie stellt der elektromechanische Kopplungsfaktor dar. Bei
Bimorphstrukturen setzt sich der effektive Kopplungsfaktor $k_{eff}$
aus einem material- {\em und} einem geometrieabhängigen Anteil zusammen:
\begin{eqnarray}
\label{kgeo}
 k_{eff} & = & k_{mat} \cdot k_{geo}
\end{eqnarray}
Bei komplexen Wandlergeometrien kann er nur mit Hilfe gekoppelter
FE-Berechnungen unter Berücksichtigung der {\em mechanischen} und
{\em elektrischen} Randbedingungen ermittelt werden (siehe
Gleichung~\ref{kfem}), weil die elastische und dielektrische
Energieverteilung im Festkörper von der Wandlergeometrie, den
Einspannbedingungen und den elektrischen Abschlußbedingungen abhängig ist.
In der einfachsten Näherung läßt sich $k_{eff}$ durch die
Serienresonanzfrequenz $f_{s}$ und die Parallelresonanzfrequenz $f_{p}$
des elektrischen Ersatzschaltbildes\footnote{Siehe hierzu Abbildung~5.2 in
Kapitel~5.1.} des piezoelektrischen Wandlerelementes
ausdrücken \cite{Ler90}:
\begin{eqnarray}
\label{keff}
 k_{eff} & = & \sqrt{ \frac{f_{p}^{2} - f_{s}^{2}}{f_{p}^{2}}}
\end{eqnarray}
In Kapitel~5 werden verschiedene Bimorphwandler mit piezoelektrischen
Dünnschichten numerisch und experimentell untersucht, um den Einfluß von
Geometrie und Elektrodenform auf den effektiven elektromechanischen
Kopplungsfaktor zu bestimmen.\\
%
Für den {\em materialabhängigen} Kopplungsfaktor $k^{mat}$ gilt bei
Transversal- oder Planarschwingern \cite{VIB}:
\begin{eqnarray}
\label{kmat}
 \left( k_{31}^{mat} \right)^{2} & = &
    \displaystyle \frac{ d_{31}^{2} }{ S_{11}^{E} \cdot
                     \epsilon_{33}^{\sigma} } \\
 k_{p}^{mat} & = & \sqrt{ \frac{2}{(1 - \nu)}} \, k_{31}^{mat}
\end{eqnarray}
Diese Werte sind von dem mechanischen Steifigkeitskoeffizienten
$S_{11}^{E}$, der {\sl Poisson}--Zahl $\nu$ sowie den piezo- und
dielektrischen Konstanten $d_{31}$ und $\epsilon_{33}^{\sigma}$
($\sigma = const$) der piezoelektrischen Schicht abhängig.\\
%
% Für den Zusammenhang zwischen dem materialabhängigen elektromechanischen
% Kopplungsfaktor $k_{mat}$ und den Materialdaten, bei unterschiedlichen
% thermodynamischen Randbedingungen gilt \cite{Ike90a}:
% \begin{eqnarray}
%\label{kmatelekrandbed}
% 1 - k^{2}_{mat} & = & \displaystyle
%   \frac{C^{E}}{C^{D}} = \frac{S^{D}}{S^{E}} =
%   \frac{\epsilon^{\varepsilon}}{\epsilon^{\sigma}}
% \end{eqnarray}
% Auf diese Weise kann mit Hilfe von $k_{mat}$ das Verhältnis der
%Steifigkeiten bei elektrischem Kurzschluß ($\vec E=0$) oder bei offenen
%Anschlußelektroden des Wandlers ($\vec D=0$) beschrieben ausgedrückt
%werden.\\
%
Im folgenden soll eine Abschätzung der Leistungsfähigkeit mikromechanischer
Biegewandler mit piezoelektrischem Antrieb in erster Näherung
mit den {\em materialabhängigen} transversalen und planaren Kopplungsfaktoren
$k_{31}^{mat}$ und $k_{p}^{mat}$ für reine Transversal- oder Planarschwinger
erfolgen. Um einen Vergleich der elektromechanischen Kopplungsfaktoren mit
Quarz vornehmen zu können, wird der Wert
$k_{Quarz}$ = $d_{11}$/$\sqrt{S^{E}_{11} \cdot \epsilon^{\sigma}_{11}}$
verwendet. In Abhängigkeit der
verschiedenen mechanischen Steifigkeiten, der piezo- und dielektrischen
Eigenschaften ergeben sich fr Aluminiumnitrid, Zinkoxid und PZT-Keramik
(hier: {\sl VIBRIT420}, \cite{Sie81}) unterschiedlich große
Kopplungsfaktoren in der Größenordnung von etwa 18~\%, 40~\% und 60~\%.
Aufgrund des Einflusses
der Querkontraktion ($\nu$) unterscheidet sich der planare Kopplungsfaktor
$k_{p}$ um einen Faktor von 1,6 bis 1,9 vom transversalen Kopplungsfaktor
$k_{31}$.
%----------------------- Beginn: table ---------------------------
\begin{table}[htb]
\caption{\label{tabpiezoelektrika}
 Materialeigenschaften von Piezoelektrika}
 % [Fra88, Ike90a, LB82, Sie81, Tic80]}
\begin{center}
\begin{footnotesize}
\begin{tabular}{|c||c|c|c|c|c|c|c|c|}
\hline
{\bf Material} &  $ \varrho  $ &  $ S^{E}_{11}, S^{E}_{12} $ &
$ d_{31} $  &  $ d_{33} $   &  $ \epsilon^{\sigma}_{11} / \epsilon_{0} $ &
$ \epsilon^{\sigma}_{33} / \epsilon_{0} $                                &
$ \alpha_{th} $             &  $ k_{p}^{mat} $ \\
     & [kg/m$^{3}$] &  [TPa$^{-1}$]  & [pC/N]  &  [pC/N] & & & [ppm/K] & \\
\hline \hline
Quarz  &  2649  & 12,78 & $ d_{11} = 2,30 $ & $ d_{14} = -0,67 $ &
4,51   &  4,63  & $ 7,48 \, \| \, c $             & (0,1) \\
       &       &  -1,81 &  &  &  &  & $ 13,7 \, \bot \, c $ & \\
\hline \hline
$AIN$    &  3260  & 3,53  & -2,00  & 5,53 & 9,04 & 11,4 & $ 4,15 \, \| \,
c $ &
                                                              0,18 \\
       &        & -1,01 &        &       &     &      & $ 5,27 \, \bot
\, c $
                                                      & \\
\hline
$ZnO$    &  5675  & 7,91  & -5,12  &  12,3 & 9,26 & 8,20 & $ 2,92 \, \| \,
c $ &
                                                               0,40 \\
       &        & -3,30 &        &       &     &      & $ 4,75 \, \bot
\, c $
                                                      & \\
\hline
$PZT$    &  7600  & 15,5  & -160   &   355 & 1600 & 1600 & $ \approx 7,0 $
                                                       & 0,61 \\
       &        & -5,70 &        &       &      &      &           &
                                                        \\
\hline
\end{tabular}
\end{footnotesize}
\end{center}
\end{table}
%----------------------- Ende: table ---------------------------
In dieser Arbeit wird ein Schwerpunkt auf die Berechnung und
Charakterisierung mikromechanischer Bimorphstrukturen gelegt, die aus
Bulk-Silizium mit piezoelektrischen {\em ZnO}-Dünnschichten aufgebaut sind.
Daher soll kurz auf die Eigenschaften von Zinkoxid eingegangen werden.
Eine genaue Darstellung des Herstellungsverfahrens und der prozeßabhängigen
Materialeigenschaften ist in \cite{Wag94} zu finden. \\
%
Zinkoxid ist ein II-VI-Halbleiter, besitzt eine hexagonale Kristallstruktur
und weist ein Gitter vom Wurtzit-Typ auf. Aufgrund der 6mm Kristallsymmetrie,
sind die beiden $x_{1}$-- und $x_{2}$--Koordinatenachsen
($a$--Kristallachsen) senkrecht zur $x_{3}$--Koordinatenachse
($c$--Kristallachse) und kristallographisch gleichwertig, so daß die
mechanischen Eigenschaften in der $x_{1}/x_{2}$--Ebene gleich sind, d.h.\
$C_{11}$ = $C_{22}$, $C_{13}$ = $C_{23}$ und $C_{44}$ = $C_{55}$.
Für die piezoelektrischen Koeffizienten folgt aus den gleichen
Symmetrieüberlegungen $d_{31}$ = $d_{32}$ und
$d_{15}$ = $d_{24}$ = $-8,3$ pC/N \cite{LB82}. Die durch Sputtern
erzeugten polykristallinen Dünnschichten weisen eine starke Orientierung
bezüglich der $c$--Achse auf. Das Aufwachsen der {\em ZnO}-Kristallite
erfolgt entlang dieser Vorzugsrichtung senkrecht zur Silizium-Substratebene.
Neben den mechanischen sind auch die piezoelektrischen Eigenschaften der
{\em ZnO}-Schichten stark von den technologischen Prozeßparametern abhängig.
Insbesondere hat die innere Schichtspannung einen großen Einfluß auf den
elektromechanischen Kopplungsfaktor. Für den
Einsatz als aktive Wandlerschicht in mikromechanischen Bauelementen muß
das Schichtsystem passiviert werden.  Zum einen müssen nachfolgende
Prozesse vor einer Kontamination durch {\em ZnO} bewahrt werden, andererseits
muß das {\em ZnO} vor Umwelteinflüssen geschützt werden.  Hierzu sind im
Rahmen des BMFT-Verbundprojektes vom Projektpartner {\em Robert Bosch GmbH}
plasmaabgeschiedene und gesputterte dielektrische Schichten ($ SiO_{2},
SiO_{x}N_{y}, Si_{3}N_{4} $) entwickelt und untersucht worden, wobei
eine Prozeßintegration des Zinkoxids mit der Silizium-Bulkmikromechanik
erreicht werden konnte \cite{ABV93}.

