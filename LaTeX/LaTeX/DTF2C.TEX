\section[Analytische Beschreibungsweise]
{Analytische Beschreibung von Mikroresonatoren}
\label{analytischebeschreibungsweise}

Mit Hilfe mikromechanischer Herstellungsverfahren lassen sich
unterschiedliche Resonatorgeometrien realisieren, die in verschiedenen
Schwingungsmoden angeregt werden können. Bei der Sensoranwendung
beruht das Meßprinzip auf der Abhängigkeit der
Ausbreitungscharakteristika elastischer Wellen im Festkörper von
der Meßgröße oder auf der Änderung der Resonanzfrequenz des Mikroresonators.
Fr die in dieser Arbeit betrachteten Resonanzsensoren zur Erfassung der
physikalischen Meßgrößen Kraft, Druck und Strömungsgeschwindigkeit kommen
insbesondere Biegeschwinger auf der Basis von Balken- und
Membranresonatoren in Frage. Bei
einfachen Strukturen können die Eigenfrequenzen und Schwingungsformen
exakt bzw.\ in einer ausreichenden Näherung analytisch berechnet
werden. Unter der
Annahme idealer Einspannbedingungen und homogener, isotroper
Materialeigenschaften werden im folgenden die Resonanzfrequenzen und
lastabhängigen Frequenzänderungen der Resonatoren analytisch abgeleitet.
Durch eine geeignete Wahl der Sensorabmessungen kann der Arbeitspunkt der
resonanten Sensoren, das sind die Resonanzfrequenz und die
Empfindlichkeit gegenüber der Meßgröße, eingestellt werden.


\subsection{Mikromechanische Resonatorstrukturen}
\label{resonatorstrukturen}

Die elastischen Schwingungen, die sich in einem Festkörper ausbilden,
sind von der Schwingergeometrie und den Randbedingungen abhängig.  In
{\bf Abbildung~\ref{abbwellentypen}} sind die verschiedenen
Wellentypen dargestellt, die in
elastischen Festkörpern auftreten können \cite{Whi70}.  Der Festkörper kann
longitudinale Kompressionsschwingungen (a), transversale Biege- (b) und
Torsionsschwingungen, sowie Scherschwingungen (c), diese sind z.B.\
Dicken- oder Flächenscherschwingungen, ausführen. Die elastischen Raumwellen,
die sich innerhalb des Festkörpers fortpflanzen, werden bei den sogennanten
Eigenresonanzschwingern ({\sl BAW = \underline{B}ulk
\underline{A}coustic \underline{W}aves}) ausgenutzt.  Die Wellenlängen
sind von der gleichen Größenordnung wie die typischen Abmessungen des
Schwingers ($\lambda \approx d$), wobei die minimale Wellenlänge
einer Biegeschwingung $2d$ betr„gt.  Die typischen Resonanzfrequenzen
liegen zwischen 10~kHz und 20~MHz. Beispiele hierfür sind
Quarz-Resonatoren auf der Basis von Einfach- und Doppelstimmgabeln
(Uhren-Quarze, Quarz-Kraftsensoren) und die in dieser Arbeit behandelten
Resonanzsensoren auf Siliziumbasis.\\
Oberflächenwellen\footnote{auch {\sl Rayleigh}-Wellen genannt.}
({\sl SAW = \underline{S}urface \underline{A}coustic \underline{W}aves})
sind aufgrund ihrer geringen Eindringtiefen in den Festk”rper an die freie
Oberfläche des Festkörpers gebunden (d). Die Wellenlänge ist gegenüber der
Resonatordicke $d$ sehr
klein, d.h.\ $\lambda \ll d$.  Die Resonanzfrequenzen liegen im
Bereich von 20~MHz bis 2~GHz. Hauptanwendungsgebiete sind Filter und
Sensoren, die nach verschiedenen Funktionsprinzipien arbeiten. Die Frequenz
der SAW-Resonatoren ändert sich infolge von
Massenanlagerung (akusto-gravimetrische Wechselwirkung),
Leitfähigkeitsänderung (akusto-elektrische Wechselwirkung) und der
Steifigkeitsänderung, die eine rein elastische Wechselwirkung darstellt.
Zusätzlich führt die viskoelastische Wechselwirkung zu einer
Geschwindigkeitsänderung und Abschwächung der Oberflächenwelle \cite{Ric91}.
Unter Verwendung selektiv sensitiver Grenzschichten können so chemische
und biochemische Sensoren aufgebaut werden. \\
Plattenwellen\footnote{auch {\sl Lamb}- bzw.\ {\sl Love}-Wellen genannt.}
({\sl FPW = \underline{F}lexural \underline{P}late
\underline{W}aves}) treten bei sehr dünnen Membranstrukturen auf, d.h.
$\lambda \gg d$, und können sowohl in symmetrischen, als auch
antisymmetrischen Schwingungsmoden angeregt werden (e). Die
Resonanzfrequenzen liegen im mittleren Frequenzbereich von etwa
500~kHz bis 10~MHz \cite{Bue91b}. Diese Biege-Plattenwellen-Wandler werden
als chemische und biologische Sensoren auf der Grundlage des
akusto-gravimetrischen Wandlungsprinzips eingesetzt. Außer in
gasförmigen Medien können sie auch in Flüssigkeiten aufgrund geringer
Abstrahlungsverluste eingesetzt werden \cite{Gie92}. Die erreichbaren
hohen Empfindlichkeiten beruhen auf dem großen Verhältnis von
Membranoberfläche zu Volumen, das die Wechselwirkung von akustischer
Energie mit den nachzuweisenden biochemischen Komponenten begünstigt.
Weitere Anwendungen von {\sl Lamb}-Wellen sind Sensoren für die Messung von
Fluiddichte und -viskosität, Differenzdrücken, Beschleunigungen, Temperatur
und Wärmestrahlung \cite{Whi87, Wen88}.
%----------------------- Beginn: Figure-Environment ----------------------
\begin{figure}[htb]
\begin{center}
% --- Dateiname des Bildes
\input{abbzf.tex}
\setabbzf
\end{center}
\caption{\label{abbwellentypen}
 Wellentypen in elastischen Festkörpern}
\end{figure}
%----------------------- Ende: Figure-Environment ----------------------

\clearpage

\subsection{Balkenresonatoren}
\label{balkenreson}

Um das dynamische Verhalten eines doppelseitig eingespannten Biegebalkens
analytisch zu beschreiben, gengt es, den Balken in
einer eindimensionalen Näherung als konti\-nuierliches System aufzufassen.
Unter den Voraussetzungen, daá die Balkendicke wesentlich kleiner als die
Balkenbreite und -länge ist (d.h.\ $l \gg b > h$), der Balken
homogen und ideal prismatisch\footnote{d.h.\ der Balkenquerschnitt und damit
auch die Biegesteifigkeit bleiben über die Balkenlänge konstant.} ist und auf
den Balken eine axiale äußere Kraft einwirkt, können die Biegeschwingungen
für kleine Amplituden durch die folgende eindimensionale
partielle Differentialgleichung beschrieben werden \cite{Wea90}:
%
\begin{equation}
\label{balkdgl}
\fbox{$
 \displaystyle
 \begin{array}{rcl}
  \hat E I \, \displaystyle \frac{\partial^{4} \, u(x,t)}{\partial x^{4}}
  \: - \:
  F \displaystyle \frac{ \partial ^{2} \, u(x,t)}{\partial x^{2}}
  \: + \:
  \rho A \displaystyle \frac{\partial ^{2} \, u(x,t)}{\partial t^{2}}
  \: + \:
  c \displaystyle \frac{\partial \, u(x,t)}{\partial t}
  = f(x,t)
 \end{array}
 $}
\end{equation}
%
mit den Randbedingungen an den Balkenenden:
%
\begin{eqnarray*}
  \displaystyle  u(x,t) \left |_{x=0,l} \right. & = & 0
  \qquad \mbox{und} \qquad
  \displaystyle \frac{ \partial u(x,t) }{\partial x} \displaystyle
  \left |_{x=0,l} \right. \, = \, 0
\end{eqnarray*}
%
Hierbei bezeichnet $u(x,t)$ die zeit- und ortsabhängige transversale
Auslenkung des Balkens senkrecht zur Balkenausdehnung ($x$--Richtung).
Die ersten beiden Terme sind Beiträge aus der
Federsteifigkeit des Balkens, wobei $\hat E I$ die Biegesteifigkeit des
Balkens und $F$ die äußere axiale Kraft darstellen. Der dritte Term
entspricht dem Trägheitsbeitrag, hervorgerufen durch die
homogene Massendichte $\rho$. Die viskosen, d.h.\
geschwindigkeitsproportionalen Dämpfungseffekte werden
durch den Dämpfungskoeffizienten $c$ beschrieben. Auf der rechten Seite
der Gleichung stellt $f(x,t)$ eine zeitlich, eventuell auch örtlich,
variierende Anregungskraft dar. Das Materialverhalten wird als isotrop
angenommen, wobei $\hat E = E / (1-\nu^{2})$ für den
reduzierten E-Modul steht und $\nu$ die {\sl Poisson}-Zahl ist.
Die Balkenfläche $A = b \cdot h$ wird als rechteckig\footnote{Die in der
Mikromechanik in (100)-Silizium anisotrop geätzten Biegebalken weisen
trapezförmige Balkenquerschnitte mit einem Winkel von 54,74$^{\circ}$ auf,
deren äquatoriales Flächenträgheitsmoment sich nach:
\begin{eqnarray}
%\label{Itrapez}
 I & = & h^{3}(B^2+4Bb+b^2)/36(B+b)
\end{eqnarray}
berechnet, wobei $b$ die obere und $B$
die untere Balkenbreite ist. In der Regel handelt es sich um sehr breite
Balken, bei denen $B \geq b \gg h$, so daß in erster Näherung die
Querschnitte als rechteckig angenähert werden können.} angenommen, so daß
das äquatoriale Flächenträgheitsmoment $I = b \cdot h^{3} / 12$ ist.
Bei der Beschränkung auf den homogenen Fall ohne Anregung
(d.h.\ $f(x,t) = 0$) und unter Vernachlässigung von Dämpfungseffekten
($c = 0$) läßt sich die Lösung obiger Gleichung (\ref{balkdgl}) durch einen
Separationsansatz in der Form:
%
\begin{eqnarray}
  u(x,t) & = & \sum_{n} \, \Phi_{n}(x) \cdot \left
  ( A_{n} \cos \omega_{n}t \; + \; B_{n} \sin \omega _{n} t \right )
\end{eqnarray}
%
darstellen, wobei $f_{n} \, = \, \omega_{n}/{2\pi}$ die Eigenfrequenzen
und $\Phi_{n}(x)$ die ortsabhängigen Eigenschwingungsformen der $n$--ten
Schwingungsmode sind. Die $\Phi_{n}(x)$ lassen sich durch eine
šberlagerung von trigonometrischen und hyperbolischen Funktionen
ausdrücken und führen zu exakten Lösungen der homogenen Gleichung
%(\ref{balkdgl})
\cite{Bou91}. Die lastabhängigen Eigenfrequenzen können durch:
%
\begin{eqnarray}
\label{freqF}
f_{n}(F) & = & f_{n}(0) \: \sqrt{ 1 \, + \, \gamma_{n} \cdot
               \frac{Fl^{2}}{12 \hat E I}}
\end{eqnarray}
%
angenähert werden, wobei:
\begin{eqnarray*}
        f_{n}(0)   & := & f_{n}(F=0) \nonumber \\
        \gamma_{n} & = & 12(k_{n} - 2) \, / \, k_{n}^{3}
\end{eqnarray*}
Auáerdem gilt $k_{n} = (n + \frac{1}{2}) \pi$ fr $n \geq 3$ und
$k_{1}$ = 4,730 sowie $k_{2}$ = 7,853.
Der Näherungsfehler beträgt weniger als 0,5~\%, falls der Steifigkeitsbeitrag
der Axialkraft kleiner als die Biegesteifigkeit des Balkens ist, d.h.\
$\gamma_{n} Fl^{2} < 12 \hat E I$. Als Annahme geht bei der Lösung
der homogenen Gleichung            %(\ref{balkdgl})
ein, daß die Schwingungsamplituden klein
gegenüber der Balkendicke $h$ sind, d.h.\ $(u_{max}/h)^2 < 1$. Für das
Verhältnis der Resonanzfrequenzen der Oberschwingungen bezogen auf die
unbelastete Grundfrequenz $f_{0}$ gilt:
%
\begin{eqnarray}
        \frac {f_{n}}{f_{0}} & = & \left( \frac{k_{n}}{4,73} \right)^2
\end{eqnarray}
%
Die unbelastete Resonanzfrequenz $f_{1}(F = 0)$ der Grundbiegeschwingung
berechnet sich nach:
%
\begin{equation}
\label{balkfreq}
\fbox{$
 \displaystyle
 \begin{array}{rcl}
    f_{1}(0) & = & \frac{ {4,73}^{2} }{2 \pi}
        \sqrt{ \frac{ \hat E I}{\rho A l^{4}}}
        \approx 1,028 \cdot \frac {h}{l^{2}} \, \sqrt{ \frac{\hat E}{\rho}}
 \end{array}
 $}
\end{equation}
%
Das statische und dynamische Verhalten eines beidseitig eingespannten
Biegebalkens unter dem Einfluß einer Temperaturerhöhung wird in
Kapitel~\ref{temperaturverhalten} behandelt.


\subsection{Membranresonatoren}
\label{membranreson}

Die in der Literatur fr Membranen angegebenen analytischen Zusammenhänge
für den Auslenkungs- und Spannungszustand, sowie die Resonanzfrequenzen
beruhen auf der {\sl Kirchhoff}schen Platten- und Schalentheorie und gehen
von der Annahme ebener Objekte aus, deren laterale Abmessungen
wesentlicher größer als deren Dicke ist \cite{Tim87}. Bei der Ermittlung der
Eigenfrequenzen und Schwingungsformen werden verschiedene
Näherungsverfahren (z.B.\ \cite{You50}) verwendet. Für die Resonanzfrequenzen
$f_{ij}$ ebener Membranen gilt \cite{Ble84}:
%
\begin{equation}
\label{memfreq}
\fbox{$
 \displaystyle
 \begin{array}{rcl}
     f_{ij} & = & \frac{\lambda_{ij}^{2}}{2 \pi \sqrt{12}} \,
     \frac{h}{a^{2}} \, \sqrt{ \frac {\hat E}{\rho}}
 \end{array}
 $}
\end{equation}
%
wobei $a, h$ die Membranseitenl„nge bzw.\ -dicke und
$i$, $j$ die Anzahl der Halbwellen in $x$-- bzw.\ $y$--Richtung darstellen.
Die Konstante $\lambda^{2}_{ij}$ ist von den Einspannbedingungen und dem
Seitenverhältnis der Membran abhängig und wird auch als normierte
Eigenfrequenz fr die jeweilige Schwingungsmode bezeichnet. In
{\bf Tabelle~\ref{tabcij}} ist eine Zusammenstellung der $\lambda_{ij}^{2}$
für eine fest eingespannte, quadratische, homogene Membran gegeben.
Die Werte für anisotropes Materialverhalten wurden \cite{Pon91} entnommen.
%----------------------- Beginn: table ---------------------------
\begin{table}[htb]
\caption{\label{tabcij}
 Normierte Eigenfrequenzen $\lambda_{ij}^{2}$ einer ebenen Siliziummembran}
\begin{center}
\begin{tabular}{|l||c|c|c|} \hline
 {\bf Material}: & {\bf isotrop} & {\bf isotrop} & {\bf anisotrop} \\
\hline
 Literatur:      & \cite{Ble84}  &  \cite{Pon91} & \cite{Pon91} \\ \hline
\hline
{\bf Mode ij}   &               &               & \\
 1 $ \quad $ {\bf 11} & 35,99  & 35,99 & 35,16 \\
 2 $ \quad $ {\bf 21} & 73,41  & 73,39 & 71,91 \\
 3 $ \quad $ {\bf 12} & 73,41  & 73,39 & 71,91 \\
 4 $ \quad $ {\bf 22} & 108,30  & 108,22 & 104,35 \\
 5 $ \quad $ {\bf 31} & 131,60  & 131,78 & 130,09 \\
 6 $ \quad $ {\bf 13} & 132,20  & 132,41 & 130,67 \\
 7 $ \quad $ {\bf 32} & 165,15  & 165,16 & 159,73 \\
 8 $ \quad $ {\bf 23} & 165,15  & 165,16 & 159,73 \\
 9 $ \quad $ {\bf 33} & ---     & 220,32 & 211,34 \\ \hline
\end{tabular}
\end{center}
\end{table}
%----------------------- Ende: table ---------------------------
Fr die Ermittlung der Eigenfrequenzverschiebung als Folge einer
Druckeinwirkung ergibt sich die šberlagerung eines statischen
und dynamischen Problems. Im
ersten Schritt wird die Membranmittenauslenkung als Funktion des
einwirkenden Druckes ermittelt, im zweitem Schritt die
Frequenzverschiebung als Funktion der Membranauslenkung.  Die
šberlagerung ergibt die Frequenzverschiebung als Funktion des
einwirkenden Druckes. In der Literatur wird von verschiedenen Ansätzen
ausgegangen, bei denen lineare Zusammenhänge für relativ dicke Membranen
bei kleinen Auslenkungen bzw.\ nichtlineare Zusammenhänge für dünne
Membranen bei großen Auslenkungen angegeben werden.  Für quadratische,
fest eingespannte Platten der Dicke $h$ gilt fr den nichtlinearen
Zusammenhang zwischen einwirkendem Druck $p$ und der
Membranmittenauslenkung $ d $ nach \cite{Cha87}:
%
\begin{eqnarray}
\label{pvond}
 p & = & 16 \hat E \left( \frac{h}{a} \right)^{4} \, \left[ 4,19847
 \frac{d}{h} \, + 1,5816 \left( \frac{d}{h} \right)^{3} \right]
\end{eqnarray}
%
Diese Gleichung kann mit Hilfe der Lösungsformel nach {\sl Kardan} aufgelöst
und zur Berechnung der Resonanzfrequenz„nderung in Abhängigkeit von der
Auslenkung herangezogen werden $f_0 = f(d=0)$ \cite{Utt87}:
%
\begin{eqnarray}
\label{fvond}
     f(d) & = & f_{0} \, \sqrt{ 1 \, + \, c \left( \frac{d}{h} \right)^{2}}
\end{eqnarray}
%
Die Konstante $c$ ist von der Einspannung und der Membranform
abhängig. Für kreisförmige, fest eingespannte Membranen existiert eine
analytisch exakte Lösung mit dem Wert $c = 1,464$ \cite{Utt87}.



\section{Skalierungsverhalten dynamischer Größen}
\label{skalierungsverhalten}

In der Mikromechanik werden im wesentlichen die in
{\bf Abbildung~\ref{abbgrundgeometrien}}
dargestellten drei grundlegenden Strukturgeometrien für Sensor- und
Aktoranwendungen realisiert.
%----------------------- Beginn: Figure-Environment ----------------------
\begin{figure}[htb]
\begin{center}
% --- Dateiname des Bildes
\input{abbzs.tex}
\setabbzs
\end{center}
\caption{\label{abbgrundgeometrien}
 Mikromechanische Grundstrukturen}
\end{figure}
%----------------------- Ende: Figure-Environment ----------------------
Die Resonanzfrequenzen fr die
Biegeschwingungen dieser Strukturen sind von den geometrischen
Abmessungen (d.h.\ Resonatorlänge $l$ und -dicke $h$), der Einspannungsart
und den Materialparametern (bei isotropem Materialverhalten:
Elastizitätsmodul $E$, {\sl Poisson}-Zahl $ \nu $, sowie Materialdichte
$\rho$) abhängig. Für die Resonanzfrequenz der Grundbiegeschwingung
gilt unabhängig von der Geometrie:
%
\begin{eqnarray}
\label{freqskal}
      f & \approx & \frac{h}{l^{2}} \sqrt{ \frac{\hat E}{ \rho}}
         = c_{f} \frac{h}{l^{2}} \sqrt{ \frac{\hat E}{ \rho}}
         = c_{f} \frac{h}{l^{2}} c_{s}
\end{eqnarray}
%
Die Proportionalitätskonstante $c_{f}$ ist von der Resonatorgeometrie
und der Schwingungsmode abhängig und $c_{s}$ stellt die Schallgeschwindigkeit
dar und hängt damit nur von den Materialeigenschaften des Festkörpers ab.
Für die dargestellten drei
mikromechanischen Grundstrukturen sind die geometrieabhängigen
Frequenzkonstanten $c_{f}$ der jeweiligen Grundbiegeschwingung
in {\bf Tabelle~\ref{tabcf}} zusammengefaßt
(siehe Gleichungen (\ref{balkfreq})
und (\ref{memfreq})).
%----------------------- Beginn: table ---------------------------
\begin{table}[htb]
\caption{\label{tabcf}
 Frequenzkonstanten $c_{f}$ verschiedener Grundgeometrien}
\begin{center}
\begin{tabular} {|l|l||c|}
\hline
 Geometrie  &  Einspannung      &  $c_{f}$ \\
\hline \hline
 a)  &  einseitig     & 0,162 \\
 b)  &  doppelseitig  & 1,028 \\
 c)  &  allseitig     & 1,654 \\
\hline
\end{tabular}
\end{center}
\end{table}
%----------------------- Ende: table ---------------------------
Bei der einseitig eingespannten Zungengeometrie kann die Querkontraktion
vernachl„ssigt werden, so daá in diesem Fall $\hat E = E$ ist \cite{Ste91a}.
In {\bf Tabelle~\ref{tabresfreqanalyt}} sind die Resonanzfrequenzen
dieser Grundstrukturen
bis auf die Konstante $c_{f}$ zusammengestellt. Die jeweilige
Resonanzfrequenz ist durch Multiplikation mit der entsprechenden Konstante
zu erhalten. Da die Siliziumsensoren, die aus (100)-orientierten
Wafern hergestellt werden, bei den Biegeschwingungen Dehnungen in
$\langle$110$\rangle$-Kristallrichtung erfahren, gengt es in erster N„herung
die elastischen Eigenschaften isotrop anzusetzen. Fr den Elastizit„tsmodul
$E$ und die {\sl Poisson}-Zahl $\nu$ gilt nach \cite{Heu89}:
\begin{eqnarray}
\label{simat}
      E_{110}   & = & 168,9 \, \mbox{GPa}  \\
      \nu_{110} & = & 0,063 \nonumber
\end{eqnarray}
Die Dichte $\rho$ von Silizium betr„gt 2329 $kg/m^{3}$. Die
angegebenen Materialdaten beziehen sich auf Zimmertemperatur.
Unter Verwendung dieser Materialdaten betr„gt die Schallgeschwindigkeit
$c_{s}$ fr einkristallines Silizium etwa 8533~$m/s$.
%----------------------- Beginn: table ---------------------------
\begin{table}[htb]
\caption{\label{tabresfreqanalyt}
 GrӇenordnung der Resonanzfrequenzen von Grundbiegeschwingungen}
\begin{center}
\begin{tabular}{|l||c|c|c|c||c|}
\hline
 l/h & $10^{1}$ & $10^{2}$ & $10^{3}$ & $10^{4}$ & Mikromechanik \\
 Dicke h & & & & & Technologie \\
\hline \hline
 1 $\mu$m & {\bf 85,33 MHz} &  {\bf 853 kHz} & {\bf 8,53 kHz} & {\bf 85 Hz} & OFM \\
          & (10 $\mu$m)     & (100 $\mu$m)   & (1 mm)         & (10 mm) & \\
\hline \hline
 10 $\mu$m & {\em 8,53 MHz} & {\em 85,3 kHz} & {\em 853 Hz}   & 8,5 Hz &  \\
           & (100 $\mu$m)   & (1 mm)         & (10 mm)        & (10 cm) & BMM  \\
\cline{1-5}
 100 $\mu$m & {\em 853 kHz} & {\em 8,53 kHz} & 85 Hz          & 0,85 Hz & \\
            & (1 mm)        & (10 mm)        & (10 cm)        & (1 m) & \\
\hline
\end{tabular}
\end{center}
\end{table}
%----------------------- Ende: table ---------------------------
Die Strukturdimensionen im oberen Teil der Tabelle ({\bf fett}
geschrieben) lassen
sich nur durch die Technologien der Oberfl„chenmikromechanik
{\em (OFM)} realisieren, w„hrend die {\em kursiv}
gedruckten Werte im Bereich der heute {\em klassischen
Mikromechanik}\footnote{Auch als Bulk-Mikromechanik ({\em BMM}) bezeichnet.}
liegen. Um die Kraftempfindlichkeiten
von Balkenresonatoren abzusch„tzen, gengt es den linearen
Anteil\footnote{Aus dem ersten Glied der Taylorentwicklung folgt fr
$c_{\eta}$ der Wert 0,148 $N^{-1}$.} von Gleichung (\ref{freqF})
zu bercksichtigen. Fr die normierte relative Kraftempfindlichkeit
$\eta$ der Grundmode, bezogen auf die Grundresonanzfrequenz
$f_{0} := f(F=0)$, gilt:
%
\begin{eqnarray}
\label{etadef}
     \eta & := & \frac{1}{f_{0}} \cdot \frac{\partial f}{\partial F}
          \; \approx \; c_{\eta} \cdot \frac{1}{\hat E} \,
            \left( \frac{l}{h} \right)^{2} \, \frac{F}{b \cdot h}
\end{eqnarray}
%
wobei $\partial f$ die Frequenz„nderung und $\partial F$ die Kraft„nderung
bedeuten.
Die Kraftempfindlichkeit eines Balkenresonators ist demnach vom
L„ngen-Dicken-Verh„ltnis $ l/h $ abh„ngig.  Die Empfindlichkeit wird
umso gr”áer, je dnner der Resonatorbalken ist.  Abh„ngig von der Wahl
der Geometrieparameter $l, b$ und $h$ kann ein ann„hernd linearer
Verlauf der Kennlinie\footnote{Die Kennliniennichtlinearit„t kann bei
einer gnstigen Parameterwahl weniger als 1~\% bei Vollast betragen.} in
einem bestimmten Kraftbereich erreicht werden.  In obiger Gleichung
(\ref{etadef})
stellt der letzte Ausdruck auf der rechten Seite die mechanische
Spannung $\sigma_{mech}$ im Resonator dar.  Diese kann die
Bruchspannung $\sigma_{Bruch}^{mat}$ des verwendeten Materials nicht
bersteigen, so daá die theoretisch erreichbare maximale
Frequenz„nderung abgesch„tzt werden kann:
%
\begin{eqnarray}
     \frac{F}{b \cdot h} & = & \sigma_{mech}
     \; \ll \; \sigma_{Bruch}^{mat}
\end{eqnarray}
%
Ersetzt man $\sigma_{mech}$ durch $\sigma_{Bruch}^{mat}$ und betrachtet
das Verh„ltnis der Frequenzverschiebung $\Delta f$ zur Kraft„nderung
$\Delta F$, so gilt unter Verwendung von Gleichung~(\ref{freqskal}):
%
\begin{eqnarray}
\label{etaf}
    \frac{\Delta f}{\Delta F} & = & \eta \cdot f_{0} \; = \;
     \left( \frac{c_{\eta}c_{f}}{\sqrt{\hat E \rho}} \right) \cdot
     \frac{\sigma_{mech}}{h} \; \ll \; const \cdot
     \frac{\sigma_{Bruch}^{mat}}{h}
\end{eqnarray}
%
Gleichung (\ref{etaf}) sagt aus, daá das Verh„ltnis nur von der
Balkendicke, d.h.\ der Abmessung des Balkens in Schwingungsrichtung, abh„ngt.
Der Verkleinerung dieses Parameters stehen aber sowohl prozeátechnische
Grenzen (Balkendicke: $h$ $\approx$ einige $\mu$m) als auch
funktionstechnische Gesichtspunkte entgegen. Falls $h$ zu klein wird, so
nimmt bei Zugbelastung die Bruchgefahr, bei Druckbelastung die
Knickgefahr enorm zu. Als Beispiel soll ein Siliziumbalken mit den
typischen Abmessungen von $l$~=~5,0~mm und $h$~=~20~$\mu$m betrachtet
und die Materialparameter $\hat E$~$\approx$~170~GPa,
$\rho$~$\approx$~2330~$kg/m^{3}$ und $\sigma_{Bruch}$~$\approx$~200~MPa
verwendet werden. Fr die maximal erreichbare Kraftempfindlichkeit
$\eta_{max}$ folgt:
%
\begin{eqnarray}
     \eta_{max}^{Si} & \leq & \frac{c_{\eta}}{\hat E}
     \left( \frac{l}{h} \right)^{2} \sigma_{Bruch}^{Si} \,
      \approx \, 11 \, [N^{-1}]
\end{eqnarray}
%
Dieses entspricht bei einer Grundfrequenz von etwa 7 kHz einer
theoretisch maximal erreichbaren Frequenz„nderung von ann„hernd 77 kHz
pro Newton.


\newpage
\section{Grenzen der analytischen Beschreibungsweise}
\label{grenzenderanalytik}

Exakte analytische L”sungen physikalischer Problemstellungen
existieren nur fr wenige Spezialf„lle, bei denen in der Regel
vereinfachende Annahmen getroffen und idealisierte Randbedingungen
verwendet werden. Bei der analytischen Beschreibungsweise
werden daher durch die Annahmen gewisse Beschr„nkungen auferlegt:\\
%
\begin{itemize}
\item
Einspannungseffekte werden nur angen„hert bercksichtigt, so daá
Randbedingungen bei der L”sung nur unvollst„ndig bercksichtigt werden.
Bei der Behandlung von
Randwertproblemen haben aber gerade aus mathematischer Sicht die
Randbedingungen einen erheblichen Einfluá auf das L”sungsverhalten.
\item
Das Materialverhalten wird linear elastisch und isotrop angesetzt, so daá
die Richtungsabh„ngigkeit der Materialeigenschaften nicht bercksichtigt
wird. Weiterhin wird von einer homogenen Materialverteilung ausgegangen,
d.h.\ ortsabh„ngige Inhomogenit„ten werden nicht bercksichtigt. Auch der
Aufbau aus mehreren Materialien bei Multilayerstrukturen wird nicht genau
erfaát, sondern meist nur ber eine Gewichtung der Materialeigenschaften
entsprechend der Schichtdickenverh„ltnisse.
\item
Die Geometrie der Balken- und Membranschwinger wird als ideal
vorausgesetzt, d.h.\ man geht von homogenen, nicht variierenden
Abmessungen aus und setzt voraus, daá $l~\gg~b~\geq~$~h hinreichend genau
erfllt ist.
\item
Bei Belastungen bleiben die Bauelementequerschnitte eben, so daá
nur ein linearer Dehnungs- bzw.\ Spannungsverlauf ber die Dicke
der Bauelemente zugelassen wird. In der linearen Plattentheorie
k”nnen daher nur \glqq kleine\grqq Verformungen behandelt werden.
\item
Inhomogene Lastverteilungen und komplex berlagerte Bauelementebelastungen
sind nur schwer zu erfassen. Daher werden die behandelten Berechnungsf„lle
oft auf einen homogenen Belastungsfall reduziert.
\item
Bei piezoelektrischer Kopplung werden nur eindimensionale
Ersatzmodelle behandelt, die rein transversale bzw.\ rein longitudinale
Effekte (entkoppelte Schwingungsmoden) beschreiben.
% Zus„tzlich treten bei hohen
% Feldst„rken nichtlineare Effekte (z.B.\ Elektrostriktion) auf, so daá der
% Zusammenhang zwischen der elektrischen Feldst„rke und der
% mechanischen Reaktion des Systems nicht mehr linear ist. Auáerdem ist
% das Groásignalverhalten durch Hystereseeffekte \cite{Smi92a} und
% Erh”hung der dielektrischen Verluste gekennzeichnet.
\end{itemize}
%
Die FE-Methode bietet als allgemeines Berechnungsverfahren die
M”glichkeit, komplexe mikromechanische Strukturen unter Bercksichtigung
von realen Randbedingungen, Einfluá von Nichtlinearit„ten und gekoppelter
Effekte zu berechnen, auch wenn keine analytisch geschlossenen L”sungen
mehr angebbar sind.
