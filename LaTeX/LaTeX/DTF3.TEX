\chapter{Methode der finiten Elemente}
\label{femethode}

Die Finite-Elemente-Methode ({\em FEM}) hat sich in den
Ingenieurswissenschaften wegen ihrer universellen Anwendbarkeit auf
beliebig geformte Strukturen insbesondere auf dem Gebiet der Elastomechanik
gegenüber anderen numerischen Berechnungsverfahren, z.B.\ der
Finite-Differenzen-Methode ({\em FDM})
durchgesetzt. Aufgrund der allgemeinen mathematischen Formulierung
der FE-Methode auf der Grundlage von Variationsprinzipien läß sie sich auch
zur Lösung von verschiedenen physikalischen Feldproblemen einsetzen. Neben
Problemstellungen der
technischen Mechanik werden heute zunehmend auch Wärmefeldprobleme,
gekoppelte Feldprobleme (elektromagnetische, piezoelektrische Wechselwirkung)
und Probleme der Strömungsmechanik erfolgreich behandelt. Exakte analytische
Lösungen partieller Differentialgleichungen, welche die zugrundeliegenden
physikalischen Probleme beschreiben, existieren in der Regel nur für
einige einfache Sonderfälle. Als universelles numerisches Werkzeug gestattet
die FE-Methode, diese Probleme auch bei sehr komplexen Strukturgeometrien in
Verbindung mit vielfältigen Randbedingungen zu berechnen.
Das Konzept der FE-Methode ist mathematisch abgesichert und es existieren
effiziente numerische Lösungsalgorithmen sowie verifizierte finite Elemente
für verschiedene Problemklassen (Schalen-, Platten- und Volumenelemente).
Die kommerziell verfügbaren, leistungsfähigen Programmsysteme mit graphisch
orientierten Pre- und Postprozessoren\footnote{Werkzeuge für die
Modellaufbereitung und
Nachbearbeitung der numerischen Ergebnisse.} verhelfen der FE-Methode
zu einem breiten Einsatz in praktischen Anwendungen. In diesem
Kapitel werden die mathematischen Grundlagen der FE-Methode und
die in dieser Arbeit verwendeten numerischen Berechnungsverfahren zur
Lösung sowohl linearer, als auch nichtlinearer Randwertprobleme der
Mikromechanik behandelt.


\section{Mathematische Grundlagen}
\label{mathgrundlagen}

Die mathematischen Grundlagen der FE-Methode werden im folgenden an
partiellen Differentialgleichungen (pDGL) vom elliptischen Typ erläutert,
insbesondere sollen hierbei ohne Beschränkung der Allgemeinheit
skalare Gleichungen zweiter Ordnung betrachtet
werden. Die numerische Behandlung von physikalischen Problemstellungen vom
elliptischen Typ kann sowohl durch die FD-Methode \cite{Mar89}, als
auch durch die FE-Methode erfolgen. Ein wesentlicher Nachteil der
Differenzenmethode, bei der die Differentialoperatoren durch
Differenzenquotienten ersetzt werden, besteht darin, daß sie bequem und
unmittelbar nur auf rechtwinkligen, regelmäßigen Gitterdiskretisierungen
anwendbar ist, so daß nur einfache Strukturgeometrien betrachtet werden
können. Im Gegensatz dazu kann die FE-Methode auf beliebig komplexe
Geometrien angewendet werden.


\subsection{Klassisches Randwertproblem}
\label{klassRWP}

Zur Veranschaulichung der mathematischen Vorgehensweise, die der FE-Methode
zugrundeliegt, soll das kontinuierliche, n-dimensionale (n = 2 bzw.\ 3),
quasi-lineare Randwertproblem betrachtet werden, welches durch die
folgende pDGL gegeben ist:
\begin{equation}
\label{PDGL}
\fbox{$
 \displaystyle
 \begin{array}{rcl}
 \nabla \{ a(x,\phi) \nabla \phi(x) \} \; - \; f(x) & = & 0
 \end{array}
 $}
\end{equation}
wobei:
\begin{eqnarray*}
 \phi(x) & : & \mbox{gesuchte Lösungsfunktion im Gebietsinnern}
	    \; \Omega \subset {\rm R}^{n}\\
% \phi_{\Gamma}(x) & : & \mbox{Randbedingung auf dem Gebietsrand}
%	    \; \Gamma := \partial \Omega \\
 a(x,\phi) & : & \mbox{verallgemeinerter Transportkoeffizient}\\
 f(x) & : & \mbox{erzeugendes Funktionensystem}
\end{eqnarray*}
Diese Gleichung stellt die klassische Form der Erhaltungsgesetze der Physik
(Kontinuitätsgleichung) auf dem einfach zusammenhängenden Grundgebiet
$\Omega$ dar und wird erst durch physikalisch sinnvolle Randbedingungen
spezifiziert. Diese können sowohl vom {\sl Dirichlet}-Typ, als auch vom
{\sl Neumann}-Typ sein, falls einerseits die gesuchte Funktion $\phi(x)$
selbst oder die Normalenableitung der Funktion
($\partial_{\nu} := \partial / \partial \nu$) auf dem Gebietsrand
$\Gamma := \partial \Omega$
vorgegeben ist. Es können u.U.\ auch gemischte Randbedingungen gelten, so
daß sich der Gebietsrand aus $\Gamma_{D}$ := $\Gamma \backslash \Gamma_{N}$
mit $\phi(x) = g_{1}(x)$ und $\Gamma_{N}$ mit
$\partial_{\nu} \phi(x)= g_{2}(x)$ zusammensetzt.
Eine Lösung der Differentialgleichung (\ref{PDGL}) ist im {\em klassischen}
Sinne gegeben, wenn eine Funktion $\phi(x)$ existiert, die die vorgegebenen
Randwerte annimmt und zweimal stetig differenzierbar ist, d.h.\
$\phi(x) \in C^{2}(\Omega)$ für alle $x \in \Omega$. Ein wichtige
Voraussetzung für die Existenz von klassischen\footnote{Die Lösung des
klassischen Randwertproblems wird mathematisch auch als
 {\em starke} Lösung bezeichnet.}
Lösungen, ist die hinreichende Glattheit des Gebietsrandes.


\subsection{Konzept der schwachen Lösung}
\label{schwacheloesung}

Im allgemeinen existiert bei derartigen Problemen keine analytisch
geschlossen darstellbare Lösung, so daá das Grundgebiet diskretisiert
und die {\em klassische} Lösung numerisch approximiert werden muß. Diese
Approximation macht Gebrauch vom {\em Konzept der schwachen Lösung}.
Anstatt der pDGL (\ref{PDGL}) wird hierzu ein äquivalentes Variationsproblem
betrachtet, da die Lösungen $\phi(x)$ gewisse Minimalitätseigenschaften
aufweisen \cite{Bat82, Zie84}. Nach dem {\sl Lax-Milgram}-Theorem besitzt
das zu dem elliptischen Randwertproblem zugeordnete Variationsproblem eine
eindeutige, in diesem Sinne {\em schwache Lösung} \cite{Bra91}.
Sofern die Lösung
$\phi(x) \in V$ (V = $H_{0}^{1}$ : {\sl Sobolev}-Raum\footnote{Der
{\sl Sobolev}-Raum $H^{m}(\Omega)$ ist ein {\sl Hilbert}-Raum der
{\sl Lebesgue}-meßbaren Funktionen $L_{2}(\Omega)$ mit quadrat-integrablen
{\em schwachen} Ableitungen bis zur Ordnung $m$, wobei
$\dim H^{m}(\Omega) = \infty$.} mit Nullrandbedingung)
zusätzlich gewisse Regularitätsbedingungen
aufweist, d.h.\ stetig, genügend glatt und zweimal stetig differenzierbar
ist,    % ($\phi(x) \in C^{2}(\Omega)$),
läßt sich obige pDGL in das folgende äquivalente Problem umschreiben:
% \cite{Wac93b}:
\begin{eqnarray}
\label{schwach}
  \forall_{\psi \in V}: \quad \int_{\Omega} \nabla\psi(x) \cdot a(x,\phi)
  \nabla \phi(x) \, d\Omega & = & \int_{\Omega} \psi(x) f(x) d\Omega
\end{eqnarray}
Diese Integralgleichung wird auch die \glqq schwache Formulierung\grqq \,
von Gleichung (\ref{PDGL}) genannt und $\phi(x)$ ist entsprechend die
\glqq schwache Lösung\grqq \, der pDGL. Diese Formulierung mittelt die
Differentialgleichung (\ref{PDGL}) mit einer Gewichtungsfunktion\footnote{Die
Testfunktionen $\psi(x)$ sind in der Regel aus dem gleichen Raum wie die
Lösungsfunktion $\phi(x)$.} $\psi(x)$, wobei die zweite Ortsableitung der
gesuchten Lösungsfunktion $\phi(x)$ durch die Anwendung des
{\sl Green}schen Integralsatzes mit Hilfe der partiellen Integration um
eine Stufe erniedrigt wird. Der Vorteil dieser Vorgehensweise besteht
darin, daß die {\em schwache} Lösung im Gegensatz zur {\em starken} Lösung,
die durch Gleichung (\ref{PDGL}) gegeben ist, nur {\em einmal} stetig
differenzierbar sein muá, d.h.\
$\phi(x) \in C^{1}(\Omega)$ für alle $x \in \Omega$.



\subsection{Finite Elemente Diskretisierung}
\label{diskretisierung}

Die Diskretisierung des kontinuierlichen, {\em schwachen} Problems
(\ref{schwach}) kann
auf verschiedene Weisen vorgenommen werden, z.B.\ mit der
Finite-Volumen-Methode ({\em FVM}) oder der FE-Methode\footnote{Im Gegensatz
zu diesen beiden Methoden approximiert die FD-Methode die {\em starke} Lösung
des klassischen Randwertproblems (\ref{PDGL}).}, auf die im folgenden
näher eingegangen werden soll. Bei der FE-Methode wird für die Lösung
des globalen Variationsproblems das Grundgebiet $\Omega$ in endliche
({\em finite}) Elemente $\Omega_{e}$ zerlegt und die Lösung des globalen
Variationsproblems durch die Lösung des Variationsproblems für die
einzelnen Elemente ersetzt. Die schwache Lösung $\phi(x) \in V$ wird durch
den diskreten Ansatz $\phi_{h}(x) \in S_{h}$ numerisch stückweise
approximiert, der über eine endliche Summe von Formfunktionen
(engl.: {\em Shapefunctions}) $S_{k}(x)$ dargestellt werden kann:
\begin{eqnarray}
\label{Sentw}
 \phi_{h}(x) & = & \sum_{k=1}^{N_{h}} \alpha_{k} S_{k}(x)
\end{eqnarray}
Diese Formfunktionen spannen einen endlich dimensionalen Unterraum
$S_{h}(\Omega) \subset H_{0}^{1}(\Omega)$, den sogenannten
{\em Finite-Elemente-Raum}, auf, mit $\dim S_{h}(\Omega) = N_{h}$.
Der eingeschränkte Funktionenraum $S_{h}$ wird durch den
Diskretisierungsparameter $h$ charakterisiert, der den größten Durchmesser
der finiten Elemente darstellt. Man kann zeigen, daá die $\phi_{h}(x)$ eine
konvergierende {\sl Cauchy}\/-Folge in dem abgeschlossenen und vollständigen
Unterraum $S_{h}$ bilden und daá $\phi_{h}(x)$ gegen die {\em schwache}
Lösung $\phi(x)$ strebt, falls die Elementweite genügend klein wird,
d.h.: $h \rightarrow 0$ \cite{Cia78}.\\
%
Den einfachsten Ansatz für Formfunktionen stellen Polynome ersten Grades
dar, die einen stetigen, linearen Funktionsverlauf abbilden können. In diesem
Fall sind die Formfunktionen durch $S_{j}(x_{k}) \; = \; \delta_{jk}$
gegeben, so daß die Näherungslösung $\phi_{h}$ eine %stückweise affine
auf $\Omega$ stetige Funktion wird, die in den Elementknoten $x_{k}$ die
Werte der Koeffizienten $\alpha_{k}$ annimmt, d.h.\
$\phi_{h}(x_{k})=\alpha_{k}$.
In {\bf Abbildung~\ref{abbelemapprox}} ist die Approximation einer
beliebigen Funktion $\phi(x)$ unter Verwendung linearer Formfunktionen
$S_{i}(x)$ dargestellt.
%----------------------- Beginn: Figure-Environment ----------------------
\begin{figure}[htb]
\begin{center}
% --- Dateiname des Bildes
\input{abbde.tex}
\setabbde
\end{center}
\caption{\label{abbelemapprox}
  Approximation einer Funktion unter Verwendung linearer Formfunktionen}
\end{figure}
%----------------------- Ende: Figure-Environment ----------------------
Ersetzt man nun $\psi(x)$ in Gleichung~(\ref{schwach}) ebenfalls durch
eine endliche Summe von Funktionen $S_{i}(x)$ gem„á Gleichung~(\ref{Sentw}):
\begin{eqnarray*}
 \psi_{h}(x) & = & \sum_{i=1}^{N_{h}} \alpha_{i} S_{i}(x)
\end{eqnarray*}
so erhält man ein System von $N_{h}$, im allgemeinen nichtlinearen
Gleichungen:
%
\begin{eqnarray}
\label{diskr}
 \sum_{k=1}^{N_{h}} \;
 \int_{\Omega} \nabla S_{i}(x) a(x, \phi_{h}) \nabla S_{k}(x) d\Omega
 \, \alpha_{k}  & = &
 \int_{\Omega} S_{i}(x) f(x) d\Omega
 \qquad \mbox{fr} \quad i=1..N_{h}
\end{eqnarray}
%
die in Matrixnotation vereinfacht zu:
\begin{equation}
\label{nlingl}
\fbox{$
 \displaystyle
 \begin{array}{rcl}
 \sum_{k=1}^{N_{h}} \; A_{ik}(\phi_{h}) \, \alpha_{k} & = &
  f_{i} \qquad \mbox{fr} \quad i=1..N_{h}
 \end{array}
 $}
\end{equation}
zusammengefaßt werden können. Aufgrund der Elliptizität der pDGL ist die
Systemmatrix $A_{ik}$ positiv definit und infolge des Zusammenhangs
$a(x,\phi_{h})$ ferner von der gesuchten Lösung $\phi_{h}$ und damit von den
erst zu bestimmenden Koeffizienten $\alpha_{k}$ abhängig. Damit liegt ein
nichtlineares Problem vor. In Kapitel~\ref{nlstatberechnungen} wird
auf die physikalischen Hintergründe dieser Nichtlinearitäten eingegangen und
die für die Lösung der daraus resultierenden nichtlinearen Gleichungssysteme
erforderlichen numerischen Verfahren kurz beschrieben. Die Güte der
numerischen Approximation der schwachen Lösung wird in
Kapitel~\ref{fehlerabschaetzungen} diskutiert. \\


\section{Formulierung der Elementsteifigkeitsmatrix}
\label{elementsteifigkeitsmatrix}

Im vorigen Kapitel wurde das der FE-Methode zugrundeliegende mathematische
Konzept der schwachen Lösung und die Diskretisierung des kontinuierlichen
Randwertproblems dargestellt.
Die physikalische Interpretation soll im weiteren am Beispiel
der Strukturmechanik erfolgen. Als Ausgangspunkt für die FE-Formulierung
dient daher die Grundgleichung der Kontinuumsmechanik, das sogenannte
\glqq Spannungs-Divergenz-Theorem\grqq. Unter Anwendung des Konzeptes der
schwachen Lösung erfolgt die Integration der mit geeigneten Testfunktionen
$\eta_{i}$ gewichteten lokalen Gleichgewichtsaussage
(Gleichung~\ref{kontmech}) über das betrachtete Gebiet $\Omega$
($i,j: 1, 2, 3$):
\begin{eqnarray}
\label{ablelmst}
 \sum _{i} \int_{\Omega} \eta_{i} \left( \sum_{j}
 \frac{\partial \sigma_{ij}}{\partial x_{j}}  % \nabla_{j}
 \, + \, F_{i}^{\Omega} \right) \, d\Omega
 & = &  \sum_{i}
 \int_{\Omega} \eta_{i} \rho \frac{\partial^{2}u_{i}}{\partial t^{2}}d\Omega
\end{eqnarray}
Hierbei stellt $F^{\Omega}$ die Summe der von außen auf den Festkörper
einwirkenden Volumenkräfte dar. Durch partielle Integration der Divergenz
des Spannungstensors % ($\nabla_{j}\sigma_{ij}$~:=~$\sigma_{ij,j}$)
wird der Gradient auf die Testfunktionen $\eta_{i}$ abgewälzt, die mit den
Veschiebungen $u_{i}$ identifiziert werden können. Unter Ausnutzung des
Zusammenhangs\footnote{Im weiteren soll nur die lineare Näherung von
Gleichung~(\ref{dehn}) betrachtet werden.} zwischen den Dehnungen
$\varepsilon_{ij}$ und den Verschiebungen $u_{i}$ und der Auswertung des
Integrals über die Randkurve $\Gamma = \Gamma_{D} + \Gamma_{N}$, bei der
der Anteil über den eingespannten Rand $\Gamma_{D}$ verschwindet
($u_{i}=0$) und nur die mit Oberflächenkräften
($F_{i}^{\Gamma} = \sum_{j} \sigma_{ij} n_{j}, n_{j}$ : Normalenvektor)
belasteten Ränder $\Gamma_{N}$ berücksichtigt werden müssen, folgt
schließlich \cite{Mat93}:
\begin{eqnarray}
\label{grund}
  \sum_{i,j} \int_{\Omega} \varepsilon_{ij} \sigma_{ij} d\Omega
  & = &
  \sum_{i} \left\{
  \int_{\Gamma} u_{i} F^{\Gamma}_{i} d\Gamma \, + \,
  \int_{\Omega} u_{i} \left( F^{\Omega}_{i} \, - \,
  \rho \frac{\partial^{2} u_{i}}{\partial t^{2}} \right) d\Omega
  \right\}
\end{eqnarray}
Die linke Seite der Gleichung~(\ref{grund}) stellt die innere Energie
des elastischen Festkörpers dar, während auf der rechten Seite die Anteile
der durch die Oberflächenkräfte $F^{\Gamma}$ und die Volumenkräfte
$F^{\Omega}$ verrichteten Arbeit stehen. Bei der Beschränkung auf rein
statische Probleme entfällt die Trägheitskraft auf der rechten Seite der
Gleichung.\\
%
Das zu der schwachen Formulierung (Gleichung~\ref{grund}) korrespondierende
Variationsprinzip ist das {\em Prinzip der virtuellen
Arbeit} bzw.\ {\em Verschiebungen} \cite{Zie84}. Aus der Forderung nach der
Stationarität der potentiellen Energie $E_{pot}$ des elastischen Kontinuums
kann die Elementsteifigkeitsmatrix direkt abgeleitet werden:
\begin{eqnarray}
\label{epot}
 E_{pot} & = & \frac{1}{2} \int_{\Omega} \varepsilon^{t} \sigma d\Omega
           - \int_{\Omega} u^{t} F^{\Omega} d\Omega
           - \int_{\Gamma} u^{t}_{\Gamma} F^{\Gamma} d\Gamma
           - \sum_{i} u_{i} F_{i} \; \rightarrow \; Min
\end{eqnarray}
Im Gleichgewichtszustand entspricht die gesamte innere Energie
des elastischen Körpers der von auáen verrichteten Arbeit.
Hierbei sind $\varepsilon$ die Dehnungen und $\sigma$ die Spannungen,
die äußeren Kräfte sind durch die Volumenkräfte $F^{\Omega}$, die
Oberflächenkräfte $F^{\Gamma}$ und die diskreten Knotenkräfte $F_{i}$
gegeben. Entsprechend sind $u$ die Vektoren des Verschiebungsfeldes im Innern,
$u_{\Gamma}$ die Oberflächenverschiebungen und $u_{i}$ die Knotenverschiebung
des $i$-ten Knotens. Für die dreidimensionale Verschiebung
$u(\vec x)^{t}$ = ($u_x(\vec x), u_y(\vec x), u_z(\vec x)$)
im Innern eines Elementes wird ein Ansatz der Form:
\begin{eqnarray}
\label{uansatz}
 u(\vec x) & = & \sum_{k} N^{k}(\vec x) \cdot u_{e}^{k}
\end{eqnarray}
gewählt, wobei $u_{e}^{t} = (u_x^e, u_y^e, u_z^e)$ der Vektor der diskreten
Knotenverschiebungen des Elementes ist, die Matrix
$N = [N_x(\vec x), N_y(\vec x), N_z(\vec x)]$ die Form- bzw.\
Ansatzfunktionen (siehe Gleichung~\ref{Sentw}) enthält und die Summe sich
über alle Knoten erstreckt.
Die Dehnungen $\varepsilon$ ergeben sich in linearer Näherung aus der
Ableitung der Verschiebungen:
\begin{eqnarray}
\label{elemdehn}
 \varepsilon(\vec x) & = & D \cdot u(\vec x) \; - \; \varepsilon_{0}
\end{eqnarray}
Hierbei entspricht $\varepsilon_{0}$ einer Vordehnung\footnote{Unter
Vordehnungen sind spannungsunabh„ngige Verformungen zu verstehen, die als
Ursache z.B.\ Kristallwachstum oder Temperaturänderungen (siehe
Kapitel~\ref{thermoelastischekopplung}) haben.} und $D$ stellt eine Matrix
von Differentialoperatoren dar:
\begin{eqnarray}
\label{Doperator}
 D & = &
 \left[
        \begin{array}{lll}
         \partial_{x} & 0            & 0 \\
         0            & \partial_{y} & 0 \\
	 0            & 0            & \partial_{z} \\
	 \partial_{y} & \partial_{x} & 0          \\
	 0            & \partial_{z} & \partial_{y} \\
	 \partial_{z} & 0            & \partial_{x} \\
	\end{array}
 \right]
\end{eqnarray}
wobei $\partial_{x_{i}} := \frac{\partial}{\partial x_{i}}$ für die
partiellen Ableitungen steht. Mit den Gleichungen~(\ref{uansatz}) und
(\ref{elemdehn}) ergibt sich:
\begin{eqnarray}
\label{dehnversch}
 \varepsilon(x) & = & B(x) \cdot u_{e} \; - \; \varepsilon_{0}
\end{eqnarray}
mit  $B(x) := D \cdot N(x)$. Über das Materialgesetz
(Gleichung~\ref{matgesetz})
wird der Zusammenhang zwischen Dehnungen und Spannungen hergestellt.
Unter der Annahme linearen Werkstoffverhaltens, d.h.\ C ist unabhängig
von $\varepsilon$, folgt:
\begin{eqnarray}
\label{sigmaelem}
 \sigma(\vec x) & = & C \varepsilon(\vec x)
                  = C \left( B(x) \cdot u_{e} \; - \; \varepsilon_{0} \right)
\end{eqnarray}
Werden diese Beziehungen in die Gleichung~(\ref{epot}) eingesetzt, so folgt
für die potentielle Energie $E_{pot}^{e}$ eines finiten Elementes:
\begin{eqnarray}
\label{epotfem}
  E_{pot}^{e} & = &
  \frac{u_{e}^{t}}{2} \int_{\Omega} B^{t} C B d\Omega u_{e} \, - \,
  u_{e}^{t} \int_{\Omega} B^{t} C \varepsilon_{0} d\Omega \, + \,
  \frac{1}{2} \int_{\Omega} \varepsilon_{0} C \varepsilon_{0} d\Omega
  \nonumber \\
  & - & u_{e}^{t} \left( \int_{\Omega} N^{t} F^{\Omega} d\Omega
	+ \int_{\Gamma} N^{t}_{\Gamma} F^{\Gamma} d\Gamma + F_{i} \right)
\end{eqnarray}
Die Bedingung fr das Minimum der potentiellen Energie des Gesamtsystems
verlangt, daß die Ableitung der Energie nach den Knotenverschiebungen
verschwindet. Für das einzelne finite Element bedeutet dies
$\partial E_{pot}^{e} / \partial u_{e} = 0$. Damit folgt aus
Gleichung~(\ref{epotfem}) die Grundgleichung der linearen Statik auf
der Elementebene:
\begin{eqnarray}
\label{intstatik}
  \int_{\Omega} B^{t} C B d\Omega \cdot u_{e}
  & = &
  \int_{\Omega} B^{t} C \varepsilon_{0} d\Omega \, + \,
  \int_{\Omega} N^{t} F^{\Omega} d\Omega \, + \,
  \int_{\Gamma} N^{t}_{\Gamma} F^{\Gamma} d\Gamma \, + \, F_{i}
\end{eqnarray}
oder in Matrixkurzschreibweise\footnote{In Anlehnung an die bestehende
FEM-Literatur \cite{Koh92} sollen im folgenden Matrizen durch $[\cdot]$ und
Vektoren durch $\{ \cdot \}$ gekennzeichnet werden.}:
\begin{equation}
\label{linstatik}
\fbox{$
 \displaystyle
 \begin{array}{rcl}
  [K_{e}] \, \cdot \, \{u_{e}\} & = & \{F_{e}\}
 \end{array}
 $}
\end{equation}
Hierbei ist die Elementsteifigkeitsmatrix durch:
\begin{eqnarray}
\label{steif}
 [K_{e}] & = & \int_{\Omega} [B]^{t} \, [C] \, [B] \, d\Omega
\end{eqnarray}
gegeben und der Gesamtkraftvektor setzt sich aus den Einzelanteilen gemäß:
\begin{eqnarray}
\label{fges}
 \{F_{e}\} & = & \{F_{0} \; + \; F^{\Omega} \; + \; F^{\Gamma} \; + \; F_{i}\}
\end{eqnarray}
zusammen. Die Kraft $F_{e}$ entspricht den äquivalenten Knotenersatzkräften,
die durch die verteilten Lasten $F^{\Omega}$ und $F^{\Gamma}$, sowie die
diskret einwirkenden Einzelkräfte $F_{i}$ hervorgerufen wird. Die
Gesamtsteifigkeitsmatrix des Systems berechnet sich über die Summation aller
einzelnen Elementsteifigkeitsmatrizen und entspricht der Systemmatrix
$A_{ik}$ der Gleichung~(\ref{nlingl}).


\section{Nichtlinearitäten}
\label{nlstatberechnungen}

Die Aufstellung der Elementsteifigkeitsmatrix, die zur Lösung des linearen
Strukturproblems (\ref{linstatik}) benötigt wird, erfolgte unter der Annahme
eines linearen Zusammenhangs zwischen den Knotenverschiebungen $\{u\}$ und
den angreifenden Knotenkräften $\{F\}$. Infolge der Belastung können jedoch
nichtlineare Effekte auftreten, die durch eine
Geometrieänderung hervorgerufen werden, so daß die Steifigkeitsmatrix
$[K(u)]$ ihrerseits vom Auslenkungs- und damit vom Belastungszustand des
Systems abhängig wird (siehe Gleichung~\ref{nlingl}):
\begin{equation}
\label{nlstatik}
\fbox{$
 \displaystyle
 \begin{array}{rcl}
 [K(u)] \, \cdot \, \{u\} & = & \{F\}
 \end{array}
 $}
\end{equation}
Prinzipiell wird bei der Behandlung mechanischer Probleme
zwischen drei verschiedenen Typen von
Nichtlinearitäten unterschieden: der geometrischen Nichtlinearität,
der Material- und der Struktur-Nichtlinearität. Die
{\em Material-Nichtlinearit„t}, die den nichtlinearen Zusammenhang zwischen
Spannung und Dehnung (z.B.\ Plastizität, Hyperelastizität) beschreibt,
kann bei
einkristallinen Materialien im Niedertemperaturbereich vernachlässigt
werden, da Silizium bis etwa 600~$^\circ$C ein absolut lineares
{\sl Hooke}sches Verhalten zeigt \cite{Ove77}.
Ebenso verhält es sich bei den
meisten anderen mikromechanischen Werkstoffen. Unter Struktur-Nichtlinearität
werden beispielsweise Kontaktprobleme zusammengefaßt, die eine
Verformungsbehinderung infolge eines mechanischen Anschlags aufweisen.
Bei der Auslegung und Berechnung von mikromechanischen Bauelementen spielen
{\em geometrische Nichtlinearitäten} die größte Rolle. Neben den
spannungsversteifenden Effekten handelt es sich hier um das Knicken und
Beulen von Balken- und Membranstrukturen, die unter äußeren Lasteinwirkungen
stehen oder technologisch bedingte, innere Spannungen aufweisen. Die
Problemklasse der geometrischen Nichtlinearitäten werden FE-technisch wie
folgt unterteilt \cite{Koh92}:
%
\begin{itemize}
\item
{\bf Große Dehnungen} ({\em Large Strains}),
bei denen der nichtlineare Anteil des {\sl Green-Lagrange}schen
Dehnungstensors in Gleichung~(\ref{dehn}) berücksichtigt wird und z.B.\
Formänderungen oder groáe Rotationen unter der Belastung auftreten können.
\item
{\bf Große Auslenkungen} ({\em Large Deflections}), bei denen große
Rotationen auftreten, aber die mechanischen Dehnungen klein bleiben.
Die Beschreibung erfolgt durch die sogenannte
{\em Corotational}-Formulierung \cite{Ran86}, bei der ein lokales
Elementkoordinatensystem eingeführt wird, das sich unter Belastung mitdreht
und die Element\-orientierung annimmt, um die Kraftvektoren bei
Druckbelastungen entsprechend der verformten Oberfläche nachzuführen.
Es treten keine Formänderungen auf.
\item
{\bf Spannungsversteifung} ({\em Stress Stiffening}), bei der die Rotationen
und Dehnungen klein bleiben, aber eine Kopplung von Längs- und
Biegebelastung auftritt, so daß sich die Biegesteifigkeit der Struktur
lastabhängig ändert.
\end{itemize}
%
Das nichtlineare Problem wird lokal linearisiert und das zugrundeliegende
Gleichungssystem mit einem linearen Gleichungslöser iterativ gelöst. Zur
Lösung der linearen Gleichungssyteme wird im FE-Programm
{\sf ANSYS}\footnote{Im
weiteren sollen anhand des kommerziellen FE-Programmsystems {\sf ANSYS}
\cite{SASI} die numerischen Berechnungsmöglichkeiten beschrieben werden,
da {\sf ANSYS} z.Zt.\ eines der universellsten Programmsysteme ist,
welches die meisten bei mikromechanischen Strukturen auftretenden
physikalischen Effekte (siehe Abbildung~\ref{abbnye}) in ihrer Gesamtheit
zu beschreiben erlaubt.} in der Regel
das {\em Wavefront}-Lösungsverfahren eingesetzt \cite{Koh92}. Da die
Steifigkeitsmatrix von den Verschiebungen abhängig ist, wird sie
entsprechend dem Belastungszustand aufgestellt und ein elementinterner
Korrekturlastvektor gemäß:
\begin{eqnarray}
\label{largedefl}
  [K(u)]_{i-1} \, \cdot \, \{\Delta u\}_{i} & = &
  \{F^{ext}\} \, - \, \{F^{kor}\}_{i-1}
\end{eqnarray}
berechnet. $[K(u)]_{i-1}$ ist hierbei die verschiebungsabhängige
Steifigkeitsmatrix der ($i-1$)--ten Iteration,
$\{\Delta u\}_{i} = \{u\}_{i} - \{u\}_{i-1}$ der inkrementelle
Verschiebungsvektor, $\{F^{ext}\}$ der extern angreifende Lastvektor und
$\{F^{kor}\}_{i-1}$ der verschiebungsabhängige Korrekturlastvektor, der
iterativ berechnet wird. Der Ausdruck auf der rechten Seite stellt die
Differenz der angreifenden äußeren Last und der Rückstellkraft, generiert
durch die Elementspannungen, dar und wird als Ungleichgewichtskraft
(Residuum) bezeichnet. Zur Lösung des nichtlinearen Problems können
prinzi\-piell verschiedene Verfahren eingesetzt werden.
Je nach Problemstellung sind
die Verfahren unterschiedlich effektiv, so daß in FE-Programmen meist
mehrere realisiert sind. Das FE-Programm {\sf ANSYS} erlaubt entweder
die Anfangssteifigkeitsmethode % (engl.: {\em Initial-Stiffness-Procedure})
oder das {\sl Newton-Raphson}-Verfahren einzusetzen, die sich durch
verschiedene Iterationsverfahren unterscheiden.
Bei beiden Methoden wird das Verschiebungsinkrement $\{\Delta u\}_{i}$
infolge jedes Iterationsschrittes kleiner, so daß bei Konvergenz der
nichtlinearen Rechnung das Residuum beliebig klein wird.\\
%----------------------- Beginn: Figure-Environment ----------------------
\begin{figure}[htb]
\begin{center}
% --- Dateiname des Bildes
\input{abbdz.tex}
\setabbdz
\end{center}
\caption{\label{abbnr}
 Iterationsprozeá beim {\sl Newton-Raphson}-Verfahren}
\end{figure}
%----------------------- Ende: Figure-Environment ----------------------
Bei der Anfangssteifigkeitsmethode wird bei jeder
Iteration eine konstante Steifigkeitsmatrix benutzt und nur der
Korrekturlastvektor wird entsprechend dem Belastungszustand geändert.
Das {\sl Newton-Raphson}-Verfahren, das in {\bf Abbildung~\ref{abbnr}}
fr ein Problem mit einem Freiheitsgrad dargestellt ist, verwendet im
Vergleich dazu eine bei jeder Iteration neu berechnete
Tangentensteifigkeitsmatrix $[K_{T}]$, für die:
\begin{eqnarray}
\label{tangentenmatrix}
  [K_{T}(x,u)] & := &  [K_{0}(x)] \, + \, [K_{u}(x,u)] \, + \,
                       [K_{\sigma}(x,u)]
\end{eqnarray}
gilt, wobei mit Hilfe der orthogonalen Transformationsmatrix [T] und durch
Summation ber die einzelnen Elemente (e) die Einzelmatrizen:
\begin{eqnarray*}
 [K_{0}] + [K_{u}] & \approx &
   \sum_{e} \; [T(x+u)]^{t} \cdot [K_{0}^{e}(x)] \cdot [T(x+u)]
\end{eqnarray*}
\begin{eqnarray*}
 [K_{\sigma}] & \approx &
     \sum_{e} \; [T(x+u)]^{t} \cdot [S_{\sigma}^{e}(x)] \cdot [T(x+u)]
\end{eqnarray*}
gebildet werden \cite{Mat93}. Hierbei entspricht $[K_{0}]$
der geometrischen Struktursteifigkeitsmatrix (\ref{steif})
und $[K_{u}]$ ist der verschiebungsabhängige Anteil. Die Matrix
$[S_{\sigma}]$ beschreibt die zusätzliche Spannungsversteifung, die
infolge der Überlagerung von Längs- und
Biegespannungsanteilen bei dnnen Balken- und Membranstrukturen auftritt.
Während die Struktursteifigkeitsmatrix von der Geometrie und den
Elastizitätseigenschaften der Struktur abhängt, wird die
Spannungsversteifungsmatrix von der lastinduzierten, inneren Längsspannung
$\sigma$ in der Struktur hervorgerufen. Bei axial belasteten
Balkenstrukturen, wie beispielsweise resonanten Kraftsensoren,
genügt es nur den Anteil $[S_{\sigma}]$:
\begin{eqnarray}
\label{sstif}
 ([K_{0}] \, + \, [S(\sigma)]) \cdot \{u\} & = & \{F\}
\end{eqnarray}
in der nichtlinearen statischen Berechnung zu berücksichtigen
({\sf ANSYS}-Option: {\em Stress Stiffening}).



\section{Dynamische Berechnungsverfahren}
\label{dynamik}

Das dynamische Verhalten mikromechanischer Systeme ist gekennzeichnet durch
ihr sta\-tionäres Schwingungsverhalten, das Frequenzgangverhalten und das
instationäre Verhalten unter Einwirkung von transienten Störungen. Die
rechnerische Modalanalyse löst das dynamische Eigenwertproblem und liefert
die Eigenfrequenzen und -schwingungsformen. Im Zusammenhang mit resonanten
Sensoren interessieren insbesondere die Eigenfrequenzänderungen unter der
Einwirkung von äußeren statischen Belastungen. Frequenzganganalysen
ermöglichen die Modellierung unter dem Einfluß von Dämpfungseffekten und
zeitlich periodisch variierenden Lasten. Mit Hilfe der direkten
Zeitintegration kann zusätzlich das Systemverhalten im Zeitbereich unter
impulsartigen Lasteinwirkungen ermittelt werden.


\subsection{Das Eigenwertproblem}
\label{eigenwertproblem}

Zur Beschreibung der freien, stationären Schwingungen eines Systems ist die
dynamische Eigenwertgleichung:
\begin{equation}
\label{modal}
\fbox{$
 \displaystyle
 \begin{array}{rcl}
 [M] \cdot \{\stackrel{\cdot \cdot}{u}\} \: +  \:
 % [C] \cdot \{\stackrel{\cdot}{u}\} \: + \:
 [K(u)] \cdot \{u\} & = & 0
 \end{array}
 $}
\end{equation}
zu lösen. Hierbei ist [K(u)] die verschiebungsabhängige Steifigkeitsmatrix
des Gesamtsystems, die in einer nichtlinearen statischen Analyse mit den
{\sf ANSYS}-Optionen {\em Large Deflection} und/oder {\em Stress Stiffening}
ermittelt wird. Die Massenmatrix:
\begin{eqnarray}
\label{massmatrix}
 [M] & = & \int_{\Omega} [N]^{t} \, \rho \, [N] \, d\Omega
\end{eqnarray}
berechnet sich aus der Massendichte $\rho$ bei der {\em konsistenten}
Formulierung mit Hilfe der Matrix der Ansatzfunktionen $[N(\vec x)]$ im
finiten Element. Die Masse der Gesamtstruktur kann auch {\em konzentriert}
in den Elementknoten durch eine Diagonalmatrix
($\sum_{i,j} M_{ij}\delta_{ij}$)
angenähert werden. Beide Formulierungen sind in {\sf ANSYS} realisiert
und zeichen sich durch unterschiedliche numerische Genauigkeiten aus
\cite{Ram90}.
Die Vektoren $\{\stackrel{\cdot \cdot}{u}\}$ und $\{u\}$ stellen die
Knotenbeschleunigungen und -verschiebungen dar.\\
Unter der Annahme freier,
ungedämpfter harmonischer Schwingungen verhält sich das System linear und
der Lösungsansatz:
\begin{eqnarray}
 \{u(\vec x,t)\} & = & \{ \Phi(\vec x) \}_{k} \cdot \exp (i \omega_{k}t)
\end{eqnarray}
führt auf das bekannte Eigenwertproblem:
\begin{eqnarray}
\label{ewp}
 \left ( [K] \, - \, \omega_{k}^{2} \, [M] \right ) \cdot
 \{ \Phi \}_{k} & = & 0
\end{eqnarray}
Diese Gleichung ist fr alle nicht verschwindenden Eigenvektoren
$\{\Phi\}_{k}$ erfllt, wenn die Determinante:
\begin{eqnarray}
 \det ( \, [K] \, - \,\omega_{k}^{2} \, [M] \,) & = & 0
\end{eqnarray}
verschwindet. Die Lösungen sind die Eigenwerte\footnote{Quadrate der
Eigenfrequenzen $\omega_{k}$.} und die Eigenschwingungsformen
$\{\Phi\}_{k} $, wobei der Subskript $k$ für die Nummer der Schwingungsmode
steht. Die Eigenschwingungsformen bilden einen Satz linear unabhängiger
Basisvektoren und sind bezüglich der Gesamtmassenmatrix (\ref{massmatrix})
orthonormiert:
\begin{eqnarray}
\label{ons}
 \{ \Phi \}^{t}_{i} \; [M] \; \{ \Phi \}_{j} & = & \delta_{ij}
\end{eqnarray}
Für die Modalanalyse können im FE-Programm {\sf ANSYS} verschiedene
Lösungsalgorithmen verwendet werden \cite{Koh92}.
Die {\sl Householder}-Methode
benutzt ein reduziertes System von Freiheitsgraden, bei dem die
Verschiebungen auf die sogenannten Hauptfreiheitsgrade
({\em MDOF : \underline{M}aster \underline{D}egrees \underline{o}f
\underline{F}reedom}) mit Hilfe einer {\sl Guyan}-Reduktion kondensiert
werden. Im Gegensatz dazu löst das
{\em Subspace}-Iterationsverfahren die voll besetzten Matrizen
des Systems bei erhöhter Rechengenauigkeit, benötigt aber entsprechend
mehr Speicherkapazität und Rechenzeit.


\subsection{Modellierung von Dämpfungseffekten}
\label{daempfungseffekte}

Bei dynamischen Vorgängen führen verschiedene physikalische
Dämpfungsmechanismen zur Energiedissipation im Mikrosystem. Diese können
durch Kristallimperfektionen im Kristallinnern oder durch zusätzlich
aufgebrachte Dünnschichten mit andersartiger Kristallstruktur (amorph oder
polykristallin) hervorgerufen werden. Diese Energieverluste sind abhängig von
den Materialeigenschaften, der Absoluttemperatur und den Strukturdimensionen
und werden unter dem Begriff \glqq thermo-elastische Reibung\grqq \,
zusammengefaßt \cite{Ros90}.
Ein weiterer Dämpfungsmechanismus wird bei Resonatoren durch
die Schwingungen hervorgerufen, bei denen Energieverluste in die
Resonatoreinspannung verursacht werden. Diese Art der Dämpfung ist
abhängig von der jeweiligen Schwingungsmode (modale Dämpfung).
Zusätzlich wird durch akustische Abstrahlung und viskose Dämpfung
externe Fluiddämpfung verursacht. Um diese
Dämpfungseffekte in FE-Berechnungen zu berücksichtigen, wird von einem
viskosen, d.h.\ geschwindigkeitsproportionalen Dämpfungsterm entsprechend
Gleichung (\ref{balkdgl}) ausgegangen \cite{Koh92}:
\begin{equation}
\label{harmfreq}
\fbox{$
 \displaystyle
 \begin{array}{rcl}
 [M] \cdot \{\stackrel{\cdot \cdot}{u} \} \: +  \:
 [C_{D}] \cdot \{\stackrel{\cdot}{u}\} \: + \:
 [K] \cdot \{u\} & = & \{f(x,t)\}
 \end{array}
 $}
\end{equation}
Hierbei sind [M], [$C_{D}$] und [K] die Massen-, Dämpfungs-\footnote{Der
Index $D$ bei der Dämpfungsmatrix soll eine Verwechslung mit der Matrix der
Elastizitätsmoduln [C] ausschließen (siehe
Kapitel~\ref{elementsteifigkeitsmatrix}).}
und Steifigkeitsmatrix der Struktur und
$\{\stackrel{\cdot}{u}\}$ der Vektor der
Knotengeschwindigkeiten. Bei harmonischen Analysen entspricht die rechte
Seite einer harmonischen Anregungskraft
$\{f(x,t)\} = \{f_{0}(x)\} \cdot \exp(i \omega t)$. Bei transienten Analysen
ist die Anregungskraft im allgemeinen nicht periodisch und zusätzlich können
Nichtlinearitäten berücksichtigt werden. Zur Lösung nichtlinearer,
transienter Probleme wird die {\sl Newmark}-Integrationsmethode in
Verbindung mit dem {\sl Newton-Raphson}-Verfahren eingesetzt.
Die Dämpfungsmatrix wird im allgemeinsten Fall durch den Ansatz:
\begin{eqnarray}
\label{cmatrix}
 [C_{D}] & = & \alpha \cdot [M] \, + \, ( \beta \, + \, \beta_{c}) \cdot
           [K] \, + \, \sum_{j} \, \beta_{j} \cdot [K_{j}] \, + \,
            \sum_{j} \, [C_{j}]
\end{eqnarray}
bercksichtigt, wobei $\alpha$ und $\beta$ die sogenannten
{\sl Rayleigh}schen
Dämpfungskonstanten sind, die die Dämpfungsmatrix auf die Massen- und
Steifigkeitsmatrix beziehen. Der Dämpfungskoeffizient
$\beta_{c} = 2 \xi / \omega_{c}$ ist variabel und nur für die harmonische
Frequenzganganalyse verfügbar. Hierbei ist der Parameter
$\xi = c / c_{krit}$  das {\sl Lehr}sche Dämpfungsmaß und
$c_{krit} = 2 m \omega_{0} $ entspricht der kritischen Dämpfung des
Systems. Zusätzlich können Materialdämpfungsbeiträge durch $[K_{j}]$ und
spezielle Dämpferelemente durch $[C_{j}]$ berücksichtigt werden.
Die Angabe der Dämpfungsbeiträge ist nur empirisch zu ermitteln, wobei
die beiden {\sl Rayleigh}-Konstanten frequenzabhängig sind. Hierbei ist
$\alpha$ ($\sim \omega^{-1}$) für niedrige und $\beta$ ($\sim \omega$) für
hohe Frequenzen maágebend. Beispielsweise betragen für Piezokeramiken,
die im Bereich um 1~MHz betrieben werden, $\alpha = 7,5$ und
$\beta = 2 \cdot 10^{-5}$ \cite{Ler90}.



\section{Gekoppelte Feldberechnungen}
\label{feldberechnungen}

Durch die Miniaturisierung von mikromechanischen Systemen erhält man
einerseits eine hohe Integrationsdichte der Bauelemente, andererseits
aber eine stark ausgeprägte Wechselwirkung der physikalischen Größen
untereinander, die bei makroskopischen Problemstellungen im allgemeinen
vernachlässigt
werden können. Zusätzlich besitzen die mikromechanischen Bauelemente
sowohl elektrische als auch
nichtelektrische Funktionen, so daß infolge der verschiedenen
mikromechanischen Wandlungsprinzipien die Berechnung von gekoppelten
Feldproblemen beim Entwurfsprozeß unabdingbar werden. Das FE-Programm
{\sf ANSYS} bietet als eines der wenigen kommerziell verfügbaren
Programmsysteme
mit den sogenannten {\em Multi-Field}-Elementen die Möglichkeit,
Feldkopplungen bereits auf Element\-ebene numerisch durchzuführen.
Diese Elementklasse besitzt mehrere Freiheitsgrade pro Knoten, die es
erlauben, strukturmechanische, thermische und piezoelektrische Probleme
berechnen zu können. Zu den strukturmechanischen
Knotenverschiebungsfreiheitsgraden $\vec u = (u_{x}, u_{y}, u_{z})$
kommen die Knotentemperatur $T$ und das elektrische Potential $\phi$ hinzu.
Dadurch wird direkt auf Elementebene eine Kopplung ermöglicht, die die
Wechselwirkung bereits in {\em einem} Berechnungsschritt zu erfassen
erlaubt \cite{Ost89}. Eine alternative
Möglichkeit besteht über die Methode der Lastvektorkopplung, bei der die
zusätzlichen Kraftbeiträge infolge der Wechselwirkung iterativ berücksichtigt
werden. Diese Vorgehensweise wird auch erforderlich, falls verschiedene
Programmwerkzeuge miteinander gekoppelt werden müssen, wie beispielsweise
bei der Berechnung der thermischen Fluid-Struktur-Wechselwirkungen
\cite{Fab93b}.


\subsection{Thermoelastische Kopplung}
\label{thermoelastischekopplung}

Die Zustandsgleichung für rein thermodynamische Problemstellungen
lautet \cite{Bol85}:
\begin{equation}
\label{thermodyn}
\fbox{$
 \displaystyle
 \begin{array}{rcl}
 C_{th} \frac{\partial T}{\partial t} + K_{th} T  & = & Q_{th}
 \end{array}
 $}
\end{equation}
wobei $C_{th}$ und $K_{th}$ für die Wärmekapazität und die
-leitfähigkeit stehen und $Q_{th}$ die Wärmeflüsse bezeichnet.
Bei reinen Temperaturfeldberechnungen kann eine {\em inhomogene}
Temperaturverteilung in mikromechanischen Bauelementen unter Vorgabe von
Wärmequellen und -senken ermittelt werden. Zusätzlich treten durch die
thermoelastische Kopplung an Bauelementeeinspannungen thermisch
induzierte mechanische Spannungen infolge verhinderter Wärmeausdehnung auf.
Diese können im einfachsten Fall unter Vorgabe einer {\em homogenen}
Temperaturüberhöhung $\Delta T$ gegenüber der Umgebungstemperatur mit Hilfe
des Wärmeausdehnungskoeffizienten $\alpha$ über die thermischen Dehnungen:
\begin{eqnarray}
\label{thermdehn}
 \varepsilon^{th}_{ij} & = & \sum_{i,j} \Delta T \alpha_{ij} \delta_{ij}
\end{eqnarray}
berechnet und damit die thermisch induzierten Spannungen:
\begin{eqnarray}
\label{thermspan}
 \sigma_{ij} & = & \sum_{k,l} C_{ijkl} (\varepsilon_{kl} \, - \, \varepsilon^{th}_{kl})
\end{eqnarray}
ermittelt werden\footnote{Das Minuszeichen bei der thermischen Dehnung ist
Konvention, da bei einer Temperaturzunahme ($\Delta T \geq 0$) und
positiver Wärmeausdehnungsdifferenz ($\Delta \alpha \geq 0$) Druckspannungen
auftreten.}. Für das stationäre, thermisch-mechanisch gekoppelte Problem
ergibt sich unter Verwendung der ungekoppelten Zustandsgleichungen
(\ref{harmfreq}) und (\ref{thermodyn}) in der FE-Formulierung auf
Elementebene \cite{Koh92}:
\begin{small}
\begin{eqnarray}
\label{thermkopl}
 \left[
 \begin{array}{ll}
  K  & 0   \\
  0  & K_{th}
 \end{array}
 \right ]
 \left \{
 \begin{array}{c}
  \vec u \\
  T
 \end{array} \right \}
                        & = &
\left \{
\begin{array}{c}
\vec F + \vec F^{th} \\
Q_{th}
\end{array} \right \}
\end{eqnarray}
\end{small}
Die Freiheitsgrade an den Knotenpunkten der finiten Elemente sind
durch den Vektor der Knotenverschiebungen $\vec u$ und die Temperatur
$T$ gegeben. Die thermisch induzierte Kraftwirkung wird durch den
zus„tzlichen Lastvektor\footnote{Im Gegensatz zur direkten Kopplung
auf Elementebene (siehe Kapitel~\ref{elektromechanischekopplung}) erfordert
die Methode der Lastvektorkopplung eine iterative Lösung mit mindestens zwei
Berechnungsschritten, um {\em beide} feldbeschreibenden Gleichungen zu
lösen. Im {\em ersten} Schritt wird dabei das Temperaturfeldproblem
und im {\em zweiten} das mechanische Problem berechnet.}
$\vec F^{th}$ beschrieben:
\begin{eqnarray}
\label{thermkraft}
 \{F^{th}\} & = & \int_{\Omega} [B]^{t} [C] \{\varepsilon^{th}\} d\Omega
\end{eqnarray}
%
Bei mikromechanischen Multilayerstrukturen treten ferner durch die
unterschiedlichen Wärmeausdehnungskoeffizienten thermisch induzierte
Spannungen auf. Der umgekehrte Effekt, daß sich durch innere Reibung und
mechanische Deformationen Wärme bildet, ist in kommerziellen FE-Programmen
nicht implementiert. Jegliche sonstige Energiedissipationen können generell
nur durch die mechanische Dämpfung (\ref{cmatrix}) berücksichtigt werden.
Insbesondere gilt dieses auch für dielektrische Verluste bei
piezoelektrischen Medien \cite{Ecc92}. Daher können mit der FE-Methode nur
verlustfreie elektromechanische Wandler simuliert werden.


\subsection{Elektromechanische Kopplung}
\label{elektromechanischekopplung}

Die piezoelektrischen Feldgleichungen
(Gleichungen:~\ref{piezo1}--\ref{piezo3}), welche die elektromechanische
Kopplung beschreiben, sind in der Regel nur für
einfache Geometrien geschlossen lösbar. Im Laufe der Zeit wurden unter
vereinfachenden Annahmen verschiedene eindimensionale Ersatzmodelle zur
Beschreibung piezoelektrischer Medien entwickelt (\cite{Ler90},
und Referenzen darin). Um jedoch
komplexe Geometrien berechnen zu können, ist es notwendig die FE-Formulierung
auf die Beschreibung der elektromechanischen Wechselwirkung zu erweitern.
Dieses kann durch die Anwendung eines verallgemeinerten Variationsprinzips
erfolgen, z.B. durch das Prinzip der virtuellen Verschiebungen angewendet auf
ein elektroelastisches Kontinuum unter dem Einfluß von mechanischen {\em und}
elektrischen Kräften. Faßt man die elektrischen Potentiale als generalisierte
Verschiebungen und die elektrischen Ladungen als generalisierte Kräfte auf,
so können finite Elemente mit
\glqq Verschiebung-Potential\grqq-Freiheitsgraden
konstruiert werden \cite{All70}. \\
%
Die im FE-Programm {\sf ANSYS} implementierten
elektromechanischen Zustandsgleichungen basieren auf einer gemischten
Formulierung unter Verwendung der extensiven Zustandsgröße $\{\varepsilon\}$
für die mechanische Dehnung und der intensiven Zustandsgröße $\{E\}$ für das
elektrische Feld. Bei linearem Materialverhalten und Beschränkung auf
Effekte erster Ordnung gilt \cite{Koh92}:
\begin{equation}
\fbox{$
 \displaystyle
 \begin{array}{rcl}
\label{piezofem1}
 \{\sigma\} & = & [C^{E}] \cdot \{\varepsilon\} \; - \; [e] \cdot \{E\} \\
\label{piezofem2}
 \{D\} & = & [e]^{t} \cdot \{\varepsilon\} \; \: + \;
  [\epsilon^{\varepsilon}] \cdot \{E \}
 \end{array}
 $}
\end{equation}
hierbei sind $\{\sigma\}, \{\varepsilon\}$ der Spannungs- und Dehnungsvektor,
$\{E\}, \{D\}$ das elektrische Feld und die Flußdichte, $[C^{E}]$ die
Steifigkeitsmatrix (bei konstantem elektrischen Feld), $[e]$ die
piezoelektrische Kopplungsmatrix und $[\epsilon^{\varepsilon}]$ die
Permittivitätsmatrix (bei konstanter mechanischer Dehnung). Analog zur
Finite-Elemente-Diskretisierung beim rein mechanischen Problem können
unter Verwendung von Methoden der Variationsrechnung die
gekoppelten elektromechanischen Zustandsgleichungen auf Elementebene
abgeleitet werden \cite{All70}:
\begin{small}
\begin{eqnarray}
\label{piezokopl}
 \left[
  \begin{array}{ll}
  M & 0 \\
  0 & 0
  \end{array}
 \right ]  \left \{
 \begin{array}{c}
   \stackrel{\cdot \cdot}{\vec u}  \\
   \stackrel{\cdot \cdot}{\phi}
  \end{array} \right \}           \; + \;
 \left[
 \begin{array}{ll}
  C & 0 \\
  0 & 0
 \end{array}
 \right ]
 \left \{
 \begin{array}{c}
  \stackrel{\cdot}{\vec u}  \\
  \stackrel{\cdot}{\phi}
 \end{array} \right \}
                       \; + \;
% \nonumber \\
 \left[
 \begin{array}{ll}
  K_{uu}           & K_{u \phi} \\
  K_{u \phi}^{t}   & K_{\phi \phi}
 \end{array}
 \right ]
 \left \{
 \begin{array}{c}
  \vec u \\
  \phi
 \end{array} \right \}
                        & = &
\left \{
\begin{array}{c}
\vec F \\
Q
\end{array} \right \}
\end{eqnarray}
\end{small}
Die Freiheitsgrade an den Knotenpunkten der finiten Elemente sind
durch den Vektor der Knotenverschiebungen $\vec u$ und das elektrische
Potential $\phi$ gegeben. Die Untermatrizen $[M]$ und $[K_{uu}]$ stellen
die bereits bekannte Massen- (\ref{massmatrix}) und Steifigkeitsmatrix
(\ref{steif}) dar, und beziehen sich lediglich auf das mechanische Problem,
das die Ansatzfunktionen $N_{u}(\vec x)$ für die Verschiebungen im Element
(Gleichung~\ref{uansatz}) verwendet. Die Strukturdämpfungsmatrix $[C_{D}]$
enthält nur mechanische Dämpfungsbeiträge
und wird gemäß Gleichung (\ref{cmatrix}) aufgebaut. Die piezoelektrische
Kopplung wird durch die Außerdiagonalelemente, dieses sind
die symmetrischen Untermatrizen $[K_{u\phi}] = [K_{\phi u}]^{t}$, der
verallgemeinerten elektromechanischen Gesamtsteifigkeitsmatrix
bewerkstelligt:
\begin{eqnarray}
\label{piezomatrix}
 [K_{u \phi}] & = & \int_{\Omega} [B_{u}]^{t} \, [e] \, [B_{\phi}] \, d\Omega
\end{eqnarray}
Für die dielektrische Leitfähigkeitsmatrix, die als \glqq elektrische
Steifigkeitsmatrix\grqq \, gedeutet werden kann, gilt:
\begin{eqnarray}
\label{dielekmatrix}
 [K_{\phi \phi}] & = & -\int_{\Omega}
     [B_{\phi}]^{t} \, [\epsilon^{\varepsilon}] \, [B_{\phi}] \, d\Omega
\end{eqnarray}
Auf der rechten Seite der Gleichung (\ref{piezokopl}) steht $\vec F$ fr
die Summe der mechanischen konzentrierten Knoten-, sowie verteilten
Oberflächen- und
Volumenkräfte (Gleichung~\ref{fges}). Die Summe der elektrischen
konzentrierten Punktladungen, Oberflächen- und Volumenladungsdichten wird
durch $\{Q\}$ dargestellt. Das elektrische Feld ergibt sich aus
dem Gradienten des elektrischen Potentials im Element:
\begin{eqnarray}
\label{egrad}
 \{E\} & = & - [ B_{\phi} ] \cdot \{ \phi \}
\end{eqnarray}
wobei:
\begin{eqnarray}
 [B_{\phi}] & = &
 \left \{
  \begin{array}{c}
    \partial_{x} \\ \partial_{y} \\ \partial_{z} \\
   \end{array}
  \right \} \cdot  \{ N_{\phi} \}^{t}
\end{eqnarray}
Der Vektor $\{N_{\phi}\} = \{N_{1}, N_{2},..., N_{n}\}$ wird als
Ansatzfunktion fr das elektrische Potential verwendet, wobei
$\{\phi\} = \{\phi_{1}, \phi_{2},..., \phi_{n}\}$ ist und $n$ die Anzahl
der Knoten im verwendeten Element darstellt. Ein Vorteil bei der
FE-Formulierung der piezoelektrischen Zustandsgleichungen mit Hilfe der
gemischten Zustandsgrößen $\{\varepsilon\}$ und $\{E\}$ ergibt sich aus der
Tatsache, daß sich die Dehnung und das elektrische Feld unter Verwendung
der Gleichungen (\ref{dehnversch}) und (\ref{egrad}) direkt aus den
Knotenfreiheitsgraden $(\vec u$, $\phi)$ bestimmen lassen. Hieraus können
gemäß den piezoelektrischen Zustandsgleichungen (\ref{piezofem1}) die
Spannung und die Flußdichte im finiten Element berechnet werden. \\
Bei der Berechnung von piezoelektrisch betriebenen Mikrostrukturen sind neben
den lokalen Knotenergebnissen die integralen Größen, wie elektrische
Impedanz und elektromechanischer Kopplungsfaktor von besonderem
Interesse. Mit Hilfe der mechanischen, dielektrischen und
elektromechanischen Energie im finiten Element \cite{Nai83}:
\begin{eqnarray}
\label{Emech}
E_{uu}& = & \displaystyle \frac{1}{2} \{u\}^{t} \, [K_{uu}] \, \{u\}
\nonumber \\
E_{\phi\phi}& =& \displaystyle \frac{1}{2} \{\phi\}^{t} \, [K_{\phi\phi}] \, \{\phi\}
\\
E_{u\phi}& = & \displaystyle \frac{1}{4} \left( \{u\}^{t} \, [K_{u\phi}] \, \{\phi\}
	      \; + \; \{\phi\}^{t} \, [K_{u\phi}]^{t} \, \{u\} \right)
\nonumber
\end{eqnarray}
kann der elektromechanische Kopplungsfaktor:
\begin{eqnarray}
\label{kfem}
 k^{2} & := & \frac {E_{u\phi}^{2}} {E_{uu} \cdot E_{\phi\phi}}
\end{eqnarray}
ermittelt werden. Bei dieser Gleichung %(\ref{kfem})
handelt es sich um die {\em exakte} Definition des
elektromechanischen Kopplungsfaktors, während
Gleichung (\ref{keff}) eine Näherung darstellt und nur für entkoppelte
Schwingungsmoden Gültigkeit besitzt \cite{Ler90}.
In Kapitel~5 wird das elektromechanische Wandlungsprinzips modelliert und
verschiedene Einflüsse auf das Schwingungsverhalten und den
elektromechanischen Kopplungsfaktor untersucht, sowie das
frequenzabhängige Impedanzverhalten von piezoelektrisch angetriebenen
Mikrostrukturen berechnet.



\section{Fehlerabsch„tzungen}
\label{fehlerabschaetzungen}

Als N„herungsverfahren unterliegt die FE-Methode verschiedenen
Fehlereinflüssen, so daß der Gültigkeitsbereich des FE-Modells
bei jeder Problemstellung einzeln untersucht und verifiziert werden muß.
Die Wahl der Diskretisierung muß dem Problem angepaßt sein. Für
unterschiedliche mathematische Analysen muß die Orts- und
Zeitdiskretisierung eine Auflösung der Variablenänderung ermöglichen.
In der Regel erfordern statische, dynamische und thermische Berechnungen
daher verschiedene Elementvernetzungen, da die Gradienten der
einzelnen Feldgrößen sich unterschiedlich über das Gebiet
verteilen.\\
%
Die Fehler, die bei der numerischen Behandlung auftreten können,
lassen sich grob unterteilen in den: % \cite{Bra91, Goe93, Hac86, Zie87}:
%
\begin{itemize}
\item
{\bf Modellfehler}, der durch die Vereinfachung der Realität bei
der mathematischen Idealisierung des physikalischen Problems auftritt.
Neben der ungenügenden Geometrieerfassung können weiterhin die nur ungenau
bekannten Materialeigenschaften zu erheblichen Fehlereinflüssen beitragen.
Die Idealisierung der Randbedingungen kann zusätzlich an
Bauelemen\-terändern zu singulärem Lösungsverhalten führen.
\item
{\bf Approximationsfehler} bzw.\ Diskretisierungsfehler, der die
Abweichung der FE-Näherungslösung von der exakten Lösung des
physikalischen Problems darstellt und der eine Aussage über die
Vernetzungsgüte des untersuchten Gebietes erlaubt. Die verwendeten
Formfunktionen der finiten Elemente haben hierbei einen unmittelbaren
Einfluß auf den Diskretisierungsfehler. Mit Hilfe verschiedener
{\em a priori} Fehlerabschätzungen kann das Konvergenzverhalten {\em vor}
der FE-Berechnung abgeschätzt werden \cite{Goe93}.
\item
{\bf Numerischer Fehler}, der durch die
Rechengenauigkeit der verwendeten numerischen Algorithmen verursacht
wird, da die numerischen Verfahren, die zur Integration und zur Lösung
nichtlinearer Gleichungen eingesetzt werden, nicht exakt sind. Zusätzlich
kommen Rundungsfehler hinzu, die von der Güte der rechnerinternen
Zahldarstellung (Rechner-$\varepsilon$) abhängen.
\end{itemize}
%
Die Größenordnung der einzelnen Fehler sollte aufeinander abgestimmt sein.
In der Mikromechanik sind beispielsweise die Materialdaten in der Regel
nicht genau
bekannt, so daß es nicht sinnvoll ist, den numerischen Fehler aufwendig
zu minimieren.\\
%
Im weiteren soll der Approximationsfehler $e_{h}$ genauer betrachtet werden,
der die Differenz zwischen der schwachen Lösung $\phi$ der
Gleichung~(\ref{schwach}) und der besten numerischen
Approximation $\phi_{h}$ im FE-Unterraum $S_{h}$, gemessen in einer
geeignet zu wählenden
% Für Fehlerabschätzungen in
% {\em Sobolev}-Räumen $H^{m}_{0}(\Omega)$ wird beispielsweise die
% euklidische $L_{2}(\Omega)$--Norm
% $|| \phi ||_{L_{2}}:= (\int_{\Omega}[\phi(x)]^{2}dx)^{\frac{1}{2}}$
% verwendet.
Norm\footnote{Zur lokalen Fehlerabschätzung bei statischen FE-Berechnungen
wird in der Regel die \glqq Energienorm\grqq \, benutzt, die eine
anschauliche Deutung über die Feldenergie im finiten Element erlaubt
\cite{Bra91, Hac86}.} darstellt. Dieser kann durch:
\begin{eqnarray}
\label{femfehler}
 e_{h} & = & || \phi - \phi_{h} || \leq c \cdot h^{p}
 % |\phi|_{2,p}
\end{eqnarray}
abgeschätzt werden \cite{Bra91}. Hierbei stellt $c$ eine netzunabhängige
Konstante dar.
Mit abnehmender Elementweite $h$ und zunehmender FE-Ordnung $p$ der
Elementansatzfunktion ($p$ = Polynomgrad+1) nimmt der Approximationsfehler
$e_{h}$ stetig ab. Jedoch hat die Glattheit (Regularität) der Lösungsfunktion
$\phi$ einen Einfluß auf das Konvergenzverhalten der verwendeten Elemente.
Das Approximationsverhalten wird bei Randwertproblemen im allgemeinen
zum Rand hin schlechter, so daß es wenig Sinn macht den Polynomgrad
beliebig zu erhöhen, vielmehr sollte
in diesen Bereichen die Diskretisierung hinreichend fein gewählt werden. \\
%
In der Praxis der FE-Berechnungen wird zur Verbesserung der
Konvergenzeigenschaften sowohl die Elementweite $h$ verkleinert
({\em h-Methode}), als auch der Polynomgrad der Elementansatzfunktionen
erhöht ({\em p-Methode}), um den Lösungsfortschritt beim Übergang auf eine
jeweils feinere Elementvernetzung zu bestimmen.
Durch Einsatz von {\em a posteriori}
Fehlerschätzern (siehe \cite{Zie87}) können adaptive Vernetzungstechniken
eingesetzt werden und die Elementweite und/oder Polynomordnung {\em lokal}
verändert werden, um die Auflösung von großen Feldgradienten zu ermöglichen.
Ein großer Vorteil der {\em h-Methode} besteht darin, daß einfache
Ansatzfunktionen, meist linearer oder quadratischer Ordnung, verwendet
werden und somit die Anzahl der Freiheitsgrade und der Berechnungsaufwand
niedrig gehalten wird. Der Hauptnachteil gegenüber der {\em p-Methode}
ist die erneut erforderliche Vernetzung\footnote{In diesem Zusammenhang
spricht man auch von der sogenannten {\em hr-Methode}, wobei $r$ für
{\em remesh} steht.} der Struktur. Neuere FE-Programmentwicklungen gehen
dazu über, beide Methoden zu kombinieren (sogenannte {\em hp-Methode}), um
eine optimale Konvergenzbeschleunigung zu erhalten und die Vorteile beider
Verfahren ausnutzen zu können.

