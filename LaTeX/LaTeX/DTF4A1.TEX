\chapter{Resonante Mikrostrukturen}
\label{schwingungsverhalten}

Im diesem Kapitel werden unter Bercksichtigung verschiedener
Einfluágr”áen die dynamischen Eigenschaften mikromechanischer
Balken- und Membranresonatoren aus Silizium
theoretisch und experimentell untersucht. Die Modellbildung
erfolgte gesttzt durch experimentelle Daten, um einerseits die
Modellannahmen bei den FE-Berechnungen berprfen zu k”nnen,
andererseits die FE-Modelle meátechnisch zu verifizieren.
Um fre\-quenzanaloge Sensoren zum Schwingen anzuregen, werden das
piezoelektrische und das elektrothermische Wandlungsprinzip eingesetzt.
Bei beiden Anregungsprinzipien weisen die mikromechanischen
Resonatoren eine Bimorphstruktur auf, so daá eine Momentenanregung erzielt
wird, die zu Biegeschwingungen und deren Oberwellen fhrt. Im Gegensatz
zu Quarzschwingern treten aufgrund des Bimorphaufbaus und der h”heren
Kristallsymmetrie keine komplexen -schwingungsformen
(z.B.\ Dickenscherschwingungen) auf, so daá neben Volumenelementen,
die numerisch sehr effizienten Schalenelemente eingesetzt werden k”nnen.
Neben der Berechnung der Eigenfrequenzen\footnote{Im weiteren wird statt
von Eigenfrequenzen, die die numerische L”sung der
Eigenwertgleichung~(\ref{modal}) durch Modalanalyse darstellen, allgemein
von {\em Resonanzfrequenzen} mikromechanischer Bauelemente gesprochen, da
D„mpfungs- und Anregungseffekte auf die Eigenfrequenzen vernachl„ssigt
werden. In Kapitel~\ref{daempfungseinfluesse} werden die zugrundeliegenden
Annahmen diskutiert.}
und Schwingungsformen, werden die
Sensorkennlinien, d.h.\ das Resonanzverhalten der Sensoren unter Einwirkung
mechanischer Lasten simuliert. In Zusammenarbeit mit den
im BMFT-Verbundprojekt beteiligten Firmen {\em Robert Bosch GmbH},
Gerlingen, und der {\em Gesellschaft fr Mikrotechnik und Sensorik}
(GMS~mbH), Villingen, konnten unterschiedliche
Sensorprototypen fr die Druck-, Kraft- und Str”mungsmessung
realisiert werden, bei denen die numerischen FE-Berechnungen
wichtige Hinweise fr die geometrische Auslegung liefern konnten.
Das Schwingungsverhalten von piezoelektrisch angetriebenen Siliziummembranen
wurde untersucht und die Druckempfindlichkeit charakterisiert.
Beim elektrothermisch angetriebenen Kraftsensor konnte durch eine
Žnderung des Widerstandslayouts die Kraftempfindlichkeit erh”ht und
gleichzeitig die thermischen Querempfindlichkeiten reduziert werden.
Im folgenden werden die verschiedenen FE-Modelleinflsse diskutiert
und die Ergebnisse der numerischen Berechnungen im Vergleich mit
Messungen einer Fehlerbetrachtung unterworfen.


\section{Finite Elemente Modellbildung}
\label{femodellbildung}

Wie in Kapitel~3.5 beschrieben h„ngt die Genauigkeit der FE-Berechnungen
von verschiedenen Fehlereinflssen ab, wie dem Diskretisierungsfehler,
den ungenauen Material- und Geometriedaten, sowie den numerischen Fehlern.
Um die verschiedenen Einflsse bei der Berechnung von konkreten Bauelementen
zu ermitteln und die Gr”áenordnung der einzelnen Fehlereinflsse
gegeneinander abw„gen zu k”nnen, wird das statische und dynamische
Verhalten von mikromechanischen Bauelementen untersucht. Hierzu werden die
Resonanzfrequenz der Grundbiegeschwingung und die druckabh„ngige Auslenkung
einer Siliziummembran numerisch berechnet und mit den analytischen N„herungen
verglichen.


\subsection{Analytische Absch„tzungen}
\label{abschaetzungen}

Bei den Konvergenzbetrachtungen im folgenden Kapitel sollen fr
verschiedene Elementtypen die analytischen N„herungsl”sungen
aus Kapitel~2 als Vergleichsbasis herangezogen werden.
In \cite{Fab92a, Sch92} sind gesttzt durch FE-Berechnungen analytische
Beziehungen fr die Membranmittenauslenkung und die
Resonanzfrequenz„nderung in Abh„ngigkeit der Druckbeaufschlagung
abgeleitet worden. Die Zusammenh„nge lassen sich wie folgt
zusammenfassen:
%
\begin{itemize}
\item Bei sehr kleinen Auslenkungen ($d/h \leq 0,14$) einer quadratischen
Membran mit der Seitenl„nge $l$ kann von einem {\em linearen} Verhalten
zwischen
Membranmittenauslenkung $d$ und Druckbeaufschlagung $p$ ausgegangen werden.
Bei zunehmender Druckbelastung nehmen die Auslenkungen infolge
spannungsversteifender Effekte unterproportional zu, so daá ein
{\em nichtlinearer} Korrekturterm erforderlich wird:
\begin{equation}
\label{nonlin1}
 d = 0,00136 \frac{l^{4}p}{D} \left(1 - 0,00024 \frac{p}{p_{0}} \right)
\end{equation}
Diese Gleichung gilt fr Auslenkungen $d/h \leq 1,3$ mit einer Abweichung
von etwa $\pm 6~\%$. Die Druckbeaufschlagung wird hierbei auf den
\glqq geometrischen Vergleichsdruck\grqq:
\begin{eqnarray}
\label{p0D}
 p_{0}   & := & \frac{Dh}{l^{4}} \nonumber
\end{eqnarray}
bezogen, so daá das Aspektverh„ltnis $h/l$ der Membran und die
Materialeigenschaften, bercksichtigt werden \cite{Pfe89}. Die
Biegesteifigkeit\footnote{In einigen Literaturstellen, z.B.\ \cite{Pon91},
ist
die Biegesteifigkeit durch $\hat D = \hat E/12$ definiert, so daá sie nur
materialabh„ngig ist. Fr Silizium wurde meátechnisch der Wert
15,5 $\le \hat D \le 17,5~$GPa ermittelt und f„llt gegenber dem
Literaturwert von 14,13~GPa unter Verwendung der Materialdaten aus
(\ref{simat}) etwas h”her aus.} der Platte ist durch
den Ausdruck:
\begin{eqnarray}
\label{biegesteif}
 D & := & \frac{1}{12} \hat E h^{3}
\end{eqnarray}
gegeben und besitzt fr eine 50~$\mu$m dicke Siliziummembran
den Wert $1,77$~$\cdot$~$10^{-3}$~Nm.
%
\item Bei groáen Auslenkungen ($d/h \gg 1$) l„át sich aus Gleichung
(\ref{pvond}) fr die druckabh„ngige Membranmittenauslenkung der
Zusammenhang:
\begin{eqnarray}
\label{memausl}
 \frac{d}{h} & = & c_{1} - \frac{0,886}{c_{1}}
\end{eqnarray}
ableiten, wobei:
\begin{eqnarray}
 c_{1}       & = & \sqrt [3] {\frac{p}{2c_{3}} + c_{2} }
\nonumber \\
 c_{2}       & = & \sqrt{0,6957 + {\left( \frac{p}{2c_{3}} \right)}^2}
\nonumber \\
 c_{3}       & = & 303,6672 \, p_{0}
%\left( \frac{Dh}{l^{4}} \right)
\nonumber
\end{eqnarray}
%
\item Die Resonanzfrequenz„nderung einer schwingenden Membran, die eine
Auslenkung $d$ infolge einer homogenen Druckbeaufschlagung $p$ erf„hrt,
l„át sich unter Verwendung von Gleichung~(\ref{pvond}) ableiten. Fr eine
quadratische, fest eingespannte Siliziummembran gilt:
\begin{eqnarray}
\label{fvondschroth}
  f(d) & = & f_{0} \sqrt{1 + c_{d} {\left( \frac{d}{h} \right)}^{2}}
\end{eqnarray}
wobei die Konstante $c_{d}$ von der Schwingungsmode der Membran abh„ngig
ist und fr die Grundmode etwa 1,25 betr„gt. Der N„herungsfehler liegt bei
etwa 5~\%. Fr die h”heren Schwingungsmoden\footnote{Siehe
Kapitel~\ref{membranresonatoren}.} $M_{13}$, $M_{31}$ und $M_{33}$
betragen die Konstanten $c_{d}(13)$ = $c_{d}(31)$ = 0,17 und
$c_{d}(33)$ = 0,06, so daá die Druckempfindlichkeit mit zunehmender
Schwingungsmode abnimmt \cite{Fab92a}.
%
\item Bei Vorhandensein einer homogenen, planaren Spannung $\sigma$ in der
Membran gilt fr die Resonanzfrequenz \cite{Bou90}:
\begin{equation}
\label{fsigma}
  f(\sigma) = f_{0} \sqrt{1 + c_{\sigma}
                {\left( \frac{l}{h} \right)}^{2} \frac{\sigma}{\hat E}}
\end{equation}
wobei $f_{0} := f(\sigma = 0)$ die Frequenz der unverspannten Membran ist.
Die Konstante $c_{\sigma}$ ist von den Einspannbedingungen der Membran
abh„ngig und betr„gt bei ideal starrer Einspannung 0,22 fr die Grundmode.
Fr Balkenstrukturen gilt der gleiche Zusammenhang mit der Konstante
$c_{\sigma}~\approx~$0,31 %\cite{Alb84}
(vgl. Gleichung~\ref{freqF}).
%
\item Die Approximation der Materialeigenschaften bei Bimorphstrukturen
erfolgt dadurch, daá sie in erster N„herung jeweils fr Silizium
und die Dnnschicht (hier: Zinkoxid) isotrop angenommen
und mit Hilfe der Schichtdicken gewichtet werden:
\begin{eqnarray}
\label{matwich}
 \bar X & := & \frac{h_{Si}X_{Si} + h_{ZnO}X_{ZnO}}
                    {h_{Si} + h_{ZnO}}
\end{eqnarray}
Auf diese Weise lassen sich gemittelte Materialparameter
$X \, (= \hat E, \nu, \rho)$ in analytischen Absch„tzungen verwenden.
\end{itemize}


\subsection{Konvergenzverhalten}
\label{konvergenzverhalten}

Als Referenzbauelement soll im weiteren eine ebene Siliziummembran mit einer
Kantenl„nge von 10~mm und einer homogenen Dicke von 50~$\mu$m betrachtet
werden. Die Einspannung der Membran durch die „tzbegrenzenden
(111)-Siliziumebenen werden vernachl„ssigt und eine ideale, d.h.\
unendlich starre Randeinspannung angenommen. In erster N„herung wird von
isotropen Materialeigenschaften nach Gleichung (\ref{simat}) ausgegangen.
Die betrachteten FE-Modelle bilden unter
diesen Voraussetzungen das Verhalten einer ebenen Platte ab, so daá die
analytischen N„herungsberechnungen zum Vergleich herangezogen werden k”nnen.
Aufgrund des hohen Aspektverh„ltnisses
(Kantenl„nge/Membrandicke: $l/h$~=~200) der mikromechanischen
Siliziummembran k”nnen in guter N„herung Schalenelemente
(bei {\sf ANSYS}: {\em SHELL43})
mit linearen Ansatzfunktionen und jeweils sechs Freiheitsgraden pro Knoten,
drei Verschiebungen und drei Rotationen, verwendet werden.
Zus„tzlich wird das numerische Verhalten von dreidimensionalen
Volumenelementen mit linearen (bei {\sf ANSYS}: {\em SOLID45})
und quadratischen Ansatzfunktionen (bei {\sf ANSYS}: {\em SOLID95}),
die nur Verschiebungsfreiheitsgrade
aufweisen, untersucht. Bei Membranen mit geringerem Aspektverh„ltnis
($l/h \leq 50$) wird der Einsatz von Schalenelementen zunehmend kritischer,
da die Dicke nur formal ber eine dem Element zugrundeliegende interne
mathematische Formulierung mitbercksichtigt wird, so daá auf
Volumenelemente ausgewichen werden sollte.\\
%----------------------- Beginn: Figure-Environment ----------------------
\begin{figure}[htb]
\begin{center}
% --- Dateiname des Bildes
\input{abbve.tex}
\setabbve
\end{center}
\caption{\label{abbkonvfreq}
  Konvergenzverhalten der Resonanzfrequenz}
\end{figure}
%----------------------- Ende: Figure-Environment ----------------------
In {\bf Abbildung \ref{abbkonvfreq}} ist die Abh„ngigkeit der
Resonanzfrequenz von der Knotenanzahl\footnote{Beim Vergleich von Schalen-
mit Volumenelementen und von Elementen mit linearen und quadratischen
Ansatzfunktionen ist es sinnvoller die Knotenanzahl als die Elementweite
des FE-Modells als
Vergleichsbasis zu w„hlen, da bei gleicher lateraler Elementdimension
der Rechenaufwand mit der Anzahl der Freiheitsgrade, d.h.\ Knotenanzahl,
steigt.} der FE-Modelle im logarithmischen Maástab graphisch
dargestellt. Zum Vergleich ist der analytische Wert der
Grundresonanzfrequenz von 7,056~kHz, berechnet nach Gleichung
(\ref{memfreq}), gestrichelt eingezeichnet. Die Knotenanzahl
der FE-Modelle wurde jeweils um drei GrӇenordnungen variiert.
Bei Reduktion der Elementabmessungen in lateraler Richtung entlang der
Membrankante, was einer gleichzeitigen Erh”hung der Anzahl der
Freiheitsgrade entspricht, wird die Steifigkeit ($K$) des FE-Modells
exakter beschrieben, so daá sich das FE-Modell weniger \glqq steif\grqq
\, verh„lt \cite{Zie84}. Aufgrund der verminderten Eigenwerte der
Steifigkeitsmatrix konvergiert die Resonanzfrequenz ($\omega = \sqrt{K/M}$)
von oben gegen den analytischen Wert.
Beim Schalenmodell (FE-Modell: S43) reichen bereits 900 Elemente aus,
so daá die Resonanzfrequenz weniger als 0,1~\% vom analytischen
Referenzwert abweicht. Bei den Volumenmodellen konvergieren die
Resonanzfrequenzen mindestens eine GrӇenordnung schlechter gegen
den analytischen Referenzwert. Im Vergleich zu Elementen mit
linearen Ansatzfunktionen (FE-Modell: V45) weisen quadratische
Ansatzfunktionen (FE-Modell: V95) wie erwartet ein verbessertes
Konvergenzverhalten auf.
Zur Bestimmung der Frequenzeigenwerte wurde die reduzierte
{\sl Householder}-Methode eingesetzt \cite{Koh92}.\\
In {\bf Tabelle \ref{tabdiskfreq}}
ist der Einfluá der Diskretisierung und der dynamischen Hauptfreiheitsgrade
(MDOF) auf die Berechnung der Resonanzfrequenz zusammengefaát.
%----------------------- Beginn: table ---------------------------
\begin{table}[htb]
\caption{\label{tabdiskfreq}
 Einfluá der Modellparameter auf die Resonanzfrequenz beim Schalenmodell}
\begin{center}
\begin{tabular} {|c||c|c|c|c||c|c|}
\hline
Modell & \multicolumn{4}{c||}{Modellparameter}
       & \multicolumn{2}{c|}{Resonanzfrequenz} \\
\hline
S43 & h [mm] & Elemente & Knoten & MDOF & f [kHz] & Abw. \\
\hline \hline
1  & 3,33 &    9  &    12 &  24 & 8,365 & 18,55 \% \\
2  & 1,66 &   36  &    49 &  50 & 7,198 &  2,01 \% \\
3  & 1,00 &  100  &   121 & 100 & 7,098 &  0,60 \% \\
4  & 0,59 &  289  &   324 & 100 & 7,072 &  0,23 \% \\
5  & 0,33 &  900  &   961 & 100 & 7,060 &  0,06 \% \\
6  & 0,20 & 2500  &  2601 & 300 & 7,056 &  --- \\
\hline
\end{tabular}\\
\end{center}
\end{table}
%----------------------- Ende: table ---------------------------
Der Parameter $h$ bezeichnet die Elementabmessung der
„quidistanten Vernetzung.
Eine Vergleichsrechnung bei der Modellvariante~5 des Schalenmodells S43
ergab bei 900 Elementen und 961 Knoten unter Verwendung des
{\sl Subspace}-Iterationsverfahrens
eine um 4~Hz verminderte Resonanzfrequenz. Dieses entspricht dem Ergebnis
der FE-Modellvariante~6, die etwa die dreifache Element- bzw.\ Knotenanzahl
aufweist. Der Einsatz des iterativen JCG-Gleichungsl”sers
({\em \underline{J}acobian \underline{C}onjugate \underline{G}radient})
lieferte die
gleiche Ergebnisgenauigkeit, jedoch auf Kosten einer sechsfach h”heren
Rechenzeit. Daher ist der Einsatz nur bei sehr groáen Modellen sinnvoll,
bei denen das direkte {\sl Wavefront}-L”sungsverfahren die Kapazit„t des
Hauptspeichers berschreiten wrde.\\
Die druckabh„ngige Membranauslenkung wurde unter Bercksichtigung der
geometrischen Nichtlinearit„ten\footnote{Bercksichtigung von
'Large Deflection' und 'Stress Stiffening'.} nach
Kapitel~\ref{nlstatberechnungen} mit dem {\sl Newton-Raphson}-Verfahren
berechnet. Als Konvergenzkriterium wurde die Norm des Residuenkraftvektors,
die rechte Seite der Gleichung (\ref{largedefl}), verwendet.
Bei ungengender Elementunterteilung fallen die Membranauslenkungen des
FE-Modells aufgrund der zu hohen Modellsteifigkeit zu niedrig aus.
Im Gegensatz zu Elementen, die nur Verschiebungsfreiheitsgrade
(Translational DOF) besitzen, beispielsweise Volumenelemente, weisen
Schalenelemente zus„tzliche Rotationsfreiheitsgrade (Rotational DOF) auf.
Infolge der erlaubten Knotenrotationen k”nnen sich die Schalenelemente
bei sehr grober Vernetzung aus ihrer Ebene heraus verw”lben, so daá die
Membranauslenkung h”her ausf„llt. Dieses fhrt dazu, daá bei zunehmender
Ele\-mentanzahl sich die Einflsse der Verschiebungs- und
Rotationsfreiheitsgrade berlagern und ein
Konvergenzverhalten\footnote{Der
Einfluá von Schalen- und Volumenelementen auf das
statische Verhalten von mikromechanischen Drucksensoren in Bezug auf die
auftretenden Auslenkungen und den Ort der maximalen Spannungen in
Abh„ngigkeit des L„ngen-/Dickenverh„ltnisses ist in \cite{Tol92} untersucht
worden. Die Ergebnisse best„tigen das Konvergenzverhalten der durchgefhrten
Schalenberechnungen.}
von oben an den {\em wahren} Wert erfolgt. Die Ergebnisse der Schalenmodelle
mit den geringen Elementzahlen (3x3 bzw.\ 5x5 Elemente) lassen keine
quantitativen Aussagen zu, da die Biegelinie der Membran mit einer so
geringen Seitenunterteilung nur unzureichend abgebildet wird.
%\clearpage
%----------------------- Beginn: Figure-Environment ----------------------
\begin{figure}[htb]
\begin{center}
% --- Dateiname des Bildes
\input{abbvze.tex}
\setabbvze
\end{center}
\caption{\label{abbkonvausl}
 Konvergenzverhalten der Membranmittenauslenkung}
\end{figure}
%----------------------- Ende: Figure-Environment ----------------------
In {\bf Abbildung \ref{abbkonvausl}} ist die Abh„ngigkeit der
Membranmittenauslenkung von der
Knotenanzahl fr das Schalenmodell S43 und die Volumenmodelle
V45 und V95 graphisch dargestellt. Das Volumenmodell V45 weist
aufgrund der linearen Ansatzfunktionen zwei Elementlagen ber
die Membrandicke auf. Als Vergleich wurde die Membranmittenauslenkung
nach Gleichung~(\ref{memausl}) analytisch bei einer Druckbeaufschlagung
von 500~mbar (= 0,5$\cdot 10^{5}$~Pa) zu etwa 116~$\mu$m berechnet und
gestrichelt eingezeichnet. Auch hier zeichnet sich das Schalenmodell
durch sehr gute Konvergenzeigenschaften aus. Bei etwa 300 Elementen wird
bereits ein stabiler Wert fr die Membranmittenauslenkung
erreicht. Im Gegensatz dazu n„hern sich die Volumenmodelle
aufgrund der generell zu \glqq hohen\grqq \, Steifigkeit von
unten an. Bei gengend feiner Vernetzung konvergieren alle numerischen
Berechnungen an eine Auslenkung von etwa 107~$\mu$m. Die Abweichung
zum analytischen Referenzwert betr„gt rund 9~\%, so daá die abgeleitete
analytische Gleichung nur n„herungsweise zur Beschreibung herangezogen
werden kann.\\
%
In {\bf Tabelle \ref{tabfreqzus}} sind die Ergebnisse der numerischen
Resultate abschlieáend zusammengefaát.
%----------------------- Beginn: table ---------------------------
\begin{table}[htb]
\caption{\label{tabfreqzus}
 Zusammenfassung der Modellparameter und numerischen Resultate}
\begin{center}
\begin{tabular} {|c||c|c|c||c|c|}
\hline
FE-Modell & h [$\mu$m] & Elemente & Knoten & f [Hz] & $u_{z}$ [$\mu$m] \\
\hline \hline
S43  & 200 &  2500  & 2601  & 7055,6  & 106,4 \\
V45  & 200 &  5000  & 7803  & 7068,2  & 106,8 \\
V95  & 330 &  900   & 6603  & 7066,4  & 106,6 \\
\hline
V64  & 200 &  2500  & 5202  & 6798,4  & 108,0 \\
\hline
\end{tabular}\\
\end{center}
\end{table}
%----------------------- Ende: table ---------------------------
Zum Vergleich ist eine Berechnung (FE-Modell: V64, eine Elementlage)
mit anisotropen Materialdaten \cite{LB82}
durchgefhrt worden. Die Abweichungen betragen etwa 4~\% bei der
Resonanzfrequenz und etwa 1~\% bei der Auslenkung gegenber der
Beschreibung mit isotropem Materialverhalten.
Bei den Volumenmodellen wurde
zus„tzlich der Einfluá der Elementierung ber die Membrandicke
untersucht. Bei einer Verdreifachung der Elementlagen resultierte
eine Frequenzerniedrigung von lediglich 0,1~\%. Dieses ist dadurch zu
erkl„ren, daá ausschlieálich die laterale Elementierung ausschlaggebend
ist. Die Unterteilung ber die Membrandicke hat nur einen Einfluá, falls
sich der Spannungszustand ber die Dicke „ndert. In Kapitel~4.3.2
werden die FE-Modelleinflsse bei der druckabh„ngigen Membranauslenkung
und Eigenfrequenz„nderung im Vergleich mit einem vermessenen
Silizium-Drucksensor diskutiert.


\subsection{Modelleinflsse}
\label{modelleinfluesse}

Verschiedene FE-Modelle sind im Rahmen einer Diplomarbeit \cite{Sch92}
fr die Modellierung von resonanten Silizium-Drucksensoren entwickelt
und der Einfluá der Geometrie- und der Materialeigenschaften unter
verschiedenen Einspannbedingungen untersucht worden. Es wurden sowohl
zweidimensionale FE-Modelle mit isotropen Materialeigenschaften, als auch
dreidimensionale Modelle unter der Bercksichtigung der
Elastizit„tsanisotropie des Siliziums und der realen
Membraneinspannverh„ltnisse simuliert. Zus„tzlich wurden Elementtypen
und Symmetrieeigenschaften der FE-Modelle variiert und die analytischen
N„herungsgleichungen (\ref{memausl}) und (\ref{fvondschroth})
von den numerischen Berechnungsergebnissen abgeleitet. Als Referenz
wurde ein dreidimensionales FE-Modell entwickelt, das die Anisotropie
der Elastizit„tseigenschaften und die schr„ge Randeinspannung durch
die um $54,74^{\circ}$ geneigten (111)-Siliziumebenen
bercksichtigt. Die wichtigsten Ergebnisse bei der Berechnung der
Grundresonanzfrequenz lassen sich wie folgt zusammenfassen:
% ------------------------ Beginn : Liste --------------------------------
\begin{itemize}
\item
Aufgrund der schr„gen Randeinspannung der Siliziummembran durch die
(111)-Ebenen verringert sich die Resonanzfrequenz infolge verminderter
Einspannungssteifigkeit.
%Der Unterschied zwischen einer ideal starren
%Einspannung bei einem Schalenmodell und der endlich steifen Bulkeinspannung
%betr„gt etwa 2,3~\%.
\item
Die Bercksichtigung der anisotropen Materialeigenschaften hat eine
weitere Frequenzerniedrigung %von etwa 3,1~\%
zur Folge.
\item
Die Reihenfolge der Schwingungsmoden und Frequenzeigenwerte ist abh„ngig
von den Symmetriebedingungen (Viertel-, Halb- oder Vollmembran).
Insbesondere lassen sich bei symmetrischen Randbedingungen keine
antisymmetrischen Schwingungsformen ermitteln und umgekehrt.
\end{itemize}
% ------------------------ Ende : Liste ----------------------------------
Im ungnstigsten Fall berlagern sich obige Einflsse und es ergibt sich
zwischen dem genauen Referenzmodell (3D, anisotrop, schr„ge Einspannung)
und der analytischen N„herungsberechnung nach Gleichung (\ref{memfreq})
eine maximale Abweichung von bis zu 10~\%, wobei der Restfehler
auf eine ungengende Vernetzungsdichte zurckgefhrt werden kann. Bei der
Berechnung der Eigenfrequenzen h”herer Schwingungsmoden
nimmt die Abweichung erheblich zu,
da eine erh”hte Anzahl von Freiheitsgraden (DOF) in Verbindung mit
einer feineren Elementunterteilung erforderlich ist, um Schwingungsmoden
mit mehreren Schwingungsknoten und -b„uchen \glqq r„umlich\grqq \,
abzutasten\footnote{In Analogie zum {\sl Shannon}schen Abtasttheorem
der Nachrichtentechnik kann dieses als \glqq geometrisches
Abtasttheorem\grqq \, bezeichnet werden.}.\\
%
Um den Einfluá der anisotropen Materialeigenschaften auf das
Schwingungsverhalten weiterer mikromechanischer Strukturen zu
untersuchen, wurden die Resonanzfrequenzen der Grundbiegeschwingungsmode
der drei Grundstrukturen aus Abbildung~\ref{abbgrundgeometrien}
berechnet. {\bf Tabelle~\ref{tabgeomvergleich}} faát die
numerisch berechneten Resonanzfrequenzen im Vergleich mit den analytischen
Werten aus Tabelle~\ref{tabcf} zusammen.
%----------------------- Beginn: table ---------------------------
\begin{table}[htb]
\caption{\label{tabgeomvergleich}
 Einfluá der Materialanisotropie auf die Resonanzfrequenz bei
 unterschiedlicher Resonatorgeometrie}
\begin{center}
\begin{tabular}{|l||c|c|c||l|}
\hline
%Resonanzfrequenz
 & Zunge & Balken & Membran & Bemerkung \\
\hline \hline
L„nge [mm]:            & 5     & 10 & 10 & Bauelementdicke:\\
Breite [mm]:           & 1     & 1  & 10 & 50 $\mu$m\\
\hline
 $f_{analytisch}$ [Hz] &  2765 &  4386  & 7056 & isotrop (\ref{simat}) \\
\hline
 Elemente: &  365  & 730  &  900 & lineare Ansatz- \\
 Knoten:   &  690  & 1368 & 1922 & funktionen \\
\hline
$f_{isotrop}$ [Hz] & 2731,6  & 4349,7  &  7137,1  & $E_{110}$, $\nu_{110}$
(\ref{simat}) \\
$f_{anisotrop}$ [Hz] & 2430,5 & 3869,3 &  6796,7  & (100)--orientiert \\
$f_{anisotrop}$ [Hz] & 2732,4 & 4350,9 &  6869,9  & (110)--orientiert \\
\hline
Abweichung:
		& 1,22~\%  & 0,83~\%  &  0,17~\%  & analytisch/isotrop\\
                & 0,029~\% & 0,028~\% &  3,89~\%  & iso-/anisotrop(110)\\
		& 11,0~\%  & 11,0~\%  &  1,10~\%   & aniso(100)/aniso(110)\\
\hline
\end{tabular}
\end{center}
\end{table}
%----------------------- Ende: table ---------------------------
Aufgrund der Ausrichtung der Siliziumbauelemente parallel
zum Waferflat, der entlang der (110)-Siliziumebene orientiert ist, war es
erforderlich, das Elementkoordinatensystem um $45^{\circ}$
gegenber dem globalen Koordinatensystem des FE-Modells zu verdrehen.
Bei Bercksichtigung der Anisotropie weichen die Resonanzfrequenzen der
einseitig eingespannten Zunge und beim doppelseitig eingespannten Balken
gegenber der isotropen Beschreibungsweise lediglich um 1~Hz ab, weil
die effektiven Materialparameter (\ref{simat}) fr die (110)-Richtung im
Silizium bereits sehr gute Ersatzwerte fr die anisotrope Formulierung
darstellen. Fr Membranstrukturen betr„gt die Abweichung zwischen
isotroper und anisotroper Beschreibungsweise bei (110)-Orientierung
weniger als 4~\%.\\
Die Abweichungen bei anderen Kristallrichtungen im Silizium, beispielsweise
bei (100)-orientierten Bauelementen (siehe Tabelle~\ref{tabgeomvergleich}),
fallen grӇer aus. Insbesondere bei den Balken- und Zungenstrukturen hat die
Orientierung einen erheblichen Einfluá im Vergleich zu Membranstrukturen.
W„hrend bei Membranen die Frequenzwerte nur um etwa 1~\% differieren,
betr„gt der Unterschied bei (100)- und (110)-orientierten Balkenstrukturen
11~\%. Die Ursache hierfr liegt in der starken Variation der
Querkontraktion in der (100)-Kristallebene von 0,064 bis 0,279 \cite{Heu89}
und dem Umstand, daá die Membranen allseitig eingespannt sind und dadurch
der Effekt vermindert wird. Bei der Berechnung von Siliziumstrukturen
in isotroper N„herung sind daher jeweils zuerst geeignete isotrope
Ersatzmaterialdaten zu bestimmen.



\subsection{D„mpfungseinflsse}
\label{daempfungseinfluesse}

Bei der Berechnung der Eigenfrequenzen mikromechanischer Strukturen
mit Hilfe der FE-Methode wird die Eigenwertgleichung~(\ref{modal})
unter Vernachl„ssigung von D„mpfungseffekten numerisch gel”st. Auf dieser
Grundlage ist in Kapitel~\ref{konvergenzverhalten} das Konvergenzverhalten
der Eigenfrequenzen untersucht und in Kapitel~\ref{modelleinfluesse} sind
weitere verschiedene Modelleinflsse betrachtet worden. Bei der
Charakterisierung realer mikromechanischer Resonatoren haben D„mpfungs- und
Anregungseffekte einen Einfluá auf die Eigenfrequenzen, so daá diese
Effekte hier abgesch„tzt werden sollen.\\
%
Wie in Kapitel~3.4.2 dargelegt berlagern sich verschiedene
D„mpfungsmechanismen bei dynamischen Vorg„ngen und es lassen sich den
einzelnen D„mpfungsbeitr„gen wie innere Strukturd„mpfung, „uáere
Fluidd„mpfung, D„mpfung durch zus„tzliche Schichten jeweils
Einzelschwingungsgten $Q_{i}$ zuordnen, so daá fr die effektive
Gesamtschwingungsgte $Q_{ges}$ gilt \cite{Til92}:
\begin{equation}
\label{qsum}
 \frac{1}{Q_{ges}} = \sum_{i} \frac{1}{Q_{i}}
\end{equation}
Aus diesem Zusammenhang ist ersichtlich, daá der dominierende D„mpfungsterm
die geringste Einzelgte zur Folge hat und dementsprechend die anderen
D„mpfungseffekte berdeckt.\\
%
Bei der experimentellen Bestimmung der mechanischen Schwingungsgte eines
Mikroresonators wird immer die Gesamtgte $Q_{ges}$ gemessen, die auf
verschiedene Weisen ermittelt werden kann. Dieses kann einerseits durch die
Vermessung der Halbwertsbreite (3dB-Amplitudenabfall) der frequenzabh„ngigen
Amplitudenkurve $A(\omega)$, andererseits durch die Amplitudenberh”hung,
d.h.\ dem Verh„ltnis der Resonanzamplitude $A_{res} := A(\omega_{res})$
zur statischen Amplitude $A_{stat} := A(\omega=0)$ erfolgen
\cite{VIB}:
\begin{equation}
\label{qmess}
 Q = \frac{\omega_{res}}{\Delta \omega_{-3dB}}
    \approx \frac{\pi^2}{8} \frac{A_{res}}{A_{stat}}
\end{equation}
%
In der Literatur werden im wesentlichen drei Bereiche unterschieden, bei
denen unterschiedliche D„mpfungsmechanismen bei mikromechanischen Strukturen
dominieren (z.B.\ \cite{Blo92}). Im Niederdruckbereich ($p \leq 1$ Pa)
ist die „uáere Fluidd„mpfung vernachl„ssigbar und es sind nur innere
Strukturd„mpfungseffekte vorhanden, so daá die Schwingungsgte und die
Resonanzfrequenz ihre maximalen Werte erreichen.
Im Molekularbereich bzw.\ {\sl Knudsen}bereich ($p=1...10^2$ Pa)
wird die D„mpfung durch
Impulsbertrag infolge Moleklkollisionen mit der schwingenden
Resonatoroberfl„che verursacht. In diesem Bereich erh”ht sich die D„mpfung
mit steigendem Druck $p$, so daá $Q \sim 1/\sqrt{p}$ fr die Schwigungsgte
gilt \cite{Blo92}. Mit zunehmendem Druck nimmt die freie Wegl„nge der
Molekle ab und sie wechselwirken miteinander, so daá zus„tzliche Kr„fte
durch den Luftwiderstand wirksam werden. In diesem sogenannten viskosen
D„mpfungsbereich ($p=10^3...10^5$ Pa) werden die Str”mungsverh„ltnisse
komplexer und durch die {\sl Navier-Stokes}-Gleichungen beschrieben. Bei
Atmosph„rendruck ($p \approx 10^5$ Pa) bilden sich turbulente
Grenzschichten an der Resonatoroberfl„che aus, w„hrend fr geringere Drcke
laminare Str”mung vorliegt. Der šbergang vom turbulent zum laminar
dominierten D„mpfungsbereich ist von der Resonatorgeometrie anh„ngig.
Fr Balkenresonatoren aus Quarz und Silizium sind die Abh„ngigkeiten
der Schwingungsgte und der Resonanzfrequenz vom Druck durch
{\sl Blom et al.} \cite{Blo92} und {\sl Christen} \cite{Chr83}
theoretisch und experimentell untersucht worden. Die relative
Frequenz„nderung $\Delta f / f$ von Balkenresonatoren betr„gt infolge
Druckschwankungen im viskosen Druckbereich etwa 0,3--1,1~ppm/mbar
\cite{Bus94, Chr83} und ist abh„ngig vom Dichteverh„ltnis des
Resonatormaterials und des umgebenden Gases.\\
Infolge der Einspannbedingungen k”nnen die Schwingungsgten ebenfalls
erheblich beeinfluát werden. Fr Membranresonatoren aus Silizium wurden
bei Drcken von $p \approx 10$ Pa Gten bei {\em starrer} Einspannung von
etwa 900 gemessen, die sich bei {\em flexibler} Einspannung der Membranen
auf etwa 150 erniedrigten \cite{Pra91}.\\
Bei den im weiteren betrachteten Resonatoren, die bei Atmosph„rendruck
betrieben werden, ist die Luftd„mpfung der dominierende D„mpfungsmechanismus.
Infolge der endlichen Schwingungsgte $Q$ ergibt sich eine geringfgige
Erniedrigung der Vakuumeigenfrequenz $f_{0}$ infolge D„mpfung \cite{Pet79}:
\begin{equation}
\label{qpetersen}
 f_{res} = f_{0} \sqrt{1 - \frac{1}{4Q^{2}} }
\end{equation}
Da die typischen Schwingungsgten der in dieser Arbeit untersuchten
mikromechanischen Membran- und Balkenresonatoren bei Atmosph„rendruck
im Bereich von 50--1000 liegen, l„át sich der D„mpfungskorrekturfaktor
bei der Betrachtung der Resonanzfrequenz vernachl„ssigen, d.h.\
$f_{res} \approx f_{0}$.  Wie in Kapitel~4.5 gezeigt wird,
sind die herstellungstechnologiebedingten Geometrietoleranzen und die
inneren Schichtspannungen die wesentlichen Unsicherheitsfaktoren bei der
meátechnischen Bestimmung der Resonanzfrequenzen.\\
Die Schwingungsgte eines Resonators hat ebenfalls einen Einfluá auf die
erzielbare Frequenz- und damit Meágr”áenaufl”sung bei einem resonanten
Sensor und ist durch:
\begin{eqnarray}
\label{Qaufl}
 \frac{\Delta f}{f} & \sim & \frac{1}{Q} \left( \frac{\Delta A}{A} \right)
\end{eqnarray}
gegeben \cite{Til93}. Hierbei ist $\Delta A$ die minimale
Amplitudenaufl”sung des eingesetzten Detektionsmechanismus.
In Kapitel~\ref{dynamischemomentenkompensation} wird auf die Erh”hung der
Schwingungsgte eines gekoppelten Balkenschwingers eingegangen, der die
dynamische Momentenkompensation einer antisymmetrischen Schwingungsmode
ausnutzt.\\
Ein weiterer Einfluá auf die Resonanzfrequenz wird durch nichtlineare
dynamische Effekte hervorgerufen, die infolge einer Erh”hung der
Anregungsamplitude zu einer Zunahme der Resonanzfrequenz fhren \cite{Eis64}:
\begin{equation}
\label{nlampli}
 f(A_{max}) = f_{0} \sqrt{1 + \beta \left( \frac{A_{max}}{h} \right)^2 }
\end{equation}
Mit zunehmender Anregungsamplitude wird der frequenzabh„ngige
Amplitudenverlauf $A(\omega)$ asymmetrisch und es k”nnen sogar
Resonatorinstabilit„ten und eine damit verbundene Hysterese auftreten
\cite{And87, Zoo92}. Dieser Effekt konnte bei den in dieser Arbeit
vermessenen Strukturen nur bei sehr groáen Schwingungsamplituden
($A_{max} \approx h$) nachgewiesen werden. Unter normalen
Betriebsbedingungen bei resonanten Sensoren ist jedoch $A_{max} \ll h$,
so daá die Frequenzerh”hung zu vernachl„ssigen ist.


\newpage
\section{Eigenfrequenzen und Schwingungsformen}
\label{eigenfrequenzenundschwingungsformen}

In Zusammenarbeit mit der {\em GMS~mbH} wurden resonante Kraft- und
Str”mungssenoren auf der Basis von elektrothermisch angetriebenen
Silizium-Balkenresonatoren realisiert. Die Entwicklung der Prozeátechnologien
zur Herstellung der Balkenresonatoren wurde von {\em GMS} durchgefhrt,
w„hrend der Entwurf und die Simulation, sowie die meátechnische
Charakterisierung der Sensoren im Rahmen dieser Arbeit erfolgten.


\subsection{Balkenresonatoren}
\label{balkenresonatoren}

Um das dynamische Verhalten von Balken- und Membranresonatoren
zu untersuchen, wurden reine Siliziumstrukturen ohne Dnnschichtsystem
mit Hilfe von extern angebrachten Piezokeramiken ber K”rperschall
akustisch zum Schwingen angeregt. Die hierfr eingesetzten experimentellen
Meáaufbauten und verwendeten Meámethoden sind im Anhang beschrieben.\\
%
{\bf Tabelle~\ref{tabbalkenfreq}} faát die gemessenen Resonanzfrequenzen
von etwa 50~$\mu$m dicken Silizium-Balkenresonatoren,
deren Strukturgeometrie in Abbildung~\ref{abbgmssensor} dargestellt ist,
im Vergleich mit FE-Berechnungen zusammen. Die mit $f_{Z1}$ normierten
Frequenzen sind in Klammern angegeben. Die Modellierung erfolgte
dreidimensional unter Verwendung anisotroper Materialkennwerte \cite{LB82}
und Bercksichtigung der schr„gen Balkeneinspannung durch die
(111)-Siliziumebenen. Die ersten beiden Oberschwingungen aus der
Balkenebene heraus sind mit Z2 und Z3 gekennzeichnet und besitzen zwei
bzw.\ drei Schwingungsb„uche.
%----------------------- Beginn: Tabelle ---------------------------
\begin{table}[htb]
\caption{\label{tabbalkenfreq}
 Resonanzfrequenzen von Silizium-Biegebalken bei Fremdanregung
 (Vergleich: FE-Berechnungen -- Messungen)}
\begin{center}
\begin{tabular} {|c||c|c|c||c|c|c|}
\cline{1-7}
   & \multicolumn{3}{c||}{L„nge = 10 mm} & \multicolumn{3}{c|}{L„nge = 8 mm} \\
\hline
Mode  & FEM  & Exp. & Abw. & FEM & Exp. & Abw. \\
\hline \hline
Z1 & 4,338 kHz  &  4,332 kHz & 0,1~\% &  6,763 kHz & 7,185 kHz  & 5,9~\% \\
   & (1,000)    &  (1,000)   &       &  (1,000)   & (1,000)    & \\
\hline
Z2 & 11,959 kHz & 11,390 kHz & 5,0~\% & 18,646 kHz & 18,718 kHz & 0,4~\% \\
   & (2,757)    &  (2,629)   &       &  (2,757)   & (2,605)    & \\
\hline
Z3 & 23,457 kHz & 24,893 kHz & 5,8~\% & 36,582 kHz & 36,440 kHz & 0,4~\% \\
   & (5,407)    &  (5,746)   &       &  (5,409)   & (5,072)    & \\
\hline
\end{tabular}\\
\end{center}
\end{table}
%----------------------- Ende: Tabelle ---------------------------
Die Bestimmung der Resonanzfrequenzen erfolgte in einem
schmalen Frequenzbereich, so daá der Meáfehler bei der Frequenzbestimmung
etwa 5--10~Hz betr„gt. Die gemessenen Frequenzwerte variieren
allerdings aufgrund der Bauelementestreuung um 5--10~\%. Das entspricht
bei den Grundmoden etwa $\pm200$~Hz, so daá
in Tabelle~\ref{tabbalkenfreq} die Mittelwerte der Messungen angegeben sind.
Die groáe Bauelementestreuung ist im wesentlichen auf den beidseitigen
Naá„tzprozeá zurckzufhren, der zu nicht genau reproduzierbaren
Balkendicken fhrt \cite{ABV93}. Die Abweichungen der numerischen
Resultate von den gemessenen Werten betragen bis zu 6~\%. Sie lassen sich
einerseits durch die Geometrietoleranzen, andererseits durch die innere
Vorspannung, die durch die Siliziumoxidschicht bewirkt wird und fr
etwa 1~$\mu$m dnne Schichten einige MPa betragen kann \cite{Mur92},
erkl„ren. Die Schwingungsgten variierten in Abh„ngigkeit der
Sensoreinspannung stark und betrugen fr die Grundmoden $400\pm100$, fr die
beiden Oberschwingungen Z2 und Z3 etwa $900\pm200$ und $1200\pm300$.
Die analytische Berechnung der Grundresonanzfrequenz $f_{Z1}$
nach Gleichung~(\ref{balkfreq}) unter Verwendung isotroper
Materialeigenschaften (Gleichung~\ref{simat}) ergibt fr
unverspannte Siliziumbalken bei 10 und 8~mm Resonatorl„nge
Resonanzfrequenzen von 4,386~kHz und 6,853~kHz in guter šbereinstimmung
mit den numerisch berechneten Ergebnissen, so daá die Abweichungen nur etwa
1~\% betragen (siehe Tabelle~\ref{tabbalkenfreq}).
%----------------------- Beginn: Figure-Environment ----------------------
\begin{figure}[htb]
\begin{center}
% --- Dateiname des Bildes
\input{abbvda.tex}
\setabbvda
\end{center}
\caption{\label{abbbalkenmoden}
 Eigenschwingungsformen eines Silizium-Biegebalkens}
\end{figure}
%----------------------- Ende: Figure-Environment ----------------------
In {\bf Abbildung~\ref{abbbalkenmoden}} sind die experimentell ermittelten
Eigenschwingungsformen (Meápunkte) eines 10~mm langen Balkens dargestellt,
der elektrothermisch mit einer niedrigen mittleren Impulsheizleistung
($\overline{P_{Heiz}}$~=~50~mW) angeregt und optisch abgetastet
wurde \cite{Mue92}. Aufgrund der jeweils an beiden Balkenenden symmetrisch
angeordneten Heizwiderst„nde, wurden nur die symmetrischen Schwingungsmoden
Z1 und Z3 elektrothermisch selektiv angeregt. Die unsymmetrischen
Biegeschwingungsmoden Z2 und Z4,
sowie die Torsionsmoden konnten im Modenspektrum nicht nachgewiesen
werden. Die Resonanzfrequenzen\footnote{Auch mit Metallschichtsystem
stimmen die Resonanzfrequenzen der Biegebalken mit denen der unbeschichteten
Siliziumbalken hinreichend gut berein, sofern die elektrische
Verlustleistung klein bleibt und die Temperaturberh”hung auf dem Balken
wenige Kelvin nicht berschreitet (siehe Kapitel~\ref{temperaturverhalten}).}
der beiden vermessenen Moden betrugen etwa $f_{Z1} = 4,23$~kHz und
$f_{Z3} = 24,1$~kHz.\\
Weil die rechnerische Modalanalyse
die Eigenwertgleichung (\ref{modal}) unter Vernachl„ssigung von
D„mpfungsbeitr„gen und Anregungskraft l”st, k”nnen nur relative
Schwingungsamplituden ermittelt werden. Um die berechneten
Amplitudenverl„ufe (durchgehende Kurvenzge) mit den Meáwerten vergleichen
zu k”nnen, wurden die gemessenen Maximalwerte auf eins normiert.
Beim optischen Nachweis steht im Gegensatz zur elektrischen Detektion
die Phaseninformation zwischen
Anregungsspannung und resultierender Schwingungsamplitude nicht zur
Verfgung, so daá nur die Absolutbetr„ge der Schwingungsamplituden
detektierbar sind. Die mittleren Meápunkte der Obermode Z3 muáten daher
mit negativen Vorzeichen versehen werden.
Da die Nullpunktjustierung der Balkenkoordinate des FE-Modells nicht
eindeutig ist, wurden ferner die gemessenen Amplitudenmaxima der beiden
Biegeschwingungen Z1 und Z3 in Balkenmitte des FE-Modells gelegt.\\
%
Die šbereinstimmung der berechneten mit den gemessenen Werten ist sehr gut
und l„át auf eine gengende Elementunterteilung und Erfassung der
richtigen Einspannbedingungen des Balken zurckschlieáen. Aufgrund der
niedrigen Eigenspannungen (einige MPa) der $SiO_{2}$--Schicht werden die
Eigenformen in ihrem Verlauf nicht beeinfluát. Im Gegensatz dazu zeigten
$ZnO$--beschichtete Siliziummembranen (siehe Kapitel~5.3.2) sehr hohe innere
Spannungen, so daá eine merkliche Abweichung des lateralen Modenverlaufs
zwischen Messung und Simulation auftrat \cite{Fab92b}. Eine hohe Vorspannung
im Resonator bewirkt auáer einer
Frequenzverschiebung einen steiler ansteigenden lateralen Modenverlauf bei
den Einspannstellen gegenber unverspannten Resonatoren.
Quantitative Untersuchungen zum Einfluá von
Schichtspannungen auf den Verlauf der lateralen Eigenformen sind von
{\sl Prak} fr Balkenresonatoren durchgefhrt worden \cite{Pra93}. Die
Beeinflussung des Auslenkungsverhaltens ebener Membranen infolge innerer
Spannungen wurde von {\sl Steinmann et al.} analytisch und numerisch
untersucht \cite{Ste93}.

