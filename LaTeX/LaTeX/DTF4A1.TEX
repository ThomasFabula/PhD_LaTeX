\chapter{Resonante Mikrostrukturen}
\label{schwingungsverhalten}

Im diesem Kapitel werden unter Berücksichtigung verschiedener
Einflußgrößen die dynamischen Eigenschaften mikromechanischer
Balken- und Membranresonatoren aus Silizium
theoretisch und experimentell untersucht. Die Modellbildung
erfolgte gestützt durch experimentelle Daten, um einerseits die
Modellannahmen bei den FE-Berechnungen überprüfen zu können,
andererseits die FE-Modelle meátechnisch zu verifizieren.
Um fre\-quenzanaloge Sensoren zum Schwingen anzuregen, werden das
piezoelektrische und das elektrothermische Wandlungsprinzip eingesetzt.
Bei beiden Anregungsprinzipien weisen die mikromechanischen
Resonatoren eine Bimorphstruktur auf, so daß eine Momentenanregung erzielt
wird, die zu Biegeschwingungen und deren Oberwellen führt. Im Gegensatz
zu Quarzschwingern treten aufgrund des Bimorphaufbaus und der höheren
Kristallsymmetrie keine komplexen -schwingungsformen
(z.B.\ Dickenscherschwingungen) auf, so daß neben Volumenelementen,
die numerisch sehr effizienten Schalenelemente eingesetzt werden können.
Neben der Berechnung der Eigenfrequenzen\footnote{Im weiteren wird statt
von Eigenfrequenzen, die die numerische Lösung der
Eigenwertgleichung~(\ref{modal}) durch Modalanalyse darstellen, allgemein
von {\em Resonanzfrequenzen} mikromechanischer Bauelemente gesprochen, da
Dämpfungs- und Anregungseffekte auf die Eigenfrequenzen vernachlässigt
werden. In Kapitel~\ref{daempfungseinfluesse} werden die zugrundeliegenden
Annahmen diskutiert.}
und Schwingungsformen, werden die
Sensorkennlinien, d.h.\ das Resonanzverhalten der Sensoren unter Einwirkung
mechanischer Lasten simuliert. In Zusammenarbeit mit den
im BMFT-Verbundprojekt beteiligten Firmen {\em Robert Bosch GmbH},
Gerlingen, und der {\em Gesellschaft für Mikrotechnik und Sensorik}
(GMS~mbH), Villingen, konnten unterschiedliche
Sensorprototypen fr die Druck-, Kraft- und Strömungsmessung
realisiert werden, bei denen die numerischen FE-Berechnungen
wichtige Hinweise für die geometrische Auslegung liefern konnten.
Das Schwingungsverhalten von piezoelektrisch angetriebenen Siliziummembranen
wurde untersucht und die Druckempfindlichkeit charakterisiert.
Beim elektrothermisch angetriebenen Kraftsensor konnte durch eine
Änderung des Widerstandslayouts die Kraftempfindlichkeit erhöht und
gleichzeitig die thermischen Querempfindlichkeiten reduziert werden.
Im folgenden werden die verschiedenen FE-Modelleinflüsse diskutiert
und die Ergebnisse der numerischen Berechnungen im Vergleich mit
Messungen einer Fehlerbetrachtung unterworfen.


\section{Finite Elemente Modellbildung}
\label{femodellbildung}

Wie in Kapitel~3.5 beschrieben hängt die Genauigkeit der FE-Berechnungen
von verschiedenen Fehlereinflüssen ab, wie dem Diskretisierungsfehler,
den ungenauen Material- und Geometriedaten, sowie den numerischen Fehlern.
Um die verschiedenen Einflüsse bei der Berechnung von konkreten Bauelementen
zu ermitteln und die Größenordnung der einzelnen Fehlereinflüsse
gegeneinander abwägen zu können, wird das statische und dynamische
Verhalten von mikromechanischen Bauelementen untersucht. Hierzu werden die
Resonanzfrequenz der Grundbiegeschwingung und die druckabhängige Auslenkung
einer Siliziummembran numerisch berechnet und mit den analytischen Näherungen
verglichen.


\subsection{Analytische Abschätzungen}
\label{abschaetzungen}

Bei den Konvergenzbetrachtungen im folgenden Kapitel sollen für
verschiedene Elementtypen die analytischen Näherungslösungen
aus Kapitel~2 als Vergleichsbasis herangezogen werden.
In \cite{Fab92a, Sch92} sind gestützt durch FE-Berechnungen analytische
Beziehungen für die Membranmittenauslenkung und die
Resonanzfrequenzänderung in Abhängigkeit der Druckbeaufschlagung
abgeleitet worden. Die Zusammenhänge lassen sich wie folgt
zusammenfassen:
%
\begin{itemize}
\item Bei sehr kleinen Auslenkungen ($d/h \leq 0,14$) einer quadratischen
Membran mit der Seitenlänge $l$ kann von einem {\em linearen} Verhalten
zwischen
Membranmittenauslenkung $d$ und Druckbeaufschlagung $p$ ausgegangen werden.
Bei zunehmender Druckbelastung nehmen die Auslenkungen infolge
spannungsversteifender Effekte unterproportional zu, so daß ein
{\em nichtlinearer} Korrekturterm erforderlich wird:
\begin{equation}
\label{nonlin1}
 d = 0,00136 \frac{l^{4}p}{D} \left(1 - 0,00024 \frac{p}{p_{0}} \right)
\end{equation}
Diese Gleichung gilt für Auslenkungen $d/h \leq 1,3$ mit einer Abweichung
von etwa $\pm 6~\%$. Die Druckbeaufschlagung wird hierbei auf den
\glqq geometrischen Vergleichsdruck\grqq:
\begin{eqnarray}
\label{p0D}
 p_{0}   & := & \frac{Dh}{l^{4}} \nonumber
\end{eqnarray}
bezogen, so daß das Aspektverhältnis $h/l$ der Membran und die
Materialeigenschaften, berücksichtigt werden \cite{Pfe89}. Die
Biegesteifigkeit\footnote{In einigen Literaturstellen, z.B.\ \cite{Pon91},
ist
die Biegesteifigkeit durch $\hat D = \hat E/12$ definiert, so daß sie nur
materialabhängig ist. Für Silizium wurde meßtechnisch der Wert
15,5 $\le \hat D \le 17,5~$GPa ermittelt und fällt gegenüber dem
Literaturwert von 14,13~GPa unter Verwendung der Materialdaten aus
(\ref{simat}) etwas h”her aus.} der Platte ist durch
den Ausdruck:
\begin{eqnarray}
\label{biegesteif}
 D & := & \frac{1}{12} \hat E h^{3}
\end{eqnarray}
gegeben und besitzt fr eine 50~$\mu$m dicke Siliziummembran
den Wert $1,77$~$\cdot$~$10^{-3}$~Nm.
%
\item Bei großen Auslenkungen ($d/h \gg 1$) läßt sich aus Gleichung
(\ref{pvond}) für die druckabhängige Membranmittenauslenkung der
Zusammenhang:
\begin{eqnarray}
\label{memausl}
 \frac{d}{h} & = & c_{1} - \frac{0,886}{c_{1}}
\end{eqnarray}
ableiten, wobei:
\begin{eqnarray}
 c_{1}       & = & \sqrt [3] {\frac{p}{2c_{3}} + c_{2} }
\nonumber \\
 c_{2}       & = & \sqrt{0,6957 + {\left( \frac{p}{2c_{3}} \right)}^2}
\nonumber \\
 c_{3}       & = & 303,6672 \, p_{0}
%\left( \frac{Dh}{l^{4}} \right)
\nonumber
\end{eqnarray}
%
\item Die Resonanzfrequenzänderung einer schwingenden Membran, die eine
Auslenkung $d$ infolge einer homogenen Druckbeaufschlagung $p$ erfährt,
läßt sich unter Verwendung von Gleichung~(\ref{pvond}) ableiten. Für eine
quadratische, fest eingespannte Siliziummembran gilt:
\begin{eqnarray}
\label{fvondschroth}
  f(d) & = & f_{0} \sqrt{1 + c_{d} {\left( \frac{d}{h} \right)}^{2}}
\end{eqnarray}
wobei die Konstante $c_{d}$ von der Schwingungsmode der Membran abhängig
ist und für die Grundmode etwa 1,25 beträgt. Der Näherungsfehler liegt bei
etwa 5~\%. Für die höheren Schwingungsmoden\footnote{Siehe
Kapitel~\ref{membranresonatoren}.} $M_{13}$, $M_{31}$ und $M_{33}$
betragen die Konstanten $c_{d}(13)$ = $c_{d}(31)$ = 0,17 und
$c_{d}(33)$ = 0,06, so daß die Druckempfindlichkeit mit zunehmender
Schwingungsmode abnimmt \cite{Fab92a}.
%
\item Bei Vorhandensein einer homogenen, planaren Spannung $\sigma$ in der
Membran gilt für die Resonanzfrequenz \cite{Bou90}:
\begin{equation}
\label{fsigma}
  f(\sigma) = f_{0} \sqrt{1 + c_{\sigma}
                {\left( \frac{l}{h} \right)}^{2} \frac{\sigma}{\hat E}}
\end{equation}
wobei $f_{0} := f(\sigma = 0)$ die Frequenz der unverspannten Membran ist.
Die Konstante $c_{\sigma}$ ist von den Einspannbedingungen der Membran
abhängig und beträgt bei ideal starrer Einspannung 0,22 für die Grundmode.
Für Balkenstrukturen gilt der gleiche Zusammenhang mit der Konstante
$c_{\sigma}~\approx~$0,31 %\cite{Alb84}
(vgl. Gleichung~\ref{freqF}).
%
\item Die Approximation der Materialeigenschaften bei Bimorphstrukturen
erfolgt dadurch, daß sie in erster Näherung jeweils für Silizium
und die Dünnschicht (hier: Zinkoxid) isotrop angenommen
und mit Hilfe der Schichtdicken gewichtet werden:
\begin{eqnarray}
\label{matwich}
 \bar X & := & \frac{h_{Si}X_{Si} + h_{ZnO}X_{ZnO}}
                    {h_{Si} + h_{ZnO}}
\end{eqnarray}
Auf diese Weise lassen sich gemittelte Materialparameter
$X \, (= \hat E, \nu, \rho)$ in analytischen Abschätzungen verwenden.
\end{itemize}


\subsection{Konvergenzverhalten}
\label{konvergenzverhalten}

Als Referenzbauelement soll im weiteren eine ebene Siliziummembran mit einer
Kantenlänge von 10~mm und einer homogenen Dicke von 50~$\mu$m betrachtet
werden. Die Einspannung der Membran durch die ätzbegrenzenden
(111)-Siliziumebenen werden vernachlässigt und eine ideale, d.h.\
unendlich starre Randeinspannung angenommen. In erster Näherung wird von
isotropen Materialeigenschaften nach Gleichung (\ref{simat}) ausgegangen.
Die betrachteten FE-Modelle bilden unter
diesen Voraussetzungen das Verhalten einer ebenen Platte ab, so daß die
analytischen Näherungsberechnungen zum Vergleich herangezogen werden können.
Aufgrund des hohen Aspektverhältnisses
(Kantenlänge/Membrandicke: $l/h$~=~200) der mikromechanischen
Siliziummembran können in guter Näherung Schalenelemente
(bei {\sf ANSYS}: {\em SHELL43})
mit linearen Ansatzfunktionen und jeweils sechs Freiheitsgraden pro Knoten,
drei Verschiebungen und drei Rotationen, verwendet werden.
Zusätzlich wird das numerische Verhalten von dreidimensionalen
Volumenelementen mit linearen (bei {\sf ANSYS}: {\em SOLID45})
und quadratischen Ansatzfunktionen (bei {\sf ANSYS}: {\em SOLID95}),
die nur Verschiebungsfreiheitsgrade
aufweisen, untersucht. Bei Membranen mit geringerem Aspektverhältnis
($l/h \leq 50$) wird der Einsatz von Schalenelementen zunehmend kritischer,
da die Dicke nur formal über eine dem Element zugrundeliegende interne
mathematische Formulierung mitberücksichtigt wird, so daß auf
Volumenelemente ausgewichen werden sollte.\\
%----------------------- Beginn: Figure-Environment ----------------------
\begin{figure}[htb]
\begin{center}
% --- Dateiname des Bildes
\input{abbve.tex}
\setabbve
\end{center}
\caption{\label{abbkonvfreq}
  Konvergenzverhalten der Resonanzfrequenz}
\end{figure}
%----------------------- Ende: Figure-Environment ----------------------
In {\bf Abbildung \ref{abbkonvfreq}} ist die Abhängigkeit der
Resonanzfrequenz von der Knotenanzahl\footnote{Beim Vergleich von Schalen-
mit Volumenelementen und von Elementen mit linearen und quadratischen
Ansatzfunktionen ist es sinnvoller die Knotenanzahl als die Elementweite
des FE-Modells als
Vergleichsbasis zu wählen, da bei gleicher lateraler Elementdimension
der Rechenaufwand mit der Anzahl der Freiheitsgrade, d.h.\ Knotenanzahl,
steigt.} der FE-Modelle im logarithmischen Maßstab graphisch
dargestellt. Zum Vergleich ist der analytische Wert der
Grundresonanzfrequenz von 7,056~kHz, berechnet nach Gleichung
(\ref{memfreq}), gestrichelt eingezeichnet. Die Knotenanzahl
der FE-Modelle wurde jeweils um drei Größenordnungen variiert.
Bei Reduktion der Elementabmessungen in lateraler Richtung entlang der
Membrankante, was einer gleichzeitigen Erhöhung der Anzahl der
Freiheitsgrade entspricht, wird die Steifigkeit ($K$) des FE-Modells
exakter beschrieben, so daá sich das FE-Modell weniger \glqq steif\grqq
\, verhält \cite{Zie84}. Aufgrund der verminderten Eigenwerte der
Steifigkeitsmatrix konvergiert die Resonanzfrequenz ($\omega = \sqrt{K/M}$)
von oben gegen den analytischen Wert.
Beim Schalenmodell (FE-Modell: S43) reichen bereits 900 Elemente aus,
so daá die Resonanzfrequenz weniger als 0,1~\% vom analytischen
Referenzwert abweicht. Bei den Volumenmodellen konvergieren die
Resonanzfrequenzen mindestens eine Größenordnung schlechter gegen
den analytischen Referenzwert. Im Vergleich zu Elementen mit
linearen Ansatzfunktionen (FE-Modell: V45) weisen quadratische
Ansatzfunktionen (FE-Modell: V95) wie erwartet ein verbessertes
Konvergenzverhalten auf.
Zur Bestimmung der Frequenzeigenwerte wurde die reduzierte
{\sl Householder}-Methode eingesetzt \cite{Koh92}.\\
In {\bf Tabelle \ref{tabdiskfreq}}
ist der Einfluß der Diskretisierung und der dynamischen Hauptfreiheitsgrade
(MDOF) auf die Berechnung der Resonanzfrequenz zusammengefaßt.
%----------------------- Beginn: table ---------------------------
\begin{table}[htb]
\caption{\label{tabdiskfreq}
 Einfluß der Modellparameter auf die Resonanzfrequenz beim Schalenmodell}
\begin{center}
\begin{tabular} {|c||c|c|c|c||c|c|}
\hline
Modell & \multicolumn{4}{c||}{Modellparameter}
       & \multicolumn{2}{c|}{Resonanzfrequenz} \\
\hline
S43 & h [mm] & Elemente & Knoten & MDOF & f [kHz] & Abw. \\
\hline \hline
1  & 3,33 &    9  &    12 &  24 & 8,365 & 18,55 \% \\
2  & 1,66 &   36  &    49 &  50 & 7,198 &  2,01 \% \\
3  & 1,00 &  100  &   121 & 100 & 7,098 &  0,60 \% \\
4  & 0,59 &  289  &   324 & 100 & 7,072 &  0,23 \% \\
5  & 0,33 &  900  &   961 & 100 & 7,060 &  0,06 \% \\
6  & 0,20 & 2500  &  2601 & 300 & 7,056 &  --- \\
\hline
\end{tabular}\\
\end{center}
\end{table}
%----------------------- Ende: table ---------------------------
Der Parameter $h$ bezeichnet die Elementabmessung der
äquidistanten Vernetzung.
Eine Vergleichsrechnung bei der Modellvariante~5 des Schalenmodells S43
ergab bei 900 Elementen und 961 Knoten unter Verwendung des
{\sl Subspace}-Iterationsverfahrens
eine um 4~Hz verminderte Resonanzfrequenz. Dieses entspricht dem Ergebnis
der FE-Modellvariante~6, die etwa die dreifache Element- bzw.\ Knotenanzahl
aufweist. Der Einsatz des iterativen JCG-Gleichungslösers
({\em \underline{J}acobian \underline{C}onjugate \underline{G}radient})
lieferte die
gleiche Ergebnisgenauigkeit, jedoch auf Kosten einer sechsfach höheren
Rechenzeit. Daher ist der Einsatz nur bei sehr großen Modellen sinnvoll,
bei denen das direkte {\sl Wavefront}-Lösungsverfahren die Kapazität des
Hauptspeichers überschreiten würde.\\
Die druckabhängige Membranauslenkung wurde unter Berücksichtigung der
geometrischen Nichtlinearitäten\footnote{Berücksichtigung von
'Large Deflection' und 'Stress Stiffening'.} nach
Kapitel~\ref{nlstatberechnungen} mit dem {\sl Newton-Raphson}-Verfahren
berechnet. Als Konvergenzkriterium wurde die Norm des Residuenkraftvektors,
die rechte Seite der Gleichung (\ref{largedefl}), verwendet.
Bei ungenügender Elementunterteilung fallen die Membranauslenkungen des
FE-Modells aufgrund der zu hohen Modellsteifigkeit zu niedrig aus.
Im Gegensatz zu Elementen, die nur Verschiebungsfreiheitsgrade
(Translational DOF) besitzen, beispielsweise Volumenelemente, weisen
Schalenelemente zusätzliche Rotationsfreiheitsgrade (Rotational DOF) auf.
Infolge der erlaubten Knotenrotationen können sich die Schalenelemente
bei sehr grober Vernetzung aus ihrer Ebene heraus verwölben, so daß die
Membranauslenkung höher ausfällt. Dieses führt dazu, daß bei zunehmender
Ele\-mentanzahl sich die Einflüsse der Verschiebungs- und
Rotationsfreiheitsgrade überlagern und ein
Konvergenzverhalten\footnote{Der
Einfluß von Schalen- und Volumenelementen auf das
statische Verhalten von mikromechanischen Drucksensoren in Bezug auf die
auftretenden Auslenkungen und den Ort der maximalen Spannungen in
Abhängigkeit des Längen-/Dickenverhältnisses ist in \cite{Tol92} untersucht
worden. Die Ergebnisse bestätigen das Konvergenzverhalten der durchgeführten
Schalenberechnungen.}
von oben an den {\em wahren} Wert erfolgt. Die Ergebnisse der Schalenmodelle
mit den geringen Elementzahlen (3x3 bzw.\ 5x5 Elemente) lassen keine
quantitativen Aussagen zu, da die Biegelinie der Membran mit einer so
geringen Seitenunterteilung nur unzureichend abgebildet wird.
%\clearpage
%----------------------- Beginn: Figure-Environment ----------------------
\begin{figure}[htb]
\begin{center}
% --- Dateiname des Bildes
\input{abbvze.tex}
\setabbvze
\end{center}
\caption{\label{abbkonvausl}
 Konvergenzverhalten der Membranmittenauslenkung}
\end{figure}
%----------------------- Ende: Figure-Environment ----------------------
In {\bf Abbildung \ref{abbkonvausl}} ist die Abhängigkeit der
Membranmittenauslenkung von der
Knotenanzahl fr das Schalenmodell S43 und die Volumenmodelle
V45 und V95 graphisch dargestellt. Das Volumenmodell V45 weist
aufgrund der linearen Ansatzfunktionen zwei Elementlagen ber
die Membrandicke auf. Als Vergleich wurde die Membranmittenauslenkung
nach Gleichung~(\ref{memausl}) analytisch bei einer Druckbeaufschlagung
von 500~mbar (= 0,5$\cdot 10^{5}$~Pa) zu etwa 116~$\mu$m berechnet und
gestrichelt eingezeichnet. Auch hier zeichnet sich das Schalenmodell
durch sehr gute Konvergenzeigenschaften aus. Bei etwa 300 Elementen wird
bereits ein stabiler Wert für die Membranmittenauslenkung
erreicht. Im Gegensatz dazu nähern sich die Volumenmodelle
aufgrund der generell zu \glqq hohen\grqq \, Steifigkeit von
unten an. Bei genügend feiner Vernetzung konvergieren alle numerischen
Berechnungen an eine Auslenkung von etwa 107~$\mu$m. Die Abweichung
zum analytischen Referenzwert beträgt rund 9~\%, so daß die abgeleitete
analytische Gleichung nur näherungsweise zur Beschreibung herangezogen
werden kann.\\
%
In {\bf Tabelle \ref{tabfreqzus}} sind die Ergebnisse der numerischen
Resultate abschlieáend zusammengefaßt.
%----------------------- Beginn: table ---------------------------
\begin{table}[htb]
\caption{\label{tabfreqzus}
 Zusammenfassung der Modellparameter und numerischen Resultate}
\begin{center}
\begin{tabular} {|c||c|c|c||c|c|}
\hline
FE-Modell & h [$\mu$m] & Elemente & Knoten & f [Hz] & $u_{z}$ [$\mu$m] \\
\hline \hline
S43  & 200 &  2500  & 2601  & 7055,6  & 106,4 \\
V45  & 200 &  5000  & 7803  & 7068,2  & 106,8 \\
V95  & 330 &  900   & 6603  & 7066,4  & 106,6 \\
\hline
V64  & 200 &  2500  & 5202  & 6798,4  & 108,0 \\
\hline
\end{tabular}\\
\end{center}
\end{table}
%----------------------- Ende: table ---------------------------
Zum Vergleich ist eine Berechnung (FE-Modell: V64, eine Elementlage)
mit anisotropen Materialdaten \cite{LB82}
durchgeführt worden. Die Abweichungen betragen etwa 4~\% bei der
Resonanzfrequenz und etwa 1~\% bei der Auslenkung gegenüber der
Beschreibung mit isotropem Materialverhalten.
Bei den Volumenmodellen wurde
zusätzlich der Einfluß der Elementierung über die Membrandicke
untersucht. Bei einer Verdreifachung der Elementlagen resultierte
eine Frequenzerniedrigung von lediglich 0,1~\%. Dieses ist dadurch zu
erklären, daß ausschließlich die laterale Elementierung ausschlaggebend
ist. Die Unterteilung über die Membrandicke hat nur einen Einfluß, falls
sich der Spannungszustand über die Dicke ändert. In Kapitel~4.3.2
werden die FE-Modelleinflüsse bei der druckabhängigen Membranauslenkung
und Eigenfrequenzänderung im Vergleich mit einem vermessenen
Silizium-Drucksensor diskutiert.


\subsection{Modelleinflüsse}
\label{modelleinfluesse}

Verschiedene FE-Modelle sind im Rahmen einer Diplomarbeit \cite{Sch92}
fr die Modellierung von resonanten Silizium-Drucksensoren entwickelt
und der Einfluß der Geometrie- und der Materialeigenschaften unter
verschiedenen Einspannbedingungen untersucht worden. Es wurden sowohl
zweidimensionale FE-Modelle mit isotropen Materialeigenschaften, als auch
dreidimensionale Modelle unter der Berücksichtigung der
Elastizitätsanisotropie des Siliziums und der realen
Membraneinspannverhältnisse simuliert. Zusätzlich wurden Elementtypen
und Symmetrieeigenschaften der FE-Modelle variiert und die analytischen
Näherungsgleichungen (\ref{memausl}) und (\ref{fvondschroth})
von den numerischen Berechnungsergebnissen abgeleitet. Als Referenz
wurde ein dreidimensionales FE-Modell entwickelt, das die Anisotropie
der Elastizitätseigenschaften und die schräge Randeinspannung durch
die um $54,74^{\circ}$ geneigten (111)-Siliziumebenen
berücksichtigt. Die wichtigsten Ergebnisse bei der Berechnung der
Grundresonanzfrequenz lassen sich wie folgt zusammenfassen:
% ------------------------ Beginn : Liste --------------------------------
\begin{itemize}
\item
Aufgrund der schrägen Randeinspannung der Siliziummembran durch die
(111)-Ebenen verringert sich die Resonanzfrequenz infolge verminderter
Einspannungssteifigkeit.
%Der Unterschied zwischen einer ideal starren
%Einspannung bei einem Schalenmodell und der endlich steifen Bulkeinspannung
%beträgt etwa 2,3~\%.
\item
Die Berücksichtigung der anisotropen Materialeigenschaften hat eine
weitere Frequenzerniedrigung %von etwa 3,1~\%
zur Folge.
\item
Die Reihenfolge der Schwingungsmoden und Frequenzeigenwerte ist abhängig
von den Symmetriebedingungen (Viertel-, Halb- oder Vollmembran).
Insbesondere lassen sich bei symmetrischen Randbedingungen keine
antisymmetrischen Schwingungsformen ermitteln und umgekehrt.
\end{itemize}
% ------------------------ Ende : Liste ----------------------------------
Im ungünstigsten Fall überlagern sich obige Einflüsse und es ergibt sich
zwischen dem genauen Referenzmodell (3D, anisotrop, schräge Einspannung)
und der analytischen Näherungsberechnung nach Gleichung (\ref{memfreq})
eine maximale Abweichung von bis zu 10~\%, wobei der Restfehler
auf eine ungenügende Vernetzungsdichte zurückgeführt werden kann. Bei der
Berechnung der Eigenfrequenzen höherer Schwingungsmoden
nimmt die Abweichung erheblich zu,
da eine erhöhte Anzahl von Freiheitsgraden (DOF) in Verbindung mit
einer feineren Elementunterteilung erforderlich ist, um Schwingungsmoden
mit mehreren Schwingungsknoten und -bäuchen \glqq räumlich\grqq \,
abzutasten\footnote{In Analogie zum {\sl Shannon}schen Abtasttheorem
der Nachrichtentechnik kann dieses als \glqq geometrisches
Abtasttheorem\grqq \, bezeichnet werden.}.\\
%
Um den Einfluß der anisotropen Materialeigenschaften auf das
Schwingungsverhalten weiterer mikromechanischer Strukturen zu
untersuchen, wurden die Resonanzfrequenzen der Grundbiegeschwingungsmode
der drei Grundstrukturen aus Abbildung~\ref{abbgrundgeometrien}
berechnet. {\bf Tabelle~\ref{tabgeomvergleich}} faßt die
numerisch berechneten Resonanzfrequenzen im Vergleich mit den analytischen
Werten aus Tabelle~\ref{tabcf} zusammen.
%----------------------- Beginn: table ---------------------------
\begin{table}[htb]
\caption{\label{tabgeomvergleich}
 Einfluß der Materialanisotropie auf die Resonanzfrequenz bei
 unterschiedlicher Resonatorgeometrie}
\begin{center}
\begin{tabular}{|l||c|c|c||l|}
\hline
%Resonanzfrequenz
 & Zunge & Balken & Membran & Bemerkung \\
\hline \hline
Länge [mm]:            & 5     & 10 & 10 & Bauelementdicke:\\
Breite [mm]:           & 1     & 1  & 10 & 50 $\mu$m\\
\hline
 $f_{analytisch}$ [Hz] &  2765 &  4386  & 7056 & isotrop (\ref{simat}) \\
\hline
 Elemente: &  365  & 730  &  900 & lineare Ansatz- \\
 Knoten:   &  690  & 1368 & 1922 & funktionen \\
\hline
$f_{isotrop}$ [Hz] & 2731,6  & 4349,7  &  7137,1  & $E_{110}$, $\nu_{110}$
(\ref{simat}) \\
$f_{anisotrop}$ [Hz] & 2430,5 & 3869,3 &  6796,7  & (100)--orientiert \\
$f_{anisotrop}$ [Hz] & 2732,4 & 4350,9 &  6869,9  & (110)--orientiert \\
\hline
Abweichung:
		& 1,22~\%  & 0,83~\%  &  0,17~\%  & analytisch/isotrop\\
                & 0,029~\% & 0,028~\% &  3,89~\%  & iso-/anisotrop(110)\\
		& 11,0~\%  & 11,0~\%  &  1,10~\%   & aniso(100)/aniso(110)\\
\hline
\end{tabular}
\end{center}
\end{table}
%----------------------- Ende: table ---------------------------
Aufgrund der Ausrichtung der Siliziumbauelemente parallel
zum Waferflat, der entlang der (110)-Siliziumebene orientiert ist, war es
erforderlich, das Elementkoordinatensystem um $45^{\circ}$
gegenüber dem globalen Koordinatensystem des FE-Modells zu verdrehen.
Bei Berücksichtigung der Anisotropie weichen die Resonanzfrequenzen der
einseitig eingespannten Zunge und beim doppelseitig eingespannten Balken
gegenüber der isotropen Beschreibungsweise lediglich um 1~Hz ab, weil
die effektiven Materialparameter (\ref{simat}) für die (110)-Richtung im
Silizium bereits sehr gute Ersatzwerte für die anisotrope Formulierung
darstellen. Für Membranstrukturen beträgt die Abweichung zwischen
isotroper und anisotroper Beschreibungsweise bei (110)-Orientierung
weniger als 4~\%.\\
Die Abweichungen bei anderen Kristallrichtungen im Silizium, beispielsweise
bei (100)-orientierten Bauelementen (siehe Tabelle~\ref{tabgeomvergleich}),
fallen größer aus. Insbesondere bei den Balken- und Zungenstrukturen hat die
Orientierung einen erheblichen Einfluß im Vergleich zu Membranstrukturen.
Während bei Membranen die Frequenzwerte nur um etwa 1~\% differieren,
beträgt der Unterschied bei (100)- und (110)-orientierten Balkenstrukturen
11~\%. Die Ursache hierfür liegt in der starken Variation der
Querkontraktion in der (100)-Kristallebene von 0,064 bis 0,279 \cite{Heu89}
und dem Umstand, daá die Membranen allseitig eingespannt sind und dadurch
der Effekt vermindert wird. Bei der Berechnung von Siliziumstrukturen
in isotroper Näherung sind daher jeweils zuerst geeignete isotrope
Ersatzmaterialdaten zu bestimmen.



\subsection{Dämpfungseinflsse}
\label{daempfungseinfluesse}

Bei der Berechnung der Eigenfrequenzen mikromechanischer Strukturen
mit Hilfe der FE-Methode wird die Eigenwertgleichung~(\ref{modal})
unter Vernachlässigung von Dämpfungseffekten numerisch gelöst. Auf dieser
Grundlage ist in Kapitel~\ref{konvergenzverhalten} das Konvergenzverhalten
der Eigenfrequenzen untersucht und in Kapitel~\ref{modelleinfluesse} sind
weitere verschiedene Modelleinflüsse betrachtet worden. Bei der
Charakterisierung realer mikromechanischer Resonatoren haben Dämpfungs- und
Anregungseffekte einen Einfluß auf die Eigenfrequenzen, so daß diese
Effekte hier abgeschätzt werden sollen.\\
%
Wie in Kapitel~3.4.2 dargelegt überlagern sich verschiedene
Dämpfungsmechanismen bei dynamischen Vorgängen und es lassen sich den
einzelnen Dämpfungsbeiträgen wie innere Strukturdämpfung, äußere
Fluiddämpfung, Dämpfung durch zusätzliche Schichten jeweils
Einzelschwingungsgten $Q_{i}$ zuordnen, so daß für die effektive
Gesamtschwingungsgte $Q_{ges}$ gilt \cite{Til92}:
\begin{equation}
\label{qsum}
 \frac{1}{Q_{ges}} = \sum_{i} \frac{1}{Q_{i}}
\end{equation}
Aus diesem Zusammenhang ist ersichtlich, daß der dominierende Dämpfungsterm
die geringste Einzelgüte zur Folge hat und dementsprechend die anderen
Dämpfungseffekte überdeckt.\\
%
Bei der experimentellen Bestimmung der mechanischen Schwingungsgüte eines
Mikroresonators wird immer die Gesamtgüte $Q_{ges}$ gemessen, die auf
verschiedene Weisen ermittelt werden kann. Dieses kann einerseits durch die
Vermessung der Halbwertsbreite (3dB-Amplitudenabfall) der frequenzabhängigen
Amplitudenkurve $A(\omega)$, andererseits durch die Amplitudenüberhöhung,
d.h.\ dem Verhältnis der Resonanzamplitude $A_{res} := A(\omega_{res})$
zur statischen Amplitude $A_{stat} := A(\omega=0)$ erfolgen
\cite{VIB}:
\begin{equation}
\label{qmess}
 Q = \frac{\omega_{res}}{\Delta \omega_{-3dB}}
    \approx \frac{\pi^2}{8} \frac{A_{res}}{A_{stat}}
\end{equation}
%
In der Literatur werden im wesentlichen drei Bereiche unterschieden, bei
denen unterschiedliche Dämpfungsmechanismen bei mikromechanischen Strukturen
dominieren (z.B.\ \cite{Blo92}). Im Niederdruckbereich ($p \leq 1$ Pa)
ist die äußere Fluiddämpfung vernachlässigbar und es sind nur innere
Strukturdämpfungseffekte vorhanden, so daß die Schwingungsgüte und die
Resonanzfrequenz ihre maximalen Werte erreichen.
Im Molekularbereich bzw.\ {\sl Knudsen}bereich ($p=1...10^2$ Pa)
wird die Dämpfung durch
Impulsübertrag infolge Moleklkollisionen mit der schwingenden
Resonatoroberfläche verursacht. In diesem Bereich erhöht sich die Dämpfung
mit steigendem Druck $p$, so daá $Q \sim 1/\sqrt{p}$ für die Schwigungsgüte
gilt \cite{Blo92}. Mit zunehmendem Druck nimmt die freie Weglänge der
Moleküle ab und sie wechselwirken miteinander, so daß zusätzliche Kräfte
durch den Luftwiderstand wirksam werden. In diesem sogenannten viskosen
Dämpfungsbereich ($p=10^3...10^5$ Pa) werden die Strömungsverhältnisse
komplexer und durch die {\sl Navier-Stokes}-Gleichungen beschrieben. Bei
Atmosphärendruck ($p \approx 10^5$ Pa) bilden sich turbulente
Grenzschichten an der Resonatoroberfläche aus, während für geringere Drücke
laminare Strömung vorliegt. Der Übergang vom turbulent zum laminar
dominierten Dämpfungsbereich ist von der Resonatorgeometrie anhängig.
Für Balkenresonatoren aus Quarz und Silizium sind die Abhängigkeiten
der Schwingungsgüte und der Resonanzfrequenz vom Druck durch
{\sl Blom et al.} \cite{Blo92} und {\sl Christen} \cite{Chr83}
theoretisch und experimentell untersucht worden. Die relative
Frequenzänderung $\Delta f / f$ von Balkenresonatoren beträgt infolge
Druckschwankungen im viskosen Druckbereich etwa 0,3--1,1~ppm/mbar
\cite{Bus94, Chr83} und ist abhängig vom Dichteverhältnis des
Resonatormaterials und des umgebenden Gases.\\
Infolge der Einspannbedingungen können die Schwingungsgüten ebenfalls
erheblich beeinflußt werden. Für Membranresonatoren aus Silizium wurden
bei Drücken von $p \approx 10$ Pa Güten bei {\em starrer} Einspannung von
etwa 900 gemessen, die sich bei {\em flexibler} Einspannung der Membranen
auf etwa 150 erniedrigten \cite{Pra91}.\\
Bei den im weiteren betrachteten Resonatoren, die bei Atmosphärendruck
betrieben werden, ist die Luftdämpfung der dominierende Dämpfungsmechanismus.
Infolge der endlichen Schwingungsgte $Q$ ergibt sich eine geringfügige
Erniedrigung der Vakuumeigenfrequenz $f_{0}$ infolge Dämpfung \cite{Pet79}:
\begin{equation}
\label{qpetersen}
 f_{res} = f_{0} \sqrt{1 - \frac{1}{4Q^{2}} }
\end{equation}
Da die typischen Schwingungsgüten der in dieser Arbeit untersuchten
mikromechanischen Membran- und Balkenresonatoren bei Atmosphärendruck
im Bereich von 50--1000 liegen, läßt sich der Dämpfungskorrekturfaktor
bei der Betrachtung der Resonanzfrequenz vernachlässigen, d.h.\
$f_{res} \approx f_{0}$.  Wie in Kapitel~4.5 gezeigt wird,
sind die herstellungstechnologiebedingten Geometrietoleranzen und die
inneren Schichtspannungen die wesentlichen Unsicherheitsfaktoren bei der
meßtechnischen Bestimmung der Resonanzfrequenzen.\\
Die Schwingungsgte eines Resonators hat ebenfalls einen Einfluß auf die
erzielbare Frequenz- und damit Meßgrößenauflösung bei einem resonanten
Sensor und ist durch:
\begin{eqnarray}
\label{Qaufl}
 \frac{\Delta f}{f} & \sim & \frac{1}{Q} \left( \frac{\Delta A}{A} \right)
\end{eqnarray}
gegeben \cite{Til93}. Hierbei ist $\Delta A$ die minimale
Amplitudenauflösung des eingesetzten Detektionsmechanismus.
In Kapitel~\ref{dynamischemomentenkompensation} wird auf die Erhöhung der
Schwingungsgüte eines gekoppelten Balkenschwingers eingegangen, der die
dynamische Momentenkompensation einer antisymmetrischen Schwingungsmode
ausnutzt.\\
Ein weiterer Einfluá auf die Resonanzfrequenz wird durch nichtlineare
dynamische Effekte hervorgerufen, die infolge einer Erhöhung der
Anregungsamplitude zu einer Zunahme der Resonanzfrequenz führen \cite{Eis64}:
\begin{equation}
\label{nlampli}
 f(A_{max}) = f_{0} \sqrt{1 + \beta \left( \frac{A_{max}}{h} \right)^2 }
\end{equation}
Mit zunehmender Anregungsamplitude wird der frequenzabhängige
Amplitudenverlauf $A(\omega)$ asymmetrisch und es können sogar
Resonatorinstabilitäten und eine damit verbundene Hysterese auftreten
\cite{And87, Zoo92}. Dieser Effekt konnte bei den in dieser Arbeit
vermessenen Strukturen nur bei sehr großen Schwingungsamplituden
($A_{max} \approx h$) nachgewiesen werden. Unter normalen
Betriebsbedingungen bei resonanten Sensoren ist jedoch $A_{max} \ll h$,
so daß die Frequenzerhöhung zu vernachlässigen ist.


\newpage
\section{Eigenfrequenzen und Schwingungsformen}
\label{eigenfrequenzenundschwingungsformen}

In Zusammenarbeit mit der {\em GMS~mbH} wurden resonante Kraft- und
Strömungssenoren auf der Basis von elektrothermisch angetriebenen
Silizium-Balkenresonatoren realisiert. Die Entwicklung der Prozeßtechnologien
zur Herstellung der Balkenresonatoren wurde von {\em GMS} durchgeführt,
während der Entwurf und die Simulation, sowie die meßtechnische
Charakterisierung der Sensoren im Rahmen dieser Arbeit erfolgten.


\subsection{Balkenresonatoren}
\label{balkenresonatoren}

Um das dynamische Verhalten von Balken- und Membranresonatoren
zu untersuchen, wurden reine Siliziumstrukturen ohne Dünnschichtsystem
mit Hilfe von extern angebrachten Piezokeramiken über Körperschall
akustisch zum Schwingen angeregt. Die hierfür eingesetzten experimentellen
Meßaufbauten und verwendeten Meßmethoden sind im Anhang beschrieben.\\
%
{\bf Tabelle~\ref{tabbalkenfreq}} faßt die gemessenen Resonanzfrequenzen
von etwa 50~$\mu$m dicken Silizium-Balkenresonatoren,
deren Strukturgeometrie in Abbildung~\ref{abbgmssensor} dargestellt ist,
im Vergleich mit FE-Berechnungen zusammen. Die mit $f_{Z1}$ normierten
Frequenzen sind in Klammern angegeben. Die Modellierung erfolgte
dreidimensional unter Verwendung anisotroper Materialkennwerte \cite{LB82}
und Berücksichtigung der schrägen Balkeneinspannung durch die
(111)-Siliziumebenen. Die ersten beiden Oberschwingungen aus der
Balkenebene heraus sind mit Z2 und Z3 gekennzeichnet und besitzen zwei
bzw.\ drei Schwingungsbäuche.
%----------------------- Beginn: Tabelle ---------------------------
\begin{table}[htb]
\caption{\label{tabbalkenfreq}
 Resonanzfrequenzen von Silizium-Biegebalken bei Fremdanregung
 (Vergleich: FE-Berechnungen -- Messungen)}
\begin{center}
\begin{tabular} {|c||c|c|c||c|c|c|}
\cline{1-7}
   & \multicolumn{3}{c||}{Länge = 10 mm} & \multicolumn{3}{c|}{Länge = 8 mm} \\
\hline
Mode  & FEM  & Exp. & Abw. & FEM & Exp. & Abw. \\
\hline \hline
Z1 & 4,338 kHz  &  4,332 kHz & 0,1~\% &  6,763 kHz & 7,185 kHz  & 5,9~\% \\
   & (1,000)    &  (1,000)   &       &  (1,000)   & (1,000)    & \\
\hline
Z2 & 11,959 kHz & 11,390 kHz & 5,0~\% & 18,646 kHz & 18,718 kHz & 0,4~\% \\
   & (2,757)    &  (2,629)   &       &  (2,757)   & (2,605)    & \\
\hline
Z3 & 23,457 kHz & 24,893 kHz & 5,8~\% & 36,582 kHz & 36,440 kHz & 0,4~\% \\
   & (5,407)    &  (5,746)   &       &  (5,409)   & (5,072)    & \\
\hline
\end{tabular}\\
\end{center}
\end{table}
%----------------------- Ende: Tabelle ---------------------------
Die Bestimmung der Resonanzfrequenzen erfolgte in einem
schmalen Frequenzbereich, so daá der Meßfehler bei der Frequenzbestimmung
etwa 5--10~Hz beträgt. Die gemessenen Frequenzwerte variieren
allerdings aufgrund der Bauelementestreuung um 5--10~\%. Das entspricht
bei den Grundmoden etwa $\pm200$~Hz, so daß
in Tabelle~\ref{tabbalkenfreq} die Mittelwerte der Messungen angegeben sind.
Die groáe Bauelementestreuung ist im wesentlichen auf den beidseitigen
Naßätzprozeß zurückzuführen, der zu nicht genau reproduzierbaren
Balkendicken führt \cite{ABV93}. Die Abweichungen der numerischen
Resultate von den gemessenen Werten betragen bis zu 6~\%. Sie lassen sich
einerseits durch die Geometrietoleranzen, andererseits durch die innere
Vorspannung, die durch die Siliziumoxidschicht bewirkt wird und für
etwa 1~$\mu$m dnne Schichten einige MPa betragen kann \cite{Mur92},
erklären. Die Schwingungsgüten variierten in Abhängigkeit der
Sensoreinspannung stark und betrugen für die Grundmoden $400\pm100$, für die
beiden Oberschwingungen Z2 und Z3 etwa $900\pm200$ und $1200\pm300$.
Die analytische Berechnung der Grundresonanzfrequenz $f_{Z1}$
nach Gleichung~(\ref{balkfreq}) unter Verwendung isotroper
Materialeigenschaften (Gleichung~\ref{simat}) ergibt für
unverspannte Siliziumbalken bei 10 und 8~mm Resonatorlänge
Resonanzfrequenzen von 4,386~kHz und 6,853~kHz in guter Übereinstimmung
mit den numerisch berechneten Ergebnissen, so daß die Abweichungen nur etwa
1~\% betragen (siehe Tabelle~\ref{tabbalkenfreq}).
%----------------------- Beginn: Figure-Environment ----------------------
\begin{figure}[htb]
\begin{center}
% --- Dateiname des Bildes
\input{abbvda.tex}
\setabbvda
\end{center}
\caption{\label{abbbalkenmoden}
 Eigenschwingungsformen eines Silizium-Biegebalkens}
\end{figure}
%----------------------- Ende: Figure-Environment ----------------------
In {\bf Abbildung~\ref{abbbalkenmoden}} sind die experimentell ermittelten
Eigenschwingungsformen (Meápunkte) eines 10~mm langen Balkens dargestellt,
der elektrothermisch mit einer niedrigen mittleren Impulsheizleistung
($\overline{P_{Heiz}}$~=~50~mW) angeregt und optisch abgetastet
wurde \cite{Mue92}. Aufgrund der jeweils an beiden Balkenenden symmetrisch
angeordneten Heizwiderstände, wurden nur die symmetrischen Schwingungsmoden
Z1 und Z3 elektrothermisch selektiv angeregt. Die unsymmetrischen
Biegeschwingungsmoden Z2 und Z4,
sowie die Torsionsmoden konnten im Modenspektrum nicht nachgewiesen
werden. Die Resonanzfrequenzen\footnote{Auch mit Metallschichtsystem
stimmen die Resonanzfrequenzen der Biegebalken mit denen der unbeschichteten
Siliziumbalken hinreichend gut überein, sofern die elektrische
Verlustleistung klein bleibt und die Temperaturüberhöhung auf dem Balken
wenige Kelvin nicht überschreitet (siehe Kapitel~\ref{temperaturverhalten}).}
der beiden vermessenen Moden betrugen etwa $f_{Z1} = 4,23$~kHz und
$f_{Z3} = 24,1$~kHz.\\
Weil die rechnerische Modalanalyse
die Eigenwertgleichung (\ref{modal}) unter Vernachlässigung von
Dämpfungsbeiträgen und Anregungskraft läst, können nur relative
Schwingungsamplituden ermittelt werden. Um die berechneten
Amplitudenverläufe (durchgehende Kurvenzüge) mit den Meßwerten vergleichen
zu können, wurden die gemessenen Maximalwerte auf eins normiert.
Beim optischen Nachweis steht im Gegensatz zur elektrischen Detektion
die Phaseninformation zwischen
Anregungsspannung und resultierender Schwingungsamplitude nicht zur
Verfügung, so daá nur die Absolutbeträge der Schwingungsamplituden
detektierbar sind. Die mittleren Meßpunkte der Obermode Z3 mußten daher
mit negativen Vorzeichen versehen werden.
Da die Nullpunktjustierung der Balkenkoordinate des FE-Modells nicht
eindeutig ist, wurden ferner die gemessenen Amplitudenmaxima der beiden
Biegeschwingungen Z1 und Z3 in Balkenmitte des FE-Modells gelegt.\\
%
Die Übereinstimmung der berechneten mit den gemessenen Werten ist sehr gut
und läßt auf eine genügende Elementunterteilung und Erfassung der
richtigen Einspannbedingungen des Balken zurückschließen. Aufgrund der
niedrigen Eigenspannungen (einige MPa) der $SiO_{2}$--Schicht werden die
Eigenformen in ihrem Verlauf nicht beeinflußt. Im Gegensatz dazu zeigten
$ZnO$--beschichtete Siliziummembranen (siehe Kapitel~5.3.2) sehr hohe innere
Spannungen, so daß eine merkliche Abweichung des lateralen Modenverlaufs
zwischen Messung und Simulation auftrat \cite{Fab92b}. Eine hohe Vorspannung
im Resonator bewirkt außer einer
Frequenzverschiebung einen steiler ansteigenden lateralen Modenverlauf bei
den Einspannstellen gegenüber unverspannten Resonatoren.
Quantitative Untersuchungen zum Einfluß von
Schichtspannungen auf den Verlauf der lateralen Eigenformen sind von
{\sl Prak} für Balkenresonatoren durchgeführt worden \cite{Pra93}. Die
Beeinflussung des Auslenkungsverhaltens ebener Membranen infolge innerer
Spannungen wurde von {\sl Steinmann et al.} analytisch und numerisch
untersucht \cite{Ste93}.

