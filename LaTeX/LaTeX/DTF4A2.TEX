\subsection{Membranresonatoren}
\label{membranresonatoren}

Im Vergleich zu Balkenresonatoren, deren Schwingungsverhalten eindimensional
beschrieben werden kann (Gleichung \ref{balkdgl}), erfordern membranartige
Resonatorstrukturen eine zwei- oder dreidimensionale Beschreibung.
Wie in Kapitel~\ref{konvergenzverhalten} gezeigt, sind Schalenelemente
sehr gut geeignet, Biegeschwingungen ebener mikromechanischer Membranen
zu modellieren. In erster Näherung können daher Schalenelemente verwendet
werden, falls sich die Verhältnisse
über die Membrandicke nicht ändern und der Einfluß der Einspannung
vernachlässigt werden kann. Bei einer Strukturierung der Membran über
die Dicke, beispielsweise im Einspannbereich, oder bei Vorhandensein eines
zusätzlichen Dünnschichtsystems, wird eine dreidimensionale
Betrachtungsweise erforderlich.\\
In {\bf Abbildung~\ref{abbmembranmoden}} sind die mit Hilfe eines
zweidimensionalen FE-Modells berechneten Schwingungsmoden
$M_{ij}$ und die zugehörigen normierten Frequenzen bzw.\ Frequenzvielfachen
$c_{ij} = f_{ij} / f_{11}$ abgebildet. Die Indizes $i, j$ entsprechen der
Anzahl der Schwingungsextrema in der $x$-- und $y$--Richtung.
Wie erwartet nimmt bei höheren Schwingungsmoden die Zahl der Knotenlinien
zu, wobei einige Schwingungsmoden entartet sind, d.h.\
gleiche Frequenzeigenwerte aufweisen. Beispielsweise handelt es sich
hierbei um die beiden
Moden $M_{12}$ und $M_{21}$, die jedoch in der Realität infolge der
Kristallanisotropie, der Membraninhomogenitäten und nicht idealer
Einspannbedingungen getrennt auftreten können.
%----------------------- Beginn: Figure-Environment ----------------------
\begin{figure}[htb]
\begin{center}
% --- Dateiname des Bildes
\input{abbvv.tex}
\setabbvv
\end{center}
\caption{\label{abbmembranmoden}
 Numerisch ermittelte Schwingungsmoden und normierte Eigenfrequenzen
 $c_{ij}$ einer ebenen Siliziummembran}
\end{figure}
%----------------------- Ende: Figure-Environment ----------------------
Die experimentelle Ermittlung der Eigenfrequenzen und -schwingungsformen
erfolgte durch optische Vermessung der Modenspektren von reinen
Siliziummembranen mit verschiedenen Längen-Dicken-Verhältnissen
\cite{Sch92, Bra92a} und an fertig prozessierten Drucksensoren mit
integriertem piezoelektrischen Antrieb.
%
In {\bf Abbildung~\ref{abbspek}} ist das frequenz- und ortsabhängige
Amplitudenspektrum $A(f,x)$ eines $ZnO$-beschichteten piezoelektrisch
angetriebenen Silizium-Drucksensors dargestellt. Aufgrund der fehlenden
Phaseninformation sind nur die Absolutbeträge der
Membranschwingungsamplituden aufgetragen. Die harmonische Anregung
erfolgte mit einer Spannung von $U = 4~V_{SS}$ im Frequenzbereich bis
55~kHz. Die Membranseitenlänge betrug
9,2~mm, die Siliziumdicke etwa 50~$\mu$m und die $ZnO$-Schichtdicke etwa
11~$\mu$m.
Die Grundbiegeschwingungsmode $M_{11}$ wies eine Resonanzamplitude von
etwas über 1~$\mu$m auf. Die Schwingungsgüte betrug unter Normalluftdruck
bei der Grundmode etwa 100, bei den Moden $M_{13}$ und $M_{33}$ etwa
310 und 360. Ein Vergleich der experimentell und numerisch ermittelten
lateralen Modenverläufe entlang der Membranmitte zeigte eine gute
Übereinstimmung, so daß bei genügend großen Schwingungsamplituden
(einige hundert Nanometer) alle experimentell ermittelten Moden
mit Hilfe der FE-Ergebnisse eindeutig zugeordnet werden konnten.
Um möglichst alle Schwingungsmoden des Drucksensors im Modenspektrum
nachzuweisen, wurde der Laserstrahl des Laservibrometers entlang der
Membranmittellinie verfahren.
%----------------------- Beginn: Figure-Environment ----------------------
\begin{figure}[htb]
\begin{center}
% --- Dateiname des Bildes
\input{abbvsex.tex}
\setabbvsex
\end{center}
\caption{\label{abbspek}
  Optische Vermessung des frequenz- und ortsabhängigen Amplitudenspektrums
  $A(f,x)$ eines piezoelektrisch angetriebenen Silizium-Membrandrucksensors}
\end{figure}
%----------------------- Ende: Figure-Environment ----------------------
Zum Vergleich der Resonanzfrequenzen sollen im folgenden die analytischen
Werte aus Tabelle~2.4, %\ref{tab...}
die zwei- und dreidimensionalen FE-Berechnungen und
die experimentellen Meáwerte herangezogen werden. Aufgrund prozeßbedingter
innerer Spannungen in der $ZnO$-Dünnschicht weisen die Drucksensoren
im allgemeinen von den analytischen und numerischen Berechnungen
abweichende Resonanzfrequenzen auf. Um die Störbeiträge, die durch
prozeßbedingten Schichtspannungen nach Gleichung (\ref{fsigma})
hervorgerufen werden können, zu eliminieren, wurden die Resonanzfrequenzen
$f_{ij}$ auf die jeweiligen Grundfrequenzen $f_{11}$ bezogen. In
{\bf Tabelle~\ref{tabmembranfreqvgl}} sind daher die normierten
Resonanzfrequenzen $c_{ij}$ der höheren Schwingungsmoden von
Membranresonatoren gegenübergestellt.
%----------------------- Beginn: table ---------------------------
\begin{table}[htb]
\caption
 {\label{tabmembranfreqvgl}
 Normierte Resonanzfrequenzen $c_{ij}$ eines $ZnO$-beschichteten
 Silizium-Drucksensors
 (Vergleich: Analytische Beschreibung -- FE-Berechnungen -- Messungen)}
\begin{center}
\begin{tabular}{|c||c|c||c|c|c||c|c|}
\hline
 Mode & \multicolumn{2}{c||}{analytisch} & \multicolumn{3}{c||}{FE-Modellierung}
 & \multicolumn{2}{c|}{Messung} \\
\hline
 $ij$ & isotrop & anisotrop & 2D/lin. & 2D/quad. & 3D & $c_{ij}$ & Abw. \\[-1.2ex]
 & \cite{Pon91} & \cite{Pon91} & isotrop & isotrop & anisotrop &  & \\
\hline \hline
 $M_{12}$ & 2,039 & 1,998 & 2,038 & 2,047 & 2,066 & 2,10 & 1,6~\% \\
\hline
 $M_{22}$ & 3,007 & 2,899 & 3,004 & 2,996 & 3,031 & 2,86 & 6,0~\%\\
\hline
 $M_{31}$ & 3,662 & 3,615 & 3,652 & 3,699 & 3,754 & --- & --- \\
\hline
 $M_{13}$ & 3,679 & 3,631 & 3,669 & 3,717 &3,770 & 3,94 & 4,3~\%\\
\hline
 $M_{33}$ & 6,122 & 5,872 & 6,105 & 6,051 & 6,021 & 6,35 & 5,2~\%\\
\hline
\end{tabular}
\end{center}
\end{table}
%----------------------- Ende: table ---------------------------
Die analytisch ermittelten Frequenzen für isotropes und anisotropes
Materialverhalten, bei denen von idealisierten Platten mit unendlich steifer
Einspannung ausgegangen wurde, sind der Literatur \cite{Pon91} entnommen.
Die Frequenzwerte bei anisotroper Rechnung fallen systematisch niedriger aus,
da eine effektiv verminderte Steifigkeit dem analytischen
Beschreibungsansatz zugrundeliegt.
% ($\alpha_{aniso} = 0,798 < \alpha_{iso}$)
Bei den zweidimensionalen FE-Berechnungen wurde isotropes
Materialverhalten vorausgesetzt und Schalenmodelle mit linearen
({\em SHELL43}-Elemente: lin.) und quadratischen Ansatzfunktionen
({\em SHELL93}-Elemente: quad.) verwendet.
Die Übereinstimmung zwischen den Berechnungsergebnissen mit quadratischen
Elementen, die ein besseres Approximationsverhalten als lineare Elemente
aufweisen, und den isotropen analytischen Werten ist
sehr gut. Die Abweichungen liegen unterhalb 0,3~\%. Der Unterschied
zwischen den 3D-Volumenmodellen, die die (111)-Einspannung und die
Materialanisotropie ({\em SOLID64}-Elemente) berücksichtigen, und dem
linearen 2D-Schalenmodell liegt zwischen 1 und 3~\%.\\
%
Die Abweichungen der gemessenen Werte des optisch charakterisierten
Membrandrucksensors (siehe Abbildung~\ref{abbspek})
von den FE-Resultaten (3D-Modell, anisotrop) liegen im Bereich von 2--6~\%.
Im Vergleich zu den analytischen und numerischen Berechnungen fiel die
gemessene Resonanzfrequenz der Grundmode um etwa 2~kHz geringer aus und
betrug 7,24~kHz. Hieraus konnte nach Gleichung (\ref{fsigma}) die
Größenordnung\footnote{Vergleiche hierzu Kapitel~\ref{fehlerbetrachtung}.}
der technologisch bedingten inneren Druckspannung in der $ZnO$-Dünnschicht
zu etwa $\sigma \approx - 15$~MPa abgeschätzt werden. Die Schwingungsmode
$M_{31}$ konnte meßtechnisch an diesem Sensor nicht nachgewiesen werden,
da das modenselektive Elektrodenlayout zur Anregung dieser Mode ungeeignet
war (siehe auch Kapitel~5.5.1). Eine Diskussion der einzelnen
Fehlereinflüsse erfolgt in Kapitel~\ref{fehlerbetrachtung}.


\newpage

\section{Charakterisierung der Sensorkennlinien}
\label{sensorkennlinien}

\subsection{Kraftabhängige Frequenzänderung}
\label{kraftabh}

Um die Kraftempfindlichkeit der Balkenresonatoren zu ermitteln, wurden
sie mit einer Axialkraft beaufschlagt. Hierzu stand eine
Meßeinrichtung zur Verfügung, die es erlaubte, mit Hilfe einer Gewichtskraft
eine angenähert uniaxiale Krafteinleitung in den Sensor zu erreichen
\cite{Mue92}.
Die Kraftsensoren mit elektrothermischem Antrieb wurden bei einer geringen
Heizleistung von $\overline{P_{Heiz}}$~=~50~mW in Resonanz betrieben und
ohne resistive Auslesung im \glqq Niedertemperaturbereich\grqq \,
($P_{DMS}$~=~0) optisch vermessen. Beim Einsatz als Strömungssensor
betragen die elektrischen Verlustleistungen $P_{DMS}$ bis zu 650~mW und
fhren zu einer Temperaturüberhöhung in Balkenmitte
von etwa 60~K gegenüber der Umgebungstemperatur \cite{Bar93},
so daß sich eine
hohe thermische Querempfindlichkeit des Resonators einstellt. Diese
unerwünschte Temperaturquerempfindlichkeit muß bei der Anwendung als
Kraftsensor kontrolliert werden. Daher wurde der Einfluß der
Temperaturüberhöhung auf das statische und dynamische Verhalten der
Balkenresonatoren genauer untersucht. Eine Diskussion der Ergebnisse
wird in Kapitel~\ref{temperaturverhalten} gegeben.\\
%
Die lastabhängige Frequenzverschiebung wurde an Sensoren mit einer
Balkenlänge von 10~mm und einer Balkendicke von $(50\pm2)~\mu$m im
Kraftbereich bis etwa 11~N vermessen. Durch die
optische Abtastung konnten bei symmetrischer Anregung die beiden
Schwingungsmoden Z1 und Z3 gleichzeitig vermessen werden. In
{\bf Tabelle~\ref{tabkraftmessung}} sind die Ergebnisse der Messungen
übersichtlich zusammengefaßt.
%----------------------- Beginn: table ---------------------------
\begin{table}[htb]
\caption{\label{tabkraftmessung}
 Experimentelle Kenndaten des elektrothermisch angetriebenen Kraftsensors}
\begin{center}
\begin{tabular}{|l||c|c|c|c|c|c|}
\hline
 KenngrӇe & $f_{0}$ & $\Delta f$ & $\frac{\Delta f}{\Delta F}$ & $\eta$ & Q & NL \\
           & [Hz]    & [Hz] & [$\frac{Hz}{N}$] & [$N^{-1}$] & & [\%]\\
\hline \hline
 Mode Z1   & 4265    & 1823 & 168  &  0,039 & 426$\pm$40   & 1,9 \\
\hline
 Mode Z3   & 24193   & 3021 & 280  &  0,012 & 692$\pm$200  & 0,9 \\
\hline
\end{tabular}
\end{center}
\end{table}
%----------------------- Ende: table ---------------------------
Für die Grundmode betrug die Resonanzfrequenz
4,265~kHz und wies eine maximale Frequenzverschiebung von 1,82~kHz auf.
Dies entspricht einer relativen Kraftempfindichkeit $\eta$, definiert
nach Gleichung~(\ref{etadef}), von etwa 0,039~N$^{-1}$ \cite{Mue92}.
Die angegebenen Schwingungsgüten $Q$ stellen die
Mittelwerte aus den zwölf Messungen dar und decken sich mit den Werten
die bei Fremdanregung erreicht wurden.
Im Gegensatz zur Grundmode schwanken die Werte der Obermode Z3 relativ
stark, da diese Schwingungsmode mit Hilfe der zur Verfügung stehenden
Anregungsschaltung\footnote{Aus diesem Grunde wurde im Rahmen
einer Diplomarbeit \cite{Wie93} eine universelle Anregungsschaltung für den
elektrothermisch angeregten Resonator entwickelt, die bei der Untersuchung
des Temperaturverhaltens in Kapitel~\ref{temperaturverhalten} zum Einsatz
kam.} nicht stabil angeregt werden konnte.
Wird für die Gesamtquerschnittsfläche (siehe Schnitt B--B' in
Abbildung~\ref{abbgmssensor})
der Sensoren, die sich additiv aus den Querschnitten der beiden
Verstärkungsstege und der Balkenquerschnittsfläche zusammensetzt, der
Wert von etwa $7,1 \cdot 10^{-7} \; m^{2}$ zugrunde gelegt,
so entspricht dieses einer mechanischen Verspannung im Resonator von
rund 15,3~MPa. Dieser Wert führt auf eine
\glqq Spannungsempfindlichkeit\grqq \, von etwa 119~Hz/MPa.\\
Zur Überprüfung der FE-Modellannahmen wurde die Eigenfrequenzverschiebung
unter der Einwirkung einer Axialkraft berechnet. Durch die äußere
Lasteinwirkung ergibt sich eine Überlagerung eines statischen und eines
dynamischen Problems. Im ersten Schritt wird in einer nichtlinearen
statischen Analyse (Gleichung \ref{nlstatik}) die Balkendehnung als Funktion
der einwirkenden Last ermittelt. Im darauffolgenden zweiten Schritt
wird die Frequenzverschiebung in Abhängigkeit der Änderung der
Resonatorsteifigkeit aufgrund spannungsversteifender Effekte nach
Gleichung~(\ref{modal}) berechnet.
Unter Verwendung des FE-Modells aus Kapitel~\ref{balkenresonatoren}
konnte für eine äquivalente Zugspannung von etwa 15,2~MPa im Resonatorbalken
eine Frequenzänderung von 1,868~kHz berechnet werden, was einer
Empfindlichkeit von etwa 123~Hz/MPa entspricht.
%----------------------- Beginn: Figure-Environment ----------------------
\begin{figure}[htb]
\begin{center}
% --- Dateiname des Bildes
\input{abbvsi.tex}
\setabbvsi
\end{center}
\caption{\label{abbkraftfreq}
 Lastabhängige Resonanzfrequenzänderung beim
 elektrothermisch angetriebenen Silizium-Kraftsensor}
\end{figure}
%----------------------- Ende: Figure-Environment ----------------------
In {\bf Abbildung~\ref{abbkraftfreq}} sind die experimentellen und
numerischen Ergebnisse graphisch dargestellt. Die FE-Ergebnisse sind
als gestrichelte Linie und die gemessenen Werte als Punkte eingezeichnet.
Um die Nichtlinearität\footnote{Die Nichtlinearität NL gibt die
maximale Abweichung einer nichtlinearen, stetig gekrümmten Kennlinie
von einer Verbindungsgeraden
durch den Anfangs- und Endpunkt der Kennlinie an und ist
definiert als \cite{Rei89}: NL$ := \pm \frac{1}{2} \left\{
[f(x_{1})+f(x_{2})]/2-f(x_{2}/2) \right\} / [f(x_{2})-f(x_{1})]$.}
der Sensorkennlinien zu ermitteln, wurde
an die Meßwerte eine lineare Ausgleichsgerade mit der Methode
der kleinsten Quadrate ({\em Least-Square-Approximation}) angepaßt. Die
so ermittelte Nichtlinearität betrug für die Grundmode rund 1,9~\%.



\subsection{Druckabhängige Frequenzänderung}
\label{druckabh}

Das druckabhängige Resonanzverhalten von Membranen wird charakterisiert
durch den Zusammenhang zwischen der Druckbeaufschlagung und der sich
einstellenden Membranauslenkung. Wie in Kapitel~\ref{abschaetzungen}
dargelegt, treten bei genügend \glqq großen\grqq \, Auslenkungen
nichtlineare Effekte auf, die zu inneren
Membranspannungen führen. Diese sind infolge der spannungsversteifenden
Wirkung für die Frequenzänderung der Membran verantwortlich.
Die Druckempfindlichkeit der Siliziummembranen wurde analog zum Vorgehen
beim Kraftsensor berechnet.
Durch eine geometrisch nichtlineare, statische FE-Berechnung wurde
die durch die Druckdifferenz hervorgerufene Steifigkeitsänderung der
Membran ermittelt und zur Berechnung der Frequenzverschiebung herangezogen.
Auf diese Weise konnte die Frequenz-Druck-Kennlinie des Drucksensors
simuliert werden.\\
Im Gegensatz zu den Balkenstrukturen wird bei den Membranstrukturen
die Spannungsversteifung durch große Auslenkungen ($d \geq h$)
hervorgerufen, so daß die geometrische Steifigkeitsänderung des Resonators
rechentechnisch durch große Deformationen (Elementrotationen) erfaßt
werden muß. In {\sf ANSYS} wird
die auslenkungsabhängige Steifigkeitsänderung $K(u)$ durch
Gleichung~(\ref{nlstatik}) beschrieben und iterativ nach
(\ref{largedefl}) gelöst.
Für die Ermittlung der druckabhängigen Membranauslenkung und
Frequenzverschiebung wurde der in Kapitel~\ref{membranresonatoren}
optisch vermessene Silizium-Drucksensor herangezogen.
Um den Silizium-Drucksensor mit einer Siliziummembrandicke von etwa
50~$\mu$m und einer $ZnO$-Schichtdicke von etwa 11~$\mu$m sowohl analytisch
als auch numerisch mit Schalenelementen modellieren zu können, wurden
unter Annahme isotroper Elastizitätseigenschaften die Materialdaten
gemäß Gleichung (\ref{matwich}) gewichtet gemittelt. In erster Näherung
reichen beim Zinkoxid die Materialeigenschaften senkrecht zur $c$--Achse aus,
um das Biegeverhalten in der Ebene der beiden $a$--Achsen
zu beschreiben. Für Zinkoxid wurde ein E-Modul von
126~GPa, eine {\em Poisson}-Zahl von 0,36 und eine Dichte von
5675~$kg/m^{3}$ verwendet \cite{LB82}. Diese Werte erweisen sich auch
für Dünnschichten als sinnvoll,
da an gesputterten $ZnO$-Dnnschichten Indenter-Messungen nach der Methode
des Last-Eindringverfahrens \cite{Ola92} durchgeführt wurden und einen
reduzierten E-Modul von $\hat E \approx 145$~GPa bei einer Schichtdicke
von 2--3~$\mu$m ergaben. Bei obigen Schichtdicken resultieren als
gemittelten Materialparameter für den E-Modul 161~GPa, für die
{\sl Poisson}-Zahl 0,1166 und für die Dichte der Wert 2932~kg/m$^{3}$.\\
%----------------------- Beginn: Figure-Environment ----------------------
\begin{figure}[htb]
\begin{center}
% --- Dateiname des Bildes
\input{abbva.tex}
\setabbva
\end{center}
\caption{\label{abbmemauslvgl}
 Druckabhängige Membranmittenauslenkung eines Silizium-Drucksensors
 (Vergleich: analytische Beschreibung -- FE-Berechnung -- Messung)}
\end{figure}
%----------------------- Ende: Figure-Environment ----------------------
Fr eine erste Abschätzung der druckabhängigen Membranmittenauslenkung
wurden sie analytisch nach zwei verschiedenen Näherungsansätzen
(Gleichung \ref{nonlin1} und
\ref{memausl}) berechnet. In {\bf Abbildung~\ref{abbmemauslvgl}} sind die
Ergebnisse der analytischen Berechnungen im Vergleich mit den optischen
Messungen und den Ergebnissen des 2D-Schalenmodells S43 aus
Tabelle~\ref{tabfreqzus} dargestellt. Die Nichtlinearität des gemessenen
Kurvenverlaufes beträgt im Maximaldruckbereich etwa $\pm$6,0~\%.
Während Gleichung (\ref{nonlin1}) zu hohe
Membranmittenauslenkungen liefert und ab etwa 400~mbar aufgrund des
negativen quadratischen Korrekturfaktors des Linearitätsfehlers
einen degressiven Kurvenverlauf annimmt, beschreibt Gleichung
(\ref{memausl}) die druckabhängige Mittenauslenkung exakter.
Die Differenz zwischen den beiden analytischen Beschreibungsansätzen
beträgt etwa  6--11~\%. Während der quadratische Näherungsansatz
(\ref{nonlin1}) bis zu einer Druckdifferenz von etwa 100~mbar, dieses
entspricht einer Auslenkung von etwa einer halben Membrandicke, verwendet
werden kann, erweist sich der kubische Näherungsansatz (\ref{memausl})
als eine bessere Beschreibung des gemessenen Kurvenverlaufs bis 500~mbar
mit einer maximalen Abweichung von 12~\%.
Das 2D-Schalenmodell kommt den gemessenen Werten mit einer Abweichung
von etwa 5~\% am nächsten. Eine lineare analytische Vergleichsrechnung,
bei der der nichtlineare Korrekturterm aus Gleichung (\ref{nonlin1})
nicht berücksichtigt wurde, ergab für den betrachteten maximalen
Druckbereich eine etwa doppelt so hohe Membranauslenkung und ist daher
nicht mehr verwendbar.\\
%
Um die Einflüsse der realen Einspannung durch die (111)-Siliziumebenen
und der anisotropen Elastizitätseigenschaften der Siliziummembran und
der $ZnO$-Dünnschicht zu untersuchen, wurden neben dem Schalenmodell S43 mit
linearen Ansatzfunktionen auch die dreidimensionalen Modelle V95 und V64
betrachtet, die aufgrund der erhöhten Anzahl von Freiheitsgraden nur ein
Viertel der Sensorgeometrie berücksichtigen.
Das FE-Modell V95 verwendet 20-knotige
Volumenelemente ({\em SOLID95}) mit quadratischen Ansatzfunktionen bei
isotropen Materialeigenschaften, das FE-Modell V64 ({\em SOLID64}-Elemente)
berücksichtigt die Elastizitätsanisotropien von Silizium {\em und} Zinkoxid.
Die Elastizitätseigenschaften wurden
für Bulkmaterial der Literatur \cite{LB82} entnommen. Während das Modell V95
nur jeweils eine Elementlage über die Membrandicke aufweist, sind bei
Modell V64 insgesamt vier Elementlagen berücksichtigt worden.
Die Berechnungsergebnisse der dreidimensionalen FE-Modelle V64 und V95 sind
in Abbildung~\ref{abbmemauslvgl} zusätzlich gestrichelt eingezeichnet
und decken sich bis 300~mbar gut mit den Resultaten des 2D-Schalenmodells.
Ab diesen Druckbereich verhalten sich die 3D-Modelle weniger steif, so daß
die Auslenkungen höher ausfallen. Dieses läßt sich auf die mit zunehmender
Druckbeaufschlagung Verdünnung der Volumenelemente zurückführen, die eine
erhebliche Verminderung der effektiven Biegesteifigkeit $D$
zur Folge haben kann. Nach
Gleichung (\ref{biegesteif}) geht die Membrandicke $h$ mit der dritten
Potenz ein, so daß bei einer Auslenkung von fast 90~$\mu$m beim
anisotropen FE-Modell V64 eine merkliche Steifigkeitsverminderung
hervorgerufen werden kann.
%Dieses liegt zum einen an dem schlechten Aspektverhältnis von etwa 7--10
%der Volumenelemente, zum anderen verfügen die Volumenlemente gegenüber den
%Schalenelementen ein verbessertes Tragverhalten gegenüber Schubbelastungen,
%so daß sie bei bestimmten Lastfällen weicher werden können.
Wie in Kapitel~\ref{modelleinfluesse} untersucht, fallen im Vergleich zur
anisotropen Beschreibung (FE-Modell: V64) die Auslenkungen bei der
isotropen Beschreibung (FE-Modell: V95) geringer aus. Das 2D-Schalenmodell
weist trotz des überlegenen Approximationsverhaltens (siehe
Abbildung~\ref{abbkonvausl}) und ausreichender Vernetzungsfeinheit
die höchste Modellsteifigkeit auf. Dieser Sachverhalt läßt sich auf die
Verwendung der isotropen Ersatzmaterialdaten zurückführen.\\
%----------------------- Beginn: Figure-Environment ----------------------
\begin{figure}[htb]
 \begin{center}
% --- Dateiname des Bildes
\input{abbvn.tex}
\setabbvn
\end{center}
\caption{\label{abbmemfreqvgl}
 Druckabhängige Frequenzänderung eines piezoelektrisch angetriebenen
 Silizium-Drucksensors (Vergleich: FE-Berechnungen -- Messung)}
\end{figure}
%----------------------- Ende: Figure-Environment ----------------------
In {\bf Abbildung~\ref{abbmemfreqvgl}} sind die Ergebnisse
der numerisch berechneten und experimentell bestimmten, druckabhängigen
Frequenzänderung des Silizium-Drucksensors graphisch dargestellt.
Die Reihenfolge der Kennlinien ist umgekehrt zu der bei den druckabhängigen
Auslenkungen. Aufgrund der größten Modellsteifigkeit weist das
2D-Schalenmodell S43 die höchsten Frequenzen und folglich auch die größten
Abweichungen zu den Meßwerten auf. Die Approximationsgüte nimmt hingegen
vom isotropen 3D-Modell V95 zum anisotropen 3D-Modell V64 zu. Für die
unbelastete Membran (p = 0) betragen die Abweichungen zur gemessenen
Resonanzfrequenz $f_{0}$ lediglich 4,4 bzw.\ 2,4~\%. Die druckabhängige
Frequenzänderung $\Delta f$/$\Delta p$ beläuft sich bei allen drei
FE-Modellen auf etwa 14,3--14,5~Hz/mbar und weicht rund 16~\% von der
gemessenen Druckempfindlichkeit (12,4~Hz/mbar) ab.
{\bf Tabelle~\ref{tabvglm723}} faßt die analytischen und numerisch
berechneten charakteristischen Sensorkenndaten im Vergleich
mit den Messungen zusammen. Die analytischen Abschätzungen für die
unbelastete Resonanzfrequenz $f_{0}$,
die maximale Membranmittenauslenkung $u_{z}$ und
die Druckempfindlichkeit $\Delta f / \Delta p$ erfolgten nach den
Gleichungen (\ref{memfreq}), (\ref{memausl}) und (\ref{pvond}) unter
Verwendung gewichteter Materialdaten nach Gleichung (\ref{matwich}).
%----------------------- Beginn: table ---------------------------
\begin{table}[htb]
\caption
 {\label{tabvglm723}
 Charakteristische Kenndaten des piezoelektrisch angetriebenen
 Silizium-Drucksensors
 (Vergleich: analytische Beschreibung -- FE-Berechnung -- Messung)}
\begin{center}
\begin{tabular}{|c||c||c|c|c||c|}
\hline
Sensor & analytisch & \multicolumn{3}{c||}{FE-Modell: S43--V95--V64}
       & Messung \\
\cline{3-5}
 Kenndaten &     & 2D      & 3D      & 3D        & \\[-1.2ex]
           &     & isotrop & isotrop & anisotrop & \\
\hline \hline
h $[\mu$m] & --- & 184     & 511     & 341       & --- \\
Aspektverh.: & --- & 3,0 & 10,2 & 6,8 & --- \\
\hline
Elemente   & --- & 2500    & 1128    & 3844      & --- \\
Knoten     & --- & 2601    & 6565    & 5230      & --- \\
MDOF       & --- & 300     & 500     & 500       & --- \\
\hline \hline
$f_{0}$ [Hz]
           &  8920  &  8890   &  8686   &  8517   &  8317   \\
           & (7,3~\%) & (6,9 \%) & (4,4 \%) & (2,4 \%) &  $\pm$50 \\
\hline
$u_{z}$ $[\mu m]$
           & 84,26    &  79,06   & 82,21   & 89,00    & 75,0     \\
           & (12,3~\%) & (5,4 \%) & (9,6~\%) & (18,7~\%) & $\pm$0,5  \\
\hline
$\frac{\Delta f}{\Delta p}$ $[\frac{Hz}{mbar}]$
        & 14,72    & 14,30    & 14,36    & 14,53    & 12,4    \\
  	& (18,7~\%) & (15,3~\%) & (15,8~\%) & (17,2~\%) & $\pm$1,0 \\
\hline
\end{tabular}\\
\end{center}
\end{table}
%----------------------- Ende: table ---------------------------
Die Approximation der Meßkurve durch die Berechnungen hängt neben
der Kenntnis der exakten Strukturabmessungen entscheidend von der richtigen
Wahl der Materialparameter sowie der Kenntnis der inneren Spannung der
Dünnschicht auf der Siliziummembran ab.
Beim Vergleich der druckabhängigen Membranmittenauslenkung
(Abbildung~\ref{abbmemauslvgl}) mit der Frequenzänderung
(Abbildung~\ref{abbmemfreqvgl}) fällt beim Schalenmodell S43 auf, daß mit
einer geeigneten Wahl der Materialparameter beispielsweise der Kurvenverlauf
der Membranauslenkung ($u_{z}(p) \sim K^{-1}$) zwar gut approximiert
werden kann, jedoch auf Kosten der Approximationsgüte bei der
Resonanzfrequenz ($f(p) \sim \sqrt{K}$).
Somit ist bei den numerischen Berechnungen sicherzustellen, daß die
Modellsteifigkeit $K$, einerseits bedingt durch die Elementvernetzung,
andererseits durch die Wahl der Materialparameter, richtig beschrieben wird.
Der Einfluß der Modellmassenmatrix $M$ kann bei einer konsistenten
Massenformulierung vernachlässigt werden, falls die laterale
Elementunterteilung genügend fein gewählt wurde.
Die durch die Geometrietoleranzen und nicht genau bekannten
Materialeigenschaften bedingten Fehlereinflüsse werden in Kapitel~4.5
diskutiert.

