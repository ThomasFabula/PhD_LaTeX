\section{Untersuchung des Temperatureinflusses}
\label{temperaturverhalten}


Um das thermische Verhalten mikromechanischer Bauelemente zu untersuchen,
eignen sich die elektrothermisch angetriebenen Kraft- und Strömungssensoren
besonders gut, da die Heiz- und DMS-Widerstände variabel ansteuerbare
Wärmequellen darstellen. Bei einer Temperaturüberhöhung treten im
eingespannten Balkenresonator thermisch induzierte Spannungen
auf, die bei Überschreitung der für das Bauelement charakteristischen
Knickspannung zum Ausknicken bzw.\ Ausbeulen ({\em engl.:} Buckling)
des Balkens führen. Hierdurch wird das Resonanzverhalten stark
beeinfluát, so daß die Resonanzfrequenzen unterschiedliche
Temperaturempfindlichkeiten {\em vor} und {\em nach} dem Ausknicken
aufweisen. Bei Erhöhung der Temperatur bauen sich infolge verhinderter
Wärmeausdehnung Druckspannungen im Resonator auf, die zu einer
Erniedrigung der Resonanzfrequenz der Grundbiegeschwingung führen.
Bei Erreichen der kritischen Temperatur knickt der ursprünglich gerade
Balken aus und nimmt eine neue Gleichgewichtslage ein.
Bei weiterer Temperaturerhöhung können sich danach nur Zugspannungen
aufbauen, so daß die Resonanzfrequenz infolge der spannungsversteifenden
Effekte wieder ansteigt.\\
%
Ausgangspunkt zur analytischen Beschreibung der temperaturabhängigen
Effekte beim doppelseitig eingespannten Biegebalken bildet
Gleichung~(\ref{fsigma}), die die Resonanzfrequenzänderung des Balkens
infolge einer zusätzlichen axialen Spannung $\sigma$ im Resonator
beschreibt. Durch Umformung dieser Gleichung kann
formal die Knickspannung $\sigma_{B}$ eingeführt werden:
\begin{eqnarray}
\label{fbuckling}
 f(\sigma) & = & f_{0} \, \sqrt{1 + \left( \frac{\sigma}{\sigma_{B}} \right)}
\end{eqnarray}
Die axiale Spannung $\sigma$ kann hierbei durch eine technologisch bedingte
innere Vorspannung $\sigma_{0}$, eine äußere axiale Kraftbeaufschlagung
$F$ oder infolge einer Temperaturüberhöhung $\Delta T$ nach Gleichung
(\ref{thermdehn}) hervorgerufen werden.
% \begin{eqnarray}
% \label{sigmall}
%  \sigma & = & \sigma_{0} + \frac{F}{A} - {\hat E} \alpha \Delta T
% \end{eqnarray}
Das Ausknicken des Balkens findet statt, sobald die Summe der inneren
Druckspannungen den Wert der Knickspannung erreicht. Unter Verwendung von
Gleichung
(\ref{fsigma}) kann die charakteristische Knickspannung, die auch als
{\sl Euler}sche Knicklast bezeichnet wird, abgeleitet werden \cite{Tim87}:
\begin{eqnarray}
\label{sigmab}
 \sigma_{B} & = & 4 {\pi}^2 \frac{\hat E I}{A{l}^2}
     \approx 3,39 \hat E {\left( \frac{h}{l} \right) }^2
\end{eqnarray}
Die Temperaturabhängigkeit der Resonanzfrequenz nach dem Ausknicken
läßt sich unter Zugrundelegung des {\sl Rayleigh}--Quotienten beschreiben,
durch den die Resonanzfrequenz des Balkens über das Verhältnis von
potentieller zu kinetischer Energie im Resonator dargestellt werden kann
\cite{Kim86, Gei91}. Durch Betrachtung der veränderten Biegelinie im
ausgeknickten Zustand ist es möglich, das Temperaturverhalten
des Balkens auch {\em nach} dem Ausknicken zu berechnen, was durch
Gleichung (\ref{fbuckling}) nicht möglich ist, da für
$\sigma < - \sigma_{B}$ der Wurzelausdruck komplex würde und die
Resonanzfrequenz dann nicht mehr definiert ist.
Für die Resonanzfrequenz des Balkens in Abhängigkeit der
Temperaturüberhöhung gegenüber der Umgebung sind in \cite{Bar93} die
Zusammenhänge:
\begin{eqnarray}
\label{fdeltat1}
 f(\Delta T) & = & \frac{1}{l} \, \sqrt{\frac{ S - \hat E \alpha \Delta T}
     {3 \rho}} \qquad \mbox{für} \quad 0 \, \leq \, \Delta T \, \leq \, \Delta T_{kr} \\
\label{fdeltat2}
 f(\Delta T) & = & \frac{1}{l} \, \sqrt{\frac{\hat E \alpha \Delta T - S}
     {\rho}} \qquad \mbox{für} \quad \Delta T \, \geq \, \Delta T_{kr}
\end{eqnarray}
abgeleitet worden. Hierbei ist $S=\sigma_{0}+\sigma_{B}$ die Summe aus
axialer Vorspannung und Knickspannung. Bei Erreichen der kritischen
Temperaturüberhöhung $\Delta T_{kr}$ = $S/{\hat E \alpha}$ kompensiert die
thermische Druckspannung
%, hervorgerufen durch den thermischen Ausdehnungskoeffizienten $\alpha$,
die mechanische Steifigkeit $S$ des Balkens, so daß aufgrund der
verschwindenden Resonatorsteifigkeit die Resonanzfrequenz im Idealfall
gegen Null geht. \\
Für den praktischen Einsatz von resonanten Sensoren ist daher die Kenntnis
der kritischen Knickspannung
und der damit verbundenen kritischen Temperaturüberhöhung von
entscheidender Bedeutung, weil der Resonator statisch instabil wird und
die temperaturabhängige Sensorkennlinie nicht mehr eindeutig ist.
Im nachfolgenden wird der Einfluß einer homogenen Temperaturerhöhung auf
den in Kapitel~\ref{balkenresonatoren} und \ref{kraftabh} behandelten
Balkenresonator untersucht.


\subsection{Statische Instabilitäten}

Die auf ihr thermisches Verhalten hin untersuchten Silizium-Kraftsensoren
weisen eine Länge von 10~mm und eine mittlere Balkendicke von etwa 50~$\mu$m
auf. Bei der Herstellung wurden für die Ätzmaskierung der (100)-Siliziumwafer
thermische Oxidschichten von 1,5~$\mu$m Dicke verwendet, die bei der
naßchemischen Prozessierung auf etwa 1~$\mu$m abgedünnt wurden. Zusätzliche
Metallschichtsysteme sind für die Heiz- und DMS-Widerstände
($NiCr$), die lötbaren ($NiCr-Ti/Pd/Au$) bzw.\ bondbaren Außenkontakte
($NiCr-Ti/Pd/Au$ galv.) und das Kontaktsystem auf den Siliziumbalken
mittels $PVD$-Verfahren (Sputtern) abgeschieden worden. Die Schichtdicken
betrugen bei $NiCr$ 30~nm und bei $Ti/Pd/Au$ etwa 60~nm/400~nm/160~nm.
Da die Haftfestigkeit des Metallschichtsystems auf dem thermischen Oxid
kritisch ist, wurde eine 30~nm dünne $NiCr$-Schicht als zusätzlicher
Haftvermittler eingesetzt \cite{ABV93}.
Dieses Metallschichtsystem wurde in den FE-Berechnungen vernachlässigt, da
es einerseits den Balken nicht ganzflächig überdeckt und die Schichtdicken
im Gegensatz zum Siliziumbalken sehr gering sind
(50~$\mu$m/1~$\mu$m/0,65~$\mu$m).\\
%
Im Rahmen einer Diplomarbeit \cite{Mes93} wurden verschiedene FE-Modelle
des Balkenresonators erstellt und das Balkenausknicken, sowie das
temperaturabhängige Resonanzfrequenzverhalten modelliert. Insbesondere
wurde die Eignung des elektrothermisch angeregten Resonators als
frequenzanaloger Strömungssensor untersucht.
In {\bf Abbildung~\ref{abbausvgl}} sind die Balkenmittenauslenkungen
in Abhängigkeit der Temperaturüberhöhung zwischen der mittleren
Balkentemperatur und der Umgebung dargestellt. Zur Überprüfung der
FE-Berechnungen wurden zusätzlich Messungen an vier verschiedenen
Sensorexemplaren durchgeführt. Die Vermessung der Sensoren erfolgte
optisch mit Hilfe eines kommerziellen Autofocusmeß\-systems \cite{UBM91}.
Allen Kurven ist gemeinsam, daß die Balkenmittenauslenkung zwischen 20
und 30~K deutlich zunimmt und bei einer Temperaturüberhöhung
von 60~K den Betrag der Balkendicke überschreitet.
Durch geometrische Imperfektionen des Resonators und nicht ideale
Einspannbedingungen findet das Ausknicken gegenüber der idealisierten
Balkengeometrie der FE-Modelle nicht abrupt statt. Infolge der inneren
Spannungen des Schichtsystems weisen die vermessenen Siliziumsensoren
Vordehnungen auf, die bereits zu 3--9~$\mu$m Balkenmittenauslenkungen bei
$\Delta T = 0$ führen. Die Variation der gemessenen Kurvenverläufe
dokumentiert weiterhin die prozeßtechnisch bedingte Bauelementestreuung,
die im wesentlichen durch die nur
ungenau kontrollierbare Balkendicke und zusätzlich durch die
unterschiedlichen inneren Schichtspannungen hervorgerufen wird.\\
%----------------------- Beginn: Figure-Environment ----------------------
\begin{figure}[htb]
\begin{center}
% --- Dateiname des Bildes
\input{abbvzz.tex}
\setabbvzz
\end{center}
\caption{\label{abbausvgl}
 Statische Auslenkung eines Silizium-Balkenresonators in Abhängigkeit der
 Temperaturüberhöhung (Vergleich: FE-Berechnungen -- Messungen)}
\end{figure}
%----------------------- Ende: Figure-Environment ----------------------
Die beiden FE-Modelle berücksichtigen die Temperaturabhängigkeit der
Wärmeausdehnungskoeffizienten und der Elastizitätsmoduln, die
Siliziumoxidschicht und die durch (111)-Siliziumebenen bedingte schräge
Sensoreinspannung. Die mechanischen Verstärkungsstege des Sensors wurden
vernachlässigt, da sie keinen Einfluß auf das Schwingungsverhalten haben.
Beim dreidimensionalen FE-Modell wurden zusätzlich die anisotropen
Materialeigenschaften berücksichtigt. Während der Einfluß auf die kritische
Temperaturdifferenz beim Ausknicken des Balkens sehr gering ist,
unterscheiden sich die maximalen Mittenauslenkungen des Balkens um etwa
11~\%. Dieses ist im wesentlichen auf die unterschiedlichen
Temperaturkoeffizienten der E-Moduln bei der isotropen bzw.\
anisotropen Beschreibungsweise zurückzuführen, da der Einfluß der
gewählten isotropen Elastizitätsersatzdaten (\ref{simat}) gemäß
Tabelle~\ref{tabgeomvergleich} vernachlässigbar ist.



\subsection{Resonanzfrequenzverhalten}

Die statischen Berechnungen unter Einwirkung der homogenen Temperaturlasten
wurden herangezogen, um die temperaturbedingte Steifigkeitsänderung des
Balkenresonators zu ermitteln und anschließend die Resonanzfrequenz
zu berechnen. Die Resonanzfrequenzcharakteristik wurde analog zum
Vorgehen bei der Berechnung der Kraftempfindlichkeit des Balkenresonators
ermittelt. In {\bf Abbildung~\ref{abbfreqvgl}} ist der Frequenzverlauf
der Grundschwingungsmode des Balkenresonators in Abhängigkeit der
Temperaturüberhöhung dargestellt. Zum Vergleich sind die experimentellen
Resultate (Meßpunkte) zweier verschiedener Sensorexemplare mit
eingezeichnet. Beide Sensoren wiesen eine Länge von 10~mm und eine
nominelle Balkendicke von 50~$\mu$m auf. Sie wurden sowohl optisch mit
einem Laservibrometer \cite{Mes93}, als auch in einem
elektrischen Meßaufbau \cite{Wie93} vermessen. Die optisch vermessene
Resonanzcharakteristik zeichnet sich gegenüber der elektrisch vermessenen
Kurve durch geringere Streuungen aus. Die Meßfehler bei der optischen
Bestimmung der Resonanzfrequenz liegen bei etwa $\pm$(10--50)~Hz, die bei
der elektrischen Messung bei etwa $\pm$200~Hz (siehe Fehlerbalken).
Die mittleren Heizleistungen $\overline{P_{Heiz}}$ betrugen etwa 220~mW,
die Umgebungstemperatur wurde auf etwa $\pm 1~^{\circ}C$ konstant gehalten.
Die großen Schwankungen bei der elektrischen Messung können durch
Schwankungen der DMS-Speisespannung und der Schwingungsamplitude, sowie
der Dauer der Anregungsimpulse erklärt werden.
Aus diesem Grunde wurde fr die elektrischen Meßwerte eine analytische
Anpassung nach den Gleichungen (\ref{fdeltat1}) und (\ref{fdeltat2})
vorgenommen. Eine optimale Anpassung an den gemessenen Kurvenverlauf
wurde mit den Parametern S = 12,1 MPa und $\Delta T_{kr} = 30,5~K$
erreicht. Die Zuordnung der Temperaturüberhöhung zur umgesetzten
elektrischen Leistung erfolgte durch \,
$\Delta T \approx 0,084 \cdot P_{DMS} + \Delta T(\overline{P_{Heiz}})$
fr die optischen {\em und} elektrischen Messungen gleichzeitig.
Der Temperatureinfluß der mittleren Heizleistung $\overline{P_{Heiz}}$
wurde hierbei vernachlässigt, da bei den optischen Messungen (in
Kapitel~\ref{balkenresonatoren}) die Resonanzfrequenz ohne DMS-Abtastung
erfolgte
und damit $\Delta T(\overline{P_{Heiz}}) \leq 4~K$ abgeschätzt werden konnte.
Mit Hilfe der optischen Amplitudenmessungen wurden die Schwankungen
der Schwingungsamplitude und der Schwingungsgüten des Balkenresonators
ermittelt. Für die Grundmode wurden die Werte
$\Delta A = \pm5~\%$ und $\overline{Q} = 220\pm30$ gemessen.
Das schwächer ausgeprägte Frequenzminimum beim optisch vermessenen Sensor
läßt sich durch die unterschiedlichen Vorspannung beider Balkenresonatoren
erklären \cite{Lin93}. Dieser Sachverhalt wird durch die unterschiedlichen
Meßergebnisse der temperaturabhängigen Balkenmittenauslenkung in
Abbildung~\ref{abbausvgl} untermauert.\\
%----------------------- Beginn: Figure-Environment ----------------------
\begin{figure}[htb]
\begin{center}
% --- Dateiname des Bildes
\input{abbvee.tex}
\setabbvee
\end{center}
\caption{\label{abbfreqvgl}
 Frequenzcharakteristik eines Silizium-Balkenresonators in Abhängigkeit
 der Temperaturüberhöhung
 (Vergleich: Analytische Beschreibung -- FE-Berechnungen -- Messungen)}
\end{figure}
%----------------------- Ende: Figure-Environment ----------------------
In {\bf Tabelle~\ref{tabdeltatkr}} sind die rechnerisch und experimentell
ermittelten charakteristischen Kenngrößen der Silizium-Balkenresonatoren
in Abhängigkeit der Temperaturüberhöhung abschließend zusammengefaát.
Es handelt sich hierbei um die Resonanzfrequenz bei Umgebungstemperatur
$f(\Delta T=0)$, die Resonanzfrequenz $f(\Delta T_{max})$ bei 60~K
Temperaturdifferenz und der maximalen Balkenmittenauslenkung
$u_{z}$, sowie der jeweils mit Hilfe verschiedener Methoden ermittelten
kritischen Temperaturüberhöhungen $\Delta T_{kr}$. Im Rahmen der
Meßfehler und der technologisch bedingten Toleranzen der Siliziumsensoren
stimmen die gemessenen Werte mit den Simulationen gut überein.\\
%----------------------- Beginn: table ---------------------------
\begin{table}[htb]
\begin{center}
\begin{tabular}{|l||c|c|c|c|}
\hline
  & $f(\Delta T=0)$ & $f(\Delta T_{max})$ & $u_{z}$  & $\Delta T_{kr}$ \\
  &      [Hz]       &     [Hz]            & [$\mu$m] &  [K]  \\
\hline \hline
  FEM (2D-Modell)          &  4223  &  7415  & 77,3    &  $27,5\pm1,0$  \\
  FEM (3D-Modell)          &  4304  &  6851  & 69,4    &  $26,0\pm1,0$  \\
\hline
  optische Messung  &  ---   &  6970  & 56--70  &  $24,0\pm2,0$  \\
  elektr. Messung   &  4164  &  7077  & ---     &  $30,5\pm2,0$  \\
\hline
\end{tabular}
\end{center}
\caption{\label{tabdeltatkr}
 Temperaturabhängigkeit charakteristischer Kenngrößen von
 Silizium-Balkenresonatoren (Vergleich: FE-Berechnungen -- Messungen)}
\end{table}
%----------------------- Ende: table ---------------------------
Bei frequenzanalogen Sensoranwendungen, die den elektrothermischen Antrieb
einsetzen, muß darauf geachtet werden, daß der Balkenresonator
nicht instabil wird und ausknickt, da ansonsten die Zuordnung zwischen
Meßgröße und Resonanzfrequenz nicht mehr eindeutig ist. Bei der Anwendung
als Kraftsensor ist die Temperaturquerempfindlichkeit soweit wie möglich
zu minimieren, das durch eine geeignete Wahl des Arbeitspunktes
(DMS-Leistung) erfolgen
kann. Bei den 10~mm langen und 50~$\mu$m dicken Silizium-Balkenresonatoren
beträgt die mittlere kritische Temperaturdifferenz etwa ($27\pm3$)~K, die
einer mittleren elektrischen Leistung von etwa ($320\pm40$)~mW entspricht.
Für einen stabilen Betrieb als Kraftsensor ist es daher notwendig, den
Arbeitspunkt weit unterhalb dieser Leistung, bei etwa 200~mW zu wählen.
Eine untere Grenze für die DMS-Leistung wurde bei 150~mW erreicht,
unterhalb derer keine stabile Anregung mehr gewährleistet war. Bei der
Anwendung als frequenzanaloger Strömungssensor muß ein Arbeitspunkt
weit oberhalb der kritischen Temperaturdifferenz gewählt werden, um den
Abkühlungseffekt durch die thermische Fluid-Struktur-Wechselwirkung
möglichst zu erhöhen.\\
%
In {\bf Tabelle~\ref{tablayoutredesign}} sind die charakteristischen
Kenngrößen elektrothermisch angetriebener Silizium-Kraftsensoren in
Abhängigkeit des Widerstandslayouts zusammengefaßt.
Beim Startlayout wurde die DMS-Vollbrückenschaltung in Balkenmitte
angeordnet, um eine maximale Anregungseffizienz zu erreichen, was zu
einer stark inhomogenen Wärmeverteilung im Balken und einer
hohen Temperaturquerempfindlichkeit führte. Beim Redesign wurden die
beiden querliegenden Widerstände nach außen zu den Balkenenden hin verlegt.
An diesen Stellen hat die axiale Spannung auf der Balkenoberfläche einen
maximalen Wert, so daá diese Widerstände ebenfalls zur Signalgewinnung
beitrugen. Bei mittleren Impulsheizleistungen von 600~mW
($R_{Heiz}=300~\Omega$) konnten Schwingungsamplituden von etwa 1~$\mu$m
erzielt werden.
%----------------------- Beginn: table ---------------------------
\begin{table}[htb]
\begin{center}
\begin{tabular}{|l||c|c|c|c|}
\hline
 KenngrӇe  & $\Delta T_{max}$ & $T_{max}$   & $U_{DMS}$ & $\frac{\Delta f}{\Delta T}$ \\
            & [K]              & [$^\circ$C] & [mV]      & [$\frac{Hz}{K}$] \\
\hline \hline
 Startlayout & 60 & 100 & 40 & 19 \\
\hline
 Redesign    & 30  & 68 & 200 & 5 \\
\hline
\end{tabular}
\end{center}
\caption{\label{tablayoutredesign}
 Charakteristische Kenngrößen des elektrothermisch angetriebenen
 Kraftsensors in Abhängigkeit des Widerstandslayouts}
\end{table}
%----------------------- Ende: table ---------------------------
Mit der Layoutänderung konnte die DMS-Spannung ($R_{DMS}=250~\Omega$),
bei gleichzeitiger Reduktion der Temperaturquerempfindlichkeit,
verfünffacht werden.
Die maximale Temperaturdifferenz $\Delta T_{max}$ zwischen Balkenmitte
und Einspannstelle konnte um etwa 30~K gesenkt werden, was zu einer
etwa vierfach verminderten Temperaturquerempfindlichkeit des Kraftsensors
führte.
