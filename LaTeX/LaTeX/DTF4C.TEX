\section{Fehlerbetrachtung}
\label{fehlerbetrachtung}

Bei der Korrelation der experimentellen und numerischen Ergebnisse
müssen die verschiedenen Fehlereinflüsse betrachtet werden, die durch
die idealen Modellannahmen, die technologiebedingten Baulementetoleranzen
und Meßungenauigkeiten hervorgerufen werden. In diesem Kapitel sollen die
Fehler, soweit sie nicht weiter quantifizierbar sind, der Größnordnung nach
abgeschätzt werden. Falls die analytischen Zusammenhänge bekannt sind,
können mit Hilfe der {\em Gauß}schen Fehlerfortpflanzung die Fehlereinflüsse
quantitativ berechnet werden. Danach ergibt sich für eine Funktion
$f(x_{1}, x_{2},...,x_{n})$ aufgrund der mit den Fehlern $\Delta x_{i}$
behafteten unabhängigen Eingangsgrößen $x_{i}$ ein Fortpflanzungsfehler:
\begin{eqnarray}
\label{gaussfehler}
 {\Delta f} & = & \sqrt{ \sum_{i=1}^{n}
   {\left( \frac{\partial f}{\partial x_{i}} \Delta x_{i} \right)}^2 }
\end{eqnarray}
Beim dynamischen Verhalten mikromechanischer Strukturen interessieren in
diesem Zusammenhang insbesondere die Fehlergrößn der Resonanzfrequenz
$f_{0}$ der unbelasteten Mikroresonatoren und die Frequenzänderung
$\Delta f$ aufgrund der Einwirkung einer äußeren Meß- oder Störgröße.\\
%
Infolge von Geometrietoleranzen und der nur ungenau bekannten
Materialeigenschaften folgt für den Fehler der Resonanzfrequenz
unter Zugrundelegung der Gleichungen (\ref{balkfreq}) und (\ref{memfreq}):\\
\begin{eqnarray}
\label{fehlerfreq}
 \frac{{\Delta f}}{f} & = & \sqrt{
       {\left( \frac{\Delta h}{h} \right)}^2  +
       {\left( \frac{2\Delta l}{l} \right)}^2 +
       {\left( \frac{\Delta \hat E}{2 \hat E} \right)}^2 +
       {\left( \frac{\Delta \rho}{2 \rho} \right)}^2 }
\end{eqnarray}
Der Einfluß der prozeßbedingten Vorspannung einer Bimorphstruktur auf die
Resonanzfrequenz kann nach Gleichung~(\ref{fsigma}) abgeschätzt werden und
liefert bei der Auflösung nach $\sigma$ den Zusammenhang:
\begin{eqnarray}
\label{sigmageom}
 \sigma & = & \left[ \left( \frac{f_{\sigma}}{f_{0}} \right) - 1 \right]
              \cdot \frac{\hat E}{c_{\sigma}}
              \left( \frac{h}{l} \right)^2
\end{eqnarray}
Diese Gleichung gilt einerseits nur fr im Vergleich zur Substratdicke dünne
Schichten, d.h.\ $h_{Si}\geq$ 5--10~$h_{Piezo}$, da sonst die ideal starre
Einspannung als Randbedingung nicht mehr erfüllt ist. Andererseits darf der
Radikand unter dem Wurzelausdruck von Gleichung~(\ref{fsigma}) nicht
verschwinden. Unter diesen Voraussetzungen ist es möglich ausgehend von der
Frequenzverschiebung $\Delta f = f_{\sigma} - f_{0}$ auf die Größnordnung
der inneren Vorspannung zurückzuschließen, sofern andere Einflüsse
ausgeschlossen werden
können. Neben dem in Kapitel~4.4 behandelten Temperatureinfluß stellen
andererseits nach Gleichung~(\ref{fehlerfreq}) Geometrietoleranzen und nur
ungenau bekannte Materialeigenschaften mögliche Einflußgrößen dar, die zu
einer {\em scheinbaren} Frequenzverschiebung führen können. Bei einem
typischen Frequenzfehler von etwa $\Delta f/f_{0}~\approx$~$\pm$20~\% führt
dieses bereits auf eine {\em scheinbare} innere Spannung von
$\sigma~\approx~\pm15$~MPa. Innerhalb dieses Spannungsbereiches ist eine
Abschätzung der Vorspannung, die durch Dünnschichten hervorgerufen werden,
daher nicht mehr eindeutig möglich.


\newpage
{\bf Modellbedingte Fehlereinflüsse:}

Der Einfluß der Modellfehler ensteht im wesentlichen durch die getroffene
Idealisierung, sowie die verwendeten Elementtypen und eingesetzten
numerischen Verfahren. Die systematischen Untersuchungen in Kapitel~4.1
haben gezeigt, daß durch die Wahl von Schalen- und Volumenelementen das
Konvergenzverhalten stark beeinflußt werden kann. Die Abweichungen der
numerisch ermittelten Resonanzfrequenzen vom analytischen Vergleichswert
betragen bei Membranstrukturen weniger als 0,1~\% für FE-Modelle mit etwa
7000 Knoten. Im Vergleich dazu erlauben Schalenmodelle eine wesentlich
genauere Beschreibung, bereits bei weniger als 1000 Knoten.
Ferner ist der Einfluß anisotroper
Materialeigenschaften bei Zungen- und Balkenstrukturen vernachlässigbar,
sofern die Orientierung der Strukturen entlang der (110)-Siliziumebenen
berücksichtigt wird. Bei einer (100)-Orientierung der Membranen können
die Abweichungen allerdings bis zu 11~\% betragen. Den weitaus größten
Einfluß besitzen allerdings Geometrietoleranzen und innere
Schichtspannungen, die im folgenden daher ausführlicher diskutiert werden.


{\bf Technologiebedingte Bauelementtoleranzen:}

Da die erzielbare Genauigkeit der hergestellten Balken- und Membrandicken
beim anisotropen Naßätzen ohne Ätzstoppschichten ungenügend ist,
resultieren hieraus hohe Fertigungstoleranzen. Dieser Prozeßschritt ist
z.Zt.\ nur über die Ätzzeit kontrollierbar, so daß im Gegensatz zum
elektrochemischen
Naßätzen {\em kein} objektives Ätzstoppkriterium gegeben ist. Dieses
hat zur Folge, daá die Strukturdicken eine hohe Streuung von Bauelement
zu Bauelement aufweisen {\em und} es nicht möglich ist, homogene Balken- und
Membrandicken über die Bauelementlänge zu erreichen.
Eine direkte Konsequenz hiervon ist die Balligkeit der von der
Waferrückseite freigeätzten Siliziummembranen
zur Mitte hin. Weitere Ursachen für die z.T.\ erheblichen Maßabweichungen
sind zum einen die schlechte Lackhaftung auf der Waferoberfläche und
Fehlpositionierungen der Vorder- zur Rückseitenmaske bezüglich der
Siliziumkristallstruktur. Hinzu kommt, daß die Toleranz der Flatjustierung
bezüglich der (110)-Ebenen bei Siliziumwafern etwa $\pm1,5^{\circ}$ beträgt.
Die Dicken der 4''-Siliziumwafer schwanken außerdem etwa um
$\pm(2$--5)~$\mu$m über den Gesamtwafer. Zusätzlich wirken sich die
Waferdickenschwankungen, die von den Herstellern zu (525$\pm$25)~$\mu$m
spezifiziert werden, auf die Bauelementtoleranzen aus. \\
%
Die Bestimmung der Bauelementelängen $l$ und -dicken $h$ erfolgte
an Schliffbildern und Bruchkanten lichtmikroskopisch sowie unter
dem Rasterelektronenmikroskop (REM).
So betrug die Balligkeit von 9,2~mm langen Siliziummembranen mit einer
Solldicke von 50~$\mu$m etwa 14~$\mu$m. Am Membranrand wurde eine
Dicke von $(40\pm2)~\mu$m und in Membranmitte eine Dicke von
($54\pm3)~\mu$m gemessen. Die Schichtdicke der $ZnO$-Schicht betrug
bei einer Solldicke von 15~$\mu$m etwa 11--13~$\mu$m mit einer Schwankung
von etwa $\pm0,7~\mu$m über die Membrangesamtlänge.
In {\bf Tabelle~\ref{tabgeomfehler}} sind die Bauelementstreuungen und
-toleranzen der im Rahmen dieser Arbeit untersuchten mikromechanischen
Strukturen zusammengestellt.
%----------------------- Beginn: table ---------------------------
\begin{table}[htb]
\caption{\label{tabgeomfehler}
 Maßabweichungen bei mikromechanischen Strukturen}
\begin{center}
\begin{tabular}{|l||c|c|c|c|}
\hline
 {\bf Struktur} &  Größe   &  Sollmaß  & Streuung  & Toleranz \\
\hline \hline
 Balken   &  Dicke   &  50~$\mu$m  & 30--60~$\mu$m  &  $\pm$(5--10)~\%   \\
          &  Länge   &  3--10~mm   & ---            &  $\leq \pm0,5$~\%  \\
\hline
 Membran  &  Dicke   &  20--200~$\mu$m   & ---  &  $\pm$(5--30)~\%    \\
          &  Länge   &  10~mm   &  9,2--9,5~mm  &  $\leq \pm$0,5~\%   \\
\hline
\end{tabular}
\end{center}
\end{table}
%----------------------- Ende: table ---------------------------
Die Abweichungen bei den Bauelementlängen fallen kaum ins Gewicht und
können zusätzlich über Vorhaltemaße
im Maskenlayout kompensiert werden. Eine kritische Größe ist hingegen die
Bauelementdicke, deren Streuungen und Toleranzen erheblich eingehen und
im Extremfall bis zu 30~\% betragen können. Zieht man zusätzlich die
Schwankungen bei den Materialeigenschaften hinzu, so ergibt sich für die
Resonanzfrequenz ein maximaler relativer Fehler von etwa:
\begin{eqnarray}
\label{relfehlerfreq}
 \frac{\Delta f}{f} & = & \sqrt{
       {\left( \frac{\Delta h}{h} \right)}^2 +
       {\left( \frac{2\Delta l}{l} \right)}^2 +
       {\left( \frac{\Delta \hat E}{2 \hat E} \right)}^2 +
       {\left( \frac{\Delta \rho}{2 \rho} \right)}^2 }
% \nonumber \\
   \, \approx \, \frac{\Delta h}{h} \, \approx \, 31~\%
\end{eqnarray}
Somit wird im wesentlichen der Fehler durch die Dickenschwankung
$\Delta h/h$ hervorgerufen. Zur Reduktion dieses Fehlers ist es daher
dringend erforderlich, Ätzstoppschichten bei der vertikalen Strukturierung
mikromechanischer Strukturen einzusetzen. \\
%
Weitere Fehlereinflüsse resultieren aus den unberücksichtigten Klebschichten
(Epoxidharz: UHU Plus Endfest 300 bzw.\ Cyancrylatkleber) mit denen die
mikromechanischen Bauelemente auf Metallträger zur Vermessung fixiert werden.
Typische Klebschichtdicken sind hierbei etwa $(30\pm10)~\mu$m. Zusätzlich
haben die Klebermaterialeigenschaften, wie Härte und Viskosität, einen
erheblichen Einfluß auf die Starrheit der Baulementefixierung.
Inhomogenitäten der Klebschichten führen zusätzlich zu undefinierten
Einspannbedingungen, so daß die Mikroresonatoren
nicht ideal starr eingespannt sind und die Resonanzfrequenzen niedriger
ausfallen. Desweiteren wirken sich die Klebschichten auf das
Temperaturverhalten der Bauelemente aus, so daß bei Temperaturaushärtung
($\Delta T \approx 120~^{\circ}$C) thermisch induzierte Spannungen auftreten,
bei Membranbauelementen beim Abkühlen sogar zur Zerstörung führen können.
Die Einflüsse der Klebschichten können nicht direkt abgeschätzt werden, da
neben den ungenauen Klebschichtdicken vor allem die Materialeigenschaften
der Kleber z.T.\ unbekannt sind, so daß hier weitergehende, detaillierte
Untersuchungen erforderlich sind \cite{Jauch}.


{\bf Meßungenauigkeiten:}

Die Meßfehler bei der Bestimmung der Resonanzfrequenzen hängen in erster
Linie vom eingesetzten Meßverfahren ab. Die Frequenzen wurden in der Regel
mit einem Laservibrometer optisch vermessen. Ausschlaggebend für die
Genauigkeit der Frequenzmessungen im untersuchten Meßbereich ist die
Auflösung des eingesetzten Spektrumanalysators ({\sl HP3588A}).
Die Auflösungsbandbreite ({\em Resolution-Bandwidth}) betrug bei den
schmalbandigsten Messungen etwa 5--10~Hz, bei einer
überdeckten Frequenzspanne von etwa 500 bzw.\ 1000~Hz.
Die Reproduzierbarkeit der Frequenzmessungen lag im Bereich von
$\pm$2~\%.
Die Meßgenauigkeit der statischen Auslenkungen beträgt etwa
$\pm1$~$\mu$m beim UBM-Autofocusmeßsystem.
In {\bf Tabelle~\ref{tabmessfehler}} sind die
Größenordnungen der Meßfehler ab\-schließend zusammengefaßt.
%----------------------- Beginn: table ---------------------------
\begin{table}[htb]
\caption{\label{tabmessfehler}
 Größenordnung der Meßfehler}
\begin{center}
\begin{tabular}{|l||c|c|}
\hline
  Meßfehler  &  Größe   &  Wert  \\
\hline \hline
Frequenz          &  $\Delta f/f$   &  $\pm$(2--5)~\%   \\
dyn. Amplitude    &  $\Delta A/A$   &  $\pm5$~\%      \\
stat. Auslenkung  &  $\Delta u$     &  $\pm1~\mu$m    \\
Schwingungsgüte   &  $\Delta Q/Q$   &  $\pm$(10--30)~\% \\
\hline
\end{tabular}
\end{center}
\end{table}
%----------------------- Ende: table ---------------------------



\newpage
\section{Zusammenstellung der Resultate}
\label{resultateschwingungsverhalten}

Die wichtigsten Ergebnisse der Modellierung und experimentellen Untersuchung
des Schwingungsverhaltens mikromechanischer Strukturen werden im folgenden
kurz zusammengestellt:


{\bf Modellierung resonanter Mikrostrukturen:}

\begin{itemize}
\item
Die vereinfachenden, analytischen Näherungsgleichungen können für die
Membrangrobdimensionierung und als Startwerte für die numerischen
FE-Berechnungen dienen. Für eine detailiertere Beschreibung des
lastabhängigen dynamischen Verhaltens mikromechanischer Sensoren,
insbesondere der spannungsversteifenden Effekte sind geometrischen
Nichtlinearitäten zu berücksichtigen. Die {\sf ANSYS}-Programmoptionen
'Large Deflections' und 'Stress Stiffening' erlauben die Simulation
dieser überlagerten nichtlinearen Effekte.
\item
Bei der Verwendung von Volumenelementen ist eine etwa achtfache Erhöhung der
Knotenanzahl erforderlich, um die gleiche Rechengenauigkeit wie bei
Schalenelementen zu erreichen. Bei der Modellierung von mikromechanischen
Membranen mit sehr großen Aspektverhältnissen ($l/h \geq 50$) ist das
Approximationsverhalten von Schalenelementen um etwa zwei Größenordnungen
besser und im wesentlichen durch die laterale Elementunterteilung
entlang der Membranseitenkante bestimmt. Weiterhin wurde nachgewiesen, daß
die numerischen Fehler bei der Berechnung der Resonanzfrequenzen und
lastabhängigen Bauelementauslenkungen minimiert wurden und im Vergleich
zu den Material- und Geometriefehlereinflüsse vernachlässigbar sind.
\item
Die Vernetzung bei dynamischen Problemen muß so erfolgen, daß die
Steifigkeit und {\em gleichzeitig} die Massenverteilung der schwingenden
Struktur richtig beschrieben wird. Hierzu sind geeignete finite Elemente,
z.B.\ Schalenelemente oder Volumenelemente mit linearen bzw.\
quadratischen Ansatzfunktionen zu wählen. Die Anzahl der dynamischen
Freiheitsgrade sollte etwa ein zehntel der Knotenanzahl des FE-Modells
betragen.
Die Wahl der Hauptfreiheitsgrade (MDOF) sollte in Hauptverformungsrichtung
erfolgen, wo große Massen liegen (d.h.\ $K/M$ klein), an lose befestigten
Teilen mit geringer Strukursteifigkeit $K$ und nicht im Bereich der
Einspannung.
\item
Bei der Modalanalyse wird die Approximationsgüte der Resonanzfrequenzen
für höhere Schwingungsmoden schlechter, so daß bei der Abtastung höherer
Eigenschwingungsformen eine entsprechend hohe laterale
Elementunterteilung gewählt werden muß.
\item
Ein Vergleich von anisotropen und isotropen Berechnungen zeigt, daß in
guter Näherung für (100)-orientierte, entlang der [100]-Kante (Flatrichtung)
ausgerichtete Siliziumstrukturen mit den isotropen Materialersatzdaten aus
Gleichung (\ref{simat}) sowohl statische als auch dynamische FE-Berechnungen
durchgeführt werden können.
\end{itemize}


{\bf Realisierung mikromechanischer Resonanzsensoren:}

Die für Balken- und Membranresonatoren entwickelten FE-Modelle wurden
meßtechnisch verifiziert und beschreiben das statische {\em und}
dynamische Verhalten im Rahmen der Modell- und Meßfehler. Sie wurden zur
Dimensionierung frequenzanaloger Siliziumsensoren herangezogen. Die
wichtigsten Ergebnissen der realisierten Kraft- und Drucksensoren lassen
sich wie folgt zusammenfassen:
\begin{itemize}
\item
Mit Hilfe von piezoelektrischen $ZnO$-Dünnschichten wurden resonante
Silizium-Membrandrucksensoren mit Membranseitenlängen von 9,2~mm
und Gesamtdicken von etwa 61~$\mu$m realisiert. Die Drucksensoren
zeichnen sich durch eine gemessene Druckempfindlichkeit von 12,4~Hz/mbar
in einem Druckbereich bis 500~mbar aus. Die maximale Membranmittenauslenkung
beträgt bei diesem Maximaldruck etwa 75~$\mu$m. Die innere
Spannung der $ZnO$-Dünnschichten konnte durch Vermessung der unbelasteten
Resonanzfrequenz der Grundbiegeschwingungsmode abgeschätzt werden.
\item
Auf der Basis von 10~mm langen, etwa 50~$\mu$m dünnen Siliziumbiegebalken
wurden elektrothermisch mit Hilfe von $NiCr$-Dehnmeßstreifen
angetriebene Kraftsensoren realisiert, die sich durch eine
Kraftempfindlichkeit von 168 Hz/N auszeichnen. Dieses entspricht einer
relativen Kraftempfindlichkeit von etwa 0,0039~N$^{-1}$. Die
Schwingungsgüte der Grundbiegeschwingungsmode beträgt etwa 430,
die Nichtlinearität der Frequenz-Kraft-Kennlinie 1,9~\%.
\item
Durch ein Redesign des Widerstandlayouts konnte die
Temperaturquerempfindlichkeit des Sensors um das Vierfache auf
5~Hz/K erniedrigt werden. Die kritische
Temperaturdifferenz $\Delta T_{kr}$, bei der der Siliziumbalken ausknickt,
wurde meßtechnisch zu etwa (27$\pm3$)~K bestimmt.
Für den praktischen Einsatz der elektrothermisch angeregten Resonatoren
muß der Betrieb unter- {\em oder} oberhalb dieser kritischen
Temperaturüberhöhung stattfinden, um eine eindeutige
Zuordnung zwischen Meßgröße und Frequenz zu ermöglichen.
\end{itemize}
