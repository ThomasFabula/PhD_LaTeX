\chapter{Piezoelektrische Dnnschichtstrukturen}
\label{piezoelektrisch}

Dieses Kapitel besch„ftigt sich mit der Modellierung und experimentellen
Charakterisierung von piezoelektrisch angetriebenen Mikrostrukturen.
Mit Hilfe der im vorhergehenden Kapitel entwickelten FE-Modelle wird
das dynamische Verhalten von mikromechanischen Resonatoren in Bimorphaufbau
unter Bercksichtigung des elektromechanischen Wandlungsprinzips simuliert.
Fr die effiziente Anregung der Bimorphstrukturen war eine geeignete
Auslegung der Elektrodengeometrie erforderlich und das Verh„ltnis der
Schichtdicken zu optimieren. Weiterhin wird der Einfluá des piezoelektrischen
Dnnschichtsystems untersucht und die erreichbaren elektromechanischen
Kopplungsfaktoren verschiedener Wand\-lergeometrien berechnet.
Bei der Ermittlung des effektiven elektromechanischen Kopplungsfaktors, der
die Anregungseffizienz zu charakterisieren erlaubt, mssen neben den
mechanischen auch die elektrischen Randbedingungen bercksichtigt werden.
Mit Hilfe der piezoelektrischen Anregung k”nnen neben dem mechanischen
Schwingungsverhalten der Sensoren auch verschiedene elektrische
Kenngr”áen, wie z.B.\ das frequenzabh„ngige Impedanz- und Phasenverhalten
abgeleitet werden. Auf diese Weise konnte mit Hilfe der
gekoppelten Feldberechnungen bei piezoelektrisch angetriebenen
Membrandrucksensoren die Modenselektivit„t verbessert und ein unimodaler
Betrieb meátechnisch nachgewiesen werden. Weiterhin konnte rechnerisch
gezeigt werden, daá die Temperaturquerempfindlichkeit von
Membrandrucksensoren im Bimorphaufbau erheblich reduziert werden kann.


\section{FE-Modell der Bimorphstruktur}
\label{bimorphstruktur}

Im vorangegangenem Kapitel wurden mit Hilfe der Modalanalyse die
Eigenfrequenzen und -schwingungsformen von resonanten Druck- und
Kraftsensoren rechnerisch und experimentell untersucht. Bei den
piezoelektrisch betriebenen Membrandrucksensoren wurden die mechanische
Eigenschaften der $ZnO$-Dnnschichten bercksichtigt, aber der
Einfluá der elektrischen Randbedingungen vernachl„ssigt. Bei
piezoelektrischen Medien wird infolge der bidirektionalen
elektromechanischen Wechselwirkung das mechanische Verhalten der
Bimorphwandler durch die „uáere elektrische Beschaltung beeinfluát.
Daher werden im folgenden die elektrischen Randbedingungen mit
bercksichtigt.\\
Hierfr wurde ein dreidimensionales FE-Modell entwickelt, das
in {\bf Abbildung~\ref{abbbimorph}} dargestellt ist.
%----------------------- Beginn: Figure-Environment ----------------------
\begin{figure}[htb]
%\vspace*{8cm}

\begin{center}
% --- Dateiname des Bildes
\input{abbfein.tex}
\setabbfein
\end{center}
\caption{\label{abbbimorph}
 FE-Modell einer Silizium-Bimorphmembran}
\end{figure}
%----------------------- Ende: Figure-Environment ----------------------
Aus Symmetriegrnden
handelt es sich um eine Viertelmembran, die ganzfl„chig von einer
piezoelektrischen Dnnschicht berdeckt ist. Die $Al$-Elektroden wurden
im FE-Modell vernachl„ssigt, da sie im Vergleich zur piezoelektrischen
Dnnschicht sehr dnn sind ($h_{ZnO} \approx 10-100 \; h_{Al}$). Die
schr„ge
Einspannung entspricht den „tzbegrenzenden (111)-Siliziumebenen, die durch
den anisotropen naáchemischen Herstellungsprozeá der Siliziummembran
bedingt sind. Die Abmessungen der
Sensormembran variieren je nach Anwendungsbereich. Die in dieser Arbeit
betrachteten mikromechanisch gefertigten Membranen weisen eine
Kantenl„nge von 5 bzw.\ 9,2~mm und Membrandicken von 20--200~$\mu$m auf.
Der Membransockel hat eine untere Breite von 5~mm und eine H”he von
etwa 0,525~mm, die durch die Dicke des Siliziumwafers
gegeben ist.\\
Zur Eingrenzung der Gltigkeit der analytischen Beschreibungsans„tze
fr Bimorphstrukturen wurde die Resonanzfrequenz in Abh„ngigkeit der
Gesamtmembrandicke betrachtet, um den Einfluá des zus„tzlichen
Dnnschichtsystems zu untersuchen.
In {\bf Abbildung~\ref{abbvglbulktf}} ist die Abh„ngigkeit der
Resonanzfrequenz einer Silizium-Bimorphmembran mit einer piezoelektrischen
$ZnO$-Dnnschicht von der Gesamtmembrandicke dargestellt. Die Dicke der
Siliziummembran betrug 20~$\mu$m und die Dicke der $ZnO$-Schicht wurde
von 1 bis 50~$\mu$m variiert. Die analytische N„herungsformel nach
Gleichung (\ref{memfreq}) geht von einer ideal homogenen, unendlich steif
eingespannten Platte aus, wobei der Bimorphcharakter lediglich durch eine
Gewichtung der Materialdaten ($\hat E, \rho$) ber die Schichtdicken
bercksichtigt wird.
Zur Vereinfachung wurde bei Silizium mit isotropen Materialeigenschaften
gerechnet (Gleichung~\ref{simat}), so daá dreidimensionale, lineare
Volumenelemente ({\em SOLID45}) verwendet werden konnten. Aufgrund der
hexagonalen Kristallsymmetrie der piezoelektrischen Schicht
(siehe Abbildung~\ref{abbmatrix6mm}) muá die Anisotropie der Elastizit„ts-
und piezoelektrischen Eigenschaften bercksichtigt werden. Hierzu bietet
{\sf ANSYS} das dreidimensionale {\em Multi-Field}-Element {\em SOLID5} an,
bei dem die Strukturfreiheitsgrade ${\vec u}$ und der elektrische
Potential-Freiheitsgrad $\phi$ verwendet wurden. Bei der $ZnO$-Dnnschicht
sind die anisotropen Materialkennwerte aus Tabelle~\ref{tabpiezoelektrika}
herangezogen worden. \\
%----------------------- Beginn: Figure-Environment ----------------------
\begin{figure}[htb]
%\vspace*{8cm}

\begin{center}
% --- Dateiname des Bildes
\input{abbfzw.tex}
\setabbfzw
\end{center}
\caption{\label{abbvglbulktf}
 Resonanzfrequenzen einer Silizium-Bimorphmembran mit piezoelektrischer
 $ZnO$-Dnnschicht in Abh„ngigkeit der Membrangesamtdicke und elektrisches
 Ersatzschaltbild fr piezoelektrische Resonatoren}
\end{figure}
%----------------------- Ende: Figure-Environment ----------------------
Entsprechend dem elektrischen Ersatzschaltbild fr elektromechanisch
gekoppelte Schwinger \cite{IEEE}, das in Abbildung~\ref{abbvglbulktf}
skizziert ist, entspricht die numerisch berechnete Serienresonanzfrequenz
$f_{s}$ der Frequenz beim elektrischen Kurzschluá ($\vec E=0$), die
Parallelresonanzfrequenz $f_{p}$ dem elektrischen Leerlauf bei offenen
Anschluáelektroden ($\vec D=0$) \cite{Nai83}. Unter Zugrundelegung dieses
einfachen
Ersatzschaltbildes\footnote{Das elektrische Ersatzschaltbild von
piezoelektrischen Dnnschichtstrukturen ist stark vom Schichtaufbau des
Wandlers abh„ngig. Zus„tzlich ist zu unterscheiden, ob der Wandler bzw.\
Schwinger als elektrisches Zweipol- oder Vierpolnetzwerk betrieben wird.
Resonante Silizium-Kraftsensoren sind von {\sl van Mullem et al.} mit
$Si/ZnO/SiO_{2}/Si_{3}N_{4}$--Schichtaufbau im Vierpolbetrieb betrieben
und ein elektrisches Ersatzschaltbild abgeleitet worden, das insbesondere
das elektrische šbersprechverhalten zu beschreiben vermag \cite{Mul92}.
Ein Zweipolersatzschaltbild fr elektromechanische Balkenresonatoren mit
vereinfachtem $Si/ZnO$-Aufbau ist im Rahmen einer Diplomarbeit entwickelt
worden \cite{Qui93}. Es vermag die piezoelektrische Kopplung und den Einfluá
einer konstanten axialen Zugbelastung prinzipiell zu beschreiben,
bercksichtigt aber kein spezifisches Dnnschichtsystem (z.B.\
Schichtwiderst„nde und -kapazit„ten), sowie elektrische St”rpfade.},
das nur im unmittelbaren Frequenzbereich der
betrachteten Schwingungsmode Gltigkeit besitzt, kann der piezoelektrisch
angetriebene Resonator als ein elektrischer Zweipol aufgefaát werden.
Das Ersatzschaltbild geht von einem verlustbehafteten Resonanzschwingkreis
aus. Der Serienschwingkreis setzt sich aus dem in Reihe geschalteten
dynamischen Verlustwiderstand $R$, der dynamischen Induktivit„t $L$ und der
dynamischen Kapazit„t $C$, sowie einer dazu parallel geschalteten statischen
Kapazit„t $C_{0}$ zusammen. Der eigentliche Resonator wird durch die drei
dynamischen Ersatzdaten beschrieben, w„hrend $C_{0}$ durch die statische
Kapazit„t der Anschluáelektroden der piezoelektrischen Schicht und die
elektrischen Zuleitungen gebildet wird. Der komplexe Widerstand (Impedanz)
des Resonators wird bei der Serien- und Parallelresonanzfrequenz reell.
Unterhalb der Serienresonanzfrequenz und oberhalb der
Parallelresonanzfrequenz ist das Wandlerverhalten rein kapazitiv, und durch
den Wert der Kapazit„t $C_{0}$ bestimmt. Bei der
Serienresonanzfrequenz hat die Admittanz des Ersatzschaltkreises ein
Maximum, bzw.\ die Impedanz ein lokales Minimum und der Wandler weist ein
rein ohmsches Verhalten mit dem Verlustwiderstand $R$ auf. Fr Frequenzen
$f_{s} < f < f_{p}$ verh„lt sich der Schwinger induktiv. Mit Hilfe der
elektrischen Ersatzdaten $R, L, C$ und $C_{0}$ lassen sich die
beiden diskreten Resonanzfrequenzen ermitteln:
\begin{eqnarray}
\label{fserfpar}
 \omega_{s}   & = & \frac{1}{\sqrt{LC}} \nonumber \\
 \omega_{p}   & = & \omega_{s} \sqrt{1+\frac{C}{C_{0}}}
\end{eqnarray}
Die Differenz zwischen der Serienresonanzfrequenz und der
Parallelresonanzfrequenz wird durch die unterschiedlichen Steifigkeiten
der piezoelektrischen Schicht in Abh„ngigkeit der „uáeren Beschaltung
verursacht und ist ein direktes Maá fr den effektiven elektromechanischen
Kopplungsfaktor (siehe Gleichung \ref{keffnaeherung}). Beim elektrischen
Kurzschluá k”nnen die durch die mechanische Verbiegung des Bimorphs
induzierten elektrischen Ladungen ab\-flieáen, w„hrend sich bei offenen
Elektroden ein zus„tzliches
elektrisches Feld in der piezoelektrischen Schicht aufbauen kann, das zu
einer Versteifung des Wandlerelementes fhrt. Diese piezoelektrisch
bedingte Versteifung erh”ht den Wert der Parallelresonanzfrequenz beim
elektrischen Leerlauf, so daá $f_{s} < f_{p}$.
Die analytisch berechnete \glqq rein mechanische\grqq \, Resonanzfrequenz
$f_{Bulk}$ entspricht der Serienresonanzfrequenz $f_{s}$ aufgrund der
Verwendung der Steifigkeitskoeffizienten $S_{11}^{E}$ ($\vec E$~=~ $const$)
aus Tabelle~\ref{tabpiezoelektrika} fr Zinkoxid. Analog k”nnte die
Parallelresonanzfrequenz $f_{p}$ bei der Verwendung von $S_{11}^{D}$
($\vec D$~=~$const$) berechnet werden, falls der Zusammenhang zwischen
$S^{E}$ und $S^{D}$, der durch die elektromechanische Kopplung
beschrieben wird, vorab bekannt w„re.\\
W„hrend fr dnne $ZnO$-Schichtdicken die analytische L”sung mit den
FE-Ergebnissen gut bereinstimmt, wird die Abweichung ab einem
Schichtdickenverh„ltnis von $h_{ZnO}/h_{Si} \ge 1$ signifikant, da die
Einspannbedingung auch fr die $ZnO$-Schicht ungltigerweise angenommen wird.
Beim FE-Modell wird der Einfluá der seitlich frei beweglichen $ZnO$-Schicht
mit zunehmender Schichtdicke gr”áer, so daá die Abweichung zwischen dem
analytischen und dem
numerisch Frequenzwert von etwa 1~\% bis auf 7~\% bei der maximalen
Schichtdicke von 70~$\mu$m zunimmt. \\
%
Der Einfluá der herstellungsprozeáabh„ngigen Dnnschichtdaten ist
in Abbildung~\ref{abbvglbulktf} ebenfalls dargestellt. Der an reaktiv
gesputterten $ZnO$-Dnnschichten \cite{Wag94} bestimmte reduzierte E-Modul
$\hat E_{tf}$ (siehe Kapitel~\ref{druckabh}) f„llt gegenber dem
Bulkwert $\hat E_{Bulk}$ um etwa 13~\% niedriger aus, so daá die
analytisch berechneten Resonanzfrequenzen $f_{tf}$ unterhalb der
Werte $f_{Bulk}$ bleiben. Die Abweichung der Frequenzwerte infolge
der Verwendung von Dnnschichtmaterialdaten bleibt im Mittel unterhalb
von 1~\% und ist damit fr dieses spezielle Schichtsystem vernachl„ssigbar.
Die Diskussion der Modelleinflsse, die durch die Elementierung und die
mechanischen Randbedingungen hervorgerufen werden,
ist Gegenstand des folgenden Kapitels.\\



\section{Einflsse der Modellparameter}
\label{piezomodell}

Verschiedene Plattenmodelle mit idealer, unendlich steifer Einspannung
und dreidimensionale Membranmodelle mit realer Membraneinspannung
wurden erstellt, um den Einfluá der FE-Modellparameter und der
Randbedingungen separat untersuchen zu k”nnen.
Bei den FE-Modellen wurde die Elementvernetzung, d.h.\ die „quidistante
Seitenteilung entlang der Membran und damit die Aspektverh„ltnisse der
Elemente variiert. Zus„tzlich wurde der Einfluá der dynamischen
Hauptfreiheitsgrade (MDOF) untersucht. Als mechanische Randbedingungen
wurden die Knotenverschiebungen an den Membranseiten gesperrt. Damit ist
die Membranen an ihrer Unterseite mechanisch fixiert. Die Modellierung
der Elektroden erfolgte durch Definition von elektrischen Randbedingungen,
indem konstante Potentiale ($\phi = 0$) an die Knoten der Oberfl„che der
Siliziummembran und Piezoschicht gelegt wurden. Die Seitenl„nge der
untersuchten Silizium-Bimorphmembran betrug 9,2~mm. Die Schichtdicken
der Siliziummembran $h_{Si}$ und der $ZnO$-Schicht $h_{ZnO}$ wurden
zu 20~$\mu$m gesetzt.\\
%
Bei den Plattenmodellen (P1--P4) handelt es sich um einfache FE-Modelle
mit idealer Randeinspannung und jeweils einer Elementlage ber die Dicke
der Siliziummembran und der $ZnO$-Schicht.
Um den Einfluá unterschiedlicher Randbedingungen zu analysieren, wurden
verschiedene Knotenlagen des Plattenmodells eingespannt. Beim Modell P1 ist
nur die untere „uáere Knotenlage in den Verschiebungen ($\vec u=0$) gesperrt.
Daher kann die Membran sich um diese Knotenlage verdrehen. Hierdurch wird
die Einspannsteifigkeit effektiv vermindert und die Resonanzfrequenzen
fallen entsprechend niedriger aus. Beim Modell P2 wurde der gesamte Rand der
Siliziummembran eingespannt und nur die $ZnO$-Schicht frei gelassen, so daá
sie seitlich frei beweglich war. Dieses entspricht einer mittleren
Steifigkeit. Modell~P3 wies eine volle Einspannung fr alle Knotenlagen des
gesamten Membranrandes auf. Entsprechend fallen die Frequenzen am h”chsten
aus. Infolge der unterschiedlichen Randeinspannungsbedingungen „ndern sich
bei den Plattenmodellen die Resonanzfrequenzen bis zu 30~\%. \\
%
Die Membranmodelle (M1--M4) bercksichtigen zus„tzlich die reale
Membraneinspannung durch die (111)-Siliziumebenen, so daá sich die
Frequenzwerte infolge verminderter Einspannsteifigkeit erniedrigen.
Gleichzeitig steigt jedoch der Modellieraufwand betr„chtlich an, bei
etwa drei- bis fnffach erh”hter Rechenzeit. Die Resonanzfrequenzen nehmen
mit steigender Elementanzahl wie erwartet ab, wobei sich der Wert fr den
effektiven Kopplungsfaktor weitgehend unabh„ngig von der Netzverfeinerung
auf etwa 20~\% stabilisiert. Erst bei einer drastischen Erh”hung der
Element- und Knotenanzahl beim Referenzmodell MR, sowie gleichzeitiger
Bercksichtigung der Materialanisotropien von Silizium {\em und} Zinkoxid
erniedrigen sich die Frequenzwerte auf $f_{s} = 4677$~Hz und
$f_{p} = 4755$~Hz, aus denen sich ein Kopplungsfaktor von 18~\% errechnet.
Eine analytische šberschlagsrechnung nach Gleichung (\ref{memfreq}) liefert
mit gewichteten Materialdaten fr die Resonanzfrequenz einen Wert von etwa
4,91~kHz. Dieses entspricht einer Abweichung von 5~\% im Vergleich zum
numerisch ermittelten Wert der Serienresonanzfrequenz. \\
%----------------------- Beginn: table ---------------------------
\begin{table}[htb]
\caption{\label{tabmodellparamkeff}
 Einfluá der Modellparameter bei der Berechnung der Resonanzfrequenzen
 $f_{s}$, $f_{p}$ und des effektiven elektromechanischen
 Kopplungsfaktors $k_{eff}$}
\begin{center}
\begin{tabular} {|l||c|c|c|c||c|c|c|c||c|}
\hline
 FE-Modell & P1 & P2 & P3 & P4 & M1 & M2 & M3 & M4 & MR \\
\hline \hline
 Elemente  & 200 & 200 & 200 & 3200 & 450 & 450 & 578 & 1058 & 2714 \\
 Knoten    & 363 & 363 & 363 & 5043 & 795 & 795 & 999 & 1767 & 4286 \\
 MDOF      & 300 & 300 & 300 & 300  & 100 & 100 & 100 & 300  & 300 \\
\hline
 Seitenteil.: & 10 & 10 & 10 & 40 & 13 & 13 & 13 & 19 & 31 \\
 Aspektver.:  & 23 & 23 & 23 & 6  & 18 & 18 & 18 & 12 & 7  \\
\hline \hline
 $f_{s}$ [Hz] & 4016 & 5196 & 5281 & 4907 & 5015 & 4836 & 4871 & 4745 & 4677 \\
 $f_{p}$ [Hz] & 4143 & 5298 & 5378 & 5012 & 5117 & 4931 & 4967 & 4842 & 4755 \\
\hline
 $k^{lin}_{eff}$ [\%]
 & 24,8 & 19,6 & 19,0 & 20,5 & 20,0 & 19,6 & 19,7 & 20,0 & 18,1 \\
\hline
\end{tabular}\\
\end{center}
\end{table}
%----------------------- Ende: table ---------------------------
In {\bf Tabelle~\ref{tabmodellparamkeff}} sind die Ergebnisse der
Parametereinflsse auf die Resonanzfrequenzen und den effektiven
elektromechanischen Kopplungsfaktor $k_{eff}$ der Grundschwingungsmode, der
nach Gleichung~(\ref{keff}) ermittelt wurde, zusammengestellt. Fr kleine
Kopplungsfaktoren ($k_{eff}^{2} \ll 1$ ) gilt n„herungsweise \cite{Sie81}:
%
\begin{eqnarray}
\label{keffnaeherung}
 k_{eff} & = & \sqrt{ \frac{f_{p}^{2} - f_{s}^{2}}{f_{p}^{2}}}
            \approx \sqrt{ 2 \frac{f_{p} - f_{s}}{f_{p}}}
\end{eqnarray}
%
Da die Differenz zwischen beiden Werten fr die hier betrachteten
Kopplungsfaktoren in der Regel kleiner als 0,1~\% ist, wird im weiteren
die {\em lineare} N„herung $k^{lin}_{eff}$ verwendet\footnote{Bei
der Untersuchung von $PZT$-Keramiken, deren Kopplungsfaktor erheblich h”her
als der von piezoelektrischen $AlN$- und $ZnO$-Dnnschichten ist, wird die
exakte Definition nach Gleichung~(\ref{keff}) verwendet.}.\\
Um bei den folgenden FE-Berechnungen den Modellieraufwand in vertretbaren
Grenzen bei gleichzeitig ausreichender numerischer Genauigkeit zu halten,
wird das FE-Modell M1 eingesetzt. Damit l„át sich das tendenzielle Verhalten
der piezoelektrisch angetriebenen Membranstrukturen bei moderaten
Rechenzeiten systematisch untersuchen. Auf der anderen Seite stellen
vielmehr die nur ungenau bekannten Eingangsdaten der Berechnung --
hier sind insbesondere die Materialeigenschaften und die Vorspannung der
$ZnO$-Dnnschicht zu nennen -- erhebliche Einfluáfaktoren dar, die im
nachfolgenden n„her untersucht werden.



\section{Elektromechanischer Kopplungsfaktor}
\label{elektromech}

Zur Verifikation der Bimorphmodelle und šberprfung der
Modellierungsgenauigkeit der piezoelektrischen FE-Berechnungen wurden
Siliziummembranen vermessen, auf die eine
Piezokeramik\footnote{Die Siliziummembranen mit hybrid aufgeklebter
%(Cyanacrylat)
bzw.\ aufgel”teter (Goldschicht) Piezokeramik und
$ZnO$-Beschichtung wurden bereits in einer sehr frhen
Phase des BMFT-Verbundprojektes freundlicherweise von Herrn Dr. G. Flik,
{\sl Robert Bosch GmbH}, Gerlingen, fr Meázwecke zum Abgleich der
FE-Modelle zur Verfgung gestellt.} hybrid aufgebracht war. Da die
Materialeigenschaften von PZT-Piezokeramiken (hier: {\sl VIBRIT}) sehr
genau bekannt sind und im Gegensatz zu piezoelektrischen Dnnschichten
kaum Prozeáschwankungen unterliegen, konnten die Fehler der
Eingangsdaten auf ein Minimum reduziert werden. Insbesondere weisen die
Hybride keine inneren mechanischen Spannungen auf. Mit Hilfe dieser
hybriden Bimorphwandler konnten in der Entwicklungsphase der
Dnnschichtsensoren die piezoelektrischen Modellrechnungen berprft und
das frequenzanaloge Sensorprinzip an Drucksensorprototypen meátechnisch
nachgewiesen werden. Ein zus„tzlicher Vorteil der Piezokeramiken ist ihr
gut meábares elektrisches Impedanzverhalten und die Kenntnis des exakten
elektrischen Ersatzschaltbildes.\\


\subsection{Piezokeramik-Hybride}
\label{hybride}

Das dynamische Verhalten der Bimorphwandler wurde mit optischen und
elektrischen Meámethoden vermessen. Die Abmessungen der untersuchten
Siliziummembranen betrugen 9,2~x~9,2~$mm^{2}$ und die Dicke der
{\sl VIBRIT}-Piezokeramiken jeweils 200~$\mu$m. Mit dem
Laservibrometer wurde das Amplitudenspektrum optisch aufgenommen und die
mechanische Resonanzfrequenz $f_{res}$, die dynamische Resonanzamplitude
und die mechanische Schwingungsgte\footnote{Die mechanische
Schwingungsgte wurde durch die Bestimmung der Halbwertsbreite
$\Delta \omega$ (3dB-Abfall der Maximalamplitude $A_{res}$)
der frequenzabh„ngigen Amplitudenkurve nach Gleichung~(\ref{qmess})
ermittelt.} $Q_{mech}$
der Grundbiegeschwingung $M_{11}$ bestimmt. Mit Hilfe eines
Impedance/Gain-Phase-Analyzers ({\sl HP4194A}) wurde das frequenzabh„ngige
Impedanz- und Phasenverhalten vermessen. Durch Anpassung einer
Impedanzcharakteristik an die gemessenen Kurvenverl„ufe gem„á dem
elektrischen Ersatzschaltbild in Abbildung~\ref{abbvglbulktf} wurden die
elektrischen Ersatzdaten $R, L, C$ und $C_{0}$ ermittelt. Mit Hilfe der
Ersatzdaten lassen sich die elektrische Schwingungsgte und
der effektive elektromechanische Kopplungsfaktor bestimmen \cite{Til93}:
\begin{eqnarray}
\label{qkmelek}
 Q_{elek}       & = & \frac{\omega_{s}L}{R} = \frac{1}{\omega_{s}RC}
                  = \frac{1}{R} \sqrt{\frac{L}{C}} \\
\label{krlc}
 k_{eff}^{RLC}  & = & \sqrt{\frac{C}{C_{0} + C}}
                \approx \sqrt{\frac{C}{C_{0}}}
\end{eqnarray}
W„hrend die Schwingungsgte $Q_{elek}$ durch den ohmschen Widerstand $R$,
der der D„mpfung entspricht, begrenzt ist, h„ngt der Kopplungsfaktor im
wesentlichen
vom Kapazit„tsverh„ltnis $C/C_{0}$ ab, da in der Regel $C_{0} \gg C$
aufgrund der Gr”áe der Anschluáelektroden erfllt ist.
Zur Charakterisierung der Wandlergte wird in der Literatur auch als
\glqq {\em Figure of Merit}\grqq \, die GrӇe $M$ angegeben, die das Produkt
aus der Schwingungsgte und dem Quadrat des Kopplungsfaktors darstellt,
falls $k_{eff} \ll 1$ \cite{IEEE}:
\begin{eqnarray}
\label{merit}
 M & = & \frac{1}{\omega_{s}RC_{0}}
     = Q \frac{k_{eff}^2}{1 - k_{eff}^2} \approx Q k_{eff}^2
\end{eqnarray}
Bei der Optimierung der Wandlereigenschaften spielen beide GrӇen eine
wichtige Rolle und sollten {\em gleichzeitig} maximiert werden.
Der Kopplungsfaktor l„át sich nach Gleichung~(\ref{krlc}) erh”hen, wenn die
Elektrodenfl„che und damit die statische Kapazit„t $C_{0}$ verringert wird.
Jedoch entspricht dieses in der Regel nicht einer modenselektiven
Anregungsgeometrie (siehe hierzu Kapitel~5.5.1), auf die es bei resonanten
Sensoren im Gegensatz zu piezoelektrisch betriebenen Aktoren ankommt.\\
%----------------------- Beginn: table ---------------------------
\begin{table}[htb]
\caption{\label{tabpiezomess}
 Experimentelle Charakterisierung von Silizium-Bimorphmembranen
 mit hybrider Piezokeramik}
\begin{center}
\begin{tabular} {|l||c|c|c|c|}
\hline
 Wandler & PA & PB & PC & PD \\
\hline \hline
% $h_{Si}$ [$\mu$m] & 200  & 200   & 150  &  150  \\
 $h_{Si}/h_{PZT}$  & $\sim$0,5 & $\sim$0,75 & $\sim$0,5 & $\sim$0,75 \\
 Abm. [$mm^{2}$]:  & 6x6  & 12x12 & 6x6  &  12x12 \\
\hline
 \multicolumn{5}{|c|}{Optische Messungen} \\
\hline
 $f_{res}$ [Hz]     & 15430 & 24428 & 16849 & 25153 \\
 $A_{SS}$ [$\mu$m]  &  2,3  &  2,1  &  4,4  &  1,4  \\
 $Q_{mech}$         &  42   &  65   &  62   &  84   \\
\hline
 \multicolumn{5}{|c|}{Elektrische Messungen} \\
\hline
 $f_{s}$ [Hz]   & 15162 & 24388 & 16854 & 25113 \\
 $f_{p}$ [Hz]   & 15888 & 24669 & 17437 & 25328 \\
\hline
% quadratisch exakte Werte
 $k_{eff}$ [\%] & 29,9  & 15,1 & 25,6 & 13,0 \\
\hline
 \multicolumn{5}{|c|}{Elektrische Ersatzdaten } \\
\hline
 $R$ [k$\Omega$] & 2,22 & 1,65  & 1,24 & 3,46  \\
 $L$ [H]         & 1,25 & 0,28  & 1,10 & 1,12  \\
 $C$ [pF]        & 86,8 & 149,7 & 80,6 & 36,0  \\
 $C_{0}$ [nF]    & 1,83 & 8,14  & 1,35 & 7,34  \\
\hline
 $Q_{elek}$      & 54   & 26    & 94    & 51 \\
 $k_{eff}^{RLC}$ [\%] & 21,3 & 13,4 & 23,7 & 7,0 \\
\hline
\end{tabular}\\
\end{center}
\end{table}
%----------------------- Ende: table ---------------------------
In {\bf Tabelle~\ref{tabpiezomess}} sind die experimentellen Ergebnisse
verschiedener hybrider Piezokeramik-Siliziummembranen zusammengefaát.
Die Anregungsspannung betrug bei den optischen Messungen etwa 4~$V_{SS}$,
bei den elektrischen Messungen maximal 1~V.
Die optisch vermessenen Resonanzfrequenzen $f_{res}$ entsprechen in
etwa den Serienresonanzfrequenzen $f_{s}$, so daá auf niederohmige
elektrische Abschluábedingungen bei den Messungen geschlossen werden kann.
Die erreichbaren Schwingungsamplituden $A_{SS}$ variieren bei der
anliegenden Anregungsspannung von 4~$V_{SS}$ zwischen 1,4--4,4 $\mu$m.
Aufgrund des linearen Zusammenhangs zwischen der Anregungsspannung und den
Schwingungsamplituden befindet man sich im Kleinsignalbereich. Die an der
Piezokeramik anliegende elektrische Feldst„rke bel„uft sich auf etwa
$E$~=~0,1~V/$\mu$m.\\
Die kleinen Piezokeramiken (6~x~6~$mm^{2}$) zeichnen sich gegenber den
groáen (12~x~12~$mm^{2}$) durch einen erheblich h”heren effektiven
elektromechanischen Kopplungsfaktor $k_{eff}$ aus. So erh”ht sich der
$k_{eff}$--Wert vom Wandler PB zum Wandler PA um das Doppelte. Den
entscheidenden Einfluá auf die Gr”áe des Kopplungsfaktors besitzt das
Schichtdickenverh„ltnis $h_{Si}/h_{PZT}$, wie in
Kapitel~\ref{schichtdickenabhaengigkeit} gezeigt wird. Zum anderen spielt
die Membranberdeckung durch die Piezokeramik eine gewisse Rolle, wodurch
das Kapazit„tsverh„ltnis $C/C_{0}$ beeinfluát wird.
In Kapitel~5.5 wird auf den Einfluá der lateralen Schichtstrukturierung
auf das Verhalten von piezoelektrischen Dnschichtstrukturen
eingegangen.\\
%Die Korrelation zwischen dem direkt aus den beiden Frequenzen $f_{s}$
%und $f_{p}$ nach Gleichung (\ref{keffnaeherung}) bestimmten
%$k_{eff}$--Wert und dem mit Hilfe der elektrischen Ersatzdaten
%nach Gleichung (\ref{qkmelek}) stimmen recht gut berein.
%Das gleiche gilt fr die šbereinstimmung der mechanisch und elektrisch
%bestimmten Schwingungsgten $Q_{mech}$ und $Q_{elek}$ nach
%Gleichung~(\ref{qkmelek}).
Die statische Kapazit„t $C_{0}$ kann durch
die Permittivit„t $\varepsilon_{33}$ und die geometrischen Abmessungen
des Kondensators, der durch die Anschluáelektroden der Piezokeramik
gebildet wird, abgesch„tzt werden. Es gilt
$C_{0}= \varepsilon_{0} \varepsilon_{33} A / h_{PZT}$, wobei
$A$ die effektive Elektrodenfl„che (6~x~6 bzw.\ 12~x~12 $mm^{2}$) ist.
Unter der Annahme von $\varepsilon_{33} / \varepsilon_{0} = 1600\pm300$
und $h_{PZT} = 200~\mu$m ($\pm10~\%$) folgt fr die beiden
Keramikabmessungen $C_{0}$=2,6 bzw.\ 10,2~nF. Diese Werte stimmen
gr”áenordnungsm„áig gut mit den gemessenen Werten berein.\\
%
%Damit kann das elektrische Ersatzschaltbild
%in erster N„herung zur Beschreibung der piezoelektrischen Bimorphe
%herangezogen werden.\\
%
%----------------------- Beginn: table ---------------------------
\begin{table}[htb]
\caption{\label{tabpiezosim}
 FE-Modellierung von Silizium-Bimorphmembranen mit hybrider Piezokeramik}
\begin{center}
\begin{tabular} {|l||c|c|c|}
\hline
 $h_{Si}$ [$\mu$m]  & 90   & 95    & 100  \\
 $h_{Si}/h_{PZT}$   & 0,45 & 0,475 & 0,5  \\
\hline \hline
 $f_{s}$ [Hz]       & 15683 & 16388 & 17074  \\
 $f_{p}$ [Hz]       & 16276 & 17041 & 17785  \\
\hline
% quadratisch exakte Werte
 $k_{eff}$ [\%]     & 26,7 & 27,4 & 28,0 \\
\hline
\end{tabular}
\end{center}
\end{table}
%----------------------- Ende: table ---------------------------
Zum Vergleich von piezoelektrischen FE-Berechnungen an Bimorphmembranen
mit den experimentellen Daten, wurde der Wandler mit den Abmessungen
6~x~6~$mm^{2}$ und der Piezokeramikdicke von 200~$\mu$m herangezogen.
Hierfr wurde eine Membrangeometrie mit strukturierten $ZnO$--Schichten
herangezogen. % \cite{Messner}.
Aufgrund der Dickenschwankungen der
Siliziummembranen, die eine nominelle Dicke von etwa 100~$\mu$m aufwiesen,
wurden die Serien- und Parallelresonanzfrequenzen, sowie die
$k_{eff}$--Werte in Abh„ngigkeit der Siliziummembrandicke $h_{Si}$
berechnet. Die Materialeigenschaften wurden anisotrop angenommen. Die
Anzahl der Elemente betrug 1970, die der Knoten 2828. Es wurde mit der
{\sl Householder}-Methode gerechnet (300 MDOF).
In {\bf Tabelle~\ref{tabpiezosim}} sind die numerischen Ergebnisse
zusammengefaát. Die gemessenen Resonanzfrequenzen $f_{s}$ und $f_{p}$ liegen
fr die beiden Wandler PA und PC im Bereich 15--17~kHz, so daá ausgehend
von den FE-Resultaten auf eine effektive Siliziummembrandicke von etwa
90--100~$\mu$m geschlossen werden kann. Die berechneten Kopplungsfaktoren
27--28~\% liegen zwischen den beiden gemessenen. Die Abweichungen zwischen
den gemessenen und berechneten Werten lassen sich durch die Toleranz der
Siliziummembrandicke und die Vernachl„ssigung der dielektrischen D„mpfung
der Piezokeramik erkl„ren.\\
%
Gegenstand der weiteren Untersuchungen von Bimorphwandlern mit
piezoelektrischen Dnnschichten sind die Einflsse:
\begin{itemize}
\item des Schichtdickenverh„ltnisses ($h_{Si}/h_{Piezo}$)
\item des Dnnschichtsystems ($AlN, PZT, ZnO$)
\item der lateralen Schichtstrukturierung.
\end{itemize}



\subsection{Zinkoxid-Dnnschichten}
\label{zno}

Neben Membranstrukturen eignen sich einseitig eingespannte Siliziumzungen
als technologische Teststrukturen zur Entwicklung der
Dnnschichtprozesse. Die Gte der piezoelektrischen
Dnnschichten ist aus der Sicht der meisten Anwendungen durch den
elektromechanischen Kopplungsfaktor bestimmt. Da der materialabh„ngige
Anteil $k_{mat}$ den Messungen nicht direkt zug„nglich ist, muá an
mikromechanischen Strukturen der effektive Wert $k_{eff}$ ermittelt
und ausgewertet werden. Im Gegensatz zu Membranen k”nnen bei den einseitig
eingespannten Zungenstrukturen zus„tzlich die statischen
%
Auslenkungen\footnote{Die Bestimmung der $k_{eff}$-Werte durch statische
Messung der Zungenauslenkungen wurde vom BMFT-Verbundpartner
{\sl Robert Bosch GmbH} durchgefhrt. Gegenber der Resonanzmethode weist
diese Methode den Nachteil auf, daá die zugrundeliegende analytische
N„herungsformel, die fr die Berechnung von $k_{eff}$ herangezogen
wird, die mechanischen und elektrischen Randbedingungen, sowie den
Bimorphcharakter nicht exakt bercksichtigt. Weiterhin kommt hinzu, daá
durch die relativ hohen Anregungsspannungen %($E \approx 3 V/\mu$m)
nichtlineare Effekte auftreten k”nnen (Groásignalverhalten). Die auf diese
Weise ermittelten $k_{eff}$-Werte fallen daher im Vergleich zu dynamisch
vermessenen Werten systematisch zu niedrig aus \cite{ABV93}.}
%
infolge elektrischer Ansteuerung gut vermessen werden, da sie im Vergleich
zu Membranstrukturen einen grӇeren Hub (einige Mikrometer) aufweisen.
Im folgenden werden die
experimentellen Daten mit den Resultaten piezoelektrischer FE-Berechnungen
verglichen und die Modellierungsgenauigkeit der numerischen Methoden, sowie
die piezoelektrischen Bimorphmodelle berprft.\\
%
Die Herstellung und Vermessung der
ganzfl„chig von einer piezoelektrischen $ZnO$-Dnnschicht berdeckten
Zungenstrukturen erfolgte vom BMFT-Verbundpartner {\sl Robert Bosch GmbH}.
Die Meáergebnisse wurden fr die Entwicklung der piezoelektrischen
FE-Modelle im Rahmen dieser Arbeit freundlicherweise zur Verfgung gestellt
\cite{Flik}.\\
%
In {\bf Tabelle~\ref{tabzungen}} sind die optisch vermessenen
Resonanzfrequenzen $f_{res}$, die durch Impedanzmessungen bestimmten
Serien- und Parallelresonanzfrequenzen $f_{s}$ und $f_{p}$, sowie die
daraus abgeleiteten effektiven Kopplungsfaktoren $k_{eff}$ zusammengestellt.
%
%Die Siliziumzungen haben eine zehnfach grӇere Dicke als die $ZnO$-Schichten,
%so daá die analytischen Absch„tzungen aus Kapitel~\ref{skalierungsverhalten}
%zum Vergleich herangezogen werden k”nnen.
%
Bei den FE-Berechnungen wurde von einem homogenen Siliziumbalken, der
ganzfl„chig von einer $ZnO$-Schicht bedeckt ist, ausgegangen. Die einseitige
Einspannung der Zungenstruktur wird durch eine schr„ge (111)-Siliziumebene
gebildet. Die L„nge der Siliziumzungen betrug 7~mm, die Breite 5~mm.
Es wurden sowohl bei Silizium, als auch bei $ZnO$ anisotrope Materialdaten
\cite{LB82} bercksichtigt. Um die numerischen Fehlereinflsse m”glichst
niedrig zu halten, wurden die Zungen lateral sehr fein unterteilt.
Das gr”bere FE-Modell weist 30~Elemente in L„ngsrichtung und 8~Elemente in
der Breite auf, w„hrend das feinere Modell entsprechend 70 bzw.\ 15~Elemente
aufweist. Die Elementunterteilung in Dickenrichtung war weniger kritisch,
wie in Kapitel~4 bereits fr die Membranstrukturen gezeigt wurde, so daá
fr die Siliziumzunge zwei Elementlagen und fr die $ZnO$-Schicht eine
Elementlage gengten. Zur Berechnung der Eigenfrequenzen wurde das
{\sl Householder}-Verfahren mit 300~MDOFs eingesetzt. Die Angaben fr die
Frequenzwerte sind aufgerundet, w„hrend die $k_{eff}$--Werte exakt
angegeben sind.\\
%----------------------- Beginn: table ---------------------------
\begin{table}[htb]
\caption{\label{tabzungen}
 Charakteristische Kenndaten $ZnO$-beschichteter Siliziumzungenstrukturen
 (Vergleich: FE-Berechnung -- Messung)}
\begin{center}
\begin{tabular}{|l||c|c|c|c|c|}
\hline
Zunge:              &  SZ1   &  SZ2   &  SZ3   &  SZ4 &  SZ5 \\
\hline \hline
 $h_{Si}$  [$\mu$m] & 123,7  & 124,5  &  137,2
 & \multicolumn{2}{|c|}{124,5} \\
 $h_{ZnO}$ [$\mu$m] & 10,4   & 10,1   &  10,7
 & \multicolumn{2}{|c|}{7,75} \\
\hline
 \multicolumn{6}{|c|}{Messungen} \\
\hline
 $f_{res}$ [Hz]  & 3330   & 3410   &  3680 & \multicolumn{2}{c|}{3455} \\
\hline
 $f_{s}$ [Hz]    & 3358   & 3416   &  3642   & 3418  & 3420  \\
 $f_{p}$ [Hz]    & 3376   & 3430   &  3654   & 3439  & 3445  \\
 $k_{eff}$ [\%]  & 10,3   & 9,0    &  8,1    & 11,0  & 12,0  \\
\hline
 \multicolumn{6}{|c|}{FE-Modellierung} \\
\hline
 Elemente  & \multicolumn{2}{|c|}{2080} & \multicolumn{3}{c|}{4410} \\
 Knoten    & \multicolumn{2}{|c|}{2781} & \multicolumn{3}{c|}{6144} \\
\hline
 $f_{s}$ [Hz]   & 3474   & 3495   &  3832   & \multicolumn{2}{c|}{3474} \\
 $f_{p}$ [Hz]   & 3502   & 3523   &  3861   & \multicolumn{2}{c|}{3497} \\
 $k_{eff}$ [\%] & 12,6   & 12,5   &  12,3   & \multicolumn{2}{c|}{11,3} \\
\hline
 $\frac{\Delta k}{k}$ [\%] & -22  & -39   &  -52    &   -3  & +6    \\
\hline
\end{tabular}
\end{center}
\end{table}
%----------------------- Ende: table ---------------------------
Die Zungen SZ1 und SZ2 weisen ein
$Si/ZnO/SiO_{2}/Si_{3}N_{4}$--Schichtsystem auf und
die Zungen SZ3 und SZ4 bestehen aus einem vereinfachten
$Si/ZnO/Al$--Schichtsystem. Die Zunge SZ5 wurde aus hochdotiertem
$p^{++}$--Silizium hergestellt und zeichnet sich bei gleichem Schichtsystem
wie Zunge SZ4 durch den h”chsten Kopplungsfaktor von etwa 12~\% aus.
Die Dicken der $SiO_{2}$-- und $Si_{3}N_{4}$--Schicht betrugen 150 und
300~nm und wurden aufgrund ihrer im Vergleich zum $ZnO$ geringen Schichtdicke
bei den FE-Berechnungen nicht bercksichtigt.\\
%
Das Schichtdickenverh„ltnis $h_{Si}/h_{ZnO}$ „ndert sich bei den
Zungenstrukturen nur unwesentlich von  12--16, so daá die effektiven
Kopplungsfaktoren davon kaum betroffen werden. Dieses belegen auch die
FE-Berechnungen, die unter der Annahme ideal homogener Schichteigenschaften
unter Verwendung der Literaturmaterialdaten (siehe
Tabelle~\ref{tabpiezoelektrika}) durchgefhrt wurden.
Die Frequenzabweichungen betragen bei den Zungen SZ4 und SZ5 etwa 1,5~\%
und nehmen bis auf 4--6~\% bei den ersten drei Zungen zu.
Die rechnerisch ermittelten Resonanzfrequenzen fallen gegenber den
gemessenen Werte alle systematisch h”her aus, was durch die Druckspannungen
(bis zu -500~MPa) in den $ZnO$-Schichten erkl„rt werden kann. Experimentell
wurde die innere Spannung der Zungen durch optische Vermessung der
Zungenauslenkung bestimmt \cite{Flik}. Insbesondere zeichnen sich die
beiden Zungen SZ4 und SZ5 durch spannungsarme $ZnO$-Schichten aus.
Nur bei diesen beiden Strukturen stimmen die Werte der effektiven
elektromechanischen Kopplungsfaktoren hinreichend gut mit den FE-Resultaten
berein. Die Abweichungen betragen nur 3 bzw.\ 6~\%, im Gegensatz zu den
anderen drei Zungen, bei denen die Abweichungen bis zu 52~\% betragen.
Ein Grund fr diese teilweise hohen Abweichungen ist der Einfluá der
mechanischen Verspannung der $ZnO$-Schichten.
Dieses belegen auch die Messungen an $ZnO$-beschichteten Membranstrukturen,
die in Abh„ngigkeit der inneren Spannung bis zu $\pm$100~\% vom numerisch
ermittelten $k_{eff}$--Wert abweichen \cite{ABV93}.
Die gemessenen maximalen Feldst„rken lagen zwischen 4--12~V/$\mu$m und
decken sich mit den in der Literatur angegebenen Wert von etwa 10~V/$\mu$m
\cite{Smi92b}.\\
%
Unter der Voraussetzung, daá bei den hier betrachteten Biegewandlern
im wesentlichen der transversale bzw.\ planare elektromechanische
Kopplungsfaktor $k_{31}$ bzw.\ $k_{p}$ ausschlaggebend ist,
kann mit Hilfe von Gleichung~(\ref{kmat}) auf einen {\em effektiven,
prozeáabh„ngigen} elektromechanischen Kopplungsfaktor $k_{31}^{eff}$ bzw.\
$k_{p}^{eff}$ geschlossen werden. Aufgrund der Proportionalit„t zwischen
$k_{31}$ und $d_{31}$ ist es m”glich durch einen Vergleich des
experimentellen $k_{31}^{eff}$--Wertes mit den FE-Resultaten einen
schichtspezifischen
$d^{eff}_{31}$-Wert\footnote{Wird bei den Zungenstrukturen SZ1--SZ3 eine
Abweichung von 22--52~\% zwischen Messung und FEM zugrundegelegt, so kann
mit dem in der FE-Rechnung verwendeten Literaturwert von
$d_{31}$~=~5,12~pC/N auf einen effektiven piezoelektrischen
Kopplungskoeffizienten von lediglich $d^{eff}_{31}$~=~2,5--4,0~pC/N
geschlossen werden. Dieser Wertebereich liegt in der GrӇenordnung von
Dnnschichtprozessen und wird auch von anderen Forschungsgruppen angegeben
\cite{Pra93}. Fr den spannungsarmen Beschichtungsprozeá, mit dem die
Zungen SZ4 und SZ5 hergestellt wurden, errechnet sich ein gegenber dem
Literaturwert leicht erh”hter Koeffizient von $d^{eff}_{31}$~=~5,43~pC/N.}
bei fester Geometrie abzuleiten.
Dieser Wert ist geeignet, um die Beschichtungsprozesse zu optimieren,
erlaubt aber keinen direkten Rckschluá auf die absolute Gr”áe des rein
materialabh„ngigen $d^{mat}_{31}$-Wertes. Zur eindeutigen Bestimmung dieses
Wertes ist es zus„tzlich erforderlich, den mechanischen
Steifigkeitskoeffizienten $S^{E}_{11}$ und die Permittivit„t
$\varepsilon^{\sigma}_{33}$ an der betrachteten Dnnschichtstruktur zu
bestimmen.\\
%
Die FE-Berechnungen an piezoelektrischen Bimorphstrukturen bercksichtigen
nicht die Schichtmorphologie und die innere Schichtspannungen im Zinkoxid
oder sonstige prozeábedingte Auswirkungen, so daá diese Rechnungen nur
qualitative Aussagen erm”glichen. Im weiteren wird daher von idealisierten
Schichtsystemen ausgegangen, die durch die Materialdaten der Literatur
beschrieben werden, und die Einflsse der Wandler- und Elektrodengeometrien
auf das statische und dynamische Verhalten prinzipiell untersucht.
