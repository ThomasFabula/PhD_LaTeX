\chapter{Piezoelektrische Dnnschichtstrukturen}
\label{piezoelektrisch}

Dieses Kapitel beschäftigt sich mit der Modellierung und experimentellen
Charakterisierung von piezoelektrisch angetriebenen Mikrostrukturen.
Mit Hilfe der im vorhergehenden Kapitel entwickelten FE-Modelle wird
das dynamische Verhalten von mikromechanischen Resonatoren in Bimorphaufbau
unter Berücksichtigung des elektromechanischen Wandlungsprinzips simuliert.
Für die effiziente Anregung der Bimorphstrukturen war eine geeignete
Auslegung der Elektrodengeometrie erforderlich und das Verhältnis der
Schichtdicken zu optimieren. Weiterhin wird der Einfluß des piezoelektrischen
Dünnschichtsystems untersucht und die erreichbaren elektromechanischen
Kopplungsfaktoren verschiedener Wand\-lergeometrien berechnet.
Bei der Ermittlung des effektiven elektromechanischen Kopplungsfaktors, der
die Anregungseffizienz zu charakterisieren erlaubt, müssen neben den
mechanischen auch die elektrischen Randbedingungen berücksichtigt werden.
Mit Hilfe der piezoelektrischen Anregung können neben dem mechanischen
Schwingungsverhalten der Sensoren auch verschiedene elektrische
Kenngrößen, wie z.B.\ das frequenzabhängige Impedanz- und Phasenverhalten
abgeleitet werden. Auf diese Weise konnte mit Hilfe der
gekoppelten Feldberechnungen bei piezoelektrisch angetriebenen
Membrandrucksensoren die Modenselektivität verbessert und ein unimodaler
Betrieb meßtechnisch nachgewiesen werden. Weiterhin konnte rechnerisch
gezeigt werden, daß die Temperaturquerempfindlichkeit von
Membrandrucksensoren im Bimorphaufbau erheblich reduziert werden kann.


\section{FE-Modell der Bimorphstruktur}
\label{bimorphstruktur}

Im vorangegangenem Kapitel wurden mit Hilfe der Modalanalyse die
Eigenfrequenzen und -schwingungsformen von resonanten Druck- und
Kraftsensoren rechnerisch und experimentell untersucht. Bei den
piezoelektrisch betriebenen Membrandrucksensoren wurden die mechanische
Eigenschaften der $ZnO$-Dünnschichten berücksichtigt, aber der
Einfluß der elektrischen Randbedingungen vernachlässigt. Bei
piezoelektrischen Medien wird infolge der bidirektionalen
elektromechanischen Wechselwirkung das mechanische Verhalten der
Bimorphwandler durch die äußere elektrische Beschaltung beeinflußt.
Daher werden im folgenden die elektrischen Randbedingungen mit
berücksichtigt.\\
Hierfür wurde ein dreidimensionales FE-Modell entwickelt, das
in {\bf Abbildung~\ref{abbbimorph}} dargestellt ist.
%----------------------- Beginn: Figure-Environment ----------------------
\begin{figure}[htb]
%\vspace*{8cm}

\begin{center}
% --- Dateiname des Bildes
\input{abbfein.tex}
\setabbfein
\end{center}
\caption{\label{abbbimorph}
 FE-Modell einer Silizium-Bimorphmembran}
\end{figure}
%----------------------- Ende: Figure-Environment ----------------------
Aus Symmetriegrnden
handelt es sich um eine Viertelmembran, die ganzflächig von einer
piezoelektrischen Dünnschicht überdeckt ist. Die $Al$-Elektroden wurden
im FE-Modell vernachlässigt, da sie im Vergleich zur piezoelektrischen
Dünnschicht sehr dünn sind ($h_{ZnO} \approx 10-100 \; h_{Al}$). Die
schräge
Einspannung entspricht den ätzbegrenzenden (111)-Siliziumebenen, die durch
den anisotropen naßchemischen Herstellungsprozeß der Siliziummembran
bedingt sind. Die Abmessungen der
Sensormembran variieren je nach Anwendungsbereich. Die in dieser Arbeit
betrachteten mikromechanisch gefertigten Membranen weisen eine
Kantenlänge von 5 bzw.\ 9,2~mm und Membrandicken von 20--200~$\mu$m auf.
Der Membransockel hat eine untere Breite von 5~mm und eine Höhe von
etwa 0,525~mm, die durch die Dicke des Siliziumwafers
gegeben ist.\\
Zur Eingrenzung der Gltigkeit der analytischen Beschreibungsansätze
für Bimorphstrukturen wurde die Resonanzfrequenz in Abhängigkeit der
Gesamtmembrandicke betrachtet, um den Einfluß des zusätzlichen
Dünnschichtsystems zu untersuchen.
In {\bf Abbildung~\ref{abbvglbulktf}} ist die Abhängigkeit der
Resonanzfrequenz einer Silizium-Bimorphmembran mit einer piezoelektrischen
$ZnO$-Dünnschicht von der Gesamtmembrandicke dargestellt. Die Dicke der
Siliziummembran betrug 20~$\mu$m und die Dicke der $ZnO$-Schicht wurde
von 1 bis 50~$\mu$m variiert. Die analytische Näherungsformel nach
Gleichung (\ref{memfreq}) geht von einer ideal homogenen, unendlich steif
eingespannten Platte aus, wobei der Bimorphcharakter lediglich durch eine
Gewichtung der Materialdaten ($\hat E, \rho$) über die Schichtdicken
berücksichtigt wird.
Zur Vereinfachung wurde bei Silizium mit isotropen Materialeigenschaften
gerechnet (Gleichung~\ref{simat}), so daß dreidimensionale, lineare
Volumenelemente ({\em SOLID45}) verwendet werden konnten. Aufgrund der
hexagonalen Kristallsymmetrie der piezoelektrischen Schicht
(siehe Abbildung~\ref{abbmatrix6mm}) muß die Anisotropie der Elastizitäts-
und piezoelektrischen Eigenschaften berücksichtigt werden. Hierzu bietet
{\sf ANSYS} das dreidimensionale {\em Multi-Field}-Element {\em SOLID5} an,
bei dem die Strukturfreiheitsgrade ${\vec u}$ und der elektrische
Potential-Freiheitsgrad $\phi$ verwendet wurden. Bei der $ZnO$-Dünnschicht
sind die anisotropen Materialkennwerte aus Tabelle~\ref{tabpiezoelektrika}
herangezogen worden. \\
%----------------------- Beginn: Figure-Environment ----------------------
\begin{figure}[htb]
%\vspace*{8cm}

\begin{center}
% --- Dateiname des Bildes
\input{abbfzw.tex}
\setabbfzw
\end{center}
\caption{\label{abbvglbulktf}
 Resonanzfrequenzen einer Silizium-Bimorphmembran mit piezoelektrischer
 $ZnO$-Dünnschicht in Abhängigkeit der Membrangesamtdicke und elektrisches
 Ersatzschaltbild für piezoelektrische Resonatoren}
\end{figure}
%----------------------- Ende: Figure-Environment ----------------------
Entsprechend dem elektrischen Ersatzschaltbild fr elektromechanisch
gekoppelte Schwinger \cite{IEEE}, das in Abbildung~\ref{abbvglbulktf}
skizziert ist, entspricht die numerisch berechnete Serienresonanzfrequenz
$f_{s}$ der Frequenz beim elektrischen Kurzschluß ($\vec E=0$), die
Parallelresonanzfrequenz $f_{p}$ dem elektrischen Leerlauf bei offenen
Anschlußelektroden ($\vec D=0$) \cite{Nai83}. Unter Zugrundelegung dieses
einfachen
Ersatzschaltbildes\footnote{Das elektrische Ersatzschaltbild von
piezoelektrischen Dünnschichtstrukturen ist stark vom Schichtaufbau des
Wandlers abhängig. Zusätzlich ist zu unterscheiden, ob der Wandler bzw.\
Schwinger als elektrisches Zweipol- oder Vierpolnetzwerk betrieben wird.
Resonante Silizium-Kraftsensoren sind von {\sl van Mullem et al.} mit
$Si/ZnO/SiO_{2}/Si_{3}N_{4}$--Schichtaufbau im Vierpolbetrieb betrieben
und ein elektrisches Ersatzschaltbild abgeleitet worden, das insbesondere
das elektrische Übersprechverhalten zu beschreiben vermag \cite{Mul92}.
Ein Zweipolersatzschaltbild für elektromechanische Balkenresonatoren mit
vereinfachtem $Si/ZnO$-Aufbau ist im Rahmen einer Diplomarbeit entwickelt
worden \cite{Qui93}. Es vermag die piezoelektrische Kopplung und den Einfluß
einer konstanten axialen Zugbelastung prinzipiell zu beschreiben,
berücksichtigt aber kein spezifisches Dünnschichtsystem (z.B.\
Schichtwiderstände und -kapazitäten), sowie elektrische Störpfade.},
das nur im unmittelbaren Frequenzbereich der
betrachteten Schwingungsmode Gültigkeit besitzt, kann der piezoelektrisch
angetriebene Resonator als ein elektrischer Zweipol aufgefaßt werden.
Das Ersatzschaltbild geht von einem verlustbehafteten Resonanzschwingkreis
aus. Der Serienschwingkreis setzt sich aus dem in Reihe geschalteten
dynamischen Verlustwiderstand $R$, der dynamischen Induktivität $L$ und der
dynamischen Kapazität $C$, sowie einer dazu parallel geschalteten statischen
Kapazität $C_{0}$ zusammen. Der eigentliche Resonator wird durch die drei
dynamischen Ersatzdaten beschrieben, während $C_{0}$ durch die statische
Kapazität der Anschlußelektroden der piezoelektrischen Schicht und die
elektrischen Zuleitungen gebildet wird. Der komplexe Widerstand (Impedanz)
des Resonators wird bei der Serien- und Parallelresonanzfrequenz reell.
Unterhalb der Serienresonanzfrequenz und oberhalb der
Parallelresonanzfrequenz ist das Wandlerverhalten rein kapazitiv, und durch
den Wert der Kapazität $C_{0}$ bestimmt. Bei der
Serienresonanzfrequenz hat die Admittanz des Ersatzschaltkreises ein
Maximum, bzw.\ die Impedanz ein lokales Minimum und der Wandler weist ein
rein ohmsches Verhalten mit dem Verlustwiderstand $R$ auf. Für Frequenzen
$f_{s} < f < f_{p}$ verhält sich der Schwinger induktiv. Mit Hilfe der
elektrischen Ersatzdaten $R, L, C$ und $C_{0}$ lassen sich die
beiden diskreten Resonanzfrequenzen ermitteln:
\begin{eqnarray}
\label{fserfpar}
 \omega_{s}   & = & \frac{1}{\sqrt{LC}} \nonumber \\
 \omega_{p}   & = & \omega_{s} \sqrt{1+\frac{C}{C_{0}}}
\end{eqnarray}
Die Differenz zwischen der Serienresonanzfrequenz und der
Parallelresonanzfrequenz wird durch die unterschiedlichen Steifigkeiten
der piezoelektrischen Schicht in Abhängigkeit der äußeren Beschaltung
verursacht und ist ein direktes Maß für den effektiven elektromechanischen
Kopplungsfaktor (siehe Gleichung \ref{keffnaeherung}). Beim elektrischen
Kurzschluß können die durch die mechanische Verbiegung des Bimorphs
induzierten elektrischen Ladungen ab\-fließen, während sich bei offenen
Elektroden ein zusätzliches
elektrisches Feld in der piezoelektrischen Schicht aufbauen kann, das zu
einer Versteifung des Wandlerelementes führt. Diese piezoelektrisch
bedingte Versteifung erhöht den Wert der Parallelresonanzfrequenz beim
elektrischen Leerlauf, so daß $f_{s} < f_{p}$.
Die analytisch berechnete \glqq rein mechanische\grqq \, Resonanzfrequenz
$f_{Bulk}$ entspricht der Serienresonanzfrequenz $f_{s}$ aufgrund der
Verwendung der Steifigkeitskoeffizienten $S_{11}^{E}$ ($\vec E$~=~ $const$)
aus Tabelle~\ref{tabpiezoelektrika} für Zinkoxid. Analog könnte die
Parallelresonanzfrequenz $f_{p}$ bei der Verwendung von $S_{11}^{D}$
($\vec D$~=~$const$) berechnet werden, falls der Zusammenhang zwischen
$S^{E}$ und $S^{D}$, der durch die elektromechanische Kopplung
beschrieben wird, vorab bekannt wäre.\\
Während für dünne $ZnO$-Schichtdicken die analytische Lösung mit den
FE-Ergebnissen gut übereinstimmt, wird die Abweichung ab einem
Schichtdickenverhältnis von $h_{ZnO}/h_{Si} \ge 1$ signifikant, da die
Einspannbedingung auch für die $ZnO$-Schicht ungültigerweise angenommen wird.
Beim FE-Modell wird der Einfluß der seitlich frei beweglichen $ZnO$-Schicht
mit zunehmender Schichtdicke größer, so daß die Abweichung zwischen dem
analytischen und dem
numerisch Frequenzwert von etwa 1~\% bis auf 7~\% bei der maximalen
Schichtdicke von 70~$\mu$m zunimmt. \\
%
Der Einfluß der herstellungsprozeßabhängigen Dünnschichtdaten ist
in Abbildung~\ref{abbvglbulktf} ebenfalls dargestellt. Der an reaktiv
gesputterten $ZnO$-Dünnschichten \cite{Wag94} bestimmte reduzierte E-Modul
$\hat E_{tf}$ (siehe Kapitel~\ref{druckabh}) fällt gegenüber dem
Bulkwert $\hat E_{Bulk}$ um etwa 13~\% niedriger aus, so daß die
analytisch berechneten Resonanzfrequenzen $f_{tf}$ unterhalb der
Werte $f_{Bulk}$ bleiben. Die Abweichung der Frequenzwerte infolge
der Verwendung von Dünnschichtmaterialdaten bleibt im Mittel unterhalb
von 1~\% und ist damit für dieses spezielle Schichtsystem vernachlässigbar.
Die Diskussion der Modelleinflüsse, die durch die Elementierung und die
mechanischen Randbedingungen hervorgerufen werden,
ist Gegenstand des folgenden Kapitels.\\



\section{Einflüsse der Modellparameter}
\label{piezomodell}

Verschiedene Plattenmodelle mit idealer, unendlich steifer Einspannung
und dreidimensionale Membranmodelle mit realer Membraneinspannung
wurden erstellt, um den Einfluß der FE-Modellparameter und der
Randbedingungen separat untersuchen zu können.
Bei den FE-Modellen wurde die Elementvernetzung, d.h.\ die äquidistante
Seitenteilung entlang der Membran und damit die Aspektverhältnisse der
Elemente variiert. Zusätzlich wurde der Einfluß der dynamischen
Hauptfreiheitsgrade (MDOF) untersucht. Als mechanische Randbedingungen
wurden die Knotenverschiebungen an den Membranseiten gesperrt. Damit ist
die Membranen an ihrer Unterseite mechanisch fixiert. Die Modellierung
der Elektroden erfolgte durch Definition von elektrischen Randbedingungen,
indem konstante Potentiale ($\phi = 0$) an die Knoten der Oberfläche der
Siliziummembran und Piezoschicht gelegt wurden. Die Seitenlänge der
untersuchten Silizium-Bimorphmembran betrug 9,2~mm. Die Schichtdicken
der Siliziummembran $h_{Si}$ und der $ZnO$-Schicht $h_{ZnO}$ wurden
zu 20~$\mu$m gesetzt.\\
%
Bei den Plattenmodellen (P1--P4) handelt es sich um einfache FE-Modelle
mit idealer Randeinspannung und jeweils einer Elementlage über die Dicke
der Siliziummembran und der $ZnO$-Schicht.
Um den Einfluß unterschiedlicher Randbedingungen zu analysieren, wurden
verschiedene Knotenlagen des Plattenmodells eingespannt. Beim Modell P1 ist
nur die untere äußere Knotenlage in den Verschiebungen ($\vec u=0$) gesperrt.
Daher kann die Membran sich um diese Knotenlage verdrehen. Hierdurch wird
die Einspannsteifigkeit effektiv vermindert und die Resonanzfrequenzen
fallen entsprechend niedriger aus. Beim Modell P2 wurde der gesamte Rand der
Siliziummembran eingespannt und nur die $ZnO$-Schicht frei gelassen, so daß
sie seitlich frei beweglich war. Dieses entspricht einer mittleren
Steifigkeit. Modell~P3 wies eine volle Einspannung für alle Knotenlagen des
gesamten Membranrandes auf. Entsprechend fallen die Frequenzen am höchsten
aus. Infolge der unterschiedlichen Randeinspannungsbedingungen ändern sich
bei den Plattenmodellen die Resonanzfrequenzen bis zu 30~\%. \\
%
Die Membranmodelle (M1--M4) berücksichtigen zusätzlich die reale
Membraneinspannung durch die (111)-Siliziumebenen, so daß sich die
Frequenzwerte infolge verminderter Einspannsteifigkeit erniedrigen.
Gleichzeitig steigt jedoch der Modellieraufwand beträchtlich an, bei
etwa drei- bis fünffach erhöhter Rechenzeit. Die Resonanzfrequenzen nehmen
mit steigender Elementanzahl wie erwartet ab, wobei sich der Wert für den
effektiven Kopplungsfaktor weitgehend unabhängig von der Netzverfeinerung
auf etwa 20~\% stabilisiert. Erst bei einer drastischen Erhöhung der
Element- und Knotenanzahl beim Referenzmodell MR, sowie gleichzeitiger
Berücksichtigung der Materialanisotropien von Silizium {\em und} Zinkoxid
erniedrigen sich die Frequenzwerte auf $f_{s} = 4677$~Hz und
$f_{p} = 4755$~Hz, aus denen sich ein Kopplungsfaktor von 18~\% errechnet.
Eine analytische Überschlagsrechnung nach Gleichung (\ref{memfreq}) liefert
mit gewichteten Materialdaten für die Resonanzfrequenz einen Wert von etwa
4,91~kHz. Dieses entspricht einer Abweichung von 5~\% im Vergleich zum
numerisch ermittelten Wert der Serienresonanzfrequenz. \\
%----------------------- Beginn: table ---------------------------
\begin{table}[htb]
\caption{\label{tabmodellparamkeff}
 Einfluß der Modellparameter bei der Berechnung der Resonanzfrequenzen
 $f_{s}$, $f_{p}$ und des effektiven elektromechanischen
 Kopplungsfaktors $k_{eff}$}
\begin{center}
\begin{tabular} {|l||c|c|c|c||c|c|c|c||c|}
\hline
 FE-Modell & P1 & P2 & P3 & P4 & M1 & M2 & M3 & M4 & MR \\
\hline \hline
 Elemente  & 200 & 200 & 200 & 3200 & 450 & 450 & 578 & 1058 & 2714 \\
 Knoten    & 363 & 363 & 363 & 5043 & 795 & 795 & 999 & 1767 & 4286 \\
 MDOF      & 300 & 300 & 300 & 300  & 100 & 100 & 100 & 300  & 300 \\
\hline
 Seitenteil.: & 10 & 10 & 10 & 40 & 13 & 13 & 13 & 19 & 31 \\
 Aspektver.:  & 23 & 23 & 23 & 6  & 18 & 18 & 18 & 12 & 7  \\
\hline \hline
 $f_{s}$ [Hz] & 4016 & 5196 & 5281 & 4907 & 5015 & 4836 & 4871 & 4745 & 4677 \\
 $f_{p}$ [Hz] & 4143 & 5298 & 5378 & 5012 & 5117 & 4931 & 4967 & 4842 & 4755 \\
\hline
 $k^{lin}_{eff}$ [\%]
 & 24,8 & 19,6 & 19,0 & 20,5 & 20,0 & 19,6 & 19,7 & 20,0 & 18,1 \\
\hline
\end{tabular}\\
\end{center}
\end{table}
%----------------------- Ende: table ---------------------------
In {\bf Tabelle~\ref{tabmodellparamkeff}} sind die Ergebnisse der
Parametereinflüsse auf die Resonanzfrequenzen und den effektiven
elektromechanischen Kopplungsfaktor $k_{eff}$ der Grundschwingungsmode, der
nach Gleichung~(\ref{keff}) ermittelt wurde, zusammengestellt. Für kleine
Kopplungsfaktoren ($k_{eff}^{2} \ll 1$ ) gilt näherungsweise \cite{Sie81}:
%
\begin{eqnarray}
\label{keffnaeherung}
 k_{eff} & = & \sqrt{ \frac{f_{p}^{2} - f_{s}^{2}}{f_{p}^{2}}}
            \approx \sqrt{ 2 \frac{f_{p} - f_{s}}{f_{p}}}
\end{eqnarray}
%
Da die Differenz zwischen beiden Werten für die hier betrachteten
Kopplungsfaktoren in der Regel kleiner als 0,1~\% ist, wird im weiteren
die {\em lineare} Näherung $k^{lin}_{eff}$ verwendet\footnote{Bei
der Untersuchung von $PZT$-Keramiken, deren Kopplungsfaktor erheblich höher
als der von piezoelektrischen $AlN$- und $ZnO$-Dünnschichten ist, wird die
exakte Definition nach Gleichung~(\ref{keff}) verwendet.}.\\
Um bei den folgenden FE-Berechnungen den Modellieraufwand in vertretbaren
Grenzen bei gleichzeitig ausreichender numerischer Genauigkeit zu halten,
wird das FE-Modell M1 eingesetzt. Damit läßt sich das tendenzielle Verhalten
der piezoelektrisch angetriebenen Membranstrukturen bei moderaten
Rechenzeiten systematisch untersuchen. Auf der anderen Seite stellen
vielmehr die nur ungenau bekannten Eingangsdaten der Berechnung --
hier sind insbesondere die Materialeigenschaften und die Vorspannung der
$ZnO$-Dünnschicht zu nennen -- erhebliche Einflußfaktoren dar, die im
nachfolgenden näher untersucht werden.



\section{Elektromechanischer Kopplungsfaktor}
\label{elektromech}

Zur Verifikation der Bimorphmodelle und Überprüfung der
Modellierungsgenauigkeit der piezoelektrischen FE-Berechnungen wurden
Siliziummembranen vermessen, auf die eine
Piezokeramik\footnote{Die Siliziummembranen mit hybrid aufgeklebter
%(Cyanacrylat)
bzw.\ aufgelöteter (Goldschicht) Piezokeramik und
$ZnO$-Beschichtung wurden bereits in einer sehr frühen
Phase des BMFT-Verbundprojektes freundlicherweise von Herrn Dr. G. Flik,
{\sl Robert Bosch GmbH}, Gerlingen, für Meßzwecke zum Abgleich der
FE-Modelle zur Verfügung gestellt.} hybrid aufgebracht war. Da die
Materialeigenschaften von PZT-Piezokeramiken (hier: {\sl VIBRIT}) sehr
genau bekannt sind und im Gegensatz zu piezoelektrischen Dünnschichten
kaum Prozeßschwankungen unterliegen, konnten die Fehler der
Eingangsdaten auf ein Minimum reduziert werden. Insbesondere weisen die
Hybride keine inneren mechanischen Spannungen auf. Mit Hilfe dieser
hybriden Bimorphwandler konnten in der Entwicklungsphase der
Dünnschichtsensoren die piezoelektrischen Modellrechnungen überprüft und
das frequenzanaloge Sensorprinzip an Drucksensorprototypen meßtechnisch
nachgewiesen werden. Ein zusätzlicher Vorteil der Piezokeramiken ist ihr
gut meßbares elektrisches Impedanzverhalten und die Kenntnis des exakten
elektrischen Ersatzschaltbildes.\\


\subsection{Piezokeramik-Hybride}
\label{hybride}

Das dynamische Verhalten der Bimorphwandler wurde mit optischen und
elektrischen Meßmethoden vermessen. Die Abmessungen der untersuchten
Siliziummembranen betrugen 9,2~x~9,2~$mm^{2}$ und die Dicke der
{\sl VIBRIT}-Piezokeramiken jeweils 200~$\mu$m. Mit dem
Laservibrometer wurde das Amplitudenspektrum optisch aufgenommen und die
mechanische Resonanzfrequenz $f_{res}$, die dynamische Resonanzamplitude
und die mechanische Schwingungsgüte\footnote{Die mechanische
Schwingungsgüte wurde durch die Bestimmung der Halbwertsbreite
$\Delta \omega$ (3dB-Abfall der Maximalamplitude $A_{res}$)
der frequenzabhängigen Amplitudenkurve nach Gleichung~(\ref{qmess})
ermittelt.} $Q_{mech}$
der Grundbiegeschwingung $M_{11}$ bestimmt. Mit Hilfe eines
Impedance/Gain-Phase-Analyzers ({\sl HP4194A}) wurde das frequenzabhängige
Impedanz- und Phasenverhalten vermessen. Durch Anpassung einer
Impedanzcharakteristik an die gemessenen Kurvenverläufe gemäß dem
elektrischen Ersatzschaltbild in Abbildung~\ref{abbvglbulktf} wurden die
elektrischen Ersatzdaten $R, L, C$ und $C_{0}$ ermittelt. Mit Hilfe der
Ersatzdaten lassen sich die elektrische Schwingungsgüte und
der effektive elektromechanische Kopplungsfaktor bestimmen \cite{Til93}:
\begin{eqnarray}
\label{qkmelek}
 Q_{elek}       & = & \frac{\omega_{s}L}{R} = \frac{1}{\omega_{s}RC}
                  = \frac{1}{R} \sqrt{\frac{L}{C}} \\
\label{krlc}
 k_{eff}^{RLC}  & = & \sqrt{\frac{C}{C_{0} + C}}
                \approx \sqrt{\frac{C}{C_{0}}}
\end{eqnarray}
Während die Schwingungsgüte $Q_{elek}$ durch den ohmschen Widerstand $R$,
der der Dämpfung entspricht, begrenzt ist, hängt der Kopplungsfaktor im
wesentlichen
vom Kapazitätsverhältnis $C/C_{0}$ ab, da in der Regel $C_{0} \gg C$
aufgrund der Größe der Anschlußelektroden erfüllt ist.
Zur Charakterisierung der Wandlergüte wird in der Literatur auch als
\glqq {\em Figure of Merit}\grqq \, die Größe $M$ angegeben, die das Produkt
aus der Schwingungsgüte und dem Quadrat des Kopplungsfaktors darstellt,
falls $k_{eff} \ll 1$ \cite{IEEE}:
\begin{eqnarray}
\label{merit}
 M & = & \frac{1}{\omega_{s}RC_{0}}
     = Q \frac{k_{eff}^2}{1 - k_{eff}^2} \approx Q k_{eff}^2
\end{eqnarray}
Bei der Optimierung der Wandlereigenschaften spielen beide Größen eine
wichtige Rolle und sollten {\em gleichzeitig} maximiert werden.
Der Kopplungsfaktor läßt sich nach Gleichung~(\ref{krlc}) erhöhen, wenn die
Elektrodenfläche und damit die statische Kapazität $C_{0}$ verringert wird.
Jedoch entspricht dieses in der Regel nicht einer modenselektiven
Anregungsgeometrie (siehe hierzu Kapitel~5.5.1), auf die es bei resonanten
Sensoren im Gegensatz zu piezoelektrisch betriebenen Aktoren ankommt.\\
%----------------------- Beginn: table ---------------------------
\begin{table}[htb]
\caption{\label{tabpiezomess}
 Experimentelle Charakterisierung von Silizium-Bimorphmembranen
 mit hybrider Piezokeramik}
\begin{center}
\begin{tabular} {|l||c|c|c|c|}
\hline
 Wandler & PA & PB & PC & PD \\
\hline \hline
% $h_{Si}$ [$\mu$m] & 200  & 200   & 150  &  150  \\
 $h_{Si}/h_{PZT}$  & $\sim$0,5 & $\sim$0,75 & $\sim$0,5 & $\sim$0,75 \\
 Abm. [$mm^{2}$]:  & 6x6  & 12x12 & 6x6  &  12x12 \\
\hline
 \multicolumn{5}{|c|}{Optische Messungen} \\
\hline
 $f_{res}$ [Hz]     & 15430 & 24428 & 16849 & 25153 \\
 $A_{SS}$ [$\mu$m]  &  2,3  &  2,1  &  4,4  &  1,4  \\
 $Q_{mech}$         &  42   &  65   &  62   &  84   \\
\hline
 \multicolumn{5}{|c|}{Elektrische Messungen} \\
\hline
 $f_{s}$ [Hz]   & 15162 & 24388 & 16854 & 25113 \\
 $f_{p}$ [Hz]   & 15888 & 24669 & 17437 & 25328 \\
\hline
% quadratisch exakte Werte
 $k_{eff}$ [\%] & 29,9  & 15,1 & 25,6 & 13,0 \\
\hline
 \multicolumn{5}{|c|}{Elektrische Ersatzdaten } \\
\hline
 $R$ [k$\Omega$] & 2,22 & 1,65  & 1,24 & 3,46  \\
 $L$ [H]         & 1,25 & 0,28  & 1,10 & 1,12  \\
 $C$ [pF]        & 86,8 & 149,7 & 80,6 & 36,0  \\
 $C_{0}$ [nF]    & 1,83 & 8,14  & 1,35 & 7,34  \\
\hline
 $Q_{elek}$      & 54   & 26    & 94    & 51 \\
 $k_{eff}^{RLC}$ [\%] & 21,3 & 13,4 & 23,7 & 7,0 \\
\hline
\end{tabular}\\
\end{center}
\end{table}
%----------------------- Ende: table ---------------------------
In {\bf Tabelle~\ref{tabpiezomess}} sind die experimentellen Ergebnisse
verschiedener hybrider Piezokeramik-Siliziummembranen zusammengefaßt.
Die Anregungsspannung betrug bei den optischen Messungen etwa 4~$V_{SS}$,
bei den elektrischen Messungen maximal 1~V.
Die optisch vermessenen Resonanzfrequenzen $f_{res}$ entsprechen in
etwa den Serienresonanzfrequenzen $f_{s}$, so daß auf niederohmige
elektrische Abschluábedingungen bei den Messungen geschlossen werden kann.
Die erreichbaren Schwingungsamplituden $A_{SS}$ variieren bei der
anliegenden Anregungsspannung von 4~$V_{SS}$ zwischen 1,4--4,4 $\mu$m.
Aufgrund des linearen Zusammenhangs zwischen der Anregungsspannung und den
Schwingungsamplituden befindet man sich im Kleinsignalbereich. Die an der
Piezokeramik anliegende elektrische Feldstärke beläuft sich auf etwa
$E$~=~0,1~V/$\mu$m.\\
Die kleinen Piezokeramiken (6~x~6~$mm^{2}$) zeichnen sich gegenüber den
großen (12~x~12~$mm^{2}$) durch einen erheblich höheren effektiven
elektromechanischen Kopplungsfaktor $k_{eff}$ aus. So erhöht sich der
$k_{eff}$--Wert vom Wandler PB zum Wandler PA um das Doppelte. Den
entscheidenden Einfluß auf die Größe des Kopplungsfaktors besitzt das
Schichtdickenverhältnis $h_{Si}/h_{PZT}$, wie in
Kapitel~\ref{schichtdickenabhaengigkeit} gezeigt wird. Zum anderen spielt
die Membranüberdeckung durch die Piezokeramik eine gewisse Rolle, wodurch
das Kapazitätsverhältnis $C/C_{0}$ beeinflußt wird.
In Kapitel~5.5 wird auf den Einfluß der lateralen Schichtstrukturierung
auf das Verhalten von piezoelektrischen Dünschichtstrukturen
eingegangen.\\
%Die Korrelation zwischen dem direkt aus den beiden Frequenzen $f_{s}$
%und $f_{p}$ nach Gleichung (\ref{keffnaeherung}) bestimmten
%$k_{eff}$--Wert und dem mit Hilfe der elektrischen Ersatzdaten
%nach Gleichung (\ref{qkmelek}) stimmen recht gut überein.
%Das gleiche gilt für die Übereinstimmung der mechanisch und elektrisch
%bestimmten Schwingungsgüten $Q_{mech}$ und $Q_{elek}$ nach
%Gleichung~(\ref{qkmelek}).
Die statische Kapazität $C_{0}$ kann durch
die Permittivität $\varepsilon_{33}$ und die geometrischen Abmessungen
des Kondensators, der durch die Anschlußelektroden der Piezokeramik
gebildet wird, abgeschätzt werden. Es gilt
$C_{0}= \varepsilon_{0} \varepsilon_{33} A / h_{PZT}$, wobei
$A$ die effektive Elektrodenfläche (6~x~6 bzw.\ 12~x~12 $mm^{2}$) ist.
Unter der Annahme von $\varepsilon_{33} / \varepsilon_{0} = 1600\pm300$
und $h_{PZT} = 200~\mu$m ($\pm10~\%$) folgt für die beiden
Keramikabmessungen $C_{0}$=2,6 bzw.\ 10,2~nF. Diese Werte stimmen
größenordnungsmäßig gut mit den gemessenen Werten überein.\\
%
%Damit kann das elektrische Ersatzschaltbild
%in erster Näherung zur Beschreibung der piezoelektrischen Bimorphe
%herangezogen werden.\\
%
%----------------------- Beginn: table ---------------------------
\begin{table}[htb]
\caption{\label{tabpiezosim}
 FE-Modellierung von Silizium-Bimorphmembranen mit hybrider Piezokeramik}
\begin{center}
\begin{tabular} {|l||c|c|c|}
\hline
 $h_{Si}$ [$\mu$m]  & 90   & 95    & 100  \\
 $h_{Si}/h_{PZT}$   & 0,45 & 0,475 & 0,5  \\
\hline \hline
 $f_{s}$ [Hz]       & 15683 & 16388 & 17074  \\
 $f_{p}$ [Hz]       & 16276 & 17041 & 17785  \\
\hline
% quadratisch exakte Werte
 $k_{eff}$ [\%]     & 26,7 & 27,4 & 28,0 \\
\hline
\end{tabular}
\end{center}
\end{table}
%----------------------- Ende: table ---------------------------
Zum Vergleich von piezoelektrischen FE-Berechnungen an Bimorphmembranen
mit den experimentellen Daten, wurde der Wandler mit den Abmessungen
6~x~6~$mm^{2}$ und der Piezokeramikdicke von 200~$\mu$m herangezogen.
Hierfür wurde eine Membrangeometrie mit strukturierten $ZnO$--Schichten
herangezogen. % \cite{Messner}.
Aufgrund der Dickenschwankungen der
Siliziummembranen, die eine nominelle Dicke von etwa 100~$\mu$m aufwiesen,
wurden die Serien- und Parallelresonanzfrequenzen, sowie die
$k_{eff}$--Werte in Abhängigkeit der Siliziummembrandicke $h_{Si}$
berechnet. Die Materialeigenschaften wurden anisotrop angenommen. Die
Anzahl der Elemente betrug 1970, die der Knoten 2828. Es wurde mit der
{\sl Householder}-Methode gerechnet (300 MDOF).
In {\bf Tabelle~\ref{tabpiezosim}} sind die numerischen Ergebnisse
zusammengefaßt. Die gemessenen Resonanzfrequenzen $f_{s}$ und $f_{p}$ liegen
für die beiden Wandler PA und PC im Bereich 15--17~kHz, so daß ausgehend
von den FE-Resultaten auf eine effektive Siliziummembrandicke von etwa
90--100~$\mu$m geschlossen werden kann. Die berechneten Kopplungsfaktoren
27--28~\% liegen zwischen den beiden gemessenen. Die Abweichungen zwischen
den gemessenen und berechneten Werten lassen sich durch die Toleranz der
Siliziummembrandicke und die Vernachlässigung der dielektrischen Dämpfung
der Piezokeramik erklären.\\
%
Gegenstand der weiteren Untersuchungen von Bimorphwandlern mit
piezoelektrischen Dünnschichten sind die Einflüsse:
\begin{itemize}
\item des Schichtdickenverhältnisses ($h_{Si}/h_{Piezo}$)
\item des Dünnschichtsystems ($AlN, PZT, ZnO$)
\item der lateralen Schichtstrukturierung.
\end{itemize}



\subsection{Zinkoxid-Dünnschichten}
\label{zno}

Neben Membranstrukturen eignen sich einseitig eingespannte Siliziumzungen
als technologische Teststrukturen zur Entwicklung der
Dünnschichtprozesse. Die Güte der piezoelektrischen
Dünnschichten ist aus der Sicht der meisten Anwendungen durch den
elektromechanischen Kopplungsfaktor bestimmt. Da der materialabhängige
Anteil $k_{mat}$ den Messungen nicht direkt zugänglich ist, muß an
mikromechanischen Strukturen der effektive Wert $k_{eff}$ ermittelt
und ausgewertet werden. Im Gegensatz zu Membranen können bei den einseitig
eingespannten Zungenstrukturen zusätzlich die statischen
%
Auslenkungen\footnote{Die Bestimmung der $k_{eff}$-Werte durch statische
Messung der Zungenauslenkungen wurde vom BMFT-Verbundpartner
{\sl Robert Bosch GmbH} durchgeführt. Gegenüber der Resonanzmethode weist
diese Methode den Nachteil auf, daß die zugrundeliegende analytische
Näherungsformel, die für die Berechnung von $k_{eff}$ herangezogen
wird, die mechanischen und elektrischen Randbedingungen, sowie den
Bimorphcharakter nicht exakt berücksichtigt. Weiterhin kommt hinzu, daß
durch die relativ hohen Anregungsspannungen %($E \approx 3 V/\mu$m)
nichtlineare Effekte auftreten können (Großsignalverhalten). Die auf diese
Weise ermittelten $k_{eff}$-Werte fallen daher im Vergleich zu dynamisch
vermessenen Werten systematisch zu niedrig aus \cite{ABV93}.}
%
infolge elektrischer Ansteuerung gut vermessen werden, da sie im Vergleich
zu Membranstrukturen einen größeren Hub (einige Mikrometer) aufweisen.
Im folgenden werden die
experimentellen Daten mit den Resultaten piezoelektrischer FE-Berechnungen
verglichen und die Modellierungsgenauigkeit der numerischen Methoden, sowie
die piezoelektrischen Bimorphmodelle überprüft.\\
%
Die Herstellung und Vermessung der
ganzflächig von einer piezoelektrischen $ZnO$-Dünnschicht überdeckten
Zungenstrukturen erfolgte vom BMFT-Verbundpartner {\sl Robert Bosch GmbH}.
Die Meßergebnisse wurden für die Entwicklung der piezoelektrischen
FE-Modelle im Rahmen dieser Arbeit freundlicherweise zur Verfügung gestellt
\cite{Flik}.\\
%
In {\bf Tabelle~\ref{tabzungen}} sind die optisch vermessenen
Resonanzfrequenzen $f_{res}$, die durch Impedanzmessungen bestimmten
Serien- und Parallelresonanzfrequenzen $f_{s}$ und $f_{p}$, sowie die
daraus abgeleiteten effektiven Kopplungsfaktoren $k_{eff}$ zusammengestellt.
%
%Die Siliziumzungen haben eine zehnfach größere Dicke als die $ZnO$-Schichten,
%so daß die analytischen Abschätzungen aus Kapitel~\ref{skalierungsverhalten}
%zum Vergleich herangezogen werden können.
%
Bei den FE-Berechnungen wurde von einem homogenen Siliziumbalken, der
ganzflächig von einer $ZnO$-Schicht bedeckt ist, ausgegangen. Die einseitige
Einspannung der Zungenstruktur wird durch eine schräge (111)-Siliziumebene
gebildet. Die Länge der Siliziumzungen betrug 7~mm, die Breite 5~mm.
Es wurden sowohl bei Silizium, als auch bei $ZnO$ anisotrope Materialdaten
\cite{LB82} berücksichtigt. Um die numerischen Fehlereinflüsse möglichst
niedrig zu halten, wurden die Zungen lateral sehr fein unterteilt.
Das gröbere FE-Modell weist 30~Elemente in Längsrichtung und 8~Elemente in
der Breite auf, während das feinere Modell entsprechend 70 bzw.\ 15~Elemente
aufweist. Die Elementunterteilung in Dickenrichtung war weniger kritisch,
wie in Kapitel~4 bereits für die Membranstrukturen gezeigt wurde, so daß
für die Siliziumzunge zwei Elementlagen und für die $ZnO$-Schicht eine
Elementlage genügten. Zur Berechnung der Eigenfrequenzen wurde das
{\sl Householder}-Verfahren mit 300~MDOFs eingesetzt. Die Angaben für die
Frequenzwerte sind aufgerundet, während die $k_{eff}$--Werte exakt
angegeben sind.\\
%----------------------- Beginn: table ---------------------------
\begin{table}[htb]
\caption{\label{tabzungen}
 Charakteristische Kenndaten $ZnO$-beschichteter Siliziumzungenstrukturen
 (Vergleich: FE-Berechnung -- Messung)}
\begin{center}
\begin{tabular}{|l||c|c|c|c|c|}
\hline
Zunge:              &  SZ1   &  SZ2   &  SZ3   &  SZ4 &  SZ5 \\
\hline \hline
 $h_{Si}$  [$\mu$m] & 123,7  & 124,5  &  137,2
 & \multicolumn{2}{|c|}{124,5} \\
 $h_{ZnO}$ [$\mu$m] & 10,4   & 10,1   &  10,7
 & \multicolumn{2}{|c|}{7,75} \\
\hline
 \multicolumn{6}{|c|}{Messungen} \\
\hline
 $f_{res}$ [Hz]  & 3330   & 3410   &  3680 & \multicolumn{2}{c|}{3455} \\
\hline
 $f_{s}$ [Hz]    & 3358   & 3416   &  3642   & 3418  & 3420  \\
 $f_{p}$ [Hz]    & 3376   & 3430   &  3654   & 3439  & 3445  \\
 $k_{eff}$ [\%]  & 10,3   & 9,0    &  8,1    & 11,0  & 12,0  \\
\hline
 \multicolumn{6}{|c|}{FE-Modellierung} \\
\hline
 Elemente  & \multicolumn{2}{|c|}{2080} & \multicolumn{3}{c|}{4410} \\
 Knoten    & \multicolumn{2}{|c|}{2781} & \multicolumn{3}{c|}{6144} \\
\hline
 $f_{s}$ [Hz]   & 3474   & 3495   &  3832   & \multicolumn{2}{c|}{3474} \\
 $f_{p}$ [Hz]   & 3502   & 3523   &  3861   & \multicolumn{2}{c|}{3497} \\
 $k_{eff}$ [\%] & 12,6   & 12,5   &  12,3   & \multicolumn{2}{c|}{11,3} \\
\hline
 $\frac{\Delta k}{k}$ [\%] & -22  & -39   &  -52    &   -3  & +6    \\
\hline
\end{tabular}
\end{center}
\end{table}
%----------------------- Ende: table ---------------------------
Die Zungen SZ1 und SZ2 weisen ein
$Si/ZnO/SiO_{2}/Si_{3}N_{4}$--Schichtsystem auf und
die Zungen SZ3 und SZ4 bestehen aus einem vereinfachten
$Si/ZnO/Al$--Schichtsystem. Die Zunge SZ5 wurde aus hochdotiertem
$p^{++}$--Silizium hergestellt und zeichnet sich bei gleichem Schichtsystem
wie Zunge SZ4 durch den höchsten Kopplungsfaktor von etwa 12~\% aus.
Die Dicken der $SiO_{2}$-- und $Si_{3}N_{4}$--Schicht betrugen 150 und
300~nm und wurden aufgrund ihrer im Vergleich zum $ZnO$ geringen Schichtdicke
bei den FE-Berechnungen nicht berücksichtigt.\\
%
Das Schichtdickenverhältnis $h_{Si}/h_{ZnO}$ ändert sich bei den
Zungenstrukturen nur unwesentlich von  12--16, so daß die effektiven
Kopplungsfaktoren davon kaum betroffen werden. Dieses belegen auch die
FE-Berechnungen, die unter der Annahme ideal homogener Schichteigenschaften
unter Verwendung der Literaturmaterialdaten (siehe
Tabelle~\ref{tabpiezoelektrika}) durchgeführt wurden.
Die Frequenzabweichungen betragen bei den Zungen SZ4 und SZ5 etwa 1,5~\%
und nehmen bis auf 4--6~\% bei den ersten drei Zungen zu.
Die rechnerisch ermittelten Resonanzfrequenzen fallen gegenüber den
gemessenen Werte alle systematisch höher aus, was durch die Druckspannungen
(bis zu -500~MPa) in den $ZnO$-Schichten erklärt werden kann. Experimentell
wurde die innere Spannung der Zungen durch optische Vermessung der
Zungenauslenkung bestimmt \cite{Flik}. Insbesondere zeichnen sich die
beiden Zungen SZ4 und SZ5 durch spannungsarme $ZnO$-Schichten aus.
Nur bei diesen beiden Strukturen stimmen die Werte der effektiven
elektromechanischen Kopplungsfaktoren hinreichend gut mit den FE-Resultaten
überein. Die Abweichungen betragen nur 3 bzw.\ 6~\%, im Gegensatz zu den
anderen drei Zungen, bei denen die Abweichungen bis zu 52~\% betragen.
Ein Grund für diese teilweise hohen Abweichungen ist der Einfluß der
mechanischen Verspannung der $ZnO$-Schichten.
Dieses belegen auch die Messungen an $ZnO$-beschichteten Membranstrukturen,
die in Abhängigkeit der inneren Spannung bis zu $\pm$100~\% vom numerisch
ermittelten $k_{eff}$--Wert abweichen \cite{ABV93}.
Die gemessenen maximalen Feldstärken lagen zwischen 4--12~V/$\mu$m und
decken sich mit den in der Literatur angegebenen Wert von etwa 10~V/$\mu$m
\cite{Smi92b}.\\
%
Unter der Voraussetzung, daá bei den hier betrachteten Biegewandlern
im wesentlichen der transversale bzw.\ planare elektromechanische
Kopplungsfaktor $k_{31}$ bzw.\ $k_{p}$ ausschlaggebend ist,
kann mit Hilfe von Gleichung~(\ref{kmat}) auf einen {\em effektiven,
prozeßabhängigen} elektromechanischen Kopplungsfaktor $k_{31}^{eff}$ bzw.\
$k_{p}^{eff}$ geschlossen werden. Aufgrund der Proportionalität zwischen
$k_{31}$ und $d_{31}$ ist es möglich durch einen Vergleich des
experimentellen $k_{31}^{eff}$--Wertes mit den FE-Resultaten einen
schichtspezifischen
$d^{eff}_{31}$-Wert\footnote{Wird bei den Zungenstrukturen SZ1--SZ3 eine
Abweichung von 22--52~\% zwischen Messung und FEM zugrundegelegt, so kann
mit dem in der FE-Rechnung verwendeten Literaturwert von
$d_{31}$~=~5,12~pC/N auf einen effektiven piezoelektrischen
Kopplungskoeffizienten von lediglich $d^{eff}_{31}$~=~2,5--4,0~pC/N
geschlossen werden. Dieser Wertebereich liegt in der Größenordnung von
Dünnschichtprozessen und wird auch von anderen Forschungsgruppen angegeben
\cite{Pra93}. Für den spannungsarmen Beschichtungsprozeß, mit dem die
Zungen SZ4 und SZ5 hergestellt wurden, errechnet sich ein gegenüber dem
Literaturwert leicht erhöhter Koeffizient von $d^{eff}_{31}$~=~5,43~pC/N.}
bei fester Geometrie abzuleiten.
Dieser Wert ist geeignet, um die Beschichtungsprozesse zu optimieren,
erlaubt aber keinen direkten Rückschluß auf die absolute Größe des rein
materialabhängigen $d^{mat}_{31}$-Wertes. Zur eindeutigen Bestimmung dieses
Wertes ist es zusätzlich erforderlich, den mechanischen
Steifigkeitskoeffizienten $S^{E}_{11}$ und die Permittivität
$\varepsilon^{\sigma}_{33}$ an der betrachteten Dünnschichtstruktur zu
bestimmen.\\
%
Die FE-Berechnungen an piezoelektrischen Bimorphstrukturen berücksichtigen
nicht die Schichtmorphologie und die innere Schichtspannungen im Zinkoxid
oder sonstige prozeßbedingte Auswirkungen, so daß diese Rechnungen nur
qualitative Aussagen ermöglichen. Im weiteren wird daher von idealisierten
Schichtsystemen ausgegangen, die durch die Materialdaten der Literatur
beschrieben werden, und die Einflüsse der Wandler- und Elektrodengeometrien
auf das statische und dynamische Verhalten prinzipiell untersucht.

