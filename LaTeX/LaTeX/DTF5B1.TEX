\subsection{Schichtdickenabhängigkeit}
\label{schichtdickenabhaengigkeit}


Um eine möglichst hohe elektromechanische Kopplung bei der piezoelektrischen
Anregung von Bimorphwandlern zu erreichen, ist unter anderem die Kenntnis
des günstigsten Schichtdickenverhältnisses von Bedeutung. Einerseits ist die
Dicke des passiven Wandlersubstrates ein ausschlaggebender
Parameter, beispielswiese für die Sensorempfindlichkeit, andererseits
unterliegen die anvisierten Schichtdicken der aktiven piezoelektrischen
Dünnschicht starken technologischen Einschränkungen (z.B.\ Sputterrate,
Schichthomogenität).
Mit verschiedenen Bimorphmodellen wurde daher der Einfluß des
Schichtdickenverhältnisses der Siliziummembran $h_{Si}$ zur Piezoschicht
$h_{Piezo}$ und die Auswirkung verschiedener piezoelektrischer Materialien
untersucht. Die Dicke der Siliziummembran betrug bei allen Variationen
20~$\mu$m, die Seitenlänge 9,2~mm. Für die Berechnungen wurde das FE-Modell
M4 mit ganzflächiger Piezoschicht, das in Kapitel~\ref{bimorphstruktur}
beschrieben ist, herangezogen.\\
In {\bf Abbildung~\ref{abbkeffvontt}} ist der effektive elektromechanische
Kopplungsfaktor in Abhängigkeit des Schichtdickenverhältnisses für
piezoelektrische $AlN$- und $ZnO$-Schichten, sowie $PZT$-Keramik abgebildet.
%----------------------- Beginn: Figure-Environment ----------------------
\begin{figure}[htb]
%\vspace*{8cm}

\begin{center}
% --- Dateiname des Bildes
\input{abbfdra.tex}
\setabbfdra
\end{center}
\caption{\label{abbkeffvontt}
 Effektiver elektromechanischer Kopplungsfaktor $k_{eff}$ einer
 Silizium-Bimorphmembran in Abhängigkeit des Schichtdickenverhältnisses
 für verschiedene Piezoelektrika (Meßwerte für $ZnO$-Dünnschichten)}
\end{figure}
%----------------------- Ende: Figure-Environment ----------------------
Die Kurvenverläufe weisen ein jeweils deutlich ausgeprägtes Maximum auf,
das bei den verschiedenen piezoelektrischen Materialien auf unterschiedliche
optimale Schichtdicken führt. Das Anwachsen des elektromechanischen
Kopplungsfaktors bei zunehmender Schichtdicke ist auf eine effizientere
Energieumwandlung zurückzuführen. Ab einer bestimmten Piezoschichtdicke
nimmt die Energiekonversion allerdings nicht mehr weiter zu, da nur die
oberflächennahen Piezoschichtbereiche zur elektromechanischen Anregung
beitragen, so daß bei weiter zunehmender Schichtdicke $k_{eff}$ wieder
abnimmt. Für $AlN$, $ZnO$ und $PZT$ werden maximalen Kopplungsfaktoren
von etwa
9~\%, 22~\% und 38~\% bei Schichtdicken von etwa 5--15~$\mu$m erreicht und
liegen damit noch im realisierbaren Bereich der Dünnschichttechnik.
Bei der Oberflächenmikromechanik, bei der typische Substratdicken im
Bereich von 0,5--2~$\mu$m liegen, verbessert sich das
Schichtdickenverhältnis entscheidend zu gunsten der Dünnschichttechnik.
Zum Vergleich sind experimentell bestimmte $k_{eff}$-Werte
von $ZnO$-beschichtenen Silizium-Membrandrucksensoren in
Abbildung~\ref{abbkeffvontt} mit eingezeichnet.
Die vermessenen $Si/ZnO$-Membranen besaßen Schichtdickenverhältnisse
von etwa 3,4--4,5 und wiesen in Abhängigkeit der Prozeßparameter
verschiedene $k_{eff}$--Werte auf. Wie bei den Zungenstrukturen konnte
ein erheblicher Einfluß der inneren Schichtspannung festgestellt werden,
die ihrerseits von der Sputterrate, der Targettemperatur und
dem Argon-Sauerstoff-Gasgemisch im Rezipienten abhängig ist
\cite{ABV93}.\\
Die Zusammenfassung der numerischen Resultate ist in
{\bf Tabelle~\ref{taboptdicke}} zu finden.
%----------------------- Beginn: table ---------------------------
\begin{table}[htb]
\caption{\label{taboptdicke}
 Optimierung der Schichtdickenverhältnisse bei Silizium-Bimorphmembranen}
\begin{center}
\begin{tabular} {|l||c|c|c|}
\hline
 Schicht & $AlN$ & $ZnO$ & $PZT$ \\
\hline \hline
 $h_{Piezo}$ [$\mu$m]   &  $5\pm1$   &  $10\pm1$   &  $15\pm1$   \\
 $h_{Si}/h_{Piezo}$     &  $\sim$4,0 &  $\sim$2,0  &  $\sim$1,3  \\
\hline
 $E_{Si}/E_{Piezo}$     &  0,59    &  1,31     &  2,55  \\
\hline
 $f_{s}$ [Hz]           & 4520,5    &  3975,4    &  3286,2     \\
 $f_{p}$ [Hz]           & 4539,3    &  4072,6    &  3551,9     \\
\hline \hline
 $k_{eff}$ [\%]         &  9,1      &  21,7      &  38,0       \\
 $k^{mat}_{p}$ [\%]     &  18       &  40        &  61         \\
 $k_{eff}/k^{mat}_{p}$  &  0,51     &  0,54      &  0,62       \\
\hline
\end{tabular}\\
\end{center}
\end{table}
Bezieht man die effektiven
Kopplungsfaktoren $k_{eff}$ auf die materialabhängigen Anteile
$k^{mat}_{p}$ der jeweiligen Piezoelektrika
(siehe Tabelle~\ref{tabpiezoelektrika}), so ergibt sich für $AlN$-
und $ZnO$-Dünnschichten ein Wert von $k_{eff}/k^{mat}_{p} \approx 0,5$.
Bei der $PZT$-Keramik weicht der Wert geringfügig aufgrund des stärker
abweichenden Schichtdickenverhältnisses ab. Werden alle drei Kurvenverläufe
auf $k^{mat}_{p}$ bezogen, so gibt es einen gemeinsamen Schnittpunkt
der drei Kurven bei einem Schichtdickenverhältnis von etwa 4,2 mit dem
gemeinsamen Wert von $k_{eff}/k^{mat}_{p} \approx 0,5$. Bei diesem
Schichtdickenverhältnis verschwindet der Geometrieeinfluß der
Siliziummembran, so daß dieses eine ideale {\em materialunabhängige}
Wandlerkonfigura\-tion darstellt und sich als mikromechanische Teststruktur
zur Charakterisierung piezoelektrischer Dünnschichteigenschaften eignet.\\
%
%(... evtl. Hinweis auf Formel von Prak für: $k=f(h_{si}/h_{Piezo}$,
%           $E_{si}/E_{Piezo}$)...)


\section{Modellierung der piezoelektrischen Anregung}
\label{impedanz}

Die experimentelle Bestimmung des effektiven elektromechanischen
Kopplungsfaktors erfolgte in Kapitel~5.3 durch die Messung des
frequenzabhängigen Impedanz- und Phasenverlaufes. Bei den piezoelektrischen
FE-Berechnungen genügt es, mit Hilfe der Modal\-analyse das Eigenwertproblem
unter den beiden elektrischen Randbedingungen ($\vec E=0$)
und ($\vec D=0$) zu betrachten, dessen Lösung aus den beiden
{\em diskreten} Werten der Serien- und Parallelresonanzfrequenz besteht.
In Meáanwendungen werden die piezoelektrisch angeregten Kraft- und
Drucksensoren in einer Oszillatorschaltung betrieben, bei der das
Impedanz- und Phasenverhalten (Zweipolbetrieb) bzw.\ das elektrische
Übertragungsverhalten (Vierpolbetrieb) der Sensorelemente bekannt sein muß.
Um den frequenzabhängigen Impedanz- und Phasenverlauf zu simulieren, ist es
daher notwendig,
eine gekoppelte Feldberechnung gemäß Gleichung~(\ref{piezokopl})
durchzuführen. Unter Vorgabe einer harmonischen % elektrischen
Anregungsspannung $U(\omega)$ und einer mechanischen Dämpfung $[C_{D}]$
kann das mechanische {\em und} elektrische Verhalten der
Bimorphwandler modelliert und neben den frequenzabhängigen
Schwingungsamplituden $A(\omega)$ der Impedanz- $Z(\omega)$ und
Phasenverlauf $\theta(\omega)$ bestimmt werden.
Im folgenden soll dieses für Siliziummembranen mit $PZT$-Keramik und
$ZnO$-Dünnschicht erfolgen und mit Messungen verglichen werden.\\



\subsection{Piezokeramik-Bimorphmembran}
\label{pzthybrid}

Zur Überprüfung der gekoppelten Feldberechnungen mit piezoelektrischer
Anregung wurde erneut die Silizium-Bimorphmembran mit $PZT$-Keramik gewählt,
um die unbekannten Fehlereinflüsse seitens des Dünnschichtsystems bei der
Modellverifikation weitgehend zu minimieren.
Berechnet wurde der frequenzabhängige Impedanz- und Phasenverlauf der
Silizium-Bimorphmembran für den Frequenzbereich der ersten Biegeschwingung
mit Hilfe einer harmonischen, linearen
Frequenzganganalyse\footnote{{\sf ANSYS}-Option:
{\em Harmonic Response Analysis}.} gemäß der gekoppelten
Feldgleichung~(\ref{piezokopl}).
Die eigentliche Berechnung des Impedanz- und Phasenverlaufes wird im
{\em Postprocessing} (bei {\sf ANSYS}: {\sf POST26}-Modul) vollzogen.
Hierzu wird die von der dielektrischen Verschiebung $\vec D$ an den
Elektrodenflächen erzeugte elektrische Ladung mittels Integration
nach $Q = \int_{\Omega} \vec D d\Omega$
ermittelt\footnote{{\sf RFORCE}-Befehl zur Berechnung der
Knotenreaktionskräfte mit der
Option {\sf AMPS}.}. Anschließend wird hieraus der induzierte
elektrische Stromfluß $I(\omega) = i \omega Q$ im Frequenzbereich errechnet
und für eine vorgegebene Anregungsspannung $U = const$ die komplexwertige
Impedanz $Z(\omega) = U / I(\omega)$ ermittelt \cite{Ecc92}.
Da dielektrische Dämpfungsbeiträge nicht erfaßt werden
können, muß ein mechanischer Dämpfungsbeitrag gemäß
Gleichung~(\ref{cmatrix}) vorgegeben werden. Dieses erfolgte mit Hilfe
eines mittleren frequenzunabhängigen Dämpfungskoeffizienten.\\
%
Bei der verwendeten
Bimorphstruktur betrug die Membranseitenlänge 9,2~mm, die Dicke der
Siliziummembran 20~$\mu$m und die der Piezokeramik 200~$\mu$m. Das
FE-Modell\footnote{Elementanzahl: 1587, Knotenanzahl:
2382, MDOF: 300.} P2 wurde hierzu herangezogen und das
Materialverhalten der Piezokeramik ({\sl VIBRIT~420}) anisotrop
unter Verwendung des {\sf ANSYS} {\em Multi-Field}-Elementes
{\sl SOLID5} modelliert.
Für die elektrische Anregungsspannung wurde 1~V angenommen und an den
ganzflächig die Membran überdeckenden Elektroden eingeprägt.
Für die mechanische Schwingungsgüte $Q$
wurde ein mittlerer Wert von 1000 zugrundegelegt, aus dem ein
relatives Dämpfungsverhältnis von $10^{-3}$ folgt. Mit diesen Vorgaben
wurde der frequenzabhängige Impedanzverlauf berechnet, aus dem die
Serienresonanzfrequenz $f_{s} = 11,638$~kHz und die Parallelresonanzfrequenz
$f_{p} = 11,984$~kHz bestimmt wurden. Daraus folgt ein $k_{eff}$-Wert von
etwa 23,9~\%. An einer Silizium-PZT-Bimorphmembran wurden die Werte
$f_{s} = 11,495$~kHz und $f_{p} = 11,715$~kHz gemessen, aus denen sich
$k_{eff}$ zu 19,3~\% berechnen läßt.
Ein Vergleich der FE-Resultate mit den Messungen zeigt eine gute
Übereinstimmung. Die Abweichungen der Resonanzfrequenzen betragen nur
etwa 1--2~\%.



\subsection{Bimorphmembran mit ZnO-Dünnschicht}
\label{znobimorph}

Zur Modellierung des frequenzabhängigen Verhaltens einer $ZnO$-beschichteten
Siliziummembran, wurde das FE-Modell M3 aus Abbildung~5.1 herangezogen, das
zusätzlich die Einspannung durch die (111)-Siliziumebenen berücksichtigt.
Die Abmessungen wurden gemäß des in Kapitel~4 bereits optisch vermessenen
Drucksensors gewählt. Die Siliziummembrandicke betrug 50~$\mu$m, die Dicke
der $ZnO$--Schicht 11~$\mu$m. Das Materialverhalten wurde für Silizium und
Zinkoxid anisotrop angenommen, wobei auf Literaturangaben \cite{LB82}
zurückgegriffen wurde. Vorab wurden die Serien- und
Parallelresonanzfrequenzen $f_{s}$ und $f_{p}$ mit dem
{\sl Householder}-Verfahren (MDOF=50) zu 8,817~kHz und 8,965~kHz
berechnet, um den interessierenden Frequenzbereich für die
Frequenzganganalyse festzustellen.
%----------------------- Beginn: Figure-Environment ----------------------
\begin{figure}[htb]
%\vspace*{14cm}

\begin{center}
% --- Dateiname des Bildes
\input{abbnfv.tex}
\setabbnfv
\end{center}

\caption{\label{abbimpfem}
 Modellierung des mechanischen und elektrischen Schwingungsverhaltens einer
 Silizium-Bimorphmembran mit einer $ZnO$-Dünnschicht}
\end{figure}
%----------------------- Ende: Figure-Environment ----------------------
Bei der Berechnung des frequenzabhängigen Amplituden- und Impedanzverlaufes
wurden für die elektrische Anregungsspannung und die mechanische
Schwingungsgüte die gleichen Werte, wie bei der $PZT$-Membran verwendet.
Die Ergebnisse der transienten Berechnung (100 Iterationen) des mechanischen
und elektrischen Schwingungsverhaltens der Silizium-Bimorphmembran mit
$ZnO$-Dünnschicht sind in {\bf Abbildung~\ref{abbimpfem}} dargestellt.
Als mechanische Resonanzfrequenz $f_{res}$ ergibt sich ein Wert von
8,965~kHz bei einer Schwingungsamplitude von etwa 0,02~$\mu$m.
Die Serien- und Parallelresonanzfrequenzen sind bei diesen Berechnungen
sehr stark abhängig von der Anzahl der Elemente und MDOFs. Außerdem haben
die Dämpfung und die Anregungsspannung einen direkten Einfluß auf die Höhe
der Resonanzamplitude. Aus diesem Grunde wurde mit einem relativ kleinem
FE-Modell und den vereinfachten Annahmen gerechnet. Im folgenden soll
daher auf eine quantitative Auswertung verzichtet und lediglich ein
qualitativer Vergleich mit der experimentell charakterisierten
Sili\-zium-Bi\-morph\-mem\-bran mit $ZnO$-Beschichtung durchgeführt
werden. \\
In {\bf Abbildung~\ref{abbimpzno}} sind die Ergebnisse der optischen und
elektrischem Messungen abgebildet. Es handelt sich um den resonanten
Membrandrucksensor aus Kapitel~4.
%----------------------- Beginn: Figure-Environment ----------------------
\begin{figure}[htb]
%\vspace*{8cm}

\begin{center}
% --- Dateiname des Bildes
\input{abbfsea.tex}
\setabbfsea
\end{center}
\caption{\label{abbimpzno}
 Experimentelle Charakterisierung einer Silizium-Bimorphmembran
 mit $ZnO$-Beschichtung}
\end{figure}
%----------------------- Ende: Figure-Environment ----------------------
Die frequenzabhängige Schwingungsamplitude
wurde optisch mit Hilfe eines Laservibrometers schmalbandig in
einem engen Frequenzbereich um die Grundmode $M_{11}$ vermessen.
Die mechanische Resonanzfrequenz $f_{res}$ entspricht etwa 7,24~kHz und
die mechanische Schwingungsgüte $Q$ beträgt etwa 100.
In der unteren Hälfte der Abbildung~\ref{abbimpzno} ist der mit dem
{\sl HP4194A}--Impedanzanalysator elektrisch vermessene Impedanz- und
Phasenverlauf dargestellt. Die Messung wurde an der großflächigen
Aluminium-Zentralelektrode (siehe Abbildung~5.7) durchgeführt, wobei das
Siliziumsubstrat den Massekontakt bildet. Deutlich sind die Serien-
($f_{s} \approx 7,255$~kHz) und Parallelresonanzfrequenz
($f_{p} \approx 7,295$~kHz) zu sehen, aus denen sich ein effektiver
Kopplungsfaktor von etwa 10,5~\% ergibt. Die Abweichung gegenüber dem
theoretischen Wert von etwa 20~\% (siehe Abbildung~\ref{abbkeffvontt})
ist einerseits auf die Prozeßbedingungen beim $ZnO$-Sputtern, andererseits
möglicherweise auf Degradationsmechanismen in der $ZnO$-Dünnschicht
zurückzuführen, die durch Temperaturlastwechseluntersuchungen verursacht
wurden \cite{Flik}.
Ein Vergleich der Meßkurven (Abbildung~\ref{abbimpzno}) mit den
FE-Resultaten (Abbildung~\ref{abbimpfem}) zeigt eine qualitativ gute
Übereinstimmung. Die Abweichungen bei den Resonanzfrequenzen lassen sich
einerseits auf die getroffenen Modellvereinfachungen (z.B.\ geringe
Elementanzahl, MDOF) und der dadurch hervorgerufenen Modellversteifung,
andererseits durch die inneren Spannungen in der $ZnO$-Schicht erklären.
Im Gegensatz zu Piezokeramik-Bimorphen entspricht die gemessene mechanische
Resonanzfrequenz der Parallelresonanzfrequenz: $f_{res} \approx f_{p}$.
Ferner sind die geringen Phasenhübe von etwa 1--2$^{\circ}$ für die
$ZnO$-beschichteten Bimorphwandler charakteristisch. Im Vergleich dazu
zeichnen sich $PZT$-Keramikwandler durch Phasenhübe bis zu 90$^{\circ}$
aus.

