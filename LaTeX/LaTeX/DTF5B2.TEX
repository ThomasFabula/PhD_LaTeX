\section{Laterale Schichtstrukturierung}
\label{schichtstrukturierung}

Durch eine laterale Strukturierung der piezoelektrischen Dünnschicht
und der Elektroden kann bei Bimorphstrukturen zum einen die Modenselektion
erhöht und zum anderen eine Temperaturkompensation des Wandlerverhaltens
erreicht werden. Während für die Erhöhung der Modenselektivität von
Balkenresonatoren
eine eindimensionale Betrachtungsweise ausreicht \cite{Pra93}, erfordern
Membranresonatoren zweidimensionale Untersuchungen.
Die prinzipielle Möglichkeit der Temperaturkompensation eines
Membranschwingers wurde bereits an $AlN$-beschichteten
Siliziummembranen\footnote{TCCR (= \underline{T}emperature
\underline{C}ompensated \underline{C}omposite \underline{R}esonator)}
von {\sl Lakin et al.} gezeigt \cite{Lak82}.
Im folgenden soll mit Hilfe der FE-Methode am Beispiel
der $ZnO$-beschichteten Silizium-Bimorphmembranen die Optimierung der
Resonatorunimodalität und {\em gleichzeitig} die Reduktion der
Temperaturquerempfindlichkeit unter Berücksichtigung von
elektro-thermo-mechanischen Wechselwirkungen gezeigt werden.


\subsection{Erhöhung der Modenselektivität}
\label{modenselektivitaet}

Die Anregung resonanter Siliziumstrukturen zu Biegeschwingungen erfolgt durch
aufgesputterte $ZnO$-Dünnschichten, die infolge des piezoelektrischen Effektes
Dehnungen und Kontraktionen auf der Bauelementoberfläche erzeugen. Um eine
möglichst hohe Moden\-selektivität zu erreichen, ist es notwendig, den genauen
Spannungsverlauf auf der Bauteil\-oberfläche zu kennen und durch eine
geeignete Elektrodenformgebung sicherzustellen, daß im Bereich von
Zugspannungen nur Dehnungen und im Bereich von Druckspannungen nur
Kontraktionen
erzeugt werden. Als Entwurfsparameter für die Elektrodenauslegung ist der
Nulldurchgang des lateralen Spannungsverlaufes (d.h. der šbergang von Zug-
in Druckspannungsbereiche) auf der Oberfläche des Resonators anzusehen.
Ausgehend von einer Siliziummembran wurden verschiedene FE-Modellen
untersucht, wobei der Einfluß unterschiedlicher Membrandicken,
variabler Druckbeaufschlagung, der Einspannung infolge der ätzbegrenzenden
(111)-Ebenen und nichtlinearer Effekte, infolge Spannungsversteifung der
Membran, berücksichtigt wurden. Die FE-Berechnungen an Schalen- und
Volumenmodellen ergaben für den Nulldurchgang des lateralen
Spannungsverlaufes einen Wert von etwa
17~\% der Membranseitenlänge. Im Vergleich dazu beträgt für beidseitig
festeingespannte Biegebalken der Nulldurchgang etwas weniger als 25~\% der
Balkenlänge. In diesen Bereichen sollten daher {\em keine} Elektroden
angeordnet werden. \\
%----------------------- Beginn: Figure-Environment ----------------------
\begin{figure}[htb]
%\vspace*{8cm}

\begin{center}
% --- Dateiname des Bildes
\input{membran.tex}
\setmembran
\end{center}

\caption{\label{abbmembranspan}
 Simulierte Spannungsverteilung auf der Membranoberfläche bei
 Druckbeaufschlagung}
\end{figure}
%----------------------- Ende: Figure-Environment ----------------------
In {\bf Abbildung~\ref{abbmembranspan}} ist die symmetrische
Spannungsverteilung auf einer von unten mit Druck beaufschlagten
Membranoberfläche zu sehen. Die FE-Berechnungen wurden mit einem
Schalenmodell (S43) durchgeführt und die Abbildung zeigt die graphisch
überlagerten Spannungsverläufe $\sigma_{x}$ und $\sigma_{y}$.
Im Randbereich bilden sich auf der Schalenoberfläche Druckspannungen
(Isolinien A--B) und in Membranmitte Zugspannungen (Isolinie C) aus.
Unter der Annahme, daá die statische Membranauslenkung in erster Näherung
der Grundbiegeschwingungsmode der Membran entspricht, kann aus der
flächenhaften Spannungsverteilung direkt ein modenselektives Elektrodenlayout
abgeleitet werden. \\
In {\bf Abbildung~\ref{abbmembraneleklayout}}
ist das fr die resonanten Membrandrucksensoren verwirklichte
Elektrodenlayout für eine selektive Anregung der Grundmode $M_{11}$
dargestellt.
%----------------------- Beginn: Figure-Environment ----------------------
\begin{figure}[htb]
%\vspace*{8cm}

\begin{center}
% --- Dateiname des Bildes
\input{abbfas.tex}
\setabbfas
\end{center}

\caption{\label{abbmembraneleklayout}
 Elektrodenlayout fr die selektive Anregung der
 Grund\-biege\-schwin\-gungs\-mode von Membranen}
\end{figure}
%----------------------- Ende: Figure-Environment ----------------------
Um Störeffekte durch elektrische Zuleitungen und Streukapazitäten
möglichst gering zu halten, wurde die Zentralelektrode mit einer
vierfach symmetrischen Zuleitung versehen. Die vier Randelektroden
überlappen die Siliziummembran bis ins Bulkmaterial und besitzen kleine
Anschlußpads, an die die elektrischen Zuleitungen ($Au$-Draht) gebondet
wurden.
Im Gegensatz zu den piezoelektrisch betriebenen Balkenresonatoren weisen die
Membranelektrodenlayouts ein wesentlich günstigeres Verhältnis von aktiver
Elektrodenfläche zu Anschlußpadfläche auf.
Die rckseitige Membranätzung ist durch eine gestrichelte Linie angedeutet.
Um eine gegenphasigen Ansteuerung der Elektrodengebiete zu ermöglichen sind
die Elektrodenflächen und die ins Substrat diffundierten Masseanschlüsse
nicht miteinander verbunden. Auf diese Weise war es möglich, den Sensor
sowohl im Zweipol- als auch im Vierpolbetrieb zu testen und verschiedene
Anregungs- und Abtastkonfigurationen zu untersuchen.\\
%
Durch die Strukturierung der Elektroden und der gegenphasigen Ansteuerung
zwischen Rand- und Zentralelektrode konnte die Modenselektivität
der Grundmode deutlich erhöht und der Betrag der Resonanzamplitude bei
konstanter Ansteuerspannung verdoppelt werden. Hierdurch kann der Sensor im
Kleinsignalbereich betrieben und dynamische Nichtlinearitäten,
hervorgerufen durch große Schwingungsamplituden, vermieden
werden.\\
%----------------------- Beginn: Figure-Environment ----------------------
\begin{figure}[htb]

\vspace*{14cm}
%\begin{center}
% --- Dateiname des Bildes
%\input{bild.tex}
%\setbild
%\end{center}

\caption{\label{abbmemselekmess}
 Experimenteller Nachweis der modenselektiven Anregung der Grundmode eines
 $ZnO$-beschichteten Membrandrucksensors}
\end{figure}
%----------------------- Ende: Figure-Environment ----------------------
In {\bf Abbildung~\ref{abbmemselekmess}} ist der experimentelle Nachweis
der modenselektiven Anregung der Grundmode einer $ZnO$-beschichteten
Siliziummembran dargestellt. Vermessen wurde eine Siliziummembran mit einer
Dicke von 50~$\mu$m und einer $ZnO$-Schichtdicke von etwa 14,4~$\mu$m.
Bei einer symmetrischen Anregung an den vier Randelektroden
(oberes Spektrum) sind zwei deutlich ausgeprägte Resonanzen im
Amplitudenspektrum zu beobachten, die Grundmode $M_{11}$ bei 9,1~kHz
und die erste Oberschwingung $M_{13}$ bei etwa 33,5~kHz. Bei der Anregung
mit Hilfe der Zentralelektrode bleibt das Modenspektrum qualitativ gleich,
jedoch nimmt die Amplitude der Grundmode um etwa 30~\% zu.
Erst bei der gegenphasigen Anregung von Rand- und Zentralelektrode
(unteres Spektrum) wird die Oberschwingung stark unterdrückt und die
Schwingungsamplitude der Grundmode steigt auf etwa das zweieinhalbfache an.
Durch die vorgestellte Elektrodenstrukturierung kann die Grundmode bei
gegenphasiger Anregung gut selektiert und gleichzeitig eine
Amplitudenerhöhung erreicht werden.


\subsection{Reduktion der Temperaturquerempfindlichkeit}
\label{temperaturquerempfindlichkeit}

Neben der Modenselektion hat die laterale Schichtstrukturierung einen
wesentlichen Einfluß auf das Temperaturverhalten von Resonatoren im
Bimorphaufbau. Ausgehend vom Elektrodenlayout (Referenzlayout), das in
Abbildung~\ref{abbmembraneleklayout} dargestellt ist, wurden die äußeren
und inneren Elektrodenflächen verändert, indem die Geometrieparameter
$l_{1}$ und $l_{2}$ variiert wurden. \\
%
%Temperaturkompensation des Wandlerverhaltens
%Reduktion der Temperaturquerempfindlichkeit
%
In {\bf Abbildung~\ref{abbtempabhfreq}} ist der Einfluß der lateralen
Strukturierung der $ZnO$-Dünnschicht auf die Temperaturquerempfindlichkeit
einer Silizium-Bimorphmembran dargestellt. Die Dicke der Siliziummembran
betrug bei den Berechnungen 50~$\mu$m, die der $ZnO$-Schicht 10~$\mu$m.
Die untersuchten FE-Modellen besaßen etwa 2300--2600 Elemente mit etwa
3300--3800 Knoten und es wurde mit 50 MDOFs gerechnet.
Variiert wurde die Größe der äußeren Randelektrode während die
Zentralelektrode konstant blieb. Deutlich ist zu sehen, daß infolge
einer Verkleinerung (Durchmesser $l_{1}$) der Randelektrode die
Temperaturabhängigkeit der Resonanzfrequenz der Grundmode $M_{11}$ abnimmt.
%----------------------- Beginn: Figure-Environment ----------------------
\begin{figure}[htb]
%\vspace*{8cm}

\begin{center}
% --- Dateiname des Bildes
\input{abbfelf.tex}
\setabbfelf
\end{center}
\caption{\label{abbtempabhfreq}
 Einfluß der lateralen Schichtstrukturierung auf die
 Temperaturquerempfindlichkeit einer Silizium-Bimorphmembran }
\end{figure}
%----------------------- Ende: Figure-Environment ----------------------
Der effektive elektromechanische Kopplungsfaktor $k_{eff}$ ändert sich nur
unwesentlich, die Verminderung beträgt lediglich 2~\%, aber die relative
Temperaturquerempfindlichkeit ($\frac{1}{f}\frac{\Delta f}{\Delta T}$)
der Resonanzfrequenz der Grundmode $M_{11}$ verringert sich um fast das
Fünffache. Das Referenzlayout REF mit den Parametern $l_{1}$ = 1,85~mm
und $l_{2}$ = 0,345~mm weist eine relative Temperaturquerempfindlichkeit
von etwa 775~ppm/K auf. Die Resonanzfrequenz $f_{0}$ beträgt 8443~Hz.
Das optimierte Layout R3 mit den Parametern $l_{1}$ = 0,55~mm
und $l_{2}$ = 1,645~mm weist eine Querempfindlichkeit von nur noch etwa
176~ppm/K bei einer Frequenz von 7745~Hz auf.
Bei den thermischen Berechnungen wurden anisotrope, temperaturabhängigen
Materialdaten sowohl für Silizium, als auch für $ZnO$ benutzt \cite{LB82}.
Die ermittelten Frequenzwerte $f_{0}$=f($\Delta T$=0) stimmen mit den
Serienresonanzfrequenzen $f_{s}$ der piezoelektrischen Rechnungen sehr
gut überein. Durch die laterale Strukturierung der $ZnO$-Schicht konnte
eine Temperaturkompensation beim resonanten Membrandrucksensor
erreicht werden.\\

