\section{Entwurfsmaßnahmen für Bimorphe}

Die in diesem Kapitel durchgeführten Untersuchungen ermöglichen es,
Entwurfsmaßnahmen für piezoelektrisch betriebene Bimorphwandler abzuleiten.
Zur Erzielung anwendungsspezifischer Eigenschaften lassen sich bei
mikromechanischen Sensoren und Aktoren mit piezoelektrischem Antrieb
folgende Maßnahmen durchführen.
%----------------------- Beginn: table ---------------------------
\begin{table}[htb]
\caption{\label{tabsenakt}
 Einflußgrößen zur Erzielung anwendungsspezifischer Eigenschaften
 bei piezoelektrisch betriebenen Bimorph-Bauelementen}
\begin{center}
\begin{tabular}{|l||c|c||c|}
\hline
 Eigenschaft      &  Sensor  &  Aktor  &  Einflußgröße  \\
\hline \hline
 Resonanzfrequenz & mittel & niedrig & Strukturdicke $h$\\
\hline
 Schwingungsgüte  &  hoch   & niedrig & Einspannung \\
\hline
 Empfindlichkeit  & hoch & --- & ($l/h$)-Verhältnis \\
\hline
 Nichtlinearität  & gering   & hoch  & ($h/l$)-Verhältnis \\
\hline
 Temperaturkomp.  & \multicolumn{2}{c||}{erwünscht} & Schichtstrukturierung \\
\hline
 elektromechan.   &  mittel  & hoch  & Schichtstrukturierung \\
 Kopplungsfaktor  &          &       & $h_{Si}/h_{Piezo}$--Verhältnis \\
\hline
 Modenselektivität & hoch  & --- & Elektrodenstrukt. \\
\hline
 Multimodebetrieb  & \multicolumn{2}{c||}{anwendungsabhängig}
                   &  Elektrodenstrukt. \\
\hline
 Auslenkung        & niedrig & hoch & $h_{Si}/h_{Piezo}$--Verhältnis \\
\hline
 Stellkräfte       & ---  & hoch  &   ($l/h$)-Verhältnis \\
\hline
\end{tabular}
\end{center}
\end{table}
%----------------------- Ende: table ---------------------------
Insbesondere handelt es sich hierbei um resonante Sensoren, d.h.\ mit
frequenzanalogem Meßprinzip, und {\em nicht-resonante}
Aktoren\footnote{In Kapitel~7 wird kurz auf den
Einsatz von {\em resonant} betriebenen Mikroaktoren eingegangen.}.



\section{Fehlerdiskussion}

Aufgrund der starken Prozeßabhängigkeit der Dünnschichteigenschaften,
schwanken die in der Literatur angegebenen Materialdaten erheblich.
Die Materialangaben weichen beispielsweise für gesputterte
$ZnO$-Dünnschichten im Vergleich zu den Bulk-Daten bei den
Elastizitätsmoduln um etwa $\pm$20~\% ab, die Schichtspannungen
bewegen sich im Bereich bis einige hundert MPa
(Zug- und Druckspannung), die Abweichung bei den
Dielektrizitätskonstanten (Permittivitäten) betragen $\pm$10~\%,
die piezoelektrischen Koeffizienten $d_{31}$ variieren um etwa
$\pm$(15--20)~\% \cite{Blo90, Hei66, Pol86, Tij91}.\\
Der Fehler des materialabhängigen Anteils des elektromechanischen
Kopplungsfaktors $k_{31}$ berechnet sich infolge von Schwankungen der
Materialeigenschaften nach Gleichung (\ref{kmat}) zu:
\begin{eqnarray}
\label{fehlerkopplfac}
 \frac{{\Delta k_{31}}}{k_{31}} & = & \sqrt{
       {\left( \frac{\Delta d_{31}}{d_{31}} \right)}^2  +
       {\left( \frac{\Delta S_{11}^{E}}{2 S_{11}^{E}} \right)}^2 +
{\left( \frac{\Delta \varepsilon_{33}^{\sigma}}{2 \varepsilon_{33}^{\sigma}}
 \right)}^2 }
\end{eqnarray}
Unter Zugrundelegung der oben angegebenen Abweichungen und den im
Rahmen dieser Arbeit bestimmten Materialdaten ergibt sich für die
möglichen Abweichungen des transversalen elektromechanischen
Kopplungsfaktors $\frac{{\Delta k_{31}}}{k_{31}} \approx \pm23$~\%. \\


\section{Zusammenstellung der Resultate}

Die in diesem Kapitel vorgestellten piezoelektrischen FE-Modelle und
durchgeführten gekoppelten Feldberechnungen erlauben die Simulation des
Verhaltens piezoelektrisch angetriebener Bimorphstrukturen. Mit Hilfe
meßtechnisch charakterisierter Piezokeramik- und $ZnO$-Dünnschichtwandlern
konnten die numerischen Ergebnisse im Rahmen der Fehlereinflüsse verifiziert
werden. Somit ließen sich bereits in der Entwurfsphase wichtige Vorgaben,
wie beispielsweise die günstigste Elektrodenanordnung und das optimale
Schichtdickenverhältnis, für die nachfolgenden technologischen
Prozeßschritte ableiten. Weiterhin konnte mit Hilfe der numerischen
Modellierung ein Beitrag zur Optimierung der technologischen
Abscheideprozesse geleistet werden.
Die Ergebnisse der experimentellen und numerischen Untersuchungen an
piezoelektrisch mit $ZnO$-Dünnschichten betriebenen Balken- und
Membranstrukturen lassen sich wie folgt zusammenfassen:
%
\begin{itemize}
\item
Für gesputterte $ZnO$-Dünschichten wurden durch Vergleich der
experimentellen Messungen mit den FE-Resultaten effektive
Materialeigenschaften, wie elektromechanische Kopplungsfaktoren und
prozeáspezifische piezoelektrische Kopplungskoeffizienten, bestimmt.
\item
An mikromechanischen Bimorphmembranstrukturen wurden für unterschiedliche
Piezoelektrika ($AlN, ZnO, PZT$) optimale Schichtdickenverhältnisse
zur Erzielung maximaler effektiver elektromechanischer Kopplungsfaktoren
numerisch errechnet.
\item
Für Membrandrucksensoren wurde ein modenselektives Elektrodenlayout
entwickelt, bei dem die Unimodalität des passiven Resonators durch
Elimination der $M_{13}$-Oberschwingung erhöht wurde. Zusätzlich konnte
experimentell eine Amplitudenverdopplung der Grundbiegeschwingungsmode
$M_{11}$ bei gleichen Ansteuerbedingungen nachgewiesen werden.
\item
Durch laterale Schichtstrukturierung konnte rechnerisch eine
Reduktion der Temperaturquerempfindlichkeit der Grundresonanzfrequenz
um das Vierfache nachgewiesen werden.
\end{itemize}
%
