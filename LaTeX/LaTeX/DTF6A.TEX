\chapter{FEM-unterstützte Sensorentwicklung}
\label{alternativesensorstrukturen}

In diesem Kapitel wird die Entwicklung zweier alternativer
Sensorgeometrien für die Kraft- und Druckmessung vorgestellt.
Auf der Basis eines Dreifachbalkenschwingers wurde ein
Kraftsensor realisiert, dessen grundlegendes dynamisches Verhalten
mit Hilfe der FE-Methode simuliert werden konnte. Durch eine geeignete
Strukturierung der Resonatorbefestigung wurde eine Modenentkopplung
erreicht, die sich durch eine verbesserte Modenselektivität des passiven
Resonators ausdrückt. Zusätzlich konnte bei technologisch realisierten
Dreifachbalkenresonatoren eine dynamische Momentenkompensation experimentell
nachgewiesen werden, die zu einer Erhöhung der Schwingungsgüte der
antisymmetrischen Schwingungsmode führte. \\
%
Weiterhin wird der Entwurf und die Entwicklung eines neuartigen
Drucksensorprinzips unter Verwendung einer
\glqq Balken-auf-Membran\grqq-Struktur (engl.:
{\em Beam-on-Diaphragm}\footnote{Im weiteren wird daher die
Kurzbezeichnung {\em BOD-Struktur} verwendet.} \cite{Tho90}) vorgestellt,
das die Vorteile von Balkenresonatoren als sensitives Element in
{\em monolithischer} Kombination mit einer Siliziummembran als
druckeinleitende Struktur ermöglicht. Da die komplizierte Strukturgeometrie
der BOD-Sensoren eine analytische Behandlung nicht zuläßt, wurde die
FE-Methode eingesetzt, um die Auswirkungen von Dimensionsänderungen auf das
dynamische Sensorverhalten und die Druckempfindlichkeit zu untersuchen.



\section{Dreifachbalken-Kraftsensor}
\label{dreifachbalken}

Im Rahmen des BMFT-Verbundprojektes wurde am Institut für Mikro-
und Informationstechnik ein resonanter Kraftsensor realisiert, der
als Resonator eine Dreifachbalkenstruktur aus Silizium einsetzt und
mit Hilfe von piezoelektrischen $ZnO$-Dünnschichten angetrieben wird
\cite{ABV93}. Um den Einfluß der Einspanngeometrie auf das Modenverhalten
des Resonators zu untersuchen, wurde die FE-Methode eingesetzt.\\
In {\bf Abbildung~\ref{abbtriplebeam}} ist die Geometrie des
Dreifachbalken-Kraftsensors schematisch dargestellt. Parallel zum Resonator,
der aus drei Balken besteht, weist der Sensor zusätzlich zwei
Verstärkungsstege in Waferdicke (etwa 525~$\mu$m) auf, die als
Überlastschutz dienen. Durch die Dimensionierung der Verstärkungsstegbreiten
läßt sich der Kraftmeßbereich der Sensoren einstellen.
%----------------------- Beginn: Figure-Environment ----------------------
\begin{figure}[htb]
%\vspace*{8cm}

\begin{center}
% --- Dateiname des Bildes
\input{abbse.tex}
\setabbse
\end{center}
\caption{\label{abbtriplebeam}
 Geometrie des Dreifachbalken-Kraftsensors}
\end{figure}

%\clearpage
%----------------------- Ende: Figure-Environment ----------------------
Das erarbeitete Sensordesign ist für einen Kraftmeßbereich
bis 50~N ausgelegt und weist Verstärkungsstege mit Breiten von jeweils 1~mm
auf. Die Gesamtabmessungen des Kraftsensors betragen 15~mm x 3,75~mm,
die Länge der Balken 3~mm und die Dicke etwa 20--25~$\mu$m. Die Breite
der beiden äußeren Balken beträgt 200~$\mu$m, die des mittleren Balkens
400~$\mu$m, wobei die Abstände zwischen den Balken 50~$\mu$m betragen.
Mit Hilfe eines Nd:YAG--Lasers wurden beidseitig kreisrunde Löcher in das
Siliziumbulkmaterial geschnitten, die über Stahlstifte eine uniaxiale
Krafteinleitung in den Kraftsensor ermöglichen und dadurch das Auftreten
von Scherkräften stark reduzieren. Die mit dem Laser geschnittenen Löcher
besitzen einen Durchmesser von 1~mm und gehen durch den Gesamtsiliziumwafer
hindurch. Durch ihre ideal kreisrunde Berandung verhindern sie
Kerbwirkungen \cite{Dissaxel}.



\subsection{Dynamische Momentenkompensation}
\label{dynamischemomentenkompensation}

Bei resonanten Sensoren ist die erzielbare Meßgrößenauflösung nach
Gleichung~(\ref{Qaufl}) einerseits durch die minimale Auflösung des
Detektionsmechanismus, andererseits durch die mechanische
Schwingungsgüte des verwendeten Resonators begrenzt. Um eine
möglichst hohe Schwingungsgüte zu erreichen, kann zum einen die
Resonatorgehäusung evakuiert werden, um die äußere Dämpfung durch Luft zu
eliminieren, zum anderen ein spezielles Resonatordesign eingesetzt werden
\cite{Ste91b}. Einige in der Literatur vorgestellte Sensorkonzepte
(z.B.\ \cite{Ike90b})
machen von beiden Maßnahmen Gebrauch. Um die Energieverluste im
Einspannbereich der Resonatoren zu minimieren, werden bei resonanten
Quarz-Kraftsensoren sogenannte Doppelstimmgabel-Resonatoren verwendet, die
in einer antisymmetrischen Schwingungsmode betrieben werden. Dieses wird
erreicht, indem die beiden Stimmgabelstege gegenphasig zueinander in der
Balkenebene schwingen \cite{Eer88}. Hierdurch stellt sich eine dynamische
Momentenkompensation der beiden Stimmgabelstege ein und Schwingungsgten
von einigen Tausend werden so in Luftatmosphäre erzielt.
Beim Einsatz von Einfachbalkenschwingern als Kraftsensoren müssen
zusätzliche mechanische Konstruktionen
vorgesehen werden, die geeignet zu strukturieren sind, um eine dynamische
Momentenkompensation zu gewährleisten \cite{Alb88}.\\
%
Im Gegensatz zu Quarz ist Silizium nicht piezoelektrisch, so daß bei der
Realisierung von resonanten, piezoelektrisch betriebenen Siliziumsensoren
piezoelektrische Dünnschichten vorgesehen werden müssen. Diese können
aufgrund des Bimorphaufbaus nur Biegeschwingungen aus der Balkenebene
heraus anregen. Aus diesem Grund ist eine Momentenkompensation in der
Schwingungsebene wie bei Quarz-Doppelstimmgabeln nicht möglich.
Diese kann beispielsweise erreicht werden, wenn der verwendete Resonator
drei \cite{Kir83,Sat89} oder mehrere Balken \cite{Til93} besitzt, die
gegenphasig aus der Balkenebene herausschwingen. Bei der
Dreifachbalkenstruktur kann eine Momentenkompensation an den Einspannstellen
des Resonators erfolgen, falls der Resonator in einer antisymmetrischen
Schwingungsmode betrieben wird und der mittlere Balken bei gleicher Dicke
doppelt so breit wie die beiden äußeren ist. Diese Schwingungsmode
wird gezielt piezoelektrisch angeregt und infolge der Momentenkompensation
vermindern sich die Schwingungsamplituden in den Einspannbereichen, so daß
die Energieauskopplung minimiert wird und eine Güteerhöhung gegenüber
den symmetrischen Biegeschwingungsmoden auftritt.
Sie ermöglichen einerseits eine {\em mechanische} Entkopplung von der
Bulkeinspannung. Andererseits wird durch die flexiblen Entkopplungsbereiche
eine Verbindung der drei Balken untereinander realisiert, so daß die
Schwingungsenergie im Resonator verbleibt und nur ein minimaler Bruchteil
in der Einspannung verloren geht. Um dieses zu gewährleisten,
muá die beidseitige starre Resonatorhalterung, die durch das Bulkmaterial
gebildet wird, mittels flexibler Entkopplungsbereiche\footnote{In
Abbildung~\ref{abbtriplebeam} gestrichelt dargestellte Flächen.} ersetzt
werden. Die beidseitigen Entkopplungsbereiche entstehen bei der rückseitigen
Membranätzung und weisen die gleiche Dicke wie die drei Resonatorbalken auf.




\subsection{Herstellungsverfahren}
\label{Herstellungsverfahren}

Der Herstellungsprozeß des Dreifachbalken-Kraftsensors orientiert sich an
einem beidseitigen Prozeß, der in \cite{Mul91} beschrieben ist.
Ausgehend von einem doppelseitig polierten \{100\}-Siliziumwafer wurden die
substratseitigen Elektroden $p^{+}$--dotiert und strukturiert.
Anschließend erfolgte die Abscheidung einer etwa 2,5~$\mu$m dicken
$ZnO$-Schicht mit Hilfe eines RF-Magnetron-Sputterprozesses. % \cite{Wag94}.
Die obere Elektrode wurde durch eine Aluminiumschicht realisiert und
anschließend mit Hilfe eines $SiO_{2}/Si_{3}N_{4}$--Schichtsystems
passiviert. Um die erwünschten Balkendicken zu realisieren wurden sie von
der Rückseite in einem $KOH$--Naßätzprozeß ohne Verwendung einer
Žtzstoppschicht abgedünnt. Die Separation der drei Balken erfolgte in einem
Plasma-Trockenätzschritt von der Wafervorderseite, so daß die drei Balken
im Gegensatz zu dem in Abbildung~\ref{abbgmssensor} gezeigten Kraftsensor
rechteckige Querschnitte aufweisen. Die technologische
Charakterisierung der gesputterten $ZnO$-Schichten und eine eingehende
Beschreibung der Prozeßtechnologien ist in \cite{Wag94} zu finden.



\subsection{Schwingungsverhalten und Kraftempfindlichkeit}
\label{schwingungsverhaltenundkraftempfindlichkeit}

Das Biegeschwingungsverhalten des Dreifachbalkenresonators wurde mit einem
FE-Schalenmodell und isotropen Materialdaten (Gleichung~\ref{simat})
modelliert. Der
Einfluß des zusätzlichen $ZnO/Al$--Schichtsystems wurde vernachlässigt, da
in erster Linie nur das Modenverhalten des passiven Siliziumresonators
bei Žnderung des Einspannbereiches von Interesse war. Die Elementabmessungen
betrugen 50~$\mu$m, um die Biegeschwingungen mit einer ausreichenden
Genauigkeit abbilden zu können.
%----------------------- Beginn: Figure-Environment ----------------------
\begin{figure}[htb]
%\vspace*{8cm}
\begin{center}
% --- Dateiname des Bildes
\input{abbszw.tex}
\setabbszw
\end{center}
\caption{\label{abbtbschwing}
  FE-Schalenmodell und Biegeschwingungsmoden M1, M2 und M3
  des Dreifachbalkenresonators}
\end{figure}

%\clearpage

%----------------------- Ende: Figure-Environment ----------------------
In {\bf Abbildung~\ref{abbtbschwing}} sind
das FE-Schalenmodell des Dreifachbalkenresonators und die ersten drei
Biegeschwingungsmoden M1--M3 dargestellt. Im oberen linken Fenster (1) ist
in einer Draufsicht auf den Resonator der Einspannbereich mit einer Länge der
Entkopplungszone von 200~$\mu$m zu sehen. Die mechanische Einspannung
des Resonators erfolgte an beiden Enden durch Sperrung der Verschiebungs-
und Rotationsfreiheitsgrade der Schalenelemente ({\em SHELL43}).
Bei der Grundbiegeschwingungsmode M1 (Fenster 2) schwingen alle drei Balken
symmetrisch in Phase, während bei der Mode M2 (Fenster 3) der mittlere Balken
ruht und die beiden äußeren gegenphasig zueinander schwingen. Die dritte
Mode M3 (Fenster 4) stellt die gewünschte Sensormode dar, die aufgrund der
antisymmetrischen Schwingung die dynamische Momentenkompensation auszunutzen
erlaubt. \\
%
Zusätzlich zeichnen sich Dreifachbalkenresonatoren gegenüber
Einfachbalkenresonatoren bei gleicher Gesamtbreite durch eine höhere
Kraftempfindlichkeit aus, da eine Spannungskonzentration im
Resonatorbereich auftritt.
%
% Diese läßt sich analytisch mit durch:
% \begin{eqnarray}
% \label{spanntb}
%  \sigma & = & \frac{(l_{1}+l_{2})A_{2}}{l_{1}A_{2}+l_{2}A_{1}} E \varepsilon
% \end{eqnarray}
% beschreiben , wobei $l_{1}$ die Länge des Balken und $l_{2}$ die
% Gesamtlänge des Resoantors ist. Die Gesamtquerschnittsfläche des Sensors
% ist $A_{2}$ und $A_{1}$ ist die effektive Querschnittsfläche der Dreifachbalken.
% Ein weiterer Vorteil sind die höheren Schwingungsgüten, die bei groáem
% Längen-/Breiten-Verhältnis der Balken erreicht werden kann.
%
%----------------------- Beginn: table ---------------------------
\begin{table}[htb]
\caption{\label{tabtriple}
 Vergleich der berechneten statischen und dynamischen Eigenschaften von
 Einfachbalken- und Dreifachbalken-Kraftsensoren}
\begin{center}
\begin{tabular}{|l||c||c|c|c|}
\hline
 Resonatortyp & Einfachbalken & \multicolumn{3}{c|}{Dreifachbalken} \\
\hline
 Schwingungsmode & Grundmode & M1 & M2 & M3 \\
\hline \hline
 Frequenz [kHz] & 19,48 & 15,51 & 16,62 & 17,61 \\
\hline
 Empfindlichkeit [$\frac{kHz}{N}$] & 14,95 & 15,43 & 15,55 & 15,85 \\
\hline
 Verlängerung $u_{x}$ [$\mu$m] & 0,98 & \multicolumn{3}{c|}{1,27} \\
\hline
 Spannung $\sigma_{x}$ [MPa] & 56  & \multicolumn{3}{c|}{63} \\
\hline
\end{tabular}
\end{center}
\end{table}
%----------------------- Ende: table ---------------------------
In {\bf Tabelle~\ref{tabtriple}} sind die Resultate der FE-Berechnungen für
Einfach- und Dreifachbalkenresonatoren mit einer Entkopplungslänge
von 200~$\mu$m zusammengestellt. Die Länge der betrachteten Resonatoren
betrug 3~mm und die Dicke 20~$\mu$m. Die Breite der Balken betrug
beim Dreifachbalken-Resonator 200/400/200~$\mu$m und die Balkenabstände
50~$\mu$m. Um die Ergebnisse mit einem Einfachbalkenresonator vergleichen
zu können, wies dieser eine äquivalente Gesamtbalkenbreite von 900~$\mu$m
auf. Die Kraftbeaufschlagung betrug bei der Berechnung der
Kraftempfindlichkeiten jeweils 1~N.
Aufgrund der um mehr als 10~\% verminderten Resonatorquerschnittsfläche
erhöht sich beim Dreifachbalkenresonator die mechanische Längsspannung
$\sigma_{x}$ im Resonator um den gleichen Betrag. Die Kraftempfindlichkeit
des Dreifachbalkenresonators nimmt entsprechend zu, so daß gegenüber dem
Einfachbalkenresonator eine um bis zu 6~\% erhöhte Kraftempfindlichkeit
erzielt werden kann, wenn er in der antisymmetrischen
Schwingungsmode M3 betrieben wird. Bei den experimentell untersuchten
Dreifachbalkenresonatoren konnte die höchste Kraftempfindlichkeit bei der
Schwingungsmode M3 nachgewiesen werden. Die maximale Kraftempfindlichkeit
betrug in dem untersuchten Meßbereich (bis 5~N) etwa 8,6~kHz/N
\cite{Fab93a}.
Im Vergleich zu dem in Tabelle~\ref{tabtriple} angegebenen Wert
fällt der gemessene aufgrund von Geometrietoleranzen und den
getroffenen Idealisierungen durch die Vernachlässigung des
$ZnO/Al$-Schichtsystems niedriger aus. Zusätzlich wirken sich die hohen
inneren Spannungen in der $ZnO$-Dünnschicht einschränkend auf das
Sensorverhalten aus. Sie führen dazu, daß bei einer Zugkraftbeaufschlagung
die Resonanzfrequenz zunächst abnimmt, da die innere Druckspannung der
$ZnO$-Schicht durch die am Sensor wirkende Zugkraft erst kompensiert
werden muß. Bei zunehmender äußerer Kraft wird der Resonator spannungsfrei,
ähnlich dem temperaturabhängigen Frequenzverhalten des Balkenresonators in
Kapitel~4.4.2, so daß anschließend die Resonanzfrequenz wieder zunimmt.
Bei Meßanwendungen ist daher sicherzustellen, daß die inneren Spannungen der
$ZnO$-Schichten erheblich reduziert oder zumindest kontrolliert werden
\cite{Sathe}. Eine weitere Möglichkeit besteht darin, den Sensor in einer
entsprechenden Halterung mechanisch vorzuspannen.



\subsection{Verbesserung der Unimodalität}
\label{unimodalitaet}

Um die Auswirkungen der Änderungen des membranartigen Entkopplungsbereiches
auf das Sensorverhalten zu untersuchen, wurde der Entkopplungsbereich von
50--400~$\mu$m variiert. In {\bf Abbildung~\ref{abbentkoppl}} ist die
Abhängigkeit der Modenaufspaltung der drei Schwingungsmoden M1, M2 und
M3 des Dreifachbalkenresonators von der Länge des Entkopplungsbereiches
dargestellt.
%----------------------- Beginn: Figure-Environment ----------------------
\begin{figure}[htb]
%\vspace*{8cm}

\begin{center}
% --- Dateiname des Bildes
\input{abbsd.tex}
\setabbsd
\end{center}
\caption{\label{abbentkoppl}
 Abhängigkeit der Modenaufspaltung der Schwingungsmoden M1, M2 und M3
 von der Länge des Entkopplungsbereiches}
\end{figure}
%----------------------- Ende: Figure-Environment ----------------------
Bei verschwindender
Entkopplungslänge sind die Dreifachbalken unendlich starr eingespannt
und somit voneinander mechanisch entkoppelt. Alle drei Balken schwingen dann
mit der Frequenz des Einfachbalkens (19,48 kHz). Der flexible Membranbereich
des Entkopplungsbereiches verbindet die drei Balkenschwinger miteinander und
es können sich die verschiedenen Schwingungsmoden ausbilden.
Mit zunehmender Entkopplungslänge spalten zusätzlich die Frequenzen weiter
auf, was zu einer merklichen Erhühung der Unimodalität des Resonators führt.
Dieses ist dadurch zu
erklären, daß durch die membranartige Balkenaufhängung die Einspannung
weicher wird und folglich die Resonanzfrequenzen der verschiedenen Moden
abnehmen. Dieses macht sich insbesondere für die beiden Moden M1 und M2
unterschiedlich bemerkbar. Die Frequenz der antisymmetrische Mode M3
stabilisiert sich ab einer Entkopplungslänge von etwa 200~$\mu$m
auf einen nahezu konstanten Wert, da die dynamische Momentenkompensation
verhindert, daß der membranartige Einspannbereich zusätzlich mitschwingt und
dadurch die {\em dynamische} Länge des Resonators effektiv vergrößert
wird. Bei weiter zunehmender Entkopplungslänge besteht allerdings die
Gefahr, daß die Anfälligkeit des Resonators für Torsionsschwingungen und
überlagerte Biege-/Torsions-Schwingungsmoden zunimmt. Daher wurde eine
Entkopplungslänge von 200~$\mu$m, das sind etwa 6,7~\% der
Balkenlänge, ausgewählt, um sicherzustellen, daß an den Balken\-enden die
Schwingungsknoten der Mode M3 sitzen und eine optimale mechanische
Entkopplung des Resonators von der Bulkeinspannung gegeben ist. In dem
ursprünglichen Resonatorentwurf waren keine Entkopplungsbereiche vorgesehen,
so daß in einem Redesign Entkopplungslängen von 200~$\mu$m verwendet wurden,
um verbesserte Dreifachbalkenresonatoren mit erhöhter Modenaufspaltung zu
realisieren.
Ein weiterer Effekt der erhöhten Unimodalität des Dreifachbalkenresonators
ist es, daß ein besseres \glqq Einlocken\grqq \, der elektrischen
Oszillatorschaltung auf der gewünschten Sensormode M3 ermöglicht wird.
Zusätzlich wirkt die vergrößerte Modenaufspaltung einer durch äußere
Störeinflüsse (z.B.\ Temperatureinfluß) hervorgerufenen Modenkopplung
entgegen.\\
%----------------------- Beginn: Figure-Environment ----------------------
\begin{figure}[ht]
\begin{minipage}[t]{7cm}
%\vspace*{8cm}

\vspace*{0.25cm}
\begin{center}
% --- Dateiname des Bildes
\input{abbsv.tex}
\setabbsv
\end{center}
\caption{\label{abbmodenkoppl1}
 Amplitudenspektrum eines Dreifachbalkenresonators mit Bulkeinspannung}
%\end{figure}
\end{minipage}
\hfill
%----------------------- Ende: Figure-Environment ----------------------
%----------------------- Beginn: Figure-Environment ----------------------
\begin{minipage}[t]{7cm}
\vspace*{0.25cm}
\begin{center}
% --- Dateiname des Bildes
\input{abbsf.tex}
\setabbsf
\end{center}
\caption{\label{abbmodenkoppl2}
 Amplitudenspektrum eines Dreifachbalkenresonators mit einer
 Entkopplungslänge von 200~$\mu$m}
\end{minipage}
\end{figure}
%clearpage
%\vspace*{0.5cm}
%----------------------- Ende: Figure-Environment ----------------------
Durch Vermessung verschiedener Dreifachbalkenresonatoren konnte die erhöhte
Modenaufspaltung verifiziert und eine Güteerhöhung der antisymmetrischen
Schwingungsmode M3 gegenüber der Grundmode M1 experimentell nachgewiesen
werden. Die Güteerhöhung betrug bei den untersuchten
Dreifachbalkenresonatoren etwa das Doppelte. Als maximale Schwingungsgüte
der Mode M3 wurde in Luftatmosphäre ein Wert von etwa 400 gemessen.
In {\bf Abbildung~\ref{abbmodenkoppl1}} und
{\bf Abbildung~\ref{abbmodenkoppl2}} sind die optisch vermessenen
Amplitudenspektren zweier Dreifachbalkenresonatoren mit unterschiedlichen
Entkopplungslängen dargestellt. Während beim ersten Resonator die
Balkeneinspannung direkt durch das Bulkmaterial gebildet wird, so daß die
{\em geometrische} Entkopplungslänge gleich Null ist, weist der
zweite Resonator einen Entkopplungsbereich der Länge 200~$\mu$m auf.
Da sich eine Superposition der Schwingungsmoden beim Resonator mit der
Bulkeinspannung einstellt, ist ein Einsatz als Kraftsensor aufgrund der
stark ausgeprägten Modenkopplung nicht möglich. Bei einer Entkopplungslänge
von 200~$\mu$m erhöht sich hingegen die Modenaufspaltung um etwa das
30-fache (3~kHz). Zusätzlich tritt die oben erwähnte
Güteerhöhung der Mode M3 gegenüber der Grundmode M1 auf. Die Schwingungsmode
M2 kann optisch auf dem Mittelbalken nicht detektiert werden, da bei dieser
Mode der Mittelbalken in Ruhe bleibt, so daß diese Mode nur auf den beiden
äußeren Balken interferometrisch nachgewiesen werden konnte. Die stark
unterschiedlichen Resonanzfrequenzen der Biegeschwingungsmoden bei den
beiden untersuchten Resonatoren lassen sich durch die hohen Schwankungen
der inneren Spannung der $ZnO$-Schichten erklären. \\
In {\bf Tabelle~\ref{tabuntkoppl}} sind die Ergebnisse der FE-Berechnungen
und der Messungen für Resonatoren mit unterschiedlicher Entkopplungslänge
zusammengefaßt.
%----------------------- Beginn: table ---------------------------
\begin{table}[htb]
\caption{\label{tabuntkoppl}
 Relative Modenaufspaltung von Dreifachbalkenresonatoren in Abhängigkeit
 der Entkopplungslänge (Vergleich: FE-Berechnungen -- Messungen)}
\begin{center}
\begin{tabular}{|l||c|c||c|c|}
\hline
 & \multicolumn{2}{c||}{FE-Berechnung} & \multicolumn{2}{c|}{Messung} \\
\hline
Entkopplungslänge & 50 $\mu$m & 200 $\mu$m & Bulkeinsp. & 200 $\mu$m \\
\hline \hline
 $c_{2}-1$ & 0,3~\% & 7,2~\% & 0,5~\% & 7,0~\% \\
\hline
 $c_{3}-1$ & 0,6~\% & 13,5~\% & 0,7~\% & 13,0~\% \\
\hline
\end{tabular}
\end{center}
\end{table}
%----------------------- Ende: table ---------------------------
Um die Resonanzfrequenzen vergleichen zu können, wurden
die Frequenzwerte $f_{i}$ auf die jeweilige Frequenz der
Grundmode $f_{M1}$ normiert, da die hohe Verspannung der
$ZnO$-Dünnschichten zu großen Frequenzenverschiebungen führte.
In der Tabelle sind daher die relativen Frequenzaufspaltungen
$c_{i} = f_{i}/f_{M1}$ gegenübergestellt. Beim
Vergleich der FE-Resultate mit den Messungen korrespondiert die
Bulkeinspannung, bei der die {\em geometrische} Entkopplungslänge Null
beträgt, einer {\em dynamischen} Entkoppplungslänge von etwa 50~$\mu$m.
Die Abweichungen zwischen den
berechneten und gemessenen Werten betragen bei einer Entkoppplungslänge
von 200~$\mu$m jeweils nur 2,9~\% (Mode M2) und 3,8~\% (Mode M3), so daß
eine sehr gute Korrelation vorhanden ist.\\
