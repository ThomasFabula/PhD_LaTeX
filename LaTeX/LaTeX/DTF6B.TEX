\section{BOD-Drucksensor}
\label{BOD}


Mit Hilfe des am Institut für Mikro- und Informationstechnik entwickelten
Verfahrens des laserunterstützten anisotropen Ätzens \cite{Ala91} konnten
Balkenstrukturen
in $\langle$110$\rangle$--Siliziumwafern mit dreieckigem Balkenquerschnitt
hergestellt werden, die monolithisch auf Trägerstegen sitzend auf einer
Membran angeordnet sind \cite{Sch93b}. Diese sogenannten BOD-Strukturen
eignen sich insbesondere für die Realisierung von mechanischen Sensoren,
beispielsweise zur Messung von Drücken und Kräften.
Die BOD-Struktur zeichnet sich durch eine hohe Empfindlichkeit gegenüber
Verformungen der Membran aus. Bei der Anwendung als Drucksensor hängt die
Druckempfindlichkeit der BOD-Struktur von den Abmessungen der
Membran, der Höhe der Trägerstege und dem Längen-/Dickenverhältnis des
Balkens ab. Weiterhin kommt der Dimensionierung der Trägerstegstruktur
eine besondere Bedeutung zu, da sie die Membranverformung über eine Art
Hebelwirkung in eine mechanische Axialspannung im Balken transformiert,
der als sensitives Element dient.
Zum Betrieb können sowohl das frequenzanaloge Sensorprinzip unter
Verwendung des piezoelektrischen oder elektrothermischen Antriebs,
als auch analoge Nachweisverfahren, wie die resistive
oder piezoresistive Abtastung durch DMS- oder diffundierte Widerstände,
eingesetzt werden \cite{Ala93}.



\subsection{Herstellungsverfahren und Funktionsprinzip}
\label{bodherstellung}

Das Verfahrensprinzip des laserunterstützen anisotropen Ätzens beruht auf
der lokalen Zerstörung der Kristallordnung in langsamätzenden
\{111\}-Ebenen durch den fokussierten Strahl eines Nd:YAG-Lasers
und nachfolgendes naßchemisches anisotropes Ätzen.
Die so durch das Laserlicht zerstörten \{111\}-Ebenen bilden keinen
Ätzstopp mehr, so daß im Gegensatz zum herkömmlichen anisotropen Naßätzen
neuartige Strukturgeometrien realisierbar sind \cite{Ala92a}.
Ausgangsmaterial sind beidseitig polierte
$\langle$110$\rangle$--Siliziumwafer mit einer Dicke\footnote{Die
Realisation und die ersten experimentellen Voruntersuchungen an
BOD-Drucksensoren erfolgte unter Verwendung von 3''--Siliziumwafern.}
von etwa 380~$\mu$m. Durch
thermische Oxidation wird eine etwa 1~$\mu$m dicke Maskierschicht
(hier: $SiO_{2}$) wie in {\bf Abbildung~\ref{abbbodprinzip}}
dargestellt beidseitig aufgebracht, die an den Stellen der späteren
Laserbearbeitung geöffnet wird. Falls der Abstand der beiden rautenförmigen
Mikrokanäle bei gegebener Ätztiefe, hier etwa (230$\pm$10)~$\mu$m, einen
bestimmten Wert unterschreitet, bilden sich Balkenresonatoren mit
dreieckigem Querschnitt aus. Durch die Maskenöffnungen wird der Balken
lithographisch definiert und mit Hilfe einer nachfolgenden flächenhaften
Laserbearbeitung die Kristallordnung thermisch zerstört. Anschließend
wird der Siliziumwafer in einem Standardnaßätzprozeß in einer KOH--Lösung
geätzt und der Balken freigelegt. Die Balkenseiten
werden durch die ätzbegrenzenden \{111\}--Siliziumebenen gebildet. In
Abhängigkeit der Orientierung der Balkenlängsachse zum Waferflat können
verschiedene Balkenquerschnittsformen realisiert werden. Falls die Balken
parallel zur [110]-Kristallrichtung (senkrecht zum Waferflat) orientiert
sind, weisen ihre Seitenflächen einen Winkel von $35,26^{\circ}$ zur
Oberfläche auf, bei einer senkrechten
Orientierung zum Waferflat einen Winkel von etwa $50^{\circ}$ \cite{Ala92b}.
Durch die flächige Bestrahlung mit Laserlicht können die seitlichen
Abmessungen der Membran bei der Herstellung der BOD-Sensorstruktur
eingestellt werden.
In einer vereinfachten Prozeßabfolge kann der Lithographieschritt
entfallen und der Siliziumwafer direkt mit dem Laser bearbeitet werden.
Eine detaillierte Beschreibung des experimentellen Aufbaus und der
automatisierten Ansteuerung des Laserschreibprozesses ist in
\cite{Dissaxel} zu finden.\\
%----------------------- Beginn: Figure-Environment ----------------------
\begin{figure}[htb]
%\vspace*{8cm}

\begin{center}
% --- Dateiname des Bildes
\input{abbsse.tex}
\setabbsse
\end{center}
\caption{\label{abbbodprinzip}
 Herstellungsprinzip der Siliziumbalken mit dreieckigem Querschnitt}
\end{figure}
%----------------------- Ende: Figure-Environment ----------------------
In einer Vorstufe zur eigentlichen BOD-Sensorstruktur wurden
{\em monolithisch integrierte} Siliziumbalken mit dreieickigem Querschnitt,
wie in Abbildung~\ref{abbbodprinzip} schematisch skizziert, hergestellt und
das dynamische Verhalten experimentell untersucht, wobei die
Balkendimensionen variiert wurden.



\subsection{Einfluß des Balkenquerschnitts auf das Schwingungsverhalten}
\label{balkenquerschnitt}
  
Die Frequenzverschiebung resonanter Sensoren ist abhängig von der durch die
Meßgröße induzierten mechanischen Spannung im Resonator, so daß durch eine
Verkleinerung des
Resonatorquerschnitts die Kraftempfindlichkeit erhöht werden kann.
Während naßchemisch in $\langle$100$\rangle$--Silizium geätzte
Balkenresonatoren einen
trapez\-förmigen Querschnitt aufweisen (siehe Abbildung~1.4), können durch
Laserstrukturierung von Silizium in Kombination mit anisotroper Naßätztechnik
dreieckförmige Balkenquerschnitte in $\langle$110$\rangle$--Silizium
hergestellt werden, die bei gleicher Balkenbreite eine
kleinere Querschnittsfläche als
trapez\-förmige Balken aufweisen \cite{Ala92b}.
Somit wird bei einer gleichen Kraftbeaufschlagung eine höhere
Resonatorverspannung in dem dreieckförmigen Balken hervorgerufen.
Im nachfolgenden wird der Einfluß des Balkenquerschnitts auf das
Schwingungsverhalten und die Eignung solcher Balkenelemente als resonante
Sensoren untersucht.\\
%
Um die Balkenresonatoren mit trapezförmigem und dreieckigem Balkenquerschnitt
vergleichen zu können, wurde von gleichen Balkendicken ausgegangen, die
sich bei einer Balkenbreite von 200~$\mu$m ergeben. Durch die vorgegebene
Balkenbreite stellen sich bei den Siliziumbalken, die einen
Resonatorquerschnitt mit einem Winkel von $35^{\circ}$ bzw.\ $50^{\circ}$
aufweisen, Balkendicken
von etwa 71~$\mu$m bzw.\ 120~$\mu$m ein. Die Balkenlängen wurden in den
FE-Berechnungen zu 3~mm gew„hlt. Bei der Berechnung der Eigenfrequenzen
und auftretenden Schwingungsmoden, wurde in erster Näherung von isotropem
Materialverhalten (Gleichung~\ref{simat}) und einer ideal starren
Balkeneinspannung ausgegangen. Die FE-Modelle besaßen etwa 720 bis 1400
Volumenelemente ({\em SOLID45}) und 300 MDOFs.
In {\bf Tabelle~\ref{tabresquersch}} sind die numerisch berechneten
Resonanzfrequenzvielfachen $c_{i}$ = $f_{i}/f_{1}$ für Balkenstrukturen mit
den zwei verschiedenen Balkenquerschnitten aufgelistet. Die Frequenzen sind
auf die jeweilige Grundmode Z1 normiert und die verschiedenen
Oberschwingungen aus der Balkenebene heraus mit Z2 bis Z5, die erste
Biegeschwingungsmode in der Balkenebene mit Y1, sowie die erste
Torsionsmode mit T1 bezeichnet.
%----------------------- Beginn: table ---------------------------
\begin{table}[htb]
\caption{\label{tabresquersch}
 Einfluá des Resonatorquerschnitts auf die Resonanzfrequenzen}
\begin{center}
\begin{tabular}{|c||c|c||c|c|}
\hline
\multicolumn{5}{|c|}{Resonanzfrequenzen [kHz]}\\
\hline \hline
Dicke   & \multicolumn{2}{c||}{ 71 $ \mu m \;
(35^{\circ}) $ } & \multicolumn{2}{c|}{ 120 $ \mu m \; (50^{\circ}) $} \\
\hline \hline
 Mode   &  dreieck & trapez & dreieck & trapez \\[2.5ex]
\hline \hline
 Z1  &  56,6   & 68,5    & 94,3    & 114,1 \\
     &  (1,00) & (1,00)  & (1,00)  & (1,00) \\ \hline
 Z2  & 155,5   & 188,3   & 258,1   & 311,5 \\
     & (2,75)  & (2,75)  & (2,74)  & (2,73) \\ \hline
 Z3  & 303,9   & 368,0   & 501,3   & 603,9 \\
     & (5,38)  & (5,37)  & (5,32)  & (5,29) \\ \hline
 Z4  & 500,6   & 606,2   & 819,8   & 986,9 \\
     & (8,85)  & (8,85)  & (8,69)  & (8,65) \\ \hline
 Z5  & 744,8   & 901,9   & 1209    & 1460 \\
     & (13,2)  & (13,2)  & (12,8)  & (12,8) \\
\hline \hline
 Y1  & 136,2   & 238,8   & 136,1   & 274,8 \\
     & (2,41)  & (3,49)  & (1,44)  & (2,41) \\
\hline \hline
 T1  & 595,6   & 491,0   & 749,7   & 634,5 \\
     & (10,5)  & (7,17)  & (7,95)  & (5,56) \\ \hline
\end{tabular}
\end{center}
\end{table}
%----------------------- Ende: table ---------------------------
Die Resonanzfrequenzen der Grundschwingung von trapezförmigen
Balkenresonatoren fallen gegenüber Resonatoren mit dreieckigen
Querschnittsflächen aufgrund der größeren Querschnittsflächen, die
ihrerseits zu einer Erhöhung der axialen
Flächenträgheitsmomente\footnote{Das äquatoriale Flächenträgheitsmoment
von Balken mit dreieckiger Querschnittsfläche der Höhe $h$ und Breite $b$
betr„gt $I=b \cdot h^{3}/36$ (vergleiche hierzu Gleichung~\ref{Itrapez}).}
führen, höher aus.
Die Frequenzvielfachen $c_{i}$ der Biegeschwingungsmoden in Z-Richtung
differieren kaum, da die konstant gehaltene Balkendicke in erster Näherung
der ausschlaggebende Parameter ist. Eine analytische Vergleichsrechnung nach
Gleichung (\ref{balkfreq}) liefert für rechteckige Siliziumbalken
Grundfrequenzen von 69,1 und 116,7~kHz bei
Resonatordicken von 71 und 120~$\mu$m. Diese Werte zeigen eine hinreichend
gute Übereinstimmung (Abweichung: 1--2~\%) mit den Frequenzwerten der
trapezförmigen Balkenschwinger.\\
%
%----------------------- Beginn: Figure-Environment ----------------------
\begin{figure}[htb]
%\vspace*{8cm}

\begin{center}
% --- Dateiname des Bildes
\input{abbssi.tex}
\setabbssi
\end{center}
\caption{\label{abbquerschspek}
 Optisch vermessenes Geschwindigkeitsspektrum eines doppelseitig
 eingespannten Siliziumbalkens mit dreieckigem Querschnitt}
\end{figure}
%----------------------- Ende: Figure-Environment ----------------------
An verschiedenen Siliziumbalken mit dreieckigem Resonatorquerschnitt
(Winkel: $35^{\circ}$) wurden Messungen durchgeführt und die experimentellen
Ergebnisse mit den FE-Berechnungen verglichen. In
{\bf Abbildung~\ref{abbquerschspek}} ist ein optisch vermessenes
Modenspektrum eines 2800~$\mu$m langen, doppelseitig eingespannten
Siliziumbalkens dargestellt.
Die Balkenbreite betrug etwa 180~$\mu$m und die Balkendicke etwa
(64$\pm$2)~$\mu$m. Um eine genügend hohe Nachweisempfindlichkeit auch
für die hochfrequenten Schwingungsmoden zu erreichen, wurde die
Geschwindigkeitsschnelle $v(f)$ in Balkenmitte aufgezeichnet. Der Nachweis
der Z2-Schwingungsmode erfolgte auf der Balkenviertellänge.
Die Anregung erfolgte über Körperschall mit Hilfe eines extern angebrachten
Piezokeramikschwingers. Im Signaluntergrund sind einzelne
Resonanzüberhöhungen der Piezokeramik zu sehen, die auch bei
günstigster Justage der Piezokeramik
und zusätzlicher Dämpfung des Siliziumwafers mit dünnen Softgummischeiben
nicht vollständig eliminiert werden konnten. Im Spektrum ist
die Grundbiegeschwingung Z1 bei einer Frequenz von etwa 60,9~kHz deutlich
zu erkennen. Die Resonanzfrequenzen der Oberschwingungen Z2 und Z3 liegen
bei etwa 168,2~kHz und 327,8~kHz. Durch Ausdehnung des Meßbereiches bis 1~MHz
konnten auch die Schwingungsmoden Z4 und Z5 nachgewiesen werden. Die
Resonanzfrequenzen betrugen 536,0~kHz und 749,0~kHz, bei einem erhöhten
Meßfehler von etwa $\pm$5~kHz. Die Frequenzvielfachen $c_{i}$ der
gemessenen Frequenzwerte (2,76/5,38/8,80/12,3) korrelieren gut mit den
berechneten Werten\footnote{Vergleiche hierzu jeweils die ersten Spalten der
Tabellen~\ref{tabresquersch} und \ref{tabquerschfreq}.}. Die Bestimmung der
Schwingungsgüte der Grundbiegeschwingung, die etwa 420 in Luft betrug,
konnte durch die schmalbandige Vermessung des Amplitudenspektrums $A(f)$
erfolgen. Die Absolutwerte der Schwingungsamplituden, die
typischerweise im Nanometerbereich liegen, sind hierbei von der
akustischen Leistung abhängig, die von der Piezokeramik übertragen wird,
und damit proportional zur Anregungsspannung.\\
%----------------------- Beginn: table ---------------------------
\begin{table}[htb]
\caption{\label{tabquerschfreq}
 Resonanzfrequenzen von Siliziumbalken mit dreieckigem Resonatorquerschnitt
 (Vergleich: FE-Berechnung -- Messung)}
\begin{center}
\begin{tabular} {|l||c|c|c|}
\hline
 Mode  & FEM  & Exp. & Abw. \\
\hline \hline
 Z1 &  58,0  kHz  &   60,9 kHz & 4,8~\% \\
    & (1,000)     &  (1,000)   &        \\
\hline
 Z2 &  159,4 kHz  &  168,2 kHz & 5,2~\% \\
    & (2,748)     &  (2,762)   &        \\
\hline
 Z3 &  311,4 kHz  & 327,8 kHz  & 5,0~\% \\
    & (5,369)     & (5,383)    &        \\
\hline
\end{tabular}
\end{center}
\end{table}
%----------------------- Ende: table ---------------------------
In {\bf Tabelle~\ref{tabquerschfreq}} ist ein Vergleich zwischen einer
Messung und einer FE-Rechnung, die unter vereinfachenden Annahmen die
komplexe Einspanngeometrie vernachlässigt, zusammengefaßt. Das FE-Modell
wies 2256~Elemente mit 3325~Knoten auf, wobei die Elementabmessungen
30~$\mu$m betrugen. Die Frequenzabweichungen liegen um 5~\% und sind
einerseits durch die Vernachlässigung der exakten Einspanngeometrie und
andererseits durch die Dickenschwankungen des Balkens zu erklären.
In Kapitel~\ref{bodmessung} wird auf die meßtechnische Charakterisierung
eines BOD-Drucksensors eingegangen, der einen monolithisch integrierten
Balken auf einer Siliziummembran aufweist. In Kapitel~\ref{bodoptimierung}
wird bei der Modellierung des Sensorverhaltens und der Optimierung der
Druckempfindlichkeit die exakte BOD-Sensorgeometrie, insbesondere
die realen Einspannverhältnisse des Siliziumbalkens, berücksichtigt.\\
%----------------------- Beginn: Figure-Environment ----------------------
\begin{figure}[htb]
%\vspace*{8cm}

\begin{center}
% --- Dateiname des Bildes
\input{abbsa.tex}
\setabbsa
\end{center}
\caption{\label{abbquerschempf}
 Vergleich der numerisch berechneten Frequenzänderungen von Balkenresonatoren
 mit dreieckigem und trapezförmigem Balkenquerschnitt}
\end{figure}
%----------------------- Ende: Figure-Environment ----------------------
Um die maximal möglichen Frequenzverschiebungen der Siliziumbalken in
Abhängigkeit der mechanischen Spannung im Resonator zu untersuchen,
genügt es bei der Modellierung, lediglich eine ideal starre
Balkeneinspannung
zu betrachten. Hierzu wurden die Balken mit einer axialen Zugspannung
$\sigma_{x}$ beaufschlagt und als maximal auftretende Spannung die
Bruchspannung von Silizium, etwa 200~MPa, festgelegt.
In {\bf Abbildung~\ref{abbquerschempf}} sind die auf die unbelastete
Grundfrequenz $f_{Z1}(\sigma=0)$ normierten Resonanzfrequenzen von
Balkenschwingern mit dreieckigen und trapezförmigen Balkenquerschnitten in
Abhängigkeit der mechanischen Zugspannung dargestellt. Hervorzuheben sind die
großen Frequenzänderungen von bis zu 40~\% der Siliziumbalken mit dreieckigen
Balkenquerschnitten bei maximaler Zugbelastung. Sie zeichnen sich um eine
etwa 30~\% höhere Empfindlichkeit als Balken mit trapezförmigen
Querschnitten aus und sind damit besonders für den Einsatz in
BOD-Drucksensoren geeignet.




\subsection{Meßtechnische Charakterisierung}
\label{bodmessung}


Mit dem in Kapitel~\ref{bodherstellung} beschriebenem laserunterstützten
Herstellungsverfahren wurden BOD-Drucksensoren in
$\langle$110$\rangle$-Silizium realisiert.
In {\bf Abbildung~\ref{abbbodgeom}} ist ein Schnitt durch die Geometrie
eines solchen BOD-Drucksensors schematisch dargestellt.
%----------------------- Beginn: Figure-Environment ----------------------
\begin{figure}[htb]
%\vspace*{8cm}

\begin{center}
% --- Dateiname des Bildes
\input{abbsn.tex}
\setabbsn
\end{center}
\caption{\label{abbbodgeom}
 Geometrie des BOD-Drucksensors}
\end{figure}
%----------------------- Ende: Figure-Environment ----------------------
Die eigentliche Membran, die zur Druckeinleitung dient, ist durch zwei
senkrecht zum schwingenden Siliziumbalken laufende Stege in einen äußeren
und inneren Bereich unterteilt. Die parallelen Stege, die mit dem Balken
eine H-förmige Trägerstruktur bilden, sind für eine spätere elektrische
Kontaktierung des Balkens durch Elektroden vorgesehen. Die Waferunterseite
bildet den Membranboden und die Balkenoberseite liegt auf der Höhe der
Waferoberseite, so daß die beidseitigen Einspannbereiche des Balkens
(in der Abbildung eingekreist) die Hebelwirkung der BOD-Struktur
festlegen. Diese bewirken, daß bei einer Druckdifferenz zwischen der
Membranunterseite und -oberseite eine Verwölbung auftritt und der Balken
in seiner Längsrichtung gedehnt wird. Die durch die Dehnung im Siliziumbalken
hervorgerufene homogene Zugspannung erhöht die Resonanzfrequenz,
die beim frequenzanalogen Sensorprinzip direkt als Sensorsignal genutzt
werden kann.\\
%
{\bf Abbildung~\ref{abbbodrem1}} zeigt in einer REM-Aufnahme die
Gesamtansicht eines in $\langle$110$\rangle$--Silizium realisierten
BOD-Drucksensors \cite{Sch93b}. Die Membranseitenlänge beträgt
5~mm. Bei einer Ätztiefe von 230~$\mu$m resultiert beim
verwendeten 3''-Siliziumwafer mit
einer Waferdicke von 380~$\mu$m eine Membrandicke von etwa 150~$\mu$m.
In {\bf Abbildung~\ref{abbbodrem2}} ist die Balkeneinspannung in einer
Nahaufnahme deutlich zu erkennen. Bei einer lithographisch festgelegten
Balkenbreite von 120~$\mu$m stellt sich bei einem Winkel von $35^{\circ}$
eine theoretische Balkendicke von 42,42~$\mu$m ein. Infolge von Überätzung
vermindert sich jedoch die Balkendicke und zusätzlich treten
Dickenschwankungen bis zu $\pm$5~$\mu$m über die Balkenlänge auf,
so daá real von einer Balkendicke von etwa 37~$\mu$m auszugehen ist
\cite{Sch94}. Die freie Länge des Balkens, gemessen
zwischen den beiden dreiecksflächigen Einspannungen, die einen
Begrenzungswinkel von $2 \cdot 54,74^{\circ} = 109,48^{\circ}$ aufweisen,
beträgt etwa (1,99$\pm$0,04)~mm und wurde mit Hilfe eines
IR-Durchlichtmikroskopes bestimmt.
Die Breite der beiden querlaufenden Stege beträgt auf der Oberseite etwa
74~$\mu$m und auf dem Membrangrund verbreitern sich die Stege beidseitig
unter einem Winkel von 45$^{\circ}$ auf etwa 400~$\mu$m.\\
%----------------------- Beginn: Figure-Environment ----------------------
\begin{figure}[ht]
\begin{minipage}[t]{7cm}
\vspace*{0.25cm}
%\vspace*{8cm}

\begin{center}
% --- Dateiname des Bildes
\input{absezeb.tex}
\setabsezeb
\end{center}
\caption{\label{abbbodrem1}
 REM-Aufnahme des BOD-Drucksensors (Gesamtansicht)}
\end{minipage}
\hfill
%----------------------- Ende: Figure-Environment ----------------------
%----------------------- Beginn: Figure-Environment ----------------------
\begin{minipage}[t]{7cm}
\vspace*{0.25cm}
%\vspace*{8cm}

\begin{center}
% --- Dateiname des Bildes
\input{absezea.tex}
\setabsezea
\end{center}
\caption{\label{abbbodrem2}
 REM-Aufnahme des BOD-Drucksensors (Balkeneinspannung)}
\end{minipage}
\end{figure}

%clearpage
%\vspace*{0.5cm}
%----------------------- Ende: Figure-Environment ----------------------
Die meátechnische Charakterisierung der BOD-Strukturen erfolgte optisch mit
dem Laservibrometer bei passiver Anregung der Gesamtstruktur mittels
Piezokeramikschwinger. Hierzu wurden einzelne BOD-Strukturen mit dem
Nd:YAG-Laser aus dem Waferverbund geschnitten und auf Aluminiumhalter
geklebt, um möglichst definierte Einspannbedingungen zu gewährleisten und
eine Druckeinleitung zu ermöglichen. Aufgrund der durch den
Herstellungsprozeß hervorgerufenen Geometrietoleranzen schwanken die
gemessenen Resonanzfrequenzen und Schwingungsgüten zum Teil erheblich. Die
Abtastung der Grundbiegeschwingungsmode Z1 erfolgte in der Balkenmitte.
Die an verschiedenen Strukturen gemessenen Frequenzen lagen aufgrund der
Geometrietoleranzen im Bereich von
77--82~kHz, die der beiden Obermoden Z2 und Z3 bei etwa 217,3 und
426,7~kHz (Meßfehler: etwa $\pm$5,0~kHz). Die an der BOD-Struktur
bestimmten Schwingungsgüten variierten bei Normalluftdruck für die
Grundmode im Bereich 370--530 und für die Z2-Mode wurde ein Wert von
etwa 1240 gemessen. Bei hohen Frequenzen zeichnet sich der
Amplitudendemodulator der Laservibrometer-Steuerung durch eine geringe
Nachweisempfindlichkeit aus, so daß bei der Z3-Mode die
Geschwindigkeitsschnelle aufgezeichnet wurde und somit eine Gütebestimmung
nicht durchgeführt werden konnte. Der eindeutige Nachweis der
Schwingungsmoden erfolgte durch eine laterale Messung entlang
des Balkens. Auf der Viertelbalkenlänge wurde die Z2-Mode nachgewiesen
und auf der Balkeneinspannung (siehe Abbildung~\ref{abbbodrem2}) der
Signaluntergrund untersucht, der einzelne Resonanzen des anregenden
Piezokeramikschwingers enthielt. Unglücklicherweise lagen im Frequenzbereich
(78--80~kHz) der Biegeschwingungsmoden
des Siliziumbalkens auch Schwingungsmoden der Piezokeramik, die jedoch
im Amplitudenspektrum aufgrund ihrer geringen Güten ($Q \leq 100$) eindeutig
identifiziert und herausgerechnet werden konnten.\\
%----------------------- Beginn: Figure-Environment ----------------------
\begin{figure}[htb]
%\vspace*{8cm}

\begin{center}
% --- Dateiname des Bildes
\input{abbmm.tex}
\setabbmm
\end{center}
\caption{\label{abbbodkennlinie}
 Gemessene Frequenz-Druck-Kennlinie eines BOD-Drucksensors}
\end{figure}
%----------------------- Ende: Figure-Environment ----------------------
Zur Vermessung der Druckempfindlichkeit wurde die BOD-Struktur mit dem
Aluminiumhalter auf einen Druckmeßtisch\footnote{Im Anhang ist der gesamte
experimentelle Meßaufbau beschrieben.} geschraubt
und sowohl mit Unterdruck (bis --0,8~bar), als auch mit Überdruck
(bis 1,0~bar) kontinuierlich beaufschlagt. Die gemessene
Frequenz-Druck-Kennlinie der Grundbiegeschwingungsmode Z1 des Balkens
ist in {\bf Abbildung~\ref{abbbodkennlinie}} zu sehen.
An den gemessenen Kurvenverlauf wurde ein Polynom zweiten Grades
$f(p) = f_{0} + a_{1} \cdot p + a_{2} \cdot p^{2}$
angepaßt. Für die Grundresonanzfrequenz $f_{0}(p=0)$ ergibt die Regression
einen Wert von 82,079~kHz und einen {\em linearen} Polynomkoeffizienten
von $a_{1} = 4,47$~kHz/bar, sowie einen {\em quadratischen} Polynomkoeffizienten
von $a_{2} = -7,28 \cdot 10^{-4}$~kHz/bar$^{2}$.
Daraus folgt für die relative Druckempfindlichkeit
$\eta = \frac{1}{f_{0}} \frac{\Delta f}{\Delta p}$ in dem untersuchten
Druckbereich ein Wert von etwa 5,45~\%/bar.
Anhand des Kurvenverlaufes ist zu sehen,
daß im Unterdruckbereich ($p < 0$) die Druckempfindlichkeit anwächst.
Die unterschiedlichen Druckempfindlichkeiten lassen sich durch
die unsymmetrische Hebelwirkung der BOD-Struktur erklären, falls sich die
BOD-Membran nach unten bzw.\ oben durchbiegt.\\
%
Um eventuell auftretende Hystereseeffekte nachzuweisen, wurde die
BOD-Struktur Druckwechsellasten ausgesetzt und der maximale Unter- und
Überdruck alternierend angelegt, um jeweils die Grundresonanzfrequenz
$f_{0}$ zu messen. Im Rahmen der Meßgenauigkeit\footnote{Die Genauigkeit
der Frequenzmessung betrug etwa $\pm$100~Hz, die der Druckmessung etwa
$\pm$(1--3)~mbar.} konnten geringe Hystereseeffekte festgestellt werden.
Die Streubreite der Frequenzwerte, die etwa $\pm 200$~Hz beträgt, läßt sich
aufgrund des ideal elastischen Materialverhaltens von Silizium \cite{Ove77}
nur durch den Einfluß der Klebschicht zwischen der BOD-Struktur und
dem Aluminiumträger erklären.
Im Vergleich zu dem in Kapitel~\ref{druckabh} vermessenen Membrandrucksensor
zeichnet sich die vermessene BOD-Kennlinie durch eine bessere
Linearität\footnote{Die Nichtlinearität der Kennlinie beträgt im
vermessenen Druckbereich (-0,8 bis 1,0 bar) etwa $\pm$3,8~\%.}, bei
gleichzeitig wesentlich erhöhter Druckempfindlichkeit aus.




%\newpage
\subsection{Optimierung der Druckempfindlichkeit}
\label{bodoptimierung}


Bei der Optimierung der Druckempfindlichkeit von BOD-Sensoren ist darauf zu
achten, daß die BOD-Membran über die Trägerstegstruktur bei gegebener
Druckdifferenz eine möglichst hohe Verspannung im Balken hervorruft.
Diese ist abhängig von den Abmessungen der Membran ($a$ : Seitenlänge,
$h_{m}$ : Dicke) und
des Balkens ($l$ : Länge, $h_{b}$ : Dicke), sowie der Steghöhe $h_{s}$.
Eine analytische Näherungsbeschreibung der druckabhängigen Frequenzänderung
von BOD-Strukturen ist unter der Annahme möglich, daß das Biegeverhalten der
als ideal starr eingespannt angenommenen Membran durch den Balken und die
Trägerstegstruktur nicht
behindert wird. Diese Voraussetzung ist im wesentlichen nur für
dicke Membranen erfüllt, bei denen die Membransteifigkeit im Verhältnis zur
Steifigkeit, die durch die Trägerstege-Konstruktion gebildet wird, sehr groß
ist. Die druckabhängige Frequenzänderung läßt sich dann näherungsweise
durch:
\begin{eqnarray}
\label{bodfreq}
 f(p) & = & f_{0} \sqrt{ 1 + const
      \left( \frac{l}{h_{b}} \right)^2
      \left( \frac {h_{s}} {{h_{m}^3}} \right)
      \left( a^2 - \frac {l^2} {4}   \right)
      \frac {p} {\hat E}  }
\end{eqnarray}
beschreiben \cite{Tho90}. Die Konstante ist abhängig von der Membrangeometrie
und beträgt für runde Membranen etwa 0,225. Größenordnungsmäßig sollte
dieses auch für quadratische Membranen erfüllt sein, so daß es mit diesem
Ausdruck möglich ist, das Skalierungsverhalten der Druckempfindlichkeit
abzuschätzen und
die wesentlichen Geometrieparameter im Entwurfsprozeß festzulegen.
Bei gegebener Balkenlänge $l$ ist zum einen die Schlankheit des Balkens
($l/h_{b}$-Verhältnis), zum anderen die Membrandicke $h_{m}$ von
entscheidender Bedeutung. Außerdem ist eine Erhöhung der
Membranseitenlänge $a$ im anwendungsspezifischen Rahmen möglichst
anzustreben. Der Einfluß der Steghöhe $h_{s}$ ist aufgrund des linearen
Einflusses nahezu vernachlässigbar.\\
%
Bei der realisierten BOD-Struktur sind die der Gleichung (\ref{bodfreq})
zugrundeliegenden Annahmen {\em nicht} erfüllt, da die über die Gesamtlänge
der Membran verlaufenden Stege,
die zur Balkenbefestigung dienen, das Biegeverhalten der Membran wesentlich
verändern. Zusätzlich beeinflußt die Steifigkeit der Trägerkonstruktion
(H-Struktur) die Hebelwirkung auf den Balken. Um die Hebelwirkung bei
der vorliegenden BOD-Struktur genauer zu untersuchen, wurde daher ein
dreidimensionales FE-Modell mit dem Volumenmodellierer {\sf I-DEAS}
\cite{SDRC} erstellt. Aus Symmetriegründen wurde nur ein Viertel der
Gesamtgeometrie betrachtet, da nur das Verhalten der
Grundbiegeschwingungsmode bei der Drucksensoranwendung von Interesse ist.
Die Materialanisotropie und die $\langle$110$\rangle$--Orientierung des
Siliziumwafers
wurden durch eine $45^{\circ}$--Drehung des Elementkoordinatensystems
berücksichtigt.\\
%----------------------- Beginn: Figure-Environment ----------------------
\begin{figure}[htb]
%\vspace*{8cm}

\begin{center}
% --- Dateiname des Bildes
\input{abbsedr.tex}
\setabbsedr
\end{center}

\caption{\label{abbbodansys}
 Berechnete Auslenkungen $u_{z}$ (1), Spannungsverteilung $\sigma_{y}$ (2),
 Elementierung (3) und Membranschwingungsmode (4) des BOD-Drucksensors}
\end{figure}
%----------------------- Ende: Figure-Environment ----------------------
In {\bf Abbildung~\ref{abbbodansys}} sind die Geometrie des Viertelmodells
und die Ergebnisse der FE-Berechnungen
abgebildet. Im oberen linken Fenster (1) sind die berechneten Auslenkungen
$u_{z}$ bei homogener Druckbeaufschlagung der BOD-Struktur zu sehen.
Bei einer Druckbeaufschlagung von 1~bar beträgt die maximale
Membranauslenkung etwa
1~$\mu$m. Im oberen rechten Fenster (2) ist die gemittelte
Spannungsverteilung $\sigma_{y}$ entlang des Balkens abgebildet.
Die maximale Zugspannung konzentriert sich aufgrund der Hebelwirkung im
Resonatorbalken und beträgt etwa 24,7~MPa. Dieses entspricht
etwa dem Doppelten
der Spannung auf der Membranoberfläche. Im Fenster (3) ist die Elementierung
im Bereich des Balkens und der Einspannung zu sehen, die sehr fein gewählt
wurde, um die Hebelwirkung und die daraus resultierende
Spannungskonzentration im Balken möglichst genau berechnen zu können.
Aufgrund der nicht mit Regelflächen begrenzten Einspanngeometrie mußte
dieser Bereich des FE-Modells mit Tetraederelementen\footnote{Das
Gesamtmodell umfaßt 1748 Volumenelemente mit 2248 Knoten.}
vernetzt werden. Bei der Berechnung der Schwingungsmoden\footnote{Um die
numerischen Fehler möglichst niedrig zu halten, wurde bei der Modalanalyse
das {\em Subspace}-Iterationsverfahren verwendet.}
der BOD-Struktur wurden nur die Biegeschwingungsmode Z1 des Balkens und
die erste gekoppelte Schwingungsmode, bei der der Balken und die Membran
gegenphasig zueinander schwingen, berücksichtigt.
Im Fenster (4) ist diese antisymmetrische Membranschwingungsmode des
BOD-Drucksensors dargestellt.\\
Gemäß Gleichung (\ref{bodfreq}) ist die Membrandicke der BOD-Struktur
aufgrund des kubischen Einflusses der ausschlaggebende Parameter,
so daß ausgehend von der untersuchten
Referenzgeometrie des BOD-Drucksensors die Membrandicke im Bereich
50--300~$\mu$m variiert wurde. Die übrigen Geometrieparameter wurden
unverändert beibehalten. Die Resonanzfrequenz der Biegeschwingungsmode Z1
des Balkens sollte unabhängig von der Membrandicke sein, während die
Frequenz der
Membranschwingungsmode proportional zur Membrandicke nach
Gleichung (\ref{memfreq}) anwachsen sollte. Dieses Verhalten konnte
zumindest für dicke Membranen (150--300~$\mu$m) rechnerisch verifiziert
werden und ist in {\bf Abbildung~\ref{abbbodfbfm}} graphisch dargestellt.
%----------------------- Beginn: Figure-Environment ----------------------
\begin{figure}[htb]
%\vspace*{8cm}

\begin{center}
% --- Dateiname des Bildes
\input{bodmem.tex}
\setbodmem
\end{center}
 \caption{\label{abbbodfbfm}
 Berechnete Modenaufspaltung beim BOD-Drucksensor in Abhängigkeit
 der Membrandicke}
\end{figure}
%----------------------- Ende: Figure-Environment ----------------------
Bei der Variation der BOD-Membrandicken von 50 auf 300~$\mu$m w„chst
die Frequenz der Membranschwingungsmode $f_{Membran}$ von 65,6 auf 174,9~kHz
an. Die Balkenresonanzfrequenz $f_{Balken}$ bewegt sich im Bereich
90,5--95,4~kHz. Die Abweichungen von 16~\% zu dem gemessenen Frequenzwert
(82~kHz) lassen sich einerseits durch die großen Dickenschwankungen des
Balkens ($\pm$5~$\mu$m), andererseits durch die ungenaue Balkenlänge
(1,99$\pm$0,04~mm) erklären. Insbesondere gibt es eine Membrandicke, bei der die Steifigkeit
der Membran und der Trägerstege-Konstruktion von gleicher Größenordnung sind,
so daß beide Resonanzfrequenzen dicht nebeneinander zu liegen kommen. Diese
Konfiguration wird bei einer Membrandicke von etwas weniger als 130~$\mu$m
erreicht. Beide Frequenzen unterscheiden sich dann nur noch
geringfügig\footnote{Bei einer Membrandicke von 130~$\mu$m beträgt die
Differenz der beiden Frequenzen 5773~Hz (siehe Tabelle~\ref{tabbodopti}).},
so daß die Energie der Balkenschwingungsmode in die Membranschwingungsmode
dissipieren und eine Modenkopplung, ähnlich wie beim
Dreifachbalkenresonator in Abbildung~\ref{abbmodenkoppl1},
auftreten kann. Bei realen Sensoranwendungen sollte solch eine ungünstige
Konfiguration daher vermieden und beim
Entwurf darauf geachtet werden, daß die Membranfrequenz möglichst weit
außerhalb des druckabhängig überstrichenen Frequenzbereiches des
Balkenresonators bleibt. Im Vergleich zu den in der Literatur vorgestellten
BOD-Drucksensoren (z.B.\ \cite{And88}), zeichnet sich die hier beschriebene
Sensorstruktur durch ein einfacheres Modenspektrum aus, was sich in der
späteren Anwendung als frequenzanaloger Drucksensor durch eine erhöhte
Unimodalität des Gesamtsystems auszeichnet.\\
%
Analog zu den rechnerischen Untersuchungen in Kapitel~\ref{druckabh} wurden
die BOD-Drucksensoren von der Membranunterseite schrittweise mit Überdruck
bis 1~bar belastet und anschließend die Resonanzfrequenzänderungen infolge
der spannungsversteifenden Wirkung ermittelt. Die Frequenz-Druck-Kennlinien
der Balkenschwingungsmoden Z1 sind in {\bf Abbildung~\ref{abbbodmem}}
für BOD-Drucksensoren mit einer Membranseitenlänge von 5~mm für
unterschiedliche Membrandicken zusammengefaßt.
%----------------------- Beginn: Figure-Environment ----------------------
\begin{figure}[htb]
%\vspace*{8cm}

\begin{center}
% --- Dateiname des Bildes
\input{abbsfa.tex}
\setabbsfa
\end{center}
\caption{\label{abbbodmem}
 Einfluß der Membrandicke auf die Druckempfindlichkeit von BOD-Drucksensoren}
\end{figure}
%----------------------- Ende: Figure-Environment ----------------------
Die unbelasteten Resonanzfrequenzen $f_{0}$ liegen um 93~kHz, sofern die
Membrandicke $h_{m}$ auáerhalb des kritischen Bereichs von etwa
130~$\mu$m $\leq$ $h_{m}$ $\leq$ 150~$\mu$m bleibt. Dieses ist in guter
Näherung für relativ dicke (200, 250, 300~$\mu$m) und für dünne
Membranen (100, 110~$\mu$m) erfüllt. Infolge des ungünstigen
Steifigkeitsverhältnisses\footnote{Eine Abnahme der Membrandicke bewirkt
eine Abnahme der Biegesteifigkeit der Membran im Verhältnis zur Steifigkeit
der H-Trägerstruktur.} von BOD-Membran zur Balken-Stege-Konstruktion
beeinflussen sich die beiden Schwingungsmoden der BOD-Struktur und es kommt
zu einer Verschiebung der Balkenresonanzfrequenz $f_{Balken}$ (vgl. hierzu
Abbildung~(\ref{abbbodfbfm})). Während die Resonanzfrequenz bei einer
Membrandicke von 130~$\mu$m auf den niedrigsten Wert (90,5~kHz) fällt,
stellt sich bei einer Membrandicke von 150~$\mu$m der maximale Wert
(95,4~kHz) ein. Membrandicken von 120 und 140~$\mu$m stellen ebenfalls
Geometriekonfigurationen aus dem kritische Übergangsbereich dar. Die
Druckempfindlichkeit der BOD-Sensorstruktur ist umso größer, je dünner die
Membran ist. Die höchste Druckempfindlichkeit von etwa 12,5~kHz/bar weist
die BOD-Struktur mit einer Membrandicke von 50~$\mu$m auf. Allerdings ist
der Druckeinsatzbereich auf einige Hundert mbar beschränkt. Bei Erhöhung
der Membrandicke auf 300~$\mu$m erweitert sich der Druckbereich bis etwa
12~bar. In {\bf Tabelle~\ref{tabbodopti}} sind die Ergebnisse der
FE-Berechnungen für die Variation der Membrandicke zusammengefaßt.
%----------------------- Beginn: table ---------------------------
\begin{table}[htb]
\caption{\label{tabbodopti}
 Charakteristische Kenngrößen verschiedener BOD-Drucksensoren in Abhängigkeit
 der Membrandicke}
\begin{center}
\begin{tabular}{|c||c|c||c|c|}
\hline
 $h_{m}$ & $f_{Balken}$ & $f_{Membran}$
 & $\frac{\Delta f}{\Delta p}$ & $p_{max}$ \\
\hline
 $[\mu$m] & [kHz] & [kHz] & $[\frac{kHz}{bar}]$ & [bar] \\
\hline \hline
  50 & 93,668 & 65,556 & 12,48  &  0,5 \\
 100 & 92,870 & 80,386 & 7,955  &  1,0 \\
 130 & 90,467 & 96,240 & 1,314  &  1,5 \\
\hline
 150 & 95,387 & 103,611 & 2,615 &  2,0 \\
\hline
 200 & 93,085 & 126,503 & 2,446 &  5,0 \\
 250 & 93,174 & 150,731 & 1,547 &  7,0 \\
 300 & 93,207 & 174,869 & 1,026 & 12,0 \\
\hline
\end{tabular}
\end{center}
\end{table}
%----------------------- Ende: table ---------------------------
In Erweiterung zu der Abbildung~\ref{abbbodmem} sind die
BOD-Strukturen bis zu ihrem jeweiligen Maximaldruck $p_{max}$ belastet
worden, bei dem die mechanischen Maximalspannungen $\sigma_{max}$ im Balken
etwa ein Viertel der Bruchspannung von Silizium nicht übersteigen durften.
Die maximalen Membranauslenkungen $u_{z}$ betragen bei den angegebenen
Maximaldrücken etwa 2~$\mu$m, während die maximalen Spannungen
$\sigma_{max}$ im Bereich um 50--60~MPa liegen. Bei dieser zugrundegelegten
vierfachen Überlastsicherheit können Drucksensoren für einen Druckbereich
von 0,5--12~bar bei {\em gleichem Layout} lediglich durch Änderung der
Membrandicke realisiert werden. Der Einsatzbereich der Sensoren läßt sich
erweitern, falls von der zugrundegelegten Überlastsicherheit abgewichen
wird.\\
Bei konstant gehaltenen Balkendimensionen und unveränderter Steghöhe
(d.h.\ Ätztiefe) ist das laterale Sensorlayout mit der Membranseitenlänge
$a$ der bestimmende Parameter für die Druckempfindlichkeit. Eine
Verkleinerung dieses Parameters ist im Rahmen der prozeßspezifischen
Forderungen bezüglich der Verringerung der Prozeßzeit anzustreben, da
einerseits bei kleinerer Membranfläche die Schreibzeiten mit dem Laser
verringert und andererseits der Einsatz eines Excimer-Lasers möglich wird.
Im Gegensatz zum Nd:YAG-Laser, der eine
fokussierte Strahlquelle (Strahldurchmesser: 5--10~$\mu$m) besitzt und ein
sequentielles Schreibverfahren bedingt, erlaubt der Excimer-Laser eine
großflächige Bearbeitung (Schreibfläche: ca. 10 mm x 20 mm) des
Siliziumwafers \cite{Dissaxel}. Durch die weitere Miniaturisierung der
BOD-Drucksensoren ist es außerdem möglich, auf einem
Siliziumwafer viele Sensorelemente herzustellen und damit bei großer Anzahl
und guter Ausbeute reduzierte Herstellungskosten zu erreichen.
Im weiteren wurde daher die Membranseitenlänge bei gleichen
Balkendimensionen
(Länge: 1,95~mm, Breite: 120~$\mu$m, Dicke: etwa 42,4~$\mu$m)
von 5 auf 2.5~mm verringert und gleichzeitig verschiedene Membrandicken
untersucht. Zusätzlich wurde ein Sensorlayout mit minimalen
Abmessungen von 2~x~2~mm$^{2}$ modelliert. Aufgrund der vorgegebenen Breite
der Trägersteg-Konstruktion wurde die Balkenlänge auf 1~mm verkürzt.
Im Vergleich mit dem realisierten Referenzdrucksensor
(Membranabmessungen: 5~x~5~mm$^{2}$, Membrandicke: 150~$\mu$m)
zeichnet sich diese miniaturisierte Drucksensorkonfiguration durch eine
hohe Resonanzfrequenzen der Balkenschwingungsmode (385,3~kHz) und der
Membranmode (627,3~kHz) aus und weist eine Druckempfindlichkeit von
0,539~kHz/bar im Bereich bis etwa 20 bar auf.\\
%----------------------- Beginn: table ---------------------------
\begin{table}[htb]
\caption{\label{tabbodabm}
 Charakteristische Kenngrößen verschiedener BOD-Drucksensoren bei
 Variation von Membranseitenabmessungen und -dicke}
\begin{center}
\begin{tabular}{|c|c||c|c||c|c|}
\hline
 \multicolumn{2}{|c||}{Membranabm.} & $f_{Balken}$ & $f_{Membran}$  &
 $\frac{\Delta f}{\Delta p}$  &  $p_{max}$ \\
\hline
 $a [mm^{2}]$   &  $h_{m} \mu$m  &  [kHz]  &  [kHz] &
 $[\frac{kHz}{bar}]$ & [bar] \\
\hline \hline
 3,0 & 150 & 93,198 & 266,913 & 11,43 & 1,5 \\
 3,5 & 100 & 96,327 & 156,130 &  3,70 & 7,5 \\
\hline
 5,0 & 150 & 95,387 & 103,611 &  2,62 & 4,0 \\
\hline
 4,5 & 250 & 96,510 & 184,730 &  1,21 & 24,0 \\
 3,5 & 250 & 96,538 & 297,220 &  0,65 & 38,0 \\
 2,5 & 250 & 96,542 & 377,380 &  0,02 & 76,0 \\
\hline
\end{tabular}
\end{center}
\end{table}
%----------------------- Ende: table ---------------------------
Die berechneten Druckempfindlichkeiten der BOD-Drucksensoren mit
verschiedenen Membranabmessungen sind in {\bf Abbildung~\ref{abbbodabm}}
graphisch dargestellt und die charakteristischen Kenngrößen in
{\bf Tabelle~\ref{tabbodabm}} zusammengefaßt. Als maximale Belastung
sind die Drücke $p_{max}$ zugelassen worden, bei denen die mechanischen
Maximalspannungen $\sigma_{max}$ im Balken etwa die Hälfte der Bruchspannung
von Silizium (etwa 100~MPa) betrugen.
%----------------------- Beginn: Figure-Environment ----------------------
\begin{figure}[htb]
%\vspace*{8cm}

\begin{center}
% --- Dateiname des Bildes
\input{abbssz.tex}
\setabbssz
\end{center}
\caption{\label{abbbodabm}
 Geometrieeinfluß (Membranseitenlänge/-dicke) auf die Druckempfindlichkeit
 von BOD-Drucksensoren}
\end{figure}
%----------------------- Ende: Figure-Environment ----------------------
Aufgrund der unveränderten Trägerstege-Konstruktion unterscheiden sich
die Hebelwirkungen der einzelnen BOD-Sensoren erheblich, da die geometrische
Anordnung der Balken auf der Membran verändert wird und sich zusätzlich
die Verringerung der Membranseitenlänge bemerkbar macht.
Durch Ausnutzung dieser beiden Effekte entsteht ein weiter Spielraum für
die Parameterwahl bei der Dimensionierung von BOD-Drucksensoren.
Wie in Abbildung~\ref{abbbodabm} dargestellt lassen sich durch eine
geeignete Wahl der Membranseitenlänge (Angaben in: mm) und der Membrandicke
(Angaben in: $\mu$m) die Druckempfindlichkeit der BOD-Drucksensoren in einem
weiten Bereich einstellen. Die FEM-Berechnungen bei zusätzlicher Variation
der Membrandicke haben allerdings gezeigt, daß es aus Sicht der Beeinflussung
von Balken- und Membranmode günstiger ist, Membrandicken von 150~$\mu$m
zu vermeiden \cite{Messner}. Der Druckeinsatzbereich läßt sich weiter
steigern, indem die Membrandicke erhöht und die Membranseitenabmessungen
gleichzeitig verringert werden. So zeichnet
sich ein BOD-Drucksensor mit einer Seitenlänge von 2~mm und einer Dicke von
300~$\mu$m durch eine Druckempfindlichkeit von etwa 141 Hz/bar in einem
Maximaldruckbereich bis 100~bar aus. Die maximal auftretenden Spannungen
betragen hierbei etwa die Hälfte der Bruchspannung von Silizium.




\newpage
\section{Zusammenstellung der Resultate}

In diesem Abschnitt wurden zwei alternative Sensorstrukturen für die Kraft-
und Druckmessung theoretisch und experimentell untersucht. Insbesondere
konnte das dynamische Verhalten unter Einwirkung von äußeren Belastungen
charakterisiert werden. Mit Hilfe der numerischen FE-Berechnungen wurden
Entwurfsvorgaben für die Optimierung der
Sensor\-eigenschaften abgeleitet. Gegenüber anderen Sensorkonzepten konnten
verschiedene Vorteile nachgewiesen werden. Im folgenden sind die Ergebnisse
der Untersuchungen zusammengefaßt:

\subsection{Dreifachbalken-Kraftsensor}
\begin{itemize}
\item
Durch Einführung einer beidseitigen mechanischen Entkopplung in Form eines
membranartigen Entkopplungsbereiches der Länge 200~$\mu$m, das sind etwa
6,7~\% der Balkenlänge, wird die Unimodalität des Dreifachbalkenresonators
wesentlich erhöht. Die Modenaufspaltung zwischen der Grundmode M1 und der
gewünschten Schwingungsmode M3, bei dem der Sensor betrieben wird, nimmt
hierdurch um den Faktor 30 zu und beträgt absolut etwa 3~kHz.
\item
Weiterhin wird durch eine geeignete Dimensionierung dieses
Entkopplungsbereiches eine dynamische Momentenkompensation bei der
antisymmetrischen Schwingungsmode M3 erreicht und es stellt sich eine etwa
doppelt so hohe Schwingungsgüte wie bei der Grundmode M1 ein.
Die maximal erzielten Güten betrugen für die vermessenen Schwinger
$Q_{M3} \approx 2 \cdot Q_{M1} \approx 400$ in Luftatmosphäre.
\item
% Durch eine Spannungskonzentration in dem Dreifachbalken-Resonator erhöht
% sich die Kraftempfindlichkeit gegenüber einem Einfachbalken-Kraftsensor um etwa 6~\%.
Die gemessene Kraftempfindlichkeit betrug 8,6~kHz/N im
Kraftmeßbereich bis 5~N. Dieses entspricht einer relativen
Kraftempfindlichkeit von etwa
$\eta~=~\frac{1}{f_{0}}\frac{\Delta f}{\Delta F}$~=~0,39~N$^{-1}$
bei einer Resonanzfrequenz von $f_{M3}$~=~22,0~kHz.
\item
Durch den Einsatz von lasergebohrten Löchern konnte eine uniaxiale
Kraft\-einleitung in den Kraftsensor ermöglicht und Scherkräfte
unterdrückt werden.
\end{itemize}
Bei der Realisierung von Kraftsensoren für Präzisionsmeáanwendungen sind
allerdings noch erheblich höhere Schwingungsgüten notwendig, um eine
entsprechende Meßgrößenauflösung (siehe Gleichung~\ref{Qaufl}) zu erreichen.
Dieses
könnte beispielsweise durch die Herstellung von Balken mit einem größeren
Längen-/Breiten-Verhältnis erzielt werden \cite{Kir83}, da
eine Evakuierung des Resonatorgehäuses bei Kraftsensoren aufgrund der
erforderlichen Kraft\-einleitung problematischer als bei Drucksensoren ist.
Außerdem ist eine technologische Optimierung der $ZnO$-Schichten im Hinblick
auf die inneren Spannungen erforderlich. Alternativ bietet sich die
elektrothermische Anregung mit Hilfe eindiffundierter Widerstände in
Verbindung mit piezoresistiver Abtastung der Dreifachbalkenresonatoren
an. Bei Resonatorlängen von 500--700~$\mu$m und Balkendicken von etwa
6~$\mu$m konnten Schwingungsmoden bis etwa 1~MHz elektrothermisch angeregt
werden \cite{Bur93}.



\subsection{BOD-Drucksensor}

Die vorgestellte BOD-Sensorstruktur wurde mit Hilfe des laserunterstützten
anisotropen Ätzens vollständig in Silizium {\em monolithisch integriert}
hergestellt. Dadurch können Ätzstopp- \cite{Gre88} oder sonstige
Verbindungstechniken \cite{Par92} entfallen. Es sind auch keine zusätzlichen
manuellen Eingriffe \cite{Bus91a} erforderlich, so daß es sich hier um
einen einfachen, aber {\em reproduzierbaren} und vor allem {\em batchfähigen}
Herstellungsprozeß handelt. Es konnte gezeigt werden, daß aufgrund der
Vielzahl von variablen Geometrieparametern sich die BOD-Struktur für die
Realisierung von Drucksensoren mit unterschiedlichen Sensorspezifikationen
in verschiedenen Druckmeßbereichen sehr gut eignet.
\begin{itemize}
\item
Konventionelle Herstellungprozesse, wie elektrochemischer Ätzstopp,
und die hierfür erforderlichen Implantierungs- und Diffusionsprozesse
bieten aufgrund der geringen Diffusionslängen nur die Möglichkeit
der Herstellung von relativ dünnen Schwingerstrukturen, typischerweise
unter einem Mikrometer. Diese weisen zwar eine hohe Empfindlichkeit auf,
besitzen aber schlechte Überlasteigenschaften, insbesondere gegenüber
dynamischen Druckstößen. Der Nachteil bei {\em zeitlich} geätzten
Siliziumbalken sind die Inhomogenitäten und Schwankungen, so daß starke
Bauelementestreuungen auftreten.
\item
BOD-Drucksensoren zeichnen sich gegenüber anderen Drucksensorkonzepten
insbesondere durch die funktionelle Trennung von sensitivem Element
(Resonatorbalken) und druckeinleitendem Medium (Druckmembran) aus.
Der hier vorgestellte BOD-Drucksensor bietet mehrere Entwurfsparameter
(Membran- und Balkenabmessungen, Steghöhe und -breite) zur Auslegung der
Druckempfindlichkeit, des Druckbereiches und des Überlastverhaltens.
\item
Ein Vorteil der mit laserunterstütztem anisotropen Ätzen hergestellten
BOD-Drucksensoren sind die erzielbaren großen Balkendicken, die bis zu
etwa 200~$\mu$m betragen können. Experimentell wurden Balkendicken im
Bereich 40--120~$\mu$m realisiert und vermessen. Die Dickenschwankungen
betragen etwa $\pm$(3--5)~$\mu$m, was bei zunehmenden Balkendicken
vernachlässigbar wird. Durch Einsatz dicker Balken und Membranen ist die
Realisierung von Hochdrucksensoren bis in einem Bereich von über 100
bar möglich.
\item
Gegenüber resonanten Membrandrucksensoren zeichnen sich resonante
BOD-Drucksensoren weiterhin durch hohe Druckempfindlichkeiten bei
gleichzeitig niedrigen Kennliniennichtlinearitäten aus. Der experimentell
realisierte BOD-Drucksensor besitzt eine Druckempfindlichkeit von
4,47~kHz/bar im Druckbereich von -0,8 bis 1,0 bar, was einer relativen
Druckempfindlichkeit von etwa 5,45~\%/bar entspricht. Die
Kennliniennichtlinearität beträgt im betrachteten Meßbereich
etwa $\pm$3,8~\%.
\item
Mit Hilfe der FE-Berechnungen konnten Entwurfsvorgaben abgeleitet
werden und durch geeignete Dimensionierung der Membran- und
Balkenabmessungen die Druckempfindlichkeit wesentlich erhöht werden.
Theoretisch wurde gezeigt, daß durch Änderung der Membrandicke
Sensoren für einen Druckbereich von 0,5--12~bar mit {\em gleichem Layout}
bei einer etwa vierfachen Überlastsicherheit herzustellen sind.
FE-Berechnungen an einem BOD-Sensor mit der Membranseitenlänge
von 2~mm und einer Dicke von 300~$\mu$m ergaben eine Druckempfindlichkeit
von etwa 141 Hz/bar in einem Maximaldruckbereich bis 100~bar. Die maximal
auftretende Spannung betrug etwa die Hälfte der Bruchspannung von Silizium.
\item
Infolge der Spannungskonzentration im Balken eignet sich die
BOD-Sensorstruktur ebenfalls sehr gut für die {\em analoge}
Signalauswertung, beispielsweise mit Hilfe auf der Balkenoberfläche
diffundierter piezoresistiver oder gesputterter Metall-DMS-Widerstände.
\end{itemize}
%
