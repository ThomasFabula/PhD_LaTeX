\chapter{Schlußbetrachtung}
\label{schluss}

Die Anforderungen an den Entwurf bei der Entwicklung von mikromechanischen
Bauelementen bedingen den Einsatz rechnergestützter, numerischer
Berechnungsmethoden, um sich überlagernde nichtlineare Effekte, anisotrope
Materialeigenschaften und die elektro-thermo-mechanischen Wechselwirkungen
bei Mehrschichtsystemen beschreiben zu können. So kann beispielsweise die
Optimierung mikromechanischer Resonanzsensoren nur unter gleichzeitiger
Betrachtung der statischen {\em und} dynamischen Eigenschaften unter
Berücksichtigung des physikalischen Anregungsprinzips erfolgen.
Mit den im Rahmen dieser Arbeit entwickelten FE-Modellen in Verbindung mit
den Möglichkeiten des kommerziellen Programmsystems {\sf ANSYS} wurden
gekoppelte Feldberechnungen durchgeführt und das Verhalten piezoelektrisch
betriebener Sensoren untersucht. Damit ließen sich bereits
in der Entwurfsphase wichtige Vorgaben, wie z.B.\ die günstigste
Elektrodenanordnung und das optimale Schichtdickenverhltnis, für die
nachfolgenden technologischen Prozeßschritte ableiten.\\
%
Die durchgeführten Untersuchungen haben gezeigt, daß die {\em Methode der
Finiten Elemente} geeignet ist, das dynamische Verhalten mikromechanischer
Strukturen zu beschreiben. Weiterhin wurde gezeigt, daß bei der
Berechnung der Resonanzfrequenzen und Schwingungsmoden, sowie der
lastabhängigen Resonanzfrequenzänderungen eine gute Übereinstimmung mit den
Meßergebnissen erzielt werden konnte. Insbesondere haben sich die Stärken
der FE-Methode bei der Behandlung folgender Problemstellungen herausgestellt:
%
\begin{itemize}
\item
Berücksichtigung nichtlinearer geometrischer Effekte, insbesondere der
Spannungsversteifung bei großen Auslenkungen und Deformationen der
Mikrostrukturen.
\item
Strukturierung der Elektroden zur selektiven Anregung von Schwingungsmoden
und Unterdrckung von unerwünschten Oberwellen. Desweiteren erlauben
Siliziumsensoren in Bimorph- und Mehrschichtaufbau durch eine geeignete
laterale Schichtstrukturierung eine Temperaturkompensation der
Sensorkennlinien.
\item
Modellierung des elektro-mechanischen Sensorverhaltens unter
Berücksichtigung des piezoelektrischen Antriebsprinzips, mit dessen
Hilfe die Ableitung elektrischer Kenngrößen (Impedanz- und Phasenverhalten)
in Abhängigkeit geometrischer Randbedingungen möglich ist. Eine Variation
der Schichtdickenverhältnisse erlaubt eine Optimierung der erzielbaren
effektiven elektromechanischen Kopplungsfaktoren.
\item
Untersuchung des Einflusses der Resonatorquerschnitte und Einspannbereiche
auf das Schwingungsverhalten mikromechanischer Resonatoren. Durch Einführung
von Entkopplungsbereichen lassen sich Modenkopplungen erheblich unterdrücken.
\item
Die Einspanneffekte und die Hebelwirkung beim BOD-Drucksensor lassen sich
bei dem vorgestellten komplexen Sensorentwurf nur noch numerisch
optimieren.
\end{itemize}
%
Die Vorteile der FE-Methode bestehen darin, komplexe Geometrien unter
vielfältigen Randbedingungen zu modellieren,
Parameterstudien durchzuführen und Geometrie- und Materialeinflsse
separat betrachten zu können. Hierbei ist die Stärke der FE-Methode
weniger die Berechnung von absoluten Größen und quantitativen Ergebnissen,
die zwar bei {\em korrekten} Eingangsdaten\footnote{Dieses sind im
wesentlichen die Strukturgeometrie, das Materialverhalten und die
zugrundeliegenden Randbedingungen.} prinzipiell möglich sind, aber oft
einen erheblichen Modellieraufwand und entsprechende Rechnerressourcen
erfordern, sondern vielmehr die Berechnung der relativen Abhängigkeiten
von den Modellparametern. Auf diese Weise können Parametervariationen und
Sensitivitätsanalysen durchgeführt und mikromechanische Strukturen am
Rechner analysiert und bereits im Vorfeld optimiert werden. Die FE-Methode
hat sich dabei als ein effizientes Werkzeug bei der Entwicklung von
mikromechanischen Strukturen erwiesen.\\
%
Einschränkend bleibt jedoch festzuhalten, daß beim mikromechanischen
Entwurf der Einsatz der FE-Methode an einigen Stellen auch an ihre Grenzen
stößt. So gehen die zugrundeliegenden Differentialgleichung, auf die in
Kapitel~3 eingegangen wurde, beispielsweise von
geschwindigkeitsproportionalen Dämpfungseffekten
aus und erlauben nur die Behandlung kleiner Schwingungsamplituden.
Dynamische Nichtlinearitäten, wie sie beim anharmonischen Oszillator oder
bei chaotischem Systemverhalten auftreten können, lassen sich nicht
erfassen. Um sie zu beschreiben muá auf vereinfachte analytische
Ersatzmodelle zurückgegriffen werden, die in der Regel von Schwingern mit
einem Freiheitsgrad ausgehen \cite{Pra93,Til93}. Weiterhin sind die
kopplungsbeschreibenden Konstanten bei
gekoppelten Feldberechnungen weitgehend unbekannt. Erschwerend kommt hinzu,
daß die Materialeigenschaften in der Regel anisotrop sind und in erster
Näherung daher isotrope Ersatzdaten ermittelt werden müssen. Auáerdem sind
die zugrundeliegenden physikalischen Effekte nichtlinear, so daá bei den
beschreibenden Zustandsgleichungen Tensorbeiträge höherer Ordnung zu
bercksichtigen sind. Diese sind in den kommerziellen und auch in den
meisten an Hochschulen entwickelten FE-Programmen nicht implementiert.
Weitere Beispiele für nichterfaáte gekoppelte Effekte ist die
Elektrostriktion oder die Pyroelektrizität, sowie die Magnetostriktion, bei
der die Kopplung zwischen einem äußeren Magnetfeld
und der Strukturmechanik in mikromechanischen Aktoranwendungen ausgenutzt
wird. Eine Möglichkeit zur Modellierung solcher Effekte besteht in der
Ausnutzung von Analogien, aufgrund der gleichen mathematischen
Beschreibungsweise. Unter
gewissen Vernachlässigung und in einigen Sonderfällen kann mit den
FE-implementierten thermodynamischen Zustandsgleichungen und den
nichtlinearen numerischen Lösungsalgorithmen gerechnet werden \cite{Goetz}.
Ein weiteres Problem stellen die Materialeigenschaften dar, die zusätzlich von
den thermodynamischen Randbedingungen abhängig sind, d.h.\ ob isotherme oder
adiabatische Bedingungen vorliegen. Zukünftig sind hier weitere Anstrengungen
notwendig, um die technologischen und meßtechnischen Arbeiten zu
koordinieren und die nötigen Informationen in den Entwurfsprozeß
einfließen zu lassen.\\
%
Als Ausblick sollen an dieser Stelle einige mögliche Anwendungen
dynamisch betriebener mikromechanischer Strukturen abschließend
aufgezeigt werden:
%
\begin{itemize}
\item
{\bf Resonante Multimode-Drucksensoren} können in verschiedenen
Schwingungsmoden betrieben werden, deren Resonanzfrequenzen sich
durch unterschiedliche Druck- und Temperaturempfindlichkeiten auszeichnen.
Auf diese Art ist eine Temperaturkompensation {\em on-line} möglich, da
zwischen den verschiedenen Schwingungsmoden während des Meßbetriebes
umgeschaltet werden kann.
\item
{\bf Mikromechanische Schwinger} lassen sich als Teststrukturen zur
Bestimmung von Dünnschichteigenschaften einsetzen, falls die
Materialeigenschaften des Substrates bekannt sind und
Biegeschwingungsmoden gezielt angeregt werden. Über die Messung der
Resonanzfrequenzverschiebung läßt sich die innere Spannung der Dünnschichten
bestimmen. Weiterhin können mit Hilfe des effektiven elektromechanischen
Kopplungsfaktors die piezoelektrischen Eigenschaften von
Dünnschicht-Piezoelektrika charakterisiert werden.
\item
{\bf Multimode-Aktoren} lassen sich resonant ansteuern und weisen durch
die verschiedenen Schwingungsmoden unterschiedliche Bewegungsmöglichkeiten
auf. Ein zusätzlicher Vorteil stellt die um die mechanische Schwingungsgte
erhöhte Amplitudenvergrößerung im Resonanzfall dar (siehe
Gleichung~\ref{qmess}). Auf diese Weise lassen sich Aktoranwendungen
realisieren, die bei statischer Ansteuerung nicht durchführbar wären.
\end{itemize}
%
Die zukünftige Entwicklung wird zeigen ob die Anwendungen mikromechanischer
Strukturen und die Integration in komplexen Mikrosystemen
die heute in sie gesteckten Erwartungen bezüglich Zuverlässigkeit,
Wirtschaftlichkeit und Funktionalität erfüllen können.
Hierzu müssen neben den technologischen Herstellungsprozessen für eine
wirtschaftliche Produktion von Mikrostrukturen und -systemen die
Leistungsfähigkeit der bestehenden Entwicklungswerkzeuge erweitert und
insbesondere im Bereich der Prozeßsimulation zum Teil neue
Simulationsverfahren und -modelle entwickelt werden.
