\subsection{Metallische Dünnschichten}

Metallische Dünnschichten dienen in erster Linie als elektrische
Leiterbahnen und Kontaktsysteme, sowie der Realisierung von
großflächigen Elektroden.  Außerdem kann bei der elektrothermischen
Anregung von mikromechanischen Komponenten der große Unterschied der
Wärmeausdehnungskoeffizienten zwischen Silizium und Aluminium (Al) oder
Gold (Au) ausgenutzt werden \cite{Rie88}.  Als weitere Anwendung werden
Dünnschicht-DMS aus Nickel-Chrom-Legierungen ({\em NiCr}) für den
resistiven Signalabgriff bei Sensoren eingesetzt. Diese NiCr-Schichten
zeichnen sich durch einen verschwindend kleinen Temperaturkoeffizienten
des Widerstandswertes aus. Die
Herstellung erfolgt in der Regel durch Sputtern, Aufdampfen oder
Elektronenstrahlverdampfen. Zur Verbesserung der Haftung der
Metallschichten auf den darunterliegenden Schichtsystemen oder dem Substrat
dienen verschiedene \glqq Haftvermittler\grqq\footnote{Bei Gold handelt
es sich um eine dünne Chromschicht mit einer Dicke von etwa 30 nm, die
gleichzeitig eine Diffusionsbarriere für das Gold darstellt.}
Die Schichtdicke der Elektroden liegen im Bereich von einem Mikrometer
und sind im Verhältnis zu den Siliziumstrukturdicken in der Regel
mechanisch vernachlässigbar.\\
In {\bf Tabelle~\ref{tabmetalle}} sind die mechanischen und thermischen
Materialeigenschaften von Metallen, wie E--Modul, Dichte $\rho$,
thermischer Wärmeausdehnungskoeffizient $\alpha$, Wärmeleitfähigkeit
$\lambda$, und spezifische Wärmekapazität $c$ zusammengefaßt.
Die {\sl Poisson}-Zahl $\nu$ kann bei den meisten Metallen in guter
Näherung zu 0,3 gesetzt werden.
%----------------------- Beginn: table ---------------------------
\begin{table}[htb]
\caption{\label{tabmetalle}
 Materialeigenschaften von Metallen}
 % [Duf92, LB82, Rie88]}
\begin{center}
\begin{tabular}{|l||c|c|c|c|c|} \hline
{\bf Metall:} & E-Modul & $\rho$   &   $\alpha$  &   $\lambda$  &  c  \\
        & $[GPa]$ & $[kg/m^{3}]$ & $[ppm/K]$ & $ [W/Km] $ & $[J/kgK]$ \\
\hline \hline
Al      & 70--108  & 2700         & 23,8      &  239    &  940 \\ \hline
Au      & 80--190  & 19300        & 14,3      &  310    &  130 \\ \hline
Cr      & 180--346 & 7100         & 7--8,4    &  69     &  460 \\ \hline
Ni      & 247      & 8900         & 12,8      & 59--81  &  450 \\ \hline
\end{tabular}
\end{center}
\end{table}
%----------------------- Ende: table ---------------------------
