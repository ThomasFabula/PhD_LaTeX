
\subsection{Abschätzung der erreichbaren Hübe und Kräfte}
\label{huebeundkraefte}

Für die Abschätzung der erreichbaren Hübe und Kräfte soll ein einseitig
eingespannter Biegebalken bzw. Zunge betrachtet werden, der piezoelektrisch
oder elektrothermisch angesteuert wird.  Um die Leistungsfähigkeit des
piezoelektrischen Antriebs an Mikrostrukturen zu untersuchen, sollen die
Materialien Aluminiumnitrid ({\em AlN}), Zinkoxid (ZnO) und PZT-Keramiken
(PZT) zugrunde gelegt werden, die auch als Dünnschichten hergestellt
werden können.  Da die Materialdaten stark von der Art des
Herstellungsprozesses und der Prozeßparameter abhängig sind, fehlen für
dünne Schichten meist genaue Materialangaben.  Für die folgenden
Abschätzungen beziehen sich die Materialdaten auf Bulk-Material
\cite{Fra88, LB82, Sie81}. Hinzu kommt, daß die Materialeigenschaften
schichtdickenabhängig und die abgeschiedenen Schichten innere
mechanische Spannungen aufweisen. Auf die unterschiedlichen Literaturangaben
bezüglich piezoelektrsicher Dünnschichteigenschaften wird in Kapitel~5.6
weiter eingegangen.\\
%
Aus den piezoelektrischen Zustandsgleichungen~\ref{piezo1}--\ref{piezo3}
ist in \cite{Smi91a, Smi91b} der Zusammenhang zwischen den externen Parametern
(Drehmoment: $M$, Kraft: $F$, Druck: $p$ und Spannung: $U$, siehe
{\bf Abbildung~\ref{abbexternparam}}) und den
internen Parametern (Ablenkungswinkel: $\alpha$,
Auslenkung: $\delta$, Verschiebungsvolumen: $V$ und Ladung: $Q$, siehe
{\bf Abbildung~\ref{abbinternparam}}) für einen einseitig
eingespannten Silizium-Biegebalken, der von einer piezoelektrischen
Dünnschicht bedeckt ist, abgeleitet.
%----------------------- Beginn: Figure-Environment ----------------------
\begin{figure}[ht]
\begin{minipage}[t]{7cm}
\vspace*{4cm}
%\vspace*{0.25cm}
%\begin{center}
% --- Dateiname des Bildes
%\input{abbzsi.tex}
%\setabbzsi
%\end{center}
\caption{\label{abbexternparam}
 Externe Parameter $M, F, p, U$}
%\end{figure}
\end{minipage}
\hfill
%----------------------- Ende: Figure-Environment ----------------------
%----------------------- Beginn: Figure-Environment ----------------------
\begin{minipage}[t]{7cm}
\vspace*{4cm}
%\vspace*{0.25cm}
%\begin{figure}[ht]
%\begin{center}
% --- Dateiname des Bildes
%\input{abbza.tex}
%\setabbza
%\end{center}
\caption{\label{abbinternparam}
 Interne Parameter $\alpha, \delta, V, Q$}
\end{minipage}
\end{figure}

%clearpage

%\vspace*{0.5cm}

%----------------------- Ende: Figure-Environment ----------------------
Die symmetrische Übertragungsmatrix [$e$] beschreibt den Biegezustand
des Balkens bei unterschiedlichen mechanischen und elektrischen
Belastungen und verknüpft die internen mit den externen Parametern:
%
\begin{eqnarray}
  \left(
  \begin{array}{c}
  \alpha \\ \delta \\ V \\ Q
  \end{array} \right) & = &
  \left[
  \begin{array}{llll}
  e_{11} & e_{12} & e_{13} & e_{14} \\
  e_{21} & e_{22} & e_{23} & e_{24} \\
  e_{31} & e_{32} & e_{33} & e_{34} \\
  e_{41} & e_{42} & e_{43} & e_{44}
  \end{array}
 \right]
 \left(
 \begin{array}{c}
 M \\ F \\ p \\ U
 \end{array}
\right)
\end{eqnarray}
%
Fr die Matrixkomponenten $e_{ij}$ gilt:
%
\begin{eqnarray}
[ e ] & = & \frac{A}{K} \cdot
\left [
\begin{array}{cccc}
\displaystyle \frac{12L}{b} &   \displaystyle \frac{6L^{2}}{b} &
\displaystyle 2L^{3}        &   6d_{31}BL \\
\\
\displaystyle               &   \displaystyle \frac{4L^{3}}{b} &
\displaystyle \frac{3L^{4}}{2}            &   3d_{31}BL^{2} \\
\\
 &  & \displaystyle \frac{3L^{5}b}{5} & 3d_{31}BL^{3}b \\
\\
%\displaystyle \frac{2L^{3}}{K} & \displaystyle \frac{3L^{4}}{2K} &
%\displaystyle \frac{3L^{5}b}{5K} & \displaystyle
%\frac{3d_{31}BL^{2}}{K}\\ \\
 & & & C
%\displaystyle \frac{Lb}{h_{p}} \cdot
%\left( \epsilon^{\sigma}_{33}K - d_{31}^{2} h_{Si}
%(S^{Si}_{11} h_{p}^{3} + S^{p}_{11} h_{Si}^{3} \right)
%\displaystyle \frac{6d_{31}BL}{K} & \displaystyle
%\frac{3d_{31}BL^{2}}{K} & \displaystyle \frac{d_{31}BL^{3}b}{K}
%                   & \quad \displaystyle \frac{Lb}{Ah_{p}}
%\cdot \left( \epsilon^{\sigma}_{33} \, - \,
%              \frac{d_{31}^{2} h_{Si} ( \, s^{Si}_{11} hp^{3} \, + \,
%s^{p}_{11} h_{Si}^{3} }{K} \right)
\end{array} \right ]
\end{eqnarray}

mit den Abkürzungen $ A, B, K $:
%
\begin{eqnarray*}
   A & = & S^{Si}_{11} S^{p}_{11} \left( S^{p}_{11} h_{Si}
           + S^{Si}_{11} h_{p} \right )
\\ \\
   B & = & \frac{h_{Si} (h_{Si}  +  h_{p}) }{ S^{p}_{11} h_{Si}
             +  S^{Si}_{11} h_{p}}
\\ \\
   C & = & \frac{Lb}{Ah_{p}} \cdot \left( \epsilon^{\sigma}_{33}K -
    d_{31}^{2} h_{Si} (S^{Si}_{11} h_{p}^{3} + S^{p}_{11} h_{Si}^{3} \right)
\\ \\
   K & = & (S^{Si}_{11})^{2} (h_{p})^{4}  +  4S^{Si}_{11}
                   S^{p}_{11} h_{Si}(h_{p})^{3} \nonumber \\
     &   & + \, 6S^{Si}_{11} S^{p}_{11} (h_{Si})^{2} (h_{p})^{2}  +
                4S^{Si}_{11} S^{p}_{11} h_{p} (h_{Si})^{3} \nonumber \\
     &   & + \, (S^{p}_{11})^{2}(h_{Si})^{4}
 \end{eqnarray*}
wobei:
\begin{eqnarray*}
 L, b                      & : &
  \mbox{L„nge, Breite des Biegebalkens [m]} \\
 h_{Si},h_{p}              & : &
  \mbox{Substratdicke (hier: Silizium), Piezoschichtdicke [m]}   \\
 S^{Si}_{11}, S^{p}_{11}   & : &
  \mbox{Steifigkeitskoeffizient Substrat, Piezoschicht} \, [Pa^{-1}] \\
 d_{31} & : & \mbox{transversaler piezoelektrischer Koeffizient} \, [C/N] \\
%  \left [ \frac{C}{N} \right ] \\
 \epsilon^{\sigma}_{33}    & : &
  \mbox{Dielektrizitätskonstante in E-Feldrichtung} \, [C/Vm]
%  \left [ \frac{C}{Vm} \right ]
\end{eqnarray*}
%
Die maximale Auslenkung $ \delta $ beträgt am Balkenende bei Anlegen
einer Spannung $ U $ an die piezoelektrische Schicht:
\begin{eqnarray}
\label{deltaU}
  \delta & = & e_{24} \cdot U \; = \;
  \left( 3 AB \frac{{L}^2}{K} \right) \cdot d_{31} U
\end{eqnarray}
Die maximal erreichbare Kraft (Klemmkraft, bei  $ \delta \, = \, 0) $,
die der Bimorphbalken ausüben kann, läßt sich durch die
Federsteifigkeit $e_{22}^{-1}$ und unter Verwendung von Gleichung
(\ref{deltaU}) ausdrücken:
\begin{eqnarray}
      F & = & (e_{22})^{-1} \cdot \delta \;
          = \; (e_{22})^{-1} \cdot (e_{24} \, U)
\end{eqnarray}
Der vollständige Zusammenhang zwischen Kraft und elektrischer Spannung
ergibt sich zu:
\begin{eqnarray}
\label{kraftU}
      F & = & \left( \frac{3}{4} \frac{bB}{L} \right) d_{31} \cdot U
\end{eqnarray}
%
In den obigen Ausdrücken gewichten die Konstanten $A$
[$m/Pa^{3}$], B [$m \cdot Pa$] und $K$ [$m^{4}/Pa^{2}$] die
Steifigkeiten mit den entsprechenden Schichtdicken.  Der Zusammenhang
zwischen der Auslenkung $\delta$ bzw. der ausübbaren Kraft $F$ und der
angelegten elektrischen Spannung $U$ ist linear, aufgrund der
zugrundeliegenden Linearität der piezoelektrischen Zustandsgleichungen.
Die Abhängigkeit von
der Balkenlänge $L$ ist bei der Auslenkung quadratisch, bei der Kraft
umgekehrt proportional. \\
Nachfolgend soll der Einfluß der Wandlergeometrie detaillierter
untersucht werden.
Variiert wurden die Balkenlänge $L$, die Substratdicke $h_{Si}$ und die
Piezoschichtdicke $h_{p}$.
Die Steifigkeitskoeffizienten vom Siliziumsubstrat $ S_{11}^{Si} $ und
der Piezoschicht $S_{11}^{p}$, sowie der transversale piezoelektrische
Koeffizient $d_{31}$ wurden für ZnO der Tabelle~\ref{tabpiezoelektrika}
entnommen. In {\bf Tabelle~\ref{tabauslkraftkonst}} sind die
Proportionalitätskonstanten $e_{24}$~[m/V]
zwischen der Auslenkung und der angelegten Spannung sowie
die Proportionalitätskonstanten $e_{24}/e_{22}$~[N/V] zwischen der Kraft
und der angelegten Spannung zusammengefaßt. Die Balkenbreite $b$ beträgt
bei den Abschätzungen 50~$\mu$m.
%----------------------- Beginn: table ---------------------------
\begin{table}[htb]
\caption{\label{tabauslkraftkonst}
 Auslenkungs-\ und Kraft-Konstanten $e_{24}$ bzw.\ $ e_{24}/e_{22} $ fr
 einen Silizium-Balken mit einer ZnO-Schicht}
\begin{center}
\begin{tabular}{|l|l||c|c|c|c|} \hline
$h_{Si}$ & $h_{ZnO}$ & L = 0,1 mm  & 1 mm  & 10 mm & Technologie \\
\hline \hline
$ 1 \, \mu m $  &  $ 1 \, \mu m $  &  19,2 nm/V  & $ 1,92 \mu m/V $ &
$ {\bf 192 \mu  m/V} $ & OFM \\  \cline{3-5}
   &  & $ 0,25  \mu N/V $ & $ 0,02  \mu N/V $ & $ 2,46 nN/V $ & \\
\hline \hline
$  10 \, \mu m $  & $ 5 \, \mu m $  & $ 0,45 nm/V $  & $ 45,3 nm/V $
& $ 4,53  \mu m/V $ &  \\  \cline{3-5}
   &  & $ 2,45  \mu N/V $  &  $ 0,25  \mu N/V $  &  $ 0,03  \mu N/V $
   & \\
\cline{1-5}
$ 100 \, \mu m $  &  $ 10 \, \mu m $  & $ 0,01 nm/V  $ & $ 1,13 nm/V $
& $ 0,11 \mu m/V $ & BMM \\  \cline{3-5}
   & & ${\bf 24,3 \mu N/V}$  &  $2,43 \mu N/V$ & $0,24 \mu N/V $ & \\
\hline
\end{tabular}
\end{center}
\end{table}
%----------------------- Ende: table ---------------------------
Legt man eine typische Feldstärke\footnote{Dieser Wert konnte durch die
Messungen (siehe Kapitel~\ref{zno}) bestätigt werden.} von $E$~=~10~V/$\mu$m
bei ZnO zugrunde \cite{Smi92b}, so ergibt sich eine maximal erreichbare
Auslenkung von 1,92~mm bei den Balkenabmessungen $L$~=~10~mm und
$h_{Si}$~=~$h_{ZnO}$~=~1~$\mu$m (Variante 1). Die maximal ausübbare
Klemmkraft
beträgt 10~V/$\mu m \cdot$10~$\mu m \cdot$24,3~$\mu$N~$\approx$~2,4~mN
für einen Balken mit den
Abmessungen $L$~=~0,1~mm, $h_{Si}$~= 100~$\mu$m und $h_{ZnO}$~=10~$\mu$m
(Variante~2).
Je länger der Balken ist, desto größer werden die Auslenkungen, während
die ausübbaren Kräfte abnehmen. Die Piezoschichtdicke $h_{p}$
hat keinen Einfluß auf die Kräfte, lediglich die Balkendicke $h_{Si}$.
Die Kraftwirkung läßt sich weiter steigern, wenn die Balkenbreite $b$
vergrößert wird:
\begin{eqnarray}
\label{Fskal}
   F & \sim & \left( \frac{b \cdot h_{Si}}{L} \right)
\end{eqnarray}
Im Gegensatz zu den Kräften geht bei den Auslenkungen die effektive
Gesamtdicke des Balkens $ h_{eff} = f(h_{Si}, h_{p}) $ quadratisch
im Nenner ein:
\begin{eqnarray}
\label{dskal}
   \delta & \sim & \left( \frac{L}{h_{eff}} \right)^{2}
\end{eqnarray}
In der {\bf Tabelle~\ref{tabvglalnznopzt}}
sind die beiden Extremfälle (Variante~1: langer
dünner Balken und Variante~2: kurzer dicker Balken) analytisch mit den
Materialien AlN und PZT (siehe Materialdaten aus Tabelle~2.3) berechnet
und verglichen worden. Während die Steifigkeitskoeffizienten $S_{11}$
von AlN, ZnO und PZT kaum differieren (Verhältnis: 1,0 / 2,24 / 4,36),
unterscheiden sich die piezoelektrischen Koeffizienten $d_{31}$
sehr stark (Verhältnis: 1,0 / 2,6 / 80).
Die maximalen elektrischen Feldstärken betragen bei AlN etwa 500~V/$\mu$m
\cite{Fra88, Ger81} und bei Piezokeramiken ist nur der Wert für gesintertes
Bulk-Material bekannt, das eine Feldstärke von $E_{PZT}$~=~0,5~V/$\mu$m
im linearen Bereich (Kleinsignalverhalten) und maximal etwa 2--3~V/$\mu$m
im Sättigungsbereich (Großsignalverhalten) zuläßt \cite{VIB}. Außerdem ist zu
beachten, daß, im Gegensatz zu Dünnschichten, gesinterte Piezokeramiken eine
Depolarisationsspannung aufweisen, die ihrerseits weit unterhalb der
eigentlichen Durchbruchspannung liegen kann. Die nachfolgenden Angaben
beziehen sich daher auf eine Vergleichsfeldstärke von 1~V/$\mu$m.
%----------------------- Beginn: table ---------------------------
\begin{table}[htb]
\caption{\label{tabvglalnznopzt}
 Einfluß des Schichtsystems auf die Auslenkungen und Kräfte}
\begin{center}
\begin{tabular}{|l||c|c|c|c||l|} \hline
Größe: & AlN & ZnO & PZT & Verhältnis & Dimension \\
\hline \hline
$ \delta_{max} \; [\mu m] $  & 72,1 & 192 & 5813 & 1/2,7/81 & Variante 1 \\
\cline{1-5}
$ F_{max} \; [nN] $  & 1,34 & 2,46 & 51,8 & 1/1,8/39 & U = 1 V
\\ \hline \hline
$ \delta_{max} \; [nm] $  & 0,007 & 0,01 & 0,21 & 1/1,4/30 & Variante 2 \\
\cline{1-5}
$ F_{max} \; [\mu N] $  & 19,2 & 24,3 & 406 & 1/1,3/21 & U = 1 V \\
\hline
\end{tabular}
\end{center}
\end{table}
%----------------------- Ende: table ---------------------------
Unter Zugrundelegung der gleichen Ansteuerspannung ermöglicht PZT die
größte Auslenkung (ca. 5,8~mm) und Stellkraft (ca. 19,2~$\mu$N). In der
5.~Spalte der Tabelle sind die Verhältnisse der bei gleicher Geometrie
und Spannung erreichbaren Werte dargestellt. Beachtet man jedoch die
unterschiedlichen maximalen Feldstärken, so differieren sie um mehrere
Größenordnungen ($E_{AlN} / E_{ZnO} / E_{PZT}) = (1000 / 20 / 1$),
so daß im Dünnschichtbereich die Verwendung von AlN neben ZnO sehr
interessant erscheint. Bei einer theoretischen Ansteuerspannung von
500~V/$\mu$m würden sich für AlN ein maximale Auslenkung von etwa 36~mm
und eine maximale Stellkraft von etwa 96~$\mu$N ergeben.


{\em Bemerkungen zum piezoelektrischen Antrieb:}
\begin{itemize}
\item Optimierung der ZnO-Eigenschaften, z.B.\ durch geeignete Wahl der
Piezoschichtdicke und eine gezielte Einstellung der Schichtspannung,
zwecks Erhöhung des elektromechanischen Kopplungsfaktors
(siehe Kapitel~\ref{schichtdickenabhaengigkeit}).
\item Flächenselektive Anregung der Bauelemente ist durch eine bipolare
Ansteuerung gegeben, so daß Kontraktionen {\em und} Dilatationen
gleichzeitig auf der Bauelement\-oberfläche erzeugt werden können.
Hieraus resultiert eine Verbesserung der Modenselektivität bei resonanten
Sensoren und eine Erhöhung der Bauteilhübe bei Aktoranwendungen
(siehe Kapitel~\ref{modenselektivitaet}).
\item Eine Prozeßintegration des ZnO ist schwierig, da eine Passivierung
gegenüber technologischen Nachfolgeprozessen und äußeren Umwelteinflüssen
notwendig ist.
\item Bei hohen Ansteuerspannungen stellen sich Degradationsmechanismen ein,
die zu irreversiblen Änderungen der Schichteigenschaften bzw.\ sogar zur
Zerstörung des Schichtsystems führen können.
\end{itemize}

\vspace*{0.5cm}

Bei der elektrothermischen Anregung handelt es sich prinzipiell um
den gleichen Antriebsmechanismus wie bei piezoelektrischen Dünnschichten,
indem in der Bimorphstruktur ein Biegemoment induziert wird.
Während beim piezoelektrischen Effekt gemäß Abbildung~\ref{abbnye} die
intensive Variable die elektrische Feldstärke $\vec E$
($|\vec E|$ = elektrische Spannung $U$ / Piezoschichtdicke $h_{p}$) und der
Kopplungskoeffizient der piezoelektrische Modul (hier: $d_{31}$) ist,
handelt es sich bei der elektrothermischen Anregung um die
Temperaturdifferenz $\Delta T$ und die Differenz der
Wärmeausdehnungskoeffizienten $\Delta \alpha$ zwischen
dem Siliziumsubstrat und der metallischen Dünnschicht.
Unter Ausnutzung dieser Analogie und dem Umstand, daß
$U \, \approx \, E \cdot h_{p}$, läßt sich das Skalierungsverhalten
gemäß den Gleichungen (\ref{Fskal}) und (\ref{dskal}) auf die
elektrothermische Anregung direkt übertragen:
\begin{eqnarray}
\label{dvont}
  \delta & \sim &
%\frac{L^{2}}{r(\Delta T)} \; \sim \;
  \left( \frac{L^{2}}{h} \right) \cdot \Delta \alpha \Delta T  \\
\label{fvont}
   F  & \sim & \left( \frac{b \cdot h^{2}}{L} \right)
               \cdot \Delta \alpha \Delta T
\end{eqnarray}

%Im Folgenden sind die Vor- und Nachteile des piezoelektrischen und
%elektrothermischen Antriebskonzeptes mittels Dünnschichten zusammengefaßt:

{\em Bemerkungen zum elektrothermischen Antrieb:}
\begin{itemize}
\item Neben Aluminium kann Gold verwendet werden, das zwar einen
geringeren Wärmeausdehnungskoeffizienten aufweist, aber aufgrund des
Edelmetallcharakters keine Passivierung erfordert.
\item Die technologische Kompatibilität mit der Siliziumtechnologie ist
sehr gut. Zusätzlich ist eine flexible technologische Realisierbarkeit
durch diffundierte Widerstände in Silizium bzw.\ NiCr-Dünnschichtwiderstände
in Metallfilmtechnologie gegeben \cite{Bar93}.
\item Im Gegensatz zum piezoelektrischen Antriebsprinzip ist keine bipolare
Ansteuerbarkeit gegeben, da nur Temperaturerhöhungen möglich sind, die zu
Wärmedehnungen führen.
\item Bei hohen Temperaturen kann es zur Ablösung oder zur Zerstörung
des Schichtsystems kommen.
\item Die Gesamtwärmebilanz beim thermischen Antrieb ist schwer zu
kontrollieren, da Wärmesenken einen sehr starken Einfluß ausüben.
\end{itemize}
